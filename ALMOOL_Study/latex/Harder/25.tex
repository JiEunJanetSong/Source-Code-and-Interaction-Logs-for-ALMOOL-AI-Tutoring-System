\SetValue{SectionAB}{}\SetValue{MainChapter}{}\SetValue{SubChapter}{}\SetValue{Contents}{%
  
The equation \( 2x^2 + 5x - 7 = 0 \) has two distinct solutions. What is the value of the smaller solution subtracted from the larger solution? 

}\SetValue{Solution}{%

Let's solve the quadratic equation step by step using the quadratic formula to find the difference between the larger and smaller solutions.\begin{alignat*}{2}
x &= \frac{-b \pm \sqrt{b^2 - 4ac}}{2a} \qquad && \{ \text{Quadratic formula} \} \\
  &= \frac{-5 \pm \sqrt{5^2 - 4(2)(-7)}}{2(2)} \qquad && \{ \text{Substitute values of } a, b, c \} \\
  &= \frac{-5 \pm \sqrt{25 + 56}}{4} \qquad && \{ \text{Simplify inside the square root} \} \\
  &= \frac{-5 \pm \sqrt{81}}{4} \qquad && \{ \text{Simplify further} \} \\
  &= \frac{-5 \pm 9}{4} \qquad && \{ \text{Square root of 81 is 9} \}
\end{alignat*}
Now, we compute both possible solutions:\begin{alignat*}{2}
x_1 &= \frac{-5 + 9}{4} = \frac{4}{4} = 1 \qquad && \{ \text{First solution with } + \} \\
x_2 &= \frac{-5 - 9}{4} = \frac{-14}{4} = -\frac{7}{2} \qquad && \{ \text{Second solution with } - \}
\end{alignat*}
Finally, we subtract the smaller solution from the larger solution:\begin{alignat*}{2}
1 - \left( -\frac{7}{2} \right) &= 1 + \frac{7}{2} = \frac{2}{2} + \frac{7}{2} = \frac{9}{2} \qquad && \{ \text{Final subtraction} \}
\end{alignat*}
Thus, the value of the larger solution subtracted from the smaller solution is $\fbox{ $\frac{9}{2}$ }$.

}\SetValue{Answer}{%
9/2
}
\ProcessDATA


