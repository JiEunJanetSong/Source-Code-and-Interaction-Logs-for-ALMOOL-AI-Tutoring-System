\SetValue{SectionAB}{}\SetValue{MainChapter}{}\SetValue{SubChapter}{}\SetValue{Contents}{%

Alex sold \( x \) bicycles in 2016. The number of bicycles he sold in 2017 was $150\%$ greater than in 2016, and the number of bicycles he sold in 2018 was $20\%$ greater than in 2017. Which of the following expressions represents the number of bicycles Alex sold in 2018?

\ansFOURsT{%start
	  \( (0.20)(2.5 x) \)	}{%<--- (A)
	  \( (1.2)(1.5 x) \)	}{%<--- (B)
	  \( (1.2)(2.5 x) \)	}{%<--- (C)
	  \( (1.5)(1.2 x) \) }%<---- (D)

}\SetValue{Solution}{%

Let's calculate the total number of bicycles sold by Alex in 2018 based on the given percentage increases.
\begin{alignat*}{2}
\text{2017 Sales} 
&= 2.5x \qquad && \{ 150\% \text{ increase from 2016} \} \\
\text{2018 Sales} 
&= 1.2 \times 2.5x  \qquad && \{ 20\% \text{ increase from 2017} \}
\end{alignat*}
We calculated that Alex sold 2.5 times the number of bicycles in 2017 compared to 2016, and in 2018, he sold 1.2 times more than in 2017.
\begin{alignat*}{2}
\text{2018 Sales} 
&= 1.2 \times 2.5x  \qquad && \{ \text{Apply the percentage increases step by step} \} \\
&= (1.2)(2.5x) && \{ \text{Final expression for 2018 sales} \}
\end{alignat*}
Thus, the correct expression representing the number of bicycles sold in 2018 is $\fbox{C) $(1.2)(2.5x)$}$.

}\SetValue{Answer}{%
C
}
\ProcessDATA


