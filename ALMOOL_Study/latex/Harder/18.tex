\SetValue{SectionAB}{}\SetValue{MainChapter}{}\SetValue{SubChapter}{}\SetValue{Contents}{%

$$
\begin{aligned}
& 5x + r = 8y + 4 \\
& 3y + s = 4x + 6
\end{aligned}
$$

In the equations above, \( r \) and \( s \) are constants. If \( r + s = 10 \), which of the following statements is true?

\ansFOURsT{%start
	 	 \( y \) minus \( x \) is 10  }{%<--- (A)
	 	 \( x \) minus \( y \) is 10  }{%<--- (B)
	 	 \( y \) is one-fifth of \( x \)  }{%<--- (C)
		 \( x \) is one-fifth of \( y \)  }%<---- (D)

}\SetValue{Solution}{%

We are given two equations. We will begin by adding them to simplify and find the relationship between \(x\) and \(y\).\begin{alignat*}{2}
(5x + r) + (3y + s) & = (8y + 4) + (4x + 6) \qquad && \text{(Add both equations)} \\
5x + 3y + r + s & = 8y + 4x + 10  && \text{(Simplify both sides)}
\end{alignat*}
Next, substitute \(r + s = 10\) into the equation and simplify further.\begin{alignat*}{2}
5x + 3y + 10 & = 8y + 4x + 10 \qquad && \text{(Substitute \(r + s = 10\))} \\
5x + 3y & = 8y + 4x  && \text{(Subtract 10 from both sides)} \\
x & = 5y  && \text{(Rearrange terms and simplify)}
\end{alignat*}
Since we found that \(x = 5y\), we can conclude that \(y\) is one-fifth of \(x\). Therefore, the correct answer is:

$\fbox{C) $y$ is one-fifth of $x$}$.

}\SetValue{Answer}{%
C
}
\ProcessDATA


