\SetValue{SectionAB}{}\SetValue{MainChapter}{}\SetValue{SubChapter}{}\SetValue{Contents}{%

The frictional force \( F \) on an object, in Newtons (N), is calculated by multiplying the normal force \( N \), in Newtons, by the coefficient of friction \( \mu \). The coefficient of static friction on ice is \( 0.1 \) and the coefficient of static friction on dry concrete is \( 0.75 \).

A crate weighing \(200\) Newtons is placed on an icy surface. It is then moved to a dry concrete surface. What is the difference in frictional force between the icy surface and the dry concrete surface acting on the crate?

\ansFOURs{%start 
10 N}{%<--- (A) 
50 N}{%<--- (B) 
130 N}{%<--- (C) 
150 N}%<---- (D)

}\SetValue{Solution}{%

The frictional force \( F \) depends on the coefficient of friction \( \mu \) and the normal force \( N \), given by the equation:
\[
F = \mu \times N
\]

1. Frictional force on ice:
   \begin{alignat*}{2}
   F_{\text{ice}} &= \mu_{\text{ice}} \times N \qquad && \{ \text{Using } \mu_{\text{ice}} = 0.1, N = 200 \, \text{N} \} \\
   &= 0.1 \times 200 \\
   &= 20 \, \text{N} 
   \end{alignat*}

2. Frictional force on dry concrete:
   \begin{alignat*}{2}
   F_{\text{concrete}} &= \mu_{\text{concrete}} \times N \qquad && \{ \text{Using } \mu_{\text{concrete}} = 0.75, N = 200 \, \text{N} \} \\
   &= 0.75 \times 200 \\
   &= 150 \, \text{N} 
   \end{alignat*}

3. Difference in frictional force:
   \begin{alignat*}{2}
   F_{\text{difference}} &= F_{\text{concrete}} - F_{\text{ice}} \qquad && \{ \text{Subtracting the two forces} \} \\
   &= 150 - 20 \\
   &= 130 \, \text{N} 
   \end{alignat*}

Thus, the difference in the frictional force is found to be 130 N.

$\fbox{C) $130 N$}$

}\SetValue{Answer}{%
D
}
\ProcessDATA


