\SetValue{SectionAB}{}\SetValue{MainChapter}{}\SetValue{SubChapter}{}\SetValue{Contents}{%
  
$$ \begin{array}{r} 10 x + 4 y = 80 \\ c x + d y = 20 \end{array} $$ In the system of equations above, \( c \) and \( d \) are constants. If the system has no solutions, what is the value of \( \frac{d}{c} \) ?

}\SetValue{Solution}{%

To solve for \( \frac{d}{c} \), we start by making the constants on the right-hand side of the equations easier to compare.
\begin{alignat*}{2}
4 \times (cx + dy) &= 4 \times 20 \qquad && \{ \text{Multiply the second equation by 4} \} \\
4cx + 4dy &= 80  && \{ \text{Resulting equation after multiplying} \}
\end{alignat*}
Now, we compare the coefficients of \(x\) and \(y\) in both equations. The first equation remains as:
\begin{alignat*}{2}
10x + 4y &= 80 \qquad && \{ \text{Original first equation} \} \\
\frac{10}{4c} &= \frac{4}{4d} \qquad && \{ \text{Equating coefficients of the terms} \}
\end{alignat*}
Simplifying both sides to find the proportionality:
\begin{alignat*}{2}
\frac{10}{c} &= \frac{4}{d} \qquad && \{ \text{Simplified equation} \} \\
10d &= 4c \qquad && \{ \text{Cross-multiplying} \} \\
\frac{d}{c} &= \frac{2}{5} \qquad && \{ \text{Solving for } \frac{d}{c} \}
\end{alignat*}
Thus, the value of \( \frac{d}{c} \) is $\fbox{ $\frac{2}{5}$ }$.

}\SetValue{Answer}{%
2/5
}
\ProcessDATA


