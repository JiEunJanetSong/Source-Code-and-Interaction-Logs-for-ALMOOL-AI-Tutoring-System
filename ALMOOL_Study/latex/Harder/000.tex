```source


```

```template1
opening paragrpah
\begin{alignat*}%display eq 1
eq
&= eq  \qquad && \{ reason \} \\
&= eq   && \{ reason \} \\
..
&= eq   && \{ reason \}
\end{alignat*}
connecting paragrpah(brief explanation or justification for the following display equation)
\begin{alignat*}%display eq 2
eq
&= eq  \qquad && \{ reason \} \\
&= eq  && \{ reason \} \\
..
&= eq   && \{ reason \}
\end{alignat*}
concluding paragraph
```

```template2 
\begin{enumerate}[label={\textbf(\Alph*)}]
\item (A) Contents of A: \textbf{True or False of A}. \\ explanation
\item (B): \textbf{True or False of B}. : True or False of B) \\  explanation
\item  (C): \textbf{True or False of C}. : True or False of C) \\  explanation
\item (D) Contents of D: \textbf{True or False of D}. \\ explanation
\end{enumerate}
concluding paragraph
```
 
Source의 solution 부분을 다음 규칙에 따라 작성하라.
1. template1 또는 template 2 중 더 적절한 것에 맞추어 작성한다.
2. opening, connecting, explanation, concluding은 한 두 문장 정도로만 간결하게 작성한다.
3.  ```solution과  ``` 사이에 LaTeX 문법으로 작성한다. 마크다운의 # heading은 사용하지 않는다.
4. 마지막에 답을 \boxed{\text{option)} math}에 적어준다. 
5. 영어로 말한다.
