\SetValue{SectionAB}{}\SetValue{MainChapter}{}\SetValue{SubChapter}{}\SetValue{Contents}{%

A local gym conducted a survey to determine whether its members were satisfied with the cleanliness of the facilities. The gym sent surveys to 600 randomly selected members who visited the gym in the past year, and 420 members responded. Which of the following factors makes it least likely that a reliable conclusion can be drawn about the cleanliness satisfaction of all gym members?

\ansFOURsT{%start
	  The survey distribution method}{%<--- (A)
	  The number of people who responded}{%<--- (B)
	  The size of the sample}{%<--- (C)
	  The time elapsed since the visit}%<---- (D)

}\SetValue{Solution}{%

We will analyze the factors to determine which one makes it least likely that a reliable conclusion can be drawn.\begin{alignat*}{2}
\text{A)} \text{The survey distribution method} & \quad && \{ \text{The method could introduce some bias, but it's not the most important factor.} \} \\
\text{B)} \text{The number of people who responded} & \quad && \{ \text{Non-response bias could skew results, making it a significant issue.} \} \\
\text{C)} \text{The size of the sample} & \quad && \{ \text{The sample size is large and representative, so it's unlikely to be a major issue.} \} \\
\text{D)} \text{The time elapsed since the visit} & \quad && \{ \text{While the time could affect accuracy, it is not the primary concern.} \} 
\end{alignat*}
After evaluating the options, it is clear that the most critical factor affecting the reliability of the survey is non-response bias.
\begin{alignat*}{2}
\text{Conclusion:} & \quad &&  \text{The factor that makes it least likely to draw a reliable conclusion is } \boxed {\text{D) The time elapsed since the visit.}} 
\end{alignat*}
}\SetValue{Answer}{%
Answer Goes Here
}
\ProcessDATA


