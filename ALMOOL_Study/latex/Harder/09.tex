\SetValue{SectionAB}{}\SetValue{MainChapter}{}\SetValue{SubChapter}{}\SetValue{Contents}{%
  
Brentwood High School has \( 850 \) sophomores and juniors. Out of them, \( 380 \) are enrolled in one or more Advanced Placement (AP) courses. Among these AP students, \( 90 \) are in AP Chemistry, \( 70 \) are in AP World History, and \( 30 \) are in both AP Chemistry and AP World History. Approximately what percent of the sophomores and juniors at Brentwood High School take AP courses other than Chemistry and World History? 

\ansFOURsT{%start 
$10 \%$}{%<--- (A) 
$20 \%$}{%<--- (B) 
$30 \%$}{%<--- (C) 
$40 \%$}%<---- (D)

}\SetValue{Solution}{%

Given that Brentwood High School has 850 sophomores and juniors, and 380 of them are enrolled in one or more AP courses, we need to find the percentage of students taking AP courses other than Chemistry and World History.

\begin{alignat*}{2}
\text{Number of students in AP Chemistry or World History} &= 90 + 70 - 30 \qquad && \{ \text{Inclusion-Exclusion Principle} \} \\
&= 130 && \{ \text{Total in AP Chemistry, AP World History, or both} \}
\end{alignat*}

Next, we find the number of students taking AP courses other than Chemistry and World History.

\begin{alignat*}{2}
\text{Number of students in other AP courses} &= 380 - 130 \qquad && \{ \text{Subtracting AP Chemistry and World History students} \} \\
&= 250 && \{ \text{Students in other AP courses} \}
\end{alignat*}

Now, we calculate the percentage of these students out of the total number of sophomores and juniors.

\begin{alignat*}{2}
\text{Percentage} &= \left( \frac{250}{850} \right) \times 100 \qquad && \{ \text{Finding the percentage} \} \\
&\approx 29.41\% && \{ \text{Rounding to the nearest whole number} \} \\
&\approx 30\% && \{ \text{Final approximation} \}
\end{alignat*}

Thus, the percentage of students taking AP courses other than Chemistry and World History is approximately:
\[
\fbox{C) $30\%$}
\]

}\SetValue{Answer}{%
A
}
\ProcessDATA


