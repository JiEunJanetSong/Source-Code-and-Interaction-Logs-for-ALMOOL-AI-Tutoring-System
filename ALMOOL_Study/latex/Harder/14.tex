\SetValue{SectionAB}{}\SetValue{MainChapter}{}\SetValue{SubChapter}{}\SetValue{Contents}{%

%\begin{tabular}{|l|c|c|}
%    \hline
%    Name of Skyscraper & $\begin{array}{c}\text { Floors } \\ \text { (Count) }\end{array}$ & $\begin{array}{c}\text { Height } \\ \text { (Meters) }\end{array}$ \\ \hline
%    Tower Alpha & 100 & 450 \\ \hline
%    Building Beta & 80 & 360 \\ \hline
%    Skyscraper Gamma & 70 & 315 \\ \hline
%    Edifice Delta & 60 & 270 \\ \hline
%    Monument Epsilon & 50 & 225 \\ \hline
%\end{tabular}

<p align='center'><img src="14.png"></p>

The chart above shows the names, number of floors, and heights of $5$ of the tallest skyscrapers in a city. 

 The skyscraper with the greatest height is what percent taller than the skyscraper with the lowest height? 

\ansFOURs{%start 
50}{%<--- (A) 
80}{%<--- (B) 
100}{%<--- (C) 
200}%<---- (D)
		 
}\SetValue{Solution}{%

The problem asks for the percentage by which the tallest skyscraper is greater in height than the shortest skyscraper.

\begin{alignat*}{2}
\text{Difference in height} &= 450 - 225 && \{ \text{Subtracting the height of Monument Epsilon from Tower Alpha} \} \\
&= 225 \, \text{meters} && \{ \text{The difference in height} \}
\end{alignat*}

Now, we calculate the percentage increase based on the height of the shorter skyscraper:

\begin{alignat*}{2}
\text{Percentage increase} &= \left( \frac{225}{225} \right) \times 100 && \{ \text{Dividing the difference by the height of the shorter skyscraper} \} \\
&= 100\% && \{ \text{Resulting percentage} \}
\end{alignat*}

Thus, the tallest skyscraper is 100\% taller than the shortest skyscraper.

The correct answer is: $\fbox{C) $100$}$.

}\SetValue{Answer}{%
B
}
\ProcessDATA








