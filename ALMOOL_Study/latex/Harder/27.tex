\SetValue{SectionAB}{}\SetValue{MainChapter}{}\SetValue{SubChapter}{}\SetValue{Contents}{%
  
If \( 6x + 6y = 20 \) and \( x^2 - y^2 = -\frac{5}{12} \), what is the value of \( 2x - 2y \)? 

\ansFOURs{%start
	 	 -\frac{1}{4}  }{%<--- (A)
	 	 -\frac{1}{8}  }{%<--- (B)
	 	 \frac{1}{4}  }{%<--- (C)
		 \frac{1}{8} }%<---- (D)

}\SetValue{Solution}{%

We are given two equations and need to simplify and solve for \( 2x - 2y \).\begin{alignat*}{2}
6x + 6y &= 20  \qquad && \{ \text{Given equation} \} \\
x + y &= \frac{10}{3}  && \{ \text{Dividing by 6 to simplify} \}
\end{alignat*}
Next, we apply the difference of squares identity to the second equation.\begin{alignat*}{2}
x^2 - y^2 &= -\frac{5}{12} \qquad && \{ \text{Given equation} \} \\
(x - y)(x + y) &= -\frac{5}{12}  && \{ \text{Applying difference of squares identity} \} \\
(x - y)\left(\frac{10}{3}\right) &= -\frac{5}{12} && \{ \text{Substituting } x + y = \frac{10}{3} \}
\end{alignat*}
Now, solve for \( x - y \) by dividing both sides by \( \frac{10}{3} \).\begin{alignat*}{2}
x - y &= \frac{-\frac{5}{12}}{\frac{10}{3}} \qquad && \{ \text{Dividing both sides} \} \\
x - y &= -\frac{1}{8} && \{ \text{Simplifying the fraction} \}
\end{alignat*}
Finally, we multiply both sides by 2 to find \( 2x - 2y \).\begin{alignat*}{2}
2x - 2y &= 2 \times \left(-\frac{1}{8}\right) \qquad && \{ \text{Multiplying both sides by 2} \} \\
2x - 2y &= -\frac{1}{4} && \{ \text{Simplifying} \}
\end{alignat*}
Thus, the value of \( 2x - 2y \) is $\fbox{A) $-\frac{1}{4}$}$.

}\SetValue{Answer}{%
A
}
\ProcessDATA


