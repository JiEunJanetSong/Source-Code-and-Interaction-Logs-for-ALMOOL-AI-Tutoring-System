\documentclass[showtrims,svgnames,chapter,openany,oneside]{oblivoir}
\usepackage{ibmath}%수식 명령 설정파일
\usepackage{memtikzpagenodes}% 페이지 레이아웃
\usepackage{SATdesign}% 디자인 설정파일
\usepackage{bigints} %큰 적분기호
\usepackage{dhucs-enumitem}
\usepackage{tabularray}
\usepackage{bookmark}
\usepackage{multirow}

%\usepackage{amsmath}
%%%%%%
\newcommand{\autoinput}{\input{AMC\expandafter\AMClevel/\expandafter\AMCyear AMC\expandafter\AMClevel\expandafter\AMCtype.tex}}
\newcommand\AMCyear{2001}%
\newcommand\AMClevel{10}%
\newcommand\AMCtype{B}%
%%%%%%%%%%%% 줄바꿈 or 페이지바꿈 설정하기
\newcommand{\Jnewpage}{%\JNewpage라는 이름의 기존에 없던 새로운 명령을 만들고, 그 명령을 입력하면 무엇이 실행될지 명시
%\vskip30pt %%% 줄바꾸려면 주석해제
 %%% 페이지바꾸려면 주석해제
}
%%%%%%%%%%% 불러올 파일 설정
%%%%%%%%%%% 머릿말 설정\text{a parabola tangent to the }x\text{-axis} 
\headersetting{%left start
\colormouth[\MYCOLOR]{ALMOOL-Log-P05}%{USSCE-2018-M$_\textbf{G1}$}% PDF출력시 왼쪽 head
}{%middle start 
%\today%middle contents
}{%{right start
\colormouth[\MYCOLOR]{EOS-ResearchTeam}%right contents
}

\begin{document}\pagestyle{mathclass2021}

%%% 어떤파일 부를지

%\input{AMC12/2000AMC12-}

%Problem 문제지 Solution 해설지 Teacher 문제+해설
\SetValue{mainTitle}{AlMool-EOS-GFMaP}%%%%왼쪽상단
\SetValue{Book}{Problem}
%\SetValue{Book}{Solution}
%\SetValue{Book}{Teacher}

%\begin{comment}
\newpage 
\null\vfill 
\hfill {\color{\MYCOLOR}\bfseries User Log Analysis}\hfill\null%PDF출력시 표지
\vfill\null

\newpage 
\SetValue{examTitle}{User Log Analysis}%test체크박스
\SetValue{examTime}{2700}
\setcounter{SATQuestionNum}{0}
\SetValue{examMode}{test}
\SetValue{examMode}{study}
\SetValue{ReferenceSwitch}{0}
\SetValue{referenceImage}{DB/Calculus/Stewart_single_7th/Reference/3-3.png}
\SetValue{NoticeSwitch}{0}%%%주관식 공지사항 표시여부

\newpage
\setcounter{SATQuestionNum}{0}

\begin{GPTQ}
    왜 c가 답인지 한국어로 설명해줘 
\end{GPTQ}

\begin{GPTA}
    물론, 문제에서 주어진 표현식을 살펴보겠습니다. $6 x^8 y^2+12 x^2 y^2$ 라는 표현식이 있습니다.
    이 표현식은 각 항의 공퉁 인수를 묶어낼 수 있습니다. 두 항 모두 $6 x^2 y^2$ 라는 공통 인수를 가지고 있습니다. 이를 묶어내면 다음과 같이 쓸 수 있습니다:
    
    $$
    6 x^2 y^2\left(x^6\right)+6 x^2 y^2(2)
    $$
    
    
    이 식을 좀 더 간단하게 표현하면:
    
    $$
    6 x^2 y^2\left(x^6+2\right)
    $$
    
    
    따라서 위의 과정을 통혜 문제에서 주어진 선택지 중 $6 x^2 y^2\left(x^6+2\right)$ 에 해당하는 (C) 선택지가 정 답이 됩니다. 공퉁 인수를 묶어념으로써 이렇게 간단하게 표현할 수 있다는 점이 중요합니다. 
\end{GPTA}

%


\newpage
\setcounter{SATQuestionNum}{1}



\newpage
\setcounter{SATQuestionNum}{2}



\newpage%0227
\setcounter{SATQuestionNum}{3}


\newpage
\setcounter{SATQuestionNum}{4}


\newpage
\setcounter{SATQuestionNum}{5}



\newpage
\setcounter{SATQuestionNum}{6}



\newpage
\setcounter{SATQuestionNum}{7}



\newpage
\setcounter{SATQuestionNum}{8}




\newpage
\setcounter{SATQuestionNum}{9}


\newpage
\setcounter{SATQuestionNum}{10}



\newpage
\setcounter{SATQuestionNum}{11}




\newpage
\setcounter{SATQuestionNum}{12}






\newpage
\setcounter{SATQuestionNum}{13}







\newpage
\setcounter{SATQuestionNum}{14}







\newpage
\setcounter{SATQuestionNum}{15}


\newpage
\setcounter{SATQuestionNum}{16}


\newpage
\setcounter{SATQuestionNum}{17}


\newpage
\setcounter{SATQuestionNum}{18}


\end{document} 