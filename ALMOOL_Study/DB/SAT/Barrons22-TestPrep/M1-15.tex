\SetValue{SectionAB}{}\SetValue{MainChapter}{}\SetValue{SubChapter}{}\SetValue{Contents}{%

College football programs are permitted to pay a maximum of 1 head coach, 9 assistant coaches, and 2 graduate assistant coaches. If ABC University wishes to have at least 1 coach for every 4 players, which of the following systems of inequalities expresses the total number of coaches, $C$, and total number of players, $P$, possible?

\ansFOURsT{%start
 $C=12$ and $P \leq 4 C$}{%<--- (A)
 $C \leq 12$ and $P \leq 4 C$}{%<--- (B)
 $C \leq 10$ and $P \leq 3 C$}{%<--- (C)
 $C=1$ and $P \leq 4$}%<---- (D)

}\SetValue{Solution}{%

Answer: \textbf{(B)} 

The programs are permitted a maximum of 12 coaches. So $C$ must be equal to or less than 12. Choice (B) is the only option with this statement. All of the other options do not allow for the full range of possible values for the number of coaches. Moreover, the number of players will be limited by the number of coaches if the university is to maintain a ratio of a maximum of 4 players per coach. By making the number of players less than or equal to 4 times the number of coaches, $P \leq 4 C$, the university will ensure that it has at least 1 coach for every 4 players.

}\SetValue{Answer}{%
B
}
\ProcessDATA


