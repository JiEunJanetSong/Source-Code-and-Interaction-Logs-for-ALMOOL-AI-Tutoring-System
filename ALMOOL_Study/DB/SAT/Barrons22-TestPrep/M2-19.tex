\SetValue{SectionAB}{}\SetValue{MainChapter}{}\SetValue{SubChapter}{}\SetValue{Contents}{%

A particular black hole has a density of $1.0 \times 10^6 \mathrm{~kg} / \mathrm{m}^3$. A physicist is conducting a thought experiment in which she would like to approximate how much she would weigh if she had the density of a black hole rather than her current weight of 150 pounds, assuming her volume remained the same. Given that her overall body density is approximately $990 \mathrm{~kg} / \mathrm{m}^3$ and that there are approximately 2.2 pounds in a kilogram, approximately how many pounds would she weigh in her thought experiment?

\ansFOURs{%start
2,178}{%<--- (A)
151,500}{%<--- (B)
990,000,000}{%<--- (C)
2,178,000,000}%<---- (D)

}\SetValue{Solution}{%
Answer: \textbf{(B)} 

Divide $1.0 \times 10^6 \mathrm{~kg} / \mathrm{m}^3$ by $990 \mathrm{~kg} / \mathrm{m}^3$ to determine by what multiple her weight will increase. Her weight will be 1,010 times greater. Then multiply 1,010 by 150 pounds to get 151,500 pounds.

}\SetValue{Answer}{%
B
}
\ProcessDATA


