\SetValue{SectionAB}{}\SetValue{MainChapter}{}\SetValue{SubChapter}{}\SetValue{Contents}{%

The approximate relationship between Kelvin (K) and degrees Celsius (C) is given by this equation:

$$
\mathrm{K}=273+\mathrm{C}
$$

The freezing point of water is 0 degrees Celsius, and the boiling point of water is 100 degrees Celsius. What are the approximate freezing and boiling points of water in Kelvin?

\ansFOURsT{%start
Freezing: $0$; boiling: $100$}{%<--- (A)
Freezing: $-273$; boiling: $-173$}{%<--- (B)
Freezing: $273$; boiling: $373$}{%<--- (C)
Freezing: $473$; boiling: $573$}%<---- (D)

}\SetValue{Solution}{%

Answer: \textbf{(C)}

Plug the freezing point of water in degrees $C$ into the equation:

\begin{align*}
    K &= 273 + C \\
    K &= 273 + 0 = 273
    \end{align*}
    


Choice (C) is the only option that has 273 for the freezing point. You can also plug in 100 for the boiling point in degrees $C$. You will get 373 as the result.


}\SetValue{Answer}{%
C
}
\ProcessDATA


