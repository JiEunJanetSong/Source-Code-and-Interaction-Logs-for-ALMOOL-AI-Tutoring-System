\SetValue{SectionAB}{}\SetValue{MainChapter}{}\SetValue{SubChapter}{}\SetValue{Contents}{%

A high school is evaluating two different DJs for its prom-DJ A and DJ B. The first one, DJ A, has a $\$ 300$ rental fee and charges $\$ 150$ per hour. The second one, DJ B, has a $\$ 200$ rental fee and charges $\$ 175$ per hour. After how many hours of performing will the cost for both DJs be the same?

}\SetValue{Solution}{%

Answer: $4$ 

We are given two DJs with different cost structures:

\begin{itemize}
    \item DJ A: \$300 rental fee + \$150 per hour  
    \item DJ B: \$200 rental fee + \$175 per hour  
\end{itemize}

We need to determine after how many hours $ h $ their total costs will be the same.

 Step 1: Set Up the Equation  
Let $ h $ be the number of hours. The total cost for each DJ is:

$
\text{Cost of DJ A} = 300 + 150h
$

$
\text{Cost of DJ B} = 200 + 175h
$

Since the costs must be equal:

$
300 + 150h = 200 + 175h
$

 Step 2: Solve for $ h $  
Subtract $ 150h $ from both sides:

$
300 = 200 + 25h
$

Subtract 200 from both sides:

$
100 = 25h
$

Divide by 25:

$
h = 4
$

 Step 3: Select the Correct Answer  
The costs for both DJs will be the same after 4 hours.


}\SetValue{Answer}{%
4
}
\ProcessDATA


