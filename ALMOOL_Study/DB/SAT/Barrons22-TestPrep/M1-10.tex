\SetValue{SectionAB}{}\SetValue{MainChapter}{}\SetValue{SubChapter}{}\SetValue{Contents}{%

\begin{center}		
    \includegraphics[scale=0.25]{SATpics/BP22-M1-10.png}
        \end{center}	

Which equation expresses the relationship between $x$ and $y$ shown in the graph above?

\ansFOURs{%start
y=\frac{1}{2} x-3}{%<--- (A)
y=3 x-3}{%<--- (B)
y=2 x+6}{%<--- (C)
y=-\frac{2}{3} x+6}%<---- (D)

}\SetValue{Solution}{%

Answer: \textbf{(A)}

This line has a $y$-intercept of -3 because that is where the line intersects the $y$-axis. You can calculate the slope using points on the line, $(0,-3)$ and $(6,0)$ :

$
\text { Slope }=\dfrac{\text { Rise }}{\text { Run }}=\dfrac{y_2-y_1}{x_2-x_1}=\dfrac{-3-0}{0-6}=\dfrac{1}{2}
$


Putting this in slope-intercept form $(y=m x+b)$ results in the equation of the line:

$
y=\frac{1}{2} x-3
$

}\SetValue{Answer}{%
A
}
\ProcessDATA


