\SetValue{SectionAB}{}\SetValue{MainChapter}{}\SetValue{SubChapter}{}\SetValue{Contents}{%

A company has four different stores at four different locations throughout a large city. The company gathered data on the initial prices, sale prices, and respective quantities sold of a particular item.

\begin{center}
    \begin{tabular}{|c|c|c|c|c|}
\hline
\textbf{Store} & \textbf{Price Before Sale} & \textbf{Quantity Sold Before Sale} & \textbf{Sale Price} & \textbf{Quantity Sold After Sale} \\ \hline
Store A        & \$12.50                   & 350                                & \$11.00             & 400                                \\ \hline
Store B        & \$13.25                   & 260                                & \$10.00             & 520                                \\ \hline
Store C        & \$11.75                   & 550                                & \$9.50              & 625                                \\ \hline
Store D        & \$14.00                   & 220                                & \$10.25             & 460                                \\ \hline
\end{tabular}
\end{center}


Look at how many of the particular item Stores A and B sold at the presale price. What is the arithmetic mean of this set of values to the nearest tenth?

\ansFOURs{%start
 8.1}{%<--- (A)
 9.7}{%<--- (B)
 10.5}{%<--- (C)
 12.8}%<---- (D)

}\SetValue{Solution}{%
Answer: \textbf{(D)} 

The arithmetic mean is the average. Solve this problem using a weighted average. In other words, calculate the average by factoring in the relative amounts of the item sold by each store.
\begin{tabular}{|c|c|c|c|c|}
    \hline
    \textbf{Store} & \textbf{Price Before Sale} & \textbf{Quantity Sold Before Sale} & \textbf{Sale Price} & \textbf{Quantity Sold After Sale} \\ \hline
    Store A        & \$12.50                   & 350                                & \$11.00             & 400                                \\ \hline
    Store B        & \$13.25                   & 260                                & \$10.00             & 520                                \\ \hline
    Store C        & \$11.75                   & 550                                & \$9.50              & 625                                \\ \hline
    Store D        & \$14.00                   & 220                                & \$10.25             & 460                                \\ \hline
    \end{tabular}
    
    Store A sold 350 items for $\$ 12.50$ each, and Store B sold 260 items for $\$ 13.25$ each. The total number of items sold is $350+260$. You can compute the average as follows:
    
    Total Income/Total Number of Items Sold $=\dfrac{[(12.50 \times 350)+(13.25 \times 260)]}{(350+260)} \approx 12.8$

}\SetValue{Answer}{%
D
}
\ProcessDATA


