\SetValue{SectionAB}{}\SetValue{MainChapter}{}\SetValue{SubChapter}{}\SetValue{Contents}{%

Consider the following system of equations with variables $A$ and $B$ and constant integers $X$ and $Y$ :


\begin{align*}
A + 2B &= 4 \\
XA + YB &= 4X
\end{align*}



By what number must the sum of $X$ and $Y$ be divisible in order for the two equations to have infinitely many solutions?

}\SetValue{Solution}{%
Answer: $3$ 

3 There will be infinitely many solutions if the two equations are multiples of the same equation. The coefficients of the $A$ and $B$ terms in $A+2 B=4$ add up to 3 because they are 1 and 2. Since $X A+Y B=4 X$ is divisible by 4 on the right-hand side, as is the other equation, the sum of $X$ and $Y$ must also be divisible by 3 in order for the two equations to be multiples of one another. To replicate the structure of the first equation, $Y$ must equal $2 X$ so that the two equations will be multiples of one another. To see this with greater clarity, consider this example:


\begin{align*}
A + 2B &= 4 \\
XA + YB &= 4X
\end{align*}



If the second equation had $X=2$ and $Y=4$, the equation would be twice the first equation: $2 A+4 B=8$. This equation is simply a multiple of the first one, making them essentially identical. As a result, there are infinitely many solutions since the equations overlap each other when graphed.
}\SetValue{Answer}{%
3
}
\ProcessDATA


