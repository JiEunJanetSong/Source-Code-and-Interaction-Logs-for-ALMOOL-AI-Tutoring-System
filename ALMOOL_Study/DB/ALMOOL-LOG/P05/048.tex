\begin{GPTQ}
    조금만 더 자세히 설명해줘
\end{GPTQ}

\begin{GPTA}
   

    물론입니다. 각 단계를 조금 더 자세히 설명해드리겠습니다.

    1. $\triangle M Q N$ 에서 $|\overline{N Q}|$ 계산:

    - 삼각형 $\triangle M Q N$ 에서, 주어진 $\angle M Q N=72^{\circ}$ 입니다.

    - 우리는 $\tan \theta$ 는 직각 삼각형에서 반대변 $(M N)$ 의 길이를 인접변 $(N Q)$ 의 길이로 나누 어 구할 수 있음을 알고 있습니다. 따라서:
    
    $$
    \tan 72^{\circ}=\frac{|\overline{M N}|}{|\overline{N Q}|}=\frac{15}{|\overline{N Q}|}
    $$
    
    - 이 식을 $|\overline{N Q}|$ 에 대해 풀면:
    
    $$
    |\overline{N Q}|=\frac{15}{\tan 72^{\circ}} \approx 4.8738 \mathrm{~m}
    $$
    
    2. $\triangle M R T$ 에서 피타고라스 정리를 사용하여 $|\overline{M R}|$ 계산:

    - 주어진 정보로부터 $\triangle M R T$ 에서 피타고라스의 정리를 이용하여 $M R$ 의 길이를 구합 니다. $M R$ 은 이미 주어졌으나, 이제 $|\overline{M T}|$ 길이를 찾아야 합니다.

    - 피타고라스의 정리:
    
    $$
    (|\overline{M R}|)^2=|\overline{M N}|^2+|\overline{N R}|^2
    $$
    
    - 그러나 $|\overline{M N}|$ 과 주어진 $|\overline{M R}|=10 \sqrt{3} \mathrm{~m}$ 만을 알지, $|\overline{N R}|$ 는 필요하지 않습니다. 대신 $|\overline{M Q}|$ 를 구합니다.

    3. 전체 길이에서 $|\overline{Q R}|$ 계산:
    - 이제 $|\overline{M R}|=|\overline{M Q}|+|\overline{Q R}|$ 이라는 관계를 사용합니다.

    - 주어진 길이에서 $|\overline{M Q}|$ 와 위에서 계산한 $|\overline{N Q}|$ 를 사용하여:
    
    $$
    5 \sqrt{3}=4.8738+|\overline{Q R}|
    $$
    
    - 이 식을 정리하여 $|\overline{Q R}|$ 를 계산하면:
    
    $$
    |\overline{Q R}|=5 \sqrt{3}-4.8738 \approx 3.7865 \mathrm{~m}
    $$
    
    - 최종적으로 반올림하면 $|\overline{Q R}| \approx 4 \mathrm{~m}$ 입니다.
    
    이러한 방식으로 문제를 풀어 우리가 원하는 답인 $|\overline{Q R}|=4 \mathrm{~m}$ 을 얻을 수 있습니다.    
\end{GPTA}

