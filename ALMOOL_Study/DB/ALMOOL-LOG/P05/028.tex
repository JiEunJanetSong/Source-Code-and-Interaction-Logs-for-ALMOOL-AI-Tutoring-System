\begin{GPTQ}
    hypotenuse 
\end{GPTQ}

\begin{GPTA}
    
    ``Hypotenuse"는 직각 삼각형에서 가장 중요한 요소 중 하나입니다. 직각 삼각형에서 ``hypotenuse" 는 다음과 같은 특징이 있습니다:

    1. 빗변: 직각을 이루는 두 변들을 제외한, 직각 반대편에 위치한 가장 긴 변을 말합니다.

    2. 피타고라스 정리: 직각 삼각형에서는 피타고라스 정리를 통해 hypotenuse의 길이를 구할 수 있습니다. 피타고라스 정리는 다음과 같습니다:
    
    $$
    a^2+b^2=c^2
    $$
    
    
    여기서 $c$ 는 hypotenuse의 길이이며, $a$ 와 $b$ 는 직각을 이루는 두 변의 길이입니다.

    3. 특수 직각 삼각형 비율:

    - $45^{\circ}-45^{\circ}-90^{\circ}$ 삼각형에서는 변의 길이 비율이 $1: 1: \sqrt{2}$ 여서 hypotenuse는 직각을 이루는 변 중 하나의 길이에 $\sqrt{2}$ 를 곱한 값입니다.

    - $30^{\circ}-60^{\circ}-90^{\circ}$ 삼각형에서는 변의 길이 비율이 $1: \sqrt{3}: 2$ 여서 hypotenuse는 직각을 이루는 가장 짧은 변의 길이에 2 를 곱한 값입니다.

    hypotenuse는 삼각 측량, 구조 해석, 물리학 등 여러 분야에서 매우 중요한 역할을 합니다. 궁금한 점이 더 있으면 언제든지 질문해 주세요!    
\end{GPTA}

