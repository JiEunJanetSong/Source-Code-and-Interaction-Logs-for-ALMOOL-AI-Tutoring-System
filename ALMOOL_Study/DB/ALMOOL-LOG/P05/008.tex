\begin{GPTQ}
그니까 설명을 해주라고 왜 바뀌는지  
\end{GPTQ}

\begin{GPTA}
물론입니다. 방금 말씀하신 부분을 다시 자세히 말씀드리겠습니다.
    
계산 과정 설명

    이제 문제에서 주어진 예산에 관한 부등식을 만들고 이를 풀어보겠습니다.

    우리는 두 가지 양초의 가격을 알고 있습니다:

    - 작은 양초 하나의 가격은 $\$ 4.90$

    - 큰 양초 하나의 가격은 $\$ 11.60$
    
    총 예산은 $\$ 2,200$ 이므로, 두 양초를 구매하는 데 드는 총비용은 예산 내여야 합니다:
    
    $$
    4.90 s+11.60 \ell \leq 2,200
    $$
    
    $s$ 를 $\ell$ 로 대체:

    문제에서 최소 200 개의 양초를 구매해야 한다고 했습니다:
    
    $$
    s+\ell \geq 200
    $$
    
    $s$ 에 대해 풀어보면:
    
    $$
    s \geq 200-\ell
    $$
    
    
    이 값을 예산 부등식에 대입합니다:
    
    $$
    4.90(200-\ell)+11.60 \ell \leq 2,200
    $$
    
    
    각 항을 계산해보면:

    1. $4.90 \times 200=980$

    2. $-4.90 \times \ell=-4.90 \ell$

    3. 합치면:
    
    $$
    980-4.90 \ell+11.60 \ell \leq 2,200
    $$
    
    
    다시 한 번, 두 항을 정리하면:
    
    $$
    \begin{gathered}
    980+(11.60 \ell-4.90 \ell) \leq 2,200 \\
    980+6.70 \ell \leq 2,200
    \end{gathered}
    $$
    
    
    위 단계에서 $-4.90 \ell$ 이 $11.60 \ell$ 과 합쳐져 $+6.70 \ell$ 로 바뀌었습니다. 이렇게 부등식을 정리하면 큰 양 초의 개수를 쉽게 계산할 수 있습니다.
    
    이해가 되셨나요? 추가적인 질문이 있으시면 언제든지 말씀해 주세요.   
\end{GPTA}

