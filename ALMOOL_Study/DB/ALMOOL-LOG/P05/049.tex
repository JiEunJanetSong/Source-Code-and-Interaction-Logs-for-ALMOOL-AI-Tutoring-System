\begin{GPTQ}
    이거랑 똑같은데 다른 문제 내줄 수 있어? 이해를 못한 것 같아
\end{GPTQ}

\begin{GPTA}
 물론이죠! 비슷한 문제를 통해 연습해 보도록 하겠습니다.
    
    문제:

    주어진 원뿔 절단체에서 아래쪽 원의 지름이 위쪽 원의 지름의 네 배입니다. 절단체의 높이가 9 cm 일 때, 전체 원뿔의 높이를 계산하세요.
    
    단계:

    1. 위쪽 원의 지름을 $d$ 라고 하고, 반지름 $r$ 은 $\frac{d}{2}$ 입니다.

    2. 아래쪽 원의 지름은 네 배이므로 $4 d^{\circ}$ 고, 반지름은 $\frac{4 d}{2}$ 입니다.

    3. 절단체의 높이가 9 cm 이므로, 상단 원뿔의 높이를 $x$ 라 할 때 전체 높이는 $x+9$ 입니다.
    
    4. 비율 관계:
    
    $$
    \frac{x+9}{\frac{4 d}{2}}=\frac{x}{\frac{d}{2}}
    $$
    
    5. 이 비례식을 풀어 전체 원뿔의 높이 $(x+9)$ 를 구하세요.
    
    이 문제를 풀어보시고 이해가 안 되시는 부분이 있으면 말씀해 주세요. 그럼 더욱 세부적으로 설 명하겠습니다!  
\end{GPTA}

