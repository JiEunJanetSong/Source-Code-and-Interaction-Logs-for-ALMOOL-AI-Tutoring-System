\begin{GPTQ}
    예산 조건에 방정식을 어떻게 세웠는지 모르겠어   
\end{GPTQ}

\begin{GPTA}
예산 조건은 다음과 같은 논리로 방정식을 세울 수 있습니다.

    예산 조건 방정식

    문제에서 주어진 정보로부터 아래와 같은 조건을 도출할 수 있습니다.

    - 작은 양초 하나의 가격은 $\$ 4.90$ 입니다.

    - 큰 양초 하나의 가격은 $\$ 11.60$ 입니다.

    - 총 예산은 $\$ 2,200$ 입니다.
    
    이제 각 종류의 양초에 대한 비용을 구하면 됩니다.

    1. 작은 양초를 $s$ 개 구입하면, 드는 비용은 $4.90 \times s$ 입니다.

    2. 큰 양초를 $\ell$ 개 구입하면, 드는 비용은 $11.60 \times \ell$ 입니다.
    
    이 두 가지 비용을 합하여 총 비용을 구합니다:
    
    $$
    4.90 s+11.60 \ell
    $$
    
    
    이 총 비용이 예산, 즉 $\$ 2,200$ 보다 크지 않아야 하므로, 우리는 다음과 같은 부등식을 세웁니다:
    
    $$
    4.90 s+11.60 \ell \leq 2,200
    $$
    
    
    이 식은 소유자가 양초를 구입할 때 지켜야 할 예산 조건을 나타냅니다. 이 방정식을 통해 양초를 구매할 수 있는 최대 수량을 계산하게 됩니다.
\end{GPTA}

