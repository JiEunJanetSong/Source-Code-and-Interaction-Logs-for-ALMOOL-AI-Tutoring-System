\begin{GPTQ}
    절단체가 뭐야?  
\end{GPTQ}

\begin{GPTA}
좋은 질문이에요! 원뿔 절단체라는 개념은 원뿔에서 생기는 특정한 도형입니다.

    절단체 란?

    - 절단체는 원뿔의 일부로, 원뿔의 아래쪽 끝부분을 자른 후 남은 윗부분을 말합니다.

    - 절단체는 두 개의 서로 다른 크기의 평행한 원으로 이루어져 있으며, 이 두 원은 한 개의 원 뿔의 윗부분과 절단된 아랫부분에 위치합니다.
    
    예시:
    
    상상해보면, 실체 원뿔에서 일정한 높이로 수평 잘라낸 형태입니다. 원뿔의 맨 꼭대기에서 맨 아 랫부분의 사이에 남아있는 부분이 절단체가 됩니다. 절단체의 윗부분과 아랫부분은 각각 둘레가 다른 두 개의 원으로 이루어져 있습니다.
    이제 절단체가 어떤 것인지 좀 더 이해가 되셨나요? 더 궁금한 점이 있다면 언제든지 물어봐 주세 요!  
\end{GPTA}

