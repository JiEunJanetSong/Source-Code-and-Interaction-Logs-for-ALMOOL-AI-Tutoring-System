\begin{GPTQ}
    왜 반지름이 $\mathrm{r}=\mathrm{d} / 2$ 로 하는거야?  
\end{GPTQ}

\begin{GPTA}
 좋은 질문이에요! 반지름 $r$ 이 지름 $d$ 의 절 반인 이유를 설명해 드릴게요.

    반지름과 지름의 관계

    어떤 원에서든지 지름은 원의 한쪽 끝에서 다른 쪽 끝까지의 거리를 말하며, 중심을 지나갑니다. 이때, 반지 름은 중심에서 원의 경계까지의 거리입니다. 따라서, 반지름은 지름의 절반입니다. 수 식으로 나타내면:
    
    $$
    r=\frac{d}{2}
    $$
    
    
    이 공식은 모든 원에 대해 일반적입니다. 
    
    반지름이 지름의 절반인 이유는 원의 중심을 기준으로 대칭적이기 때문입니다.

\end{GPTA}

