\begin{GPTQ}
    $\overline{\mathrm{x}+9 / 4 \mathrm{~d} / 2} \overline{\text { 가 왜 } \mathrm{x} / \mathrm{d} / 2}$ 가 되는지 알 수 있을까? 
\end{GPTQ}

\begin{GPTA}
 

이 표현식이 어떻게 변경되는지 단계별로 설명해 드릴게요.

주어진 표현식

우리가 다루고 있는 식은:

$$
\frac{x+9}{\frac{4 d}{2}}=\frac{x}{\frac{d}{2}}
$$


이 식은 원래 두 비율이 같음을 보여주는 비례식입니다. 이 식의 좌변과 우변 모두 원뿔의 비례 관계에 기반하여 설정됩니다.

단계별 설명  

1. 분모의 구조 이해:
- 좌변의 분모는 $\frac{4 d}{2}$ 입니다.

우변의 분모는 $\frac{d}{2}$ 입니다.

2. 기본 비례 설정:

위의 비례 관계는 절단체의 위쪽과 아래쪽 원 각각의 지름 또는 반지름을 고려하여 만 들어졌습니다.
이 구성을 통해 절단체의 전체 높이와 부분 원뿔의 높이 사이의 관계를 설명합니다.

3. 유사 삼각형의 개넘 사용:

- 위 식은 전체 원뿔과 원뿔 절단체 사이의 유사 삼각형 관계를 표현합니다.

- 원뿔의 높이에 대한 문제에서, 이 비례 관계는 원뿔의 기저와 높이의 비율이 일정함을 나타냅니다.

결론적으로, 이 식은 전체 원뿔과 절단체의 부분 원뿔들 간의 비례 관계에 기초한 것입니다. 좌변 에서 $\frac{4 d}{2}$ 는 아래쪽 면의 지름을, 우변의 $\frac{d}{2}$ 는 위쪽 면의 지름을 나타냅니다.
보다 더 구체적인 부분에 대한 설명이 필요하면 언제든지 말씀해 주세요.

\end{GPTA}

