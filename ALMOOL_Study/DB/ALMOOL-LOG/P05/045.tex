\begin{GPTQ}
    그러면 이제 문제에 대입 해줄 수 있어? 
\end{GPTQ}

\begin{GPTA}
물론입니다! 주어진 문제에 맞춰 각각의 정보를 대입하며 풀이를 진행하겠습니다.

    문제 해결 단계

    1. 삼각형 $\triangle M N Q$ 에서 각도 계산:
    - $\triangle M N Q$ 는 삼각형이므로, 내각의 합은 $180^{\circ}$ 입니다.

    - 주어진 각도는 $\angle M Q N=72^{\circ}$ 이며, $\angle M N Q+\angle M Q N+90^{\circ}=180^{\circ}$ 를 만족합니다.

    - 따라서 $\angle M N Q=180^{\circ}-72^{\circ}-90^{\circ}=18^{\circ}$ 입니다.

    2. 삼각형 $\triangle M N R$ 에서 각도 $\theta$ 계산:

    - $\sin \theta=\frac{|M N|}{|\overline{M R \mid}|}=\frac{15}{10 \sqrt{3}}$

    - 이를 통해 $\theta=\sin ^{-1}\left(\frac{15}{10 \sqrt{3}}\right)=60^{\circ}$ 입니다.

    3. 각도 $x$ 계산:

    - 삼각형 $\triangle M N R$ 에서 $\angle M N R=\angle N M R+\angle N R M=90^{\circ}$ 임을 이용합니다.

    - $\angle N M R=x, \angle N R M=60^{\circ}$ 이므로 $x+60^{\circ}+90^{\circ}=180^{\circ}$ 에서
    
    - $x=180^{\circ}-168^{\circ}=12^{\circ}$ 입니다.
    
    따라서, 각도 $x$ 의 값은 $12^{\circ}$ 로 계산됩니다. 필요한 계산이나 추가적인 설명이 더 필요하면 말씀해 주세요!  
\end{GPTA}

