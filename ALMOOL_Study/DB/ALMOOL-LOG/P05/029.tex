\begin{GPTQ}
    중간길이변이 루트 3 x 라는걸 어떻게 알아?
\end{GPTQ}

\begin{GPTA}
좋은 질문입니다! 30-60-90 삼각형에서 변들 간의 비율이 어떻게 정해지는지 설명하겠습니다. 

30-60-90 삼각형의 변의 비율 30-60-90 직각 삼각형은 특별한 비율을 가지는 이유는 다음과 같습니다:

    1. 등변삼각형에서 유도:

    - 정삼각형을 떠올려 봅시다. 각 변의 길이가 같고 모든 각이 $60^{\circ}$ 인 삼각형입니다.
    
    이 잉삼각형을 절반으로 자른다고 생각하면, 두 개의 $30-60-90$ 삼각형이 생깁니다.

    2. 변의 길이 비율:

    - 정삼각형의 한 변의 길이를 $2 x$ 라고 하면, 이때 이 길이를 반으로 나누어서 두 개의 삼 각형으로 나눕니다.

    - 정삼각형에서 한 변을 반으로 나눈 직각 변은 $x$ 입니다.

    - 이제, 중간 변인 다른 변의 길이를 $h$ 라고 하면, 피타고라스 정리를 적용할 수 있습니다
    
    $$
    h^2+x^2=(2 x)^2 .
    $$
    
    3. 피타고라스 정리를 풀어서:

    - 방정식을 풀어보면:
    
    $$
    h^2+x^2=4 x^2 h^2=3 x^2 h=\sqrt{3 x^2} h=\sqrt{3} x .
    $$
    
    
    따라서, 30-60-90 삼각형에서는 중간변의 길이가 $\sqrt{3} x$ 라는 사실을 알 수 있습니다. 이러한 비율은 각도에 따른 삼각형의 성질에 의해 정의됩니다. 이 원리를 통해 다양한 문제를 해결할 수 있습니다. 다른 질문이 있으면 언제든지 물어보세요!   
\end{GPTA}

