\begin{GPTQ}
    저 문제에서는 첫번째 단계 때 어떻게 해야해?
\end{GPTQ}

\begin{GPTA}
   

    문제를 풀기 위해 첫 번째 단계에서는 전체 원뿔의 부피 공식을 활용하여 주어진 부피를 식에 대입해서 반지름 $r$ 을 구하는 것입니다.
    주어진 정보는 다음과 같습니다:

    - 전체 원뿔의 부피 $V=39,600 \mathrm{~cm}^3$

    - 원뿔의 높이 $h=18 \mathrm{~cm}$

    - 지름은 반지름의 두 배이므로 밑면 반지름 $r$ 로 계산
    
    다음과 같이 전체 원뿔의 부피 공식을 사용할 수 있습니다:
    
    $$
    V=\frac{1}{3} \pi r^2 h
    $$
    
    
    이를 해결하기 위해 주어진 값을 대입합니다:
    
    $$
    39,600=\frac{1}{3} \times \frac{22}{7} \times r^2 \times 18
    $$
    
    
    이 식을 통해서 반지름 $r$ 을 구할 수 있습니다. 이 식을 순차적으로 풀어나가면 반지름 $r$ 의 제곱 값 을 구할 수 있습니다. 나중에 제곱근을 취하여 $r$ 을 찾게 됩니다. 혹시 계산 방법이나 다음 단계를 알고 싶으시면 말씀해 주세요!  
\end{GPTA}

