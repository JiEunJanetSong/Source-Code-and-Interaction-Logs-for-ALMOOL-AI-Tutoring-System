\begin{GPTQ}
왜 3개야?   
\end{GPTQ}

\begin{GPTA}
그림에서 함수 $y=f(x)$ 의 그래프는 $x$ 축과 교차하는 지점 을 나타냅니다. 이러한 교차점들은 $f(x)=0$ 일 때의 $x$ 값, 즉 그래프가 $x$ 축과 만나는 지점입니다.

그래프를 보면, $x$ 축과 만나는 지점이 총 세 곳 있습니다. 
    
이는 $f(x)=0$ 일 때 $x$ 의 값이 세 개 있다 는 것을 의미합니다. 이 세 교차점이 각각 $x$ 값이 되는 것이죠. 따라서 $f(x)=0$ 일 때의 $x$ 값의 개수 는 세 개입니다.  
\end{GPTA}

