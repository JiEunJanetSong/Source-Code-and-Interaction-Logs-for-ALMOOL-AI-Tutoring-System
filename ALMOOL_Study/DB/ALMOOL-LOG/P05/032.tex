\begin{GPTQ}
    여기서 뭐부터 시작해야 하는지 모르겠어. 첫 단계가 뭔지 설명 해줄래?  
\end{GPTQ}

\begin{GPTA}
물론입니다! 문제 해결을 시작하기 위해 첫 번째로 수행해야 할 단계는 주어진 정보를 분석하는 것입니다. 이 문제에서 우리는 다음 정보를 가지고 있습니다.

    1. 아래쪽 원의 지름은 위쪽 원의 지름의 세 배입니다.

    2. 절단체의 높이는 12 cm 입니다.
    
    문제는 전체 원뿔의 높이를 구하는 것입니다. 이제 문제를 풀기 위한 첫 번째 단계는 전체 원뿔과 절단체의 관계를 설정하는 것입니다.

    1. 지름과 반지름 관계 설정:

    - 위쪽 원의 지름을 $d$ 라고 하면, 반지름 $r$ 은 $r=\frac{d}{2}$ 입니다.

    - 아래쪽 원의 지름은 세 배이므로 $3 d^{\circ}$ 이고, 반지 름은 $\frac{3 d}{2}$ 입니다.

    2. 비례 관계 설정:
    
    얼단체의 높이가 12 cm 이므로, 상단 원뿔의 높이를 $x$ 라고 할 때, 전체 원뿔의 높이는 $(x+12)$ 입니다.

    3. 기하학적 비례식 작성:

    - 전체 원뿔과 둥근 절단체의 비례식을 써서 문제를 해결할 수 있습니다:
    
    $$
    \frac{x+12}{\frac{3 d}{2}}=\frac{x}{\frac{d}{2}}
    $$
    
    
    이 비례식을 풀어서 전체 원뿔의 높이를 구합니다.
    첫 번째 단계로 여기까지 진행하시면 됩니다. 이제 이 비례식을 풀어보겠습니다. 추가적인 도움 이나 다른 단계가 필요하시면 언제든지 말씀해 주세요!   
\end{GPTA}

