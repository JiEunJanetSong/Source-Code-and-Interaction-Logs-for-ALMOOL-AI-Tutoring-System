\begin{GPTQ}
    그럼 위쪽 뿔의 반지름을 사용하는 이유가 뭐야?
\end{GPTQ}

\begin{GPTA}
 좋은 질문입니다! 원뿔 절단체 문제에서 위쪽 원뿔의 반지름을 사용하는 이유는 비례 관계를 설정하기 위해서입니다. 이유를 좀 더 자세히 설명해 드릴게요.
    
    원뿔 절단체와 비례 관계

    1. 비례 관계 설정:

    - 원뿔 절단체에서 위쪽과 아래쪽 두 개의 원이 있습니다. 이 두 원은 각각 위쪽 원뿔과 아래쪽 원뿔의 기초가 되는 면입니다.

    - 전체 원뿔에서 부분적 원뿔의 높이와 기초 원의 반지름 사이의 비례 관계를 설정함으로써, 절 단체의 기하학적인 성질을 사용할 수 있습니다.
    
    2. 비례식의 필요성:

    - 위쪽 원뿔과 절단체는 동일한 축을 공유하고 있기 때문에, 비슷한 삼각형의 개념을 적 용하여 비례식을 설정할 수 있습니다.

    - 이 비례식은 주어진 조건을 사용하여 전체 원뿔의 높이나 다른 미지수를 찾는 데 유용 합니다.

    3. 비례의 일반적인 사용:

    - 이 문제에서는 위쪽 원뿔의 반지름과 높이, 그리고 아래쪽 원의 반지름의 비례 관계를 설정하여 전체 원뿔의 높이를 구하는 문제를 해결합니다.
    
    따라서 위쪽 원뿔의 반지름은 비례 관계를 설정하는 데 중요한 요소가 됩니다. 추가적인 질문이 ㅇㅣㅣㅇㅁㅁㅕ 어제든지 악려주세오    
\end{GPTA}

