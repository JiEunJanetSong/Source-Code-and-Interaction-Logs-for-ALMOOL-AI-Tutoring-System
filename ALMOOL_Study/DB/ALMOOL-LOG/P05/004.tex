\begin{GPTQ}
    왜 답이 9야?   
\end{GPTQ}

\begin{GPTA}
물체가 12 센티미터/초의 속도로 이동하고 총 108 센티미터를 이동한다고 할 때, 이동 시간을 계산할 수 있습니다.
    이를 위해 총 거리를 속도로 나눕니다:
    
    $$
    \frac{108 \text { 센티미터 }}{12 \text { 센티미터/초 }}=9 \text { 초 }
    $$
    
    
    즉, 물체가 108 센티미터를 이동하는 데 걸리는 시간은 9 초입니다. 따라서 답은 9 입니다.    
\end{GPTA}

