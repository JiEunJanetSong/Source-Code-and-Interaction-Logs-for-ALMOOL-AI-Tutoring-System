\begin{GPTQ}
    같은 문제인데 숫자가 다른걸 해주  
\end{GPTQ}

\begin{GPTA}
물론입니다! 숫자를 바꿔서 새로운 예제를 함께 풀어보겠습니다. 만약 공급과 수요 방정식이 다음과 같다면:

    - 공급: 가격 $=50-$ 수량

    - 수요: 가격 $=20+2 \times$ 수량
    
    수량을 구해보겠습니다:

    균형 상태에서는 공급과 수요가 같아야 하므로 두 식을 같게 설정합니다.
    
    $$
    50-\text { 수량 }=20+2 \times \text { 수량 }
    $$
    
    
    이 방정식을 풀어보겠습니다:

    1. 양변에서 20 을 빼고 수량을 한쪽으로 모읍니다:
    
    $$
    50-20-\text { 수량 }=2 \times \text { 수량 }
    $$
    
    1. 간단히 정리하면:
    
    $$
    30=3 \times \text { 수량 }
    $$
    
    1. 양변을 3 으로 나누면:
    
    $$
    \text { 수량 }=10
    $$
    
    
    따라서, 이 경우 램프를 10 개 판매하면 공급과 수요가 균형 상태에 있게 됩니다. 이해되셨길 바랍니다!    
\end{GPTA}

