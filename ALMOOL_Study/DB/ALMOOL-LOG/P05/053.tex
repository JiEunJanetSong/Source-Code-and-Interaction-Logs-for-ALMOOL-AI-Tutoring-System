\begin{GPTQ}
    답이 $\mathrm{x}=9 \mathrm{x} / 2 \mathrm{~d}$ 가 나왓어. 이 다음에 해야할게 뭐야  
\end{GPTQ}

\begin{GPTA}
    

    이전에 비례식을 풀다가 중간 과정에서 결과가 잘못 나왔네요. 다시 살펴봅시다. 제대로 풀어보 겠습니다.
    
    주어진 비례식

    시작하겠습니다:
    
    $$
    \frac{x+9}{\frac{4 d}{2}}=\frac{x}{\frac{d}{2}}
    $$
    
    
    단계 1: 두 식의 각 변에 대한 분모 제거

    각 양변을 분모가 없다면 다음과 같이 처리 할 수 있습니다:
    
    $$
    (x+9) \cdot \frac{d}{2}=x \cdot \frac{4 d}{2}
    $$
    
    
    단계 2: 분배법칙 적용

    식의 양변을 분배해주면,
    
    $$
    d(x+9)=4 d x
    $$
    
    
    단계 3: 분배 수행

    양변을 풀어주면:
    
    $$
    d x+9 d=4 d x
    $$
    
    
    단계 4: 상응하는 항 정리

    $d x$ 를 양쪽에서 최소화 하면:
    
    $$
    9 d=4 d x-d x
    $$
    
    
    단계 5: 비례식 정리
    
    비례식을 풀어 $x$ 를 구하면:
    
    $$
    \begin{gathered}
    3 d x=9 d \\
    3 x=9 \\
    x=\frac{9}{3}=3
    \end{gathered}
    $$
    
    
    위의 결과로, 전체 원뿔의 높이 $(x+9)$ 는 $3+9=12 \mathrm{~cm}$ 가 됩니다. 이해가 되셨나요? 추가적인 도움이 필요하면 언제든지 알려주세요!  
\end{GPTA}

