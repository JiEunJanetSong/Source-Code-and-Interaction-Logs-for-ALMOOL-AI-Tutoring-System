\begin{GPTQ}
    1-2번 넘어 갈때 왜 3 이 4가 되는거야?
\end{GPTQ}

\begin{GPTA}
좋은 질문입니다! 단계 간의 변화를 명확히 설명드리겠습니다. 주어진 방정식에서:
    
    $$
    30-\text { 수량 }=10+3 \times \text { 수량 }
    $$
    
    
    우리는 수량의 값을 양변에서 정리하고자 합니다. 이렇게 하려면, 수량 수량 관련 항목들을 식의 한쪽으로 모으게 됩니다.

    식의 양변에서 10 을 빼는 과정은 다음과 같습니다:
    
    $$
    30-10-\text { 수량 }=3 \times \text { 수량 }
    $$
    
    
    따라서, 오른쪽에서 10 을 뺀 결과로 인해 다음과 같은 식이 됩니다:

    $20-$ 수량 $=3 \times$ 수량

    이제, 수량 관련 항목을 하나의 항목으로 계산해보겠습니다.
    
    왼쪽의 -수량을 오른쪽으로 넘겨 이 동시키면:
    
    $$
    20=3 \times \text { 수량 }+ \text { 수량 }
    $$
    
    
    위 식을 간단히 하면:
    
    $$
    20=4 \times \text { 수량 }
    $$
    
    
    그래서 $20=4 \times$ 수량이 되었고, 이는 양변을 4 로 나누었을 때 다음의 결과가 나옵니다:
    수량 $=5$    
\end{GPTA}

