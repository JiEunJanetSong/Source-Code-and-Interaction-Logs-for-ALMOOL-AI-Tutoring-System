\begin{GPTQ}
    이게 왜 182 인지 설명해줘    
\end{GPTQ}

\begin{GPTA}
    문제를 단계별로 해결해 보겠습니다. 먼저 변수 설정을 합니다:

    - $s$ : 작은 양초의 개수
    
    - $\ell$ : 큰 양초의 개수
    
    주어진 조건에 따라서 방정식을 설정합니다:
    
    1. 예산 조건: $4.90 s+11.60 \ell \leq 2,200$
    
    2. 최소 구입 조건: $s+\ell \geq 200$
    
    이제 이 방정식을 결합해 보겠습니다.
    
    방정식 결합
    
    1. 두 번째 방정식에서 $s \geq 200-\ell$ 를 얻을 수 있습니다.
    2. 이 식을 $s$ 에 대한 1 식에 대입해 봅시다.
    
    $$
    4.90 s+11.60 \ell \leq 2,200
    $$
    
    3. $s$ 에 대한 표현을 대입합니다.
    
    $$
    4.90(200-\ell)+11.60 \ell \leq 2,200
    $$
    
    4. 계산을 합니다:
    
    $$
    \begin{aligned}
    & 4.90(200-\ell)=980-4.90 \ell \\
    & 980-4.90 \ell+11.60 \ell \leq 2,200
    \end{aligned}
    $$
    
    5. 간단히 정리하면:
    
    $$
    980+6.70 \ell \leq 2,200
    $$
    
    6. 양변에서 980 을 뺍니다:
    
    $$
    6.70 \ell \leq 1,220
    $$
    
    7. 양변을 6.70 으로 나눕니다:
    
    $$
    \ell \leq \frac{1,220}{6.70} \approx 182.09
    $$
    
    
    큰 양초의 개수는 정수여야 하므로, 최대 값은 $\ell=182$ 가 됩니다.
        
\end{GPTA}

