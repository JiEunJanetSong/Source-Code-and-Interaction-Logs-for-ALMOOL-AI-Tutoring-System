\begin{GPTQ}
    삼각비가 뭐야?
\end{GPTQ}

\begin{GPTA}
삼각비는 직각삼각형에서 각도와 변의 길이 사이의 관계를 나타내는 비율입니다. 기본적인 삼각 비에는 사인, 코사인, 탄젠트가 있습니다. 각 삼각비는 다음과 같이 정의됩니다:

    1. 사인 (Sine): 주어진 각의 사인은 그 각의 마주보는 변의 길이를 빗변의 길이로 나눈 값입니 다.
    
    $$
    \sin \theta=\frac{\text { 마주보는 변 }}{\text { 빗변 }}
    $$
    
    2. 코사인 (Cosine): 주어진 각의 코사인은 그 각의 인접한 변의 길이를 빗변의 길이로 나눈 값 입니다.
    
    $$
    \cos \theta=\frac{\text { 인접한 변 }}{\text { 빗변 }}
    $$
    
    3. 탄젠트 (Tangent): 주어진 각의 탄젠트는 마주보는 변의 길이를 인접한 변의 길이로 나눈 값입니다.
    
    $$
    \tan \theta=\frac{\text { 마주보는 변 }}{\text { 인접한 변 }}
    $$
    
    
    이 삼각비들은 삼각형의 각도를 알고 있을 때 변의 길이를 계산하거나, 반대로 변의 길이를 알고 있을 때 각도를 계산하는 데 사용됩니다. 직각삼각형에서 이 관계는 매우 중요하고 널리 사용됩니다. 추가적인 설명이 필요하면 언제든지 말씀해 주세요!  
\end{GPTA}

