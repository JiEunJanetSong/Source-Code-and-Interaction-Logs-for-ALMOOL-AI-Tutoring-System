\begin{GPTQ}
    frustrum 이게 뭔 뜻이야
\end{GPTQ}

\begin{GPTA}
   ``Frustrum"이라는 단어는 기하학적인 용어인 ``frustum"을 잘못 적은 것 같습니다. 원뿔 절단체를 설명할 때 사용하는 정확한 용어는 ``frustum"입니다.
    
    Frustum (절단체)의 의미

    - Frustum은 원뿔 또는 피라미드의 맨 윗부분을 평행하게 잘랐을 때 생기는 형상을 말합니다.
    
    - 절단된 부분이 평행한 두 면을 가지고 있으며, 이 두 면은 아래쪽과 위쪽의 서로 다른 지름 을 가진 원입니다.
    
    Frustum은 기하학에서 흔히 볼 수 있는 도형 중 하나입니다. 원뿔이나 피라미드의 꼭대기를 수평 으로 잘랐을 때부터 발생합니다.
    혹시 추가적인 도움이 필요하시면 언제든지 말씀해 주시기 바랍니다!   
\end{GPTA}

