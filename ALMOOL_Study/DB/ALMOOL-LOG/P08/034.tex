\begin{GPTQ}
    QR 의 값을 구하는 과정  
\end{GPTQ}

\begin{GPTA}
물론입니다. $|\overline{Q R}|$ 의 길이를 계산하는 과정을 자세히 설명드리겠습니다.

    우리는 $\triangle M N R$ 과 주어진 각도 $\angle M Q N=72^{\circ}$ 를 사용하여, 삼각함수를 통해 $|\overline{N Q}|$ 의 길이를 구하고, 피타고라스의 정리를 사용하여 $|\overline{N R}|$ 의 나머지 측면을 계산하여 $|\overline{Q R}|$ 의 길이를 구합니다.

    1. $|\overline{N Q}|$ 구하기:

    $\triangle M Q N$ 에서, $\angle M Q N=72^{\circ}$ 이고, $|\overline{M N}|=15 \mathrm{~m}$ 입니다. 이 경 우 탄젠트 함수를 사용하여 $|\overline{N Q}|$ 를 구할 수 있습니다.
    
    $$
    \tan 72^{\circ}=\frac{|\overline{M N}|}{|\overline{N Q}|}
    $$
    
    
    여기서 $|\overline{M N}|=15 \mathrm{~m}$ 이므로:
    
    $$
    |\overline{N Q}|=\frac{15}{\tan 72^{\circ}} \approx 4.8738 \mathrm{~m}
    $$
     
 2. $|\overline{N R}|$ 구하기:

우선 주어진 $|\overline{M R}|=10 \sqrt{3} \mathrm{~m}$ 를 사용해 $\triangle M R Q$ 에서 $|\overline{N R}|$ 을 구할 수 있습니다.

$$
(10 \sqrt{3})^2=15^2+|\overline{N R}|^2
$$


계산하면:

$$
\begin{gathered}
300=225+|\overline{N R}|^2 \\
|\overline{N R}|^2=75 \\
|\overline{N R}|=\sqrt{75}=5 \sqrt{3}
\end{gathered}
$$

3. $|\overline{Q R}|$ 구하기:

이제 $|\overline{N R}|=|\overline{N Q}|+|\overline{Q R}|$ 을 통해 $|\overline{Q R}|$ 를 구합니다.

$$
5 \sqrt{3}=4.8738+|\overline{Q R}|
$$


따라서:

$$
|\overline{Q R}|=5 \sqrt{3}-4.8738
$$


이를 계산하면:

$$
|\overline{Q R}| \approx 3.7865 \mathrm{~m}
$$


가장 근접한 정수로 반올림하면 $|\overline{Q R}| \approx 4 \mathrm{~m}$ 입니다.

그래서, $|\overline{Q R}|$ 의 길이는 약 4 미터가 됩니다. 추가 질문이 있으면 언제든지 물어보세요!
\end{GPTA}

