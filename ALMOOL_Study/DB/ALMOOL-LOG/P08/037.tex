\begin{GPTQ}
    위에 문제의 풀이를 말할 때 어려운 개넘들을 짧게 정리해줘      
\end{GPTQ}

\begin{GPTA}
    물론입니다! 문제 풀이에서 사용된 주요 개념들을 간단히 정리해 드리겠습니다.

    주요 개념 정리

    1. 삼각형의 내각의 합:
    
    모든 삼각형의 내각의 합은 $180^{\circ}$ 입니다.

    2. 직각삼각형:
    
    한 각이 $90^{\circ}$ 인 삼각형입니다. 나머지 두 각도의 합도 $90^{\circ}$ 이어야 합니다.

    3. 사인 함수 :

    - 직각삼각형에서, 각도 $\theta$ 에 대한 사인은 다음과 같은 비율로 정의 됩니다:
    
    $$
    \sin \theta=\frac{\text { 대변 (opposite) }}{\text { 빗변 (hypotenuse) }}
    $$
    
    4. 아크 사인 $\left(\sin ^{\wedge}\{-1\}\right)$ :

    - 주어진 사인 값을 입력으로 받아 해당 각도를 반환하는 역함수입니다. 주로 $\left[-90^{\circ}, 90^{\circ}\right]$ 범위의 각도를 출력합니다.

    5. 특정 각도의 사인 값:

    - $\sin 30^{\circ}=\frac{1}{2}$

    - $\sin 45^{\circ}=\frac{\sqrt{2}}{2}$

    - $\sin 60^{\circ}=\frac{\sqrt{3}}{2}$

    6. $\sin ^{-1}(x)$ 계산:

    - $x$ 값의 아크 사인은 그 값에 해당하는 각도를 찾을 때 사용합니다. 예를 들어, $\sin ^{-1}\left(\frac{\sqrt{3}}{2}\right)=60^{\circ}$.
    
    문제 풀이에서 적용된 부분들

    - 내각의 합 이용: 삼각형 $M N Q$ 에서 $\angle M N Q$ 를 구하기 위해 내각의 합 $180^{\circ}$ 를 사용했습니 다.

    - 사인 함수: 각도 $\theta$ 를 구하기 위해 $\sin \theta=\frac{15}{10 \sqrt{3}}$ 을 이용하여 $\theta=60^{\circ}$ 를 추출했습니다.

    - 아크 사인 함수: 주어진 사인 값을 통해 각도를 역추적하는 데 사용했습니다.
    
    이러한 개념들은 문제에서 각도를 계산하고 삼각형의 성질을 이해하는 데 중요한 역할을 합니다. 더 궁금한 것이 있으면 언제든지 알려주세요!
    
\end{GPTA}

