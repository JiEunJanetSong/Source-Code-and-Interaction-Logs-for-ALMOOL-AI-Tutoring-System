\begin{GPTQ}
    이러한 문제의 풀이 순서는 어떻게 돼? 
\end{GPTQ}

\begin{GPTA}
 좋은 질문입니다. 주어진 수식의 풀이 과정을 차근차근 설명해 드릴게요. 문제는 다음과 같습니다:

$$
2(a-4 b)\left(3+b^3\right)=?
$$


이 문제를 해결하기 위해서는 분배법칙을 사용하여 각 항을 전개합니다. 다음은 이 과정의 단계별 설명입니다:

1. 첫 번째 항 전개: $2(a-4 b)$ 와 $\left(3+b^3\right)$ 의 첫 번째 항 3 을 곱합니다.

$$
2(a-4 b) \cdot 3=6 a-24 b
$$

2. 두 번째 항 전개: $2(a-4 b)$ 와 $\left(3+b^3\right)$ 의 두 번째 항 $b^3$ 을 곱합니다.

$$
2(a-4 b) \cdot b^3=2 a b^3-8 b^4
$$

3. 모든 항 결합하기: 각각의 전개된 항들을 합칩니다.

$$
6 a-24 b+2 a b^3-8 b^4
$$

4. 정답 정리하기: 위의 항들을 문제 선택지의 형식에 맞게 정리하면,

$$
2 a b^3+6 a-8 b^4-24 b
$$

$$
2 a b^3+6 a-8 b^4-24 b
$$


따라서 풀이 과정을 퉁해 선택지 (D)가 맞다는 것을 확인할 수 있습니다. 이러한 방법으로 주어 진 문제를 해석하고 풀이할 수 있습니다. 추가적인 질문이 있으면 언제든지 말씀해 주세요!
\end{GPTA}

