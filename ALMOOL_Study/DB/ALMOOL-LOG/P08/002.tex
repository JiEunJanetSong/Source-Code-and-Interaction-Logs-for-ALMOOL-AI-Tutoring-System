\begin{GPTQ}
    4,5,6번을 자세히 알려줄래?     
\end{GPTQ}

\begin{GPTA}
    물론입니다. 등변삼각형의 높이를 구하는 방법에 대해 자세히 설명드리겠습니다.

    Step 4: 삼각형의 높이

    등변삼각형은 모든 변의 길이가 같은 삼각형으로, 높이는 삼각형의 꼭지점에서 밑변의 중점까지 수직으로 내려지는 선입니다. 문제에서 높이를 $k \sqrt{3}$ 이라고 주어졌습니다.
    
    Step 5: 30-60-90 삼각형의 특성

    등변삼각형을 중점에서 높이를 내리면, 두 개의 $30-60-90$ 직각삼각형으로 나뉩니다. 이 삼각형의 비율을 다음과 같이 이해할 수 있습니다:

    - $30^{\circ}$ 의 대변의 길이: $\frac{x}{2}$

    - $60^{\circ}$ 의 대변의 길이 (높이): $k \sqrt{3}$

    - 빗변의 길이: $s$
    
    따라서 $30^{\circ}$ 의 대변은 높이와 관계가 있고, 높이는 삼각형의 한 변으로 연걸됩니다.

    Step 6: $k$ 구하기

    주어진 수식을 이용례 $k$ 를 구해봅시다:

    - 등변삼각형의 한 변의 길이 $s$ 는 높이와 관련이 있는데, 그 관계는 $30-60-90$ 삼각형에서 $s=2 k$ 입니다 (왜냐하면 $30^{\circ}$ 의 대변이 $k \sqrt{3}$ 라고 하면, 그 두 배가 빗변 $s$ 가 되기 때문입니다 ).

    - 따라서:
    
    $$
    \begin{aligned}
    & s=208=2 k \\
    & k=\frac{208}{2}=104
    \end{aligned}
    $$
    
    
    따라서, $k$ 의 값은 104 입니다. 이러한 단계로 높이와 관계횐 값을 찾을 수 있습니다. 특수 삼각형의 관계를 활용하여 원하는 값을 구할 수 있습니다. 
\end{GPTA}

