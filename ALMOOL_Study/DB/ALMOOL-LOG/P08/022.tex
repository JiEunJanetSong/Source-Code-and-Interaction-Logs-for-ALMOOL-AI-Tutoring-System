\begin{GPTQ}
    위에서 5 번 풀이 과정에 대해 더 자세히 설명해줘 
\end{GPTQ}

\begin{GPTA}
물론입니다! 원뿔의 반지름을 구하기 위해 제곱근을 취하는 5 번 풀이 과정을 더 자세히 설명드리겠습니다.

    1. 반지름 $r$ 를 구하기 위혜 제곱근을 취하기:
    
    여기에서는 $r^2=2100$ 인 상황에서 $r$ 을 구하기 위해 제곱근을 사용합니다.
    
    $$
    r=\sqrt{2100}
    $$
    
    
    제곱근은 어떤 수를 두 번 곱해서 $r^2=2100$ 이 되는 수 $r$ 을 찾는 과정입니다. 이는 두 번 곱 했을 때 2100 이 되는 수를 구하는 것이죠.

    계산기를 사용하여 제곱근을 구하면 다음과 같이 나옵니다:
    
    $$
    \sqrt{2100} \approx 45.82575632
    $$
    
    
    문제에서는 소수 네 자리까지 답을 요구하므로 반올림하여 45.83 이 뒵니다. 따라서, 원뿔의 반지름은 약 45.83 cm 입니다.
    
    제곱근을 취하는 이유는 우리가 $r^2$ 을 알고 있을 때, 원래의 값 $r$ 을 찻기 위함입니다. 제곱을 취 했 을 때 제곱근은 그 반대 과정입니다.

    이해가 되는지요? 추가로 궁금한 점이 있으면 언제든지 말씀해 주세요!

\end{GPTA}

