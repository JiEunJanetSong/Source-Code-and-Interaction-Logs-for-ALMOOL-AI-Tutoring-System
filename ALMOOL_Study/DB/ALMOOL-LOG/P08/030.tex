\begin{GPTQ}
    그러면 다시 이 문제의 풀이를 알려줘     
\end{GPTQ}

\begin{GPTA}
    알젰습니다. 다시 문제의 풀이를 차근차근 설명드리겠습니다.
  
    문제 요약:

    다이어그램 속의 $\triangle M N R$ 은 직각삼각형입니다. $M N=15 \mathrm{~m}, M R=10 \sqrt{3} \mathrm{~m}$ 이고 $\angle M Q N=72^{\circ}$ 입니다. 이때, $\angle Q M R$ 은 $x$ 로 표시되어 있습니다.
    
    각도 $x$ 를 구하는 과정:

    1. $\angle  Q M N$ 를 구합니다:

    - $\triangle M N Q$ 에서 세 각의 합은 $180^{\circ}$ 여야 합니다.

    - $\angle M N Q+\angle M Q N+\angle N M Q=180^{\circ}$

    - 주어진 $\angle M Q N=72^{\circ}$ 이고, $\angle M N Q$ 는 직각 $\left(90^{\circ}\right)$ 입니다.

    - 따라서, $\angle N M Q+72^{\circ}+90^{\circ}=180^{\circ}$

    - $\angle N M Q+162^{\circ}=180^{\circ}$

    - $\angle N M Q=180^{\circ}-162^{\circ}=18^{\circ}$

    2. 각도 $\theta=\angle M R N$ 을 구합니다:

    - 삼각형 $M R N$ 에서는 사인 법칙을 이용합니다.

    - $\sin \theta=\frac{M N}{M R}=\frac{15}{10 \sqrt{3}}$

    - 이때 $\theta=\sin ^{-1}\left(\frac{15}{10 \sqrt{3}}\right)$ 입니다.

    - 계산하면 $\theta=60^{\circ}$ 입니다.

    3. $\angle Q M R=x$ 를 구합니다:

    - 삼각형 $M N R$ 의 내각의 합은 $180^{\circ}$ 입니다.

    - 따라서, $x+ \angle M N Q+\angle N M R+\angle M R N=180^{\circ}$

    - $x+18^{\circ}+60^{\circ}+90^{\circ}=180^{\circ}$

    - $x=12^{\circ}$
    
    따라서 $x$ 는 $12^{\circ}$ 입니다.
    이 과정들은 정확한 삼각형의 성질과 삼각함수를 이용하여 각도를 계산하는 방법을 포함하고 있습니다. 이해되지 않는 부분이 있거나 추가 질문이 있으면 언제든지 알려주세요!
   \end{GPTA}

