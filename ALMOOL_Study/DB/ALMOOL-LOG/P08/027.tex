\begin{GPTQ}
   $\sin$ 에 대해 알려줘 
\end{GPTQ}

\begin{GPTA}
$\sin$ 함수, 즉 사인(sine) 함수는 삼각 비율의 하나로, 직각삼각형에서 중요한 역할을 합니다. 사인을 이해하기 위해 다음 개념을 살펴보겠습니다.

1. 기본 정의

직각삼각형에서 사인은 다음과 같이 정의 ㅚㅚㅂ니다:
- 한 각 $\theta$ 의 사인은 그 각의 대변의 길이를 빗변의 길이로 나눈 것입니다.

수식으로 표현하면:

$$
\sin \theta=\frac{\text { 대변의 길이 }}{\text { 빗변의 길이 }}
$$

2. 단위원에서의 정의

단위원(circle with radius $=1$ )을 사용하여 사인을 정의할 수도 있습니다:

- 단위원 상의 점 $(x, y)$ 에서 각 $\theta$ 에 대 한 사인은 $y$ 좌표입니다.

3. 삼각함수의 값

사인의 값은 각도 $\theta$ 가 증가함에 따라 변합니다. 특별 한 각도에서의 사인 값은 다음과 같습니다:

- $\sin 0^{\circ}=0$

- $\sin 30^{\circ}=\frac{1}{2}$

- $\sin 45^{\circ}=\frac{\sqrt{2}}{2}$

- $\sin 60^{\circ}=\frac{\sqrt{3}}{2}$

- $\sin 90^{\circ}=1$

4. 주기성

사인 함수는 주기함수로서, 그 주기는 $360^{\circ}$ 또는 $2 \pi$ 입니다. 즉, $\sin \left(360^{\circ}+\theta\right)=\sin \theta$ 입니다.
    
5. 사인 함수의 그래프

$\sin \theta$ 함수의 그래프는 수평축으로 각도를 두고, 세로축으로 사인의 값을 두어 주기적인 파형을 그립니다. 그래프는 $\left[0,360^{\circ}\right]$ 또는 $[0,2 \pi]$ 범위에서 한 번의 완전한 파형을 완성합니다.

사인 함수는 삼각비 중 하나로, 삼각형의 각도를 사용하여 여러 길이를 구하는 데 매우 유용하며, 물리 학과 공학에서 다양한 파동과 진동을 설명하는 데 널리 사용됩니다. 추가로 궁금한 점이 있다면 언제든지 질문해주세요!
\end{GPTA}

