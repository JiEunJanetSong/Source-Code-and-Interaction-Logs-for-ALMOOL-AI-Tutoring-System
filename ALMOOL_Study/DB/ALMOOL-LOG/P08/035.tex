\begin{GPTQ}
    탄젠트를 활용하는 방법을 알려줘
\end{GPTQ}

\begin{GPTA}
물론입니다. 삼각 합수인 탄젠트를 활용하여 문제를 해걸하는 방법을 설명해 드리젰습니다. 탄젠트는 직각삼각형에서 한 각의 대변의 길이를 인접변의 길이로 나눈 값입니다. 즉, $\tan (\theta)$ 는 다음과 같은 식으로 정의됩니다:

$$
\tan (\theta)=\frac{\text { 대변의 길이 }}{\text { 인접변의 길이 }}
$$


주어진 문제에서 우리는 $\angle M Q N=72^{\circ}$ 라는 각과 그에 따른 변의 길이를 알고 있습니다. 이를 사용하여 $|\overline{N Q}|$ 를 구하는 방법은 아래와 같습니다:

$\triangle M Q N$ 에서의 탄젠트 활용:

1. 주어진 졍보:

- $|\overline{M N}|=15 \mathrm{~m}$

- $\angle M Q N=72^{\circ}$

2. 탄젠트 공식 적용:
   
2. 탄졘트 공식 적용:

$$
\tan 72^{\circ}=\frac{|\overline{M N}|}{|\overline{N Q}|}
$$

3. $|\overline{N Q}|$ 계산:

$$
|\overline{N Q}|=\frac{|\overline{M N}|}{\tan 72^{\circ}}=\frac{15}{\tan 72^{\circ}}
$$

4. 계산:
$\tan 72^{\circ}$ 의 값을 계산기에 입력하여 구하고, 15 를 이 값으로 나누면 $|\overline{N Q}|$ 의 값이 나옵니다:

$$
|\overline{N Q}| \approx 4.8738 \mathrm{~m}
$$


이와 같은 방식으로 탄젠트를 사용하여 직각삼각형에서 한 변의 길이를 구할 수 있습니다. 더 궁금한 점이 있거나 도움이 필요하면 언제든지 말씀해 주세요!

\end{GPTA}

