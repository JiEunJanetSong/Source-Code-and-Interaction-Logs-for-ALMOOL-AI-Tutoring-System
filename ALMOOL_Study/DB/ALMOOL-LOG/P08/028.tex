\begin{GPTQ}
  $\sin ^{-1}$ 은 뭐야?  
\end{GPTQ}

\begin{GPTA}
$\sin ^{-1}$, 즉 아크 사인(arc sine) 또는 인버스 사인(inverse sine)은 사인 함수의 역함수를 의미합니다. 이 함수는 주어진 사인 값에 대해 그 각도를 반환하는 함수입니다. $\sin ^{-1}(x)$ 는 ``사인의 역수"라고 혼동하지 마세요. 대신, $\sin ^{-1}(x)$ 는 아크 사인이라고 부릅니다.

1. 정의

- $\sin \theta=x$ 일 때 $\theta=\sin ^{-1}(x)$ 입니다.

- 여기서 $x$ 는 실수로서 $[-1,1]$ 사이의 값을 가집니다.

- 아크 사인 함수는 $y=\sin ^{-1}(x)$ 에서 $y$ 가 기본적으로 $\left[-\frac{\pi}{2}, \frac{\pi}{2}\right]$ 범위에 있습니다. 이는 기본 값 범위(Principal Value Range)라고 불립니다.

2. 예제

- $\sin ^{-1}(0)=0$ (예냐하면 $\sin 0=0$ 이므로)

- $\sin ^{-1}(1)=\frac{\pi}{2}\left(90^{\circ}\right)$ (왜냐하면 $\sin \frac{\pi}{2}=1$ 이므로)

- $\sin ^{-1}\left(\frac{1}{2}\right)=\frac{\pi}{6}$ (왜냐하면 $\sin \frac{\pi}{6}=\frac{1}{2}$ 이므로)

3. 사용

아크 사인 함수는 특정 사인 값과 대응하는 각도를 찾는 데 유용합니다. 특허 삼각함수의 계산을 반대로 수행하거나, 물리학에서의 진동 각 속도 등을 구하는 데 활용됩니다.
그래프적으로 보면, 아크 사인 함수는 $\sin (x)$ 함수의 그래프를 $x=y$ 직선에 대해 대칭 이동한 것 입니다.

4. 주의점

- 각도를 구할 때, 아크 사인은 여러 분기를 가질 수 있습니다. 기본적으로 이러한 합수는 $\left[-\frac{\pi}{2}, \frac{\pi}{2}\right]$ 의 범위로 제한되어 있어 한 번에 하나의 주 값을 반환합니다.

더 궁금하신 사항이 있거나 궁금한 점이 있다면 언제든지 말씀해 주세요!



\end{GPTA}

