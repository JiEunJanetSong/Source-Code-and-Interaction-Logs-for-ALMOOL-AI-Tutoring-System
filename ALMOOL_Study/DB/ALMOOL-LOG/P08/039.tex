\begin{GPTQ}
    각도 0 이 뮈야
\end{GPTQ}

\begin{GPTA}
 각도 0 도(degrees) 또는 0 라디안(radians)은 두 가지 기준점(반직선)이 동일한 방향을 가리킬 때의 각도를 의미합니다. 이는 여러 맥락에서 매우 기본적인 개념으로 사용됩니다. 여기서는 두 가지 측면에서 각도 0 을 설명 하겠습니다.
    
    0 도의 의미

    1. 연산의 기준점

    - 각도 0 은 각도 측정의 시작점입니다. 여기서부터 각도를 증가시키거나 감소시키며 다른 각도를 측정합니다.

    2. 단위원(circle)에서

    - 반지름이 1 인 탄위원의 중심에서 오른쪽으로 그린 선분이 $x$-축과 겹치도록 할 때 각도 0 입니다.

    - 각도 0 에서는 단위원 위의 점은 $(1,0)$ 이 됩니다.
    
    성질과 예시

    1. 사인 함수

    - $\sin \left(0^{\circ}\right)=0$ : 사인 함수는 대변과 빗변의 비율이므로 0 도에서는 대변이 없습니다.

    2. 코사인 함수

    - $\cos \left(0^{\circ}\right)=1$ : 코사인 함수는 인접변과 빗변의 비율로, 0 도에서는 인접변이 빗변과 동 일합니다.

    3. 탄졘트 함수

    - $\tan \left(0^{\circ}\right)=0$ : 탄젠트는 사인 값을 코사인 값으로 나눈 것이므로 0 도에서는 $\tan \left(0^{\circ}\right)=\frac{0}{1}=0$ 입니다.
    
    활용

    - 각도 0 은 삼각 측량, 물리학, 공학 등에서 초기 방향을 설정하거나 기준값을 정의할 때 사용 됩니다.

    각도 0 은 수학적 연산과 물리적 세계 모두에서 중요한 기준점으로, 여러 함수와 개념이 이로부터 출발하여 정의큅니다. 추가로 궁금한 사항이 있다면 언제든지 질문해주세요! 
\end{GPTA}

