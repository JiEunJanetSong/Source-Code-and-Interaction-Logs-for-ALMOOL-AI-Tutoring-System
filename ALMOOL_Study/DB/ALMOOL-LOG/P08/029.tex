\begin{GPTQ}
    아크 사인 함수를 쉽게 설명해줘   
\end{GPTQ}

\begin{GPTA}
    아크 사인 함수는 어떤 숫자의 사인 값이 주어졌을 때, 그 숫자가 나오는 각도를 찾아주는 함수입니다. 조금 더 쉽게 설명하자면:

    기본 개념

    - 사인 함수는 각도를 입력으로 받고, 그 각도의 사인 값을 출력으로 줍니다. 예를 들어, $\sin \left(30^{\circ}\right)=0.5$ 입니다.

    - 아크 사인 함수는 반대로, 사인 값을 입력으로 받아 그에 해당하는 각도를 출력합니다. 예를 들어, $\sin ^{-1}(0.5)=30^{\circ}$ 입니다.
    
    작동 방식

    - 범위: 아크 사인 함수는 -1 부터 1 사이의 숫자에 대해서만 작동합니다. 이는 사인 함수의 걸과가 이 범위 내에 있기 때문입니다.

    - 출력 각도: 걸과 각도는 일반적으로 $\left[-90^{\circ}, 90^{\circ}\right]$ 또는 $\left[-\frac{\pi}{2}, \frac{\pi}{2}\right]$ 범위 내에 있습니다.
    
    예시로 이해하기

    1. 숫자 0.5 의 아크 사인을 구할 때:

    - $\sin ^{-1}(0.5)=30^{\circ}$ 또는 $\frac{\pi}{6}$ 입니다.

    - 이는'어떤 각도의 사인이 0.5 인 각도는 무엇인가?'라는 질문에 답하는 것입니다.

    2. 숫자 1 의 경우:

    - $\sin ^{-1}(1)=90^{\circ}$ 또는 $\frac{\pi}{2}$ 입니다.

    - 이 각도는 사인 값이 1 일 때의 각도입니다.
    
    예 필요한가

    - 삼각형 문졔 해결: 삼각형의 변 길이로 각도를 찾아야 할 때 유용합니다.

    - 주기적인 데이터 분석: 파동이나 주기적인 신호의 특정 값을 각도로 변환하는 데 쓰입니다.
    
    아크 사인 함수는 기본적으로 `어느 각도가 이런 사인 값을 가질까?'라는 질문을 해결해주는 도구 입니다. 추가 질문이 있다면 언제든지 알려주세요!
    
\end{GPTA}

