\begin{GPTQ}
    앞에 설명을 자세히 해줘 
\end{GPTQ}

\begin{GPTA}
물론입니다. 문제의 앞부분에 대해 좀 더 자세허 설명드리젰습니다. 합수 $f(w)=6 w^2$ 는 직사각형의 넓이를 나타냅니다. 이 함수에서:

    - $w$ 는 직사각형의 너비를 나타내며, 단위는 피트(ft)입니다.

    - $6 w^2$ 는 직사각형의 넓이를 $\mathrm{ft}^2$ 로 나타냅니다.
    
    문제에서 주어진 상황은 길이가 너비의 6 배라는 것입니다. 그러므로:

    - 만약 너비가 $w$ 라면, 길이는 $6 w$ 가 됩니다.

    - 직사각형의 넓이는 너비와 길이를 곱한 값입니다. 즉, 넚이는 $w \times 6 w=6 w^2$ 가 됩니다.
    
    따라서, 함수 $f(w)=6 w^2$ 는 이 넘이를 나타내며, $f(14)=1,176$ 이라는 것은 너비가 14 피트일 때의 넓이가 1,176 제곱 피트라는 의미입니다.

    이러한 맥락에서 주어진 해석들을 검토한 결과, 선택지 1 번이 올바릅니다: ``너비가 14 ft 인 경우, 직사각형의 넓이는 $1,176 \mathrm{ft}^2$ 입니다."
    \end{GPTA}

