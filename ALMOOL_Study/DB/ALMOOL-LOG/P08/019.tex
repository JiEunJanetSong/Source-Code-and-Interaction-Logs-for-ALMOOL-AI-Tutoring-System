\begin{GPTQ}
    비율을 적용하는 식을 설명해줘 
\end{GPTQ}

\begin{GPTA}
비율을 적용하는 식은 원뿔과 원뿔대의 기하학적 특성을 이용하여 설정됩니다. 원뿔대는 원래 원뿔의 꼭짓점을 일정한 높이에서 잘라낸 형태로, 이들이 서로 유사한 삼각형 관계에 있음을 이용할 수 있습니다.
    
    삼각형 비율과

    1. 유사 삼각형 관계:

    - 원래 원뿔과 원뿔대의 잘린 부분은 유사한 삼각형을 형성합니다. 즉, 원뿔의 높이와 반지름의 비율은 동일합니다.

    - 이 유사한 삼각형의 한 쌍은 원뿔대의 아랫면과 원래 원뿔의 밑면을 잇는 삼각형과, 윗면과 원뿔의 꼭짓점을 잇는 작은 삼각형입니다.

    2. 비율 식 세우기:

    - 원래 원뿔의 높이를 $H$, 원뿔대의 높이를 $h$, 아랫면 반지름을 $R$, 윗면 반지름을 $r$ 라고 하고,

    - 원뿔의 윗면에서 원뿔대의 윗면까지의 높이는 $H-h$ 로 할 수 있습니다.

    3. 유사 삼각형 비율:
    
    $$
    \frac{H-h}{R}=\frac{H}{r}
    $$
    
    
    여기서, 각 비율은 다음과 같이 설명할 수 있습니다:

    - $\frac{H-h}{R}$ 은 윗 부분의 삼각형에서 높이와 반지름의 비율,

    - $\frac{H}{r}$ 은 원래 왼뿔의 전체 높이와 윗면 반지름의 비율입니다.

    4. 식의 해석:

    - 이 식은 원뿔과 원뿔대의 형상 간의 기하학적 유사성에 기반합니다.

    - 이를 통해 하나의 주어진 값과 비율로 원래 원뿔의 나머지 높이나 반지름을 추론할 수 있습니다.
    
    이 과정을 통헤 원뿔대 문제에서 주어진 비율을 활용하여 다양한 변수를 해결할 수 있습니다.
    \end{GPTA}

