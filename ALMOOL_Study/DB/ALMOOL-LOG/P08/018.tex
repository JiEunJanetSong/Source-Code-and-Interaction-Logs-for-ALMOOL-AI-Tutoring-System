\begin{GPTQ}
    원뿔대의 높이와 비율만 알고 있을때 어떻게 문제를 풀어?  
\end{GPTQ}

\begin{GPTA}
    원뿔대의 높이와 특정 반지름의 비율만 알고 있는 경우, 원래 원뿔의 높이 및 반지름을 이용해 문제를 해결할 수 있습니다.

    원뿔대의 경우, 원추형 원뿔 중심에서 위쪽 또는 아래쪽 비율을 사용하여 문제를 해결할 수 있습니다. 예를 들어, 원뿔대가 하단 반경이 상단 반경의 세 배라고 알려진 경우를 생각혜 보겠습니다. 이 경우 해걸 과정은 다음과 같습니다:
    
    예제 문제 설정

    - 원뿔대의 높이 $h=12 \mathrm{~cm}$

    - 아랫면 반지름 $R$ 이 윗면 반지름 $r$ 의 세 배 $(R=3 r)$ 입니다.
    
    원래 원뿔 높이 찾기

    1. 비율 설정: 원뿔높이를 $H$ 로 설정하고, 높이 비율을 사용하여 원뿔의 나머지 높이 찾기.

    2. 삼각형 비율 적용하기:
    
    $$
    \frac{H-h}{R}=\frac{H}{r}
    $$
    
    
    여기서, $R=3 r$ 을 대입하면:
    
    $$
    \frac{H-12}{3 r}=\frac{H}{r}
    $$
    
    3. 식 정리하기:
    
    $$
    r(H-12)=3 r H
    $$
    
    
    $$
    H-12=3 H
    $$
    
    
    $$
    2 H=12
    $$
    
    
    $$
    H=6
    $$
    
    
    전체 원뿔 높이 계산

    원래의 원뿔 높이는 $h+H=12+6=18 \mathrm{~cm}$ 입니다.
    이런 과정을 통해 비율과 높이를 이용한 문제도 해결할 수 있습니다. 이 계산은 비대칭 원뿔대 문제에 유용합니다.  
\end{GPTA}

