\begin{GPTQ}
    풀이를 수식으로 정리해줘   
\end{GPTQ}

\begin{GPTA}
물론입니다. 주어진 문제를 수식으로 정리하젰습니다.

    1. 등변삼각형 의 각 변의 길이를 $s$ 라고 하겠습니다.

    2. 삼각형의 둘레는 $3 s=624 \mathrm{~cm}$ 입니다.
    
    $$
    3 s=624
    $$
    
    3. 따라서 한 변의 길이 $s$ 는
    
    $$
    s=\frac{624}{3}=208 \mathrm{~cm}
    $$
    
    4. 이제 삼각형의 높이를 구해봅시다. 높이를 $k \sqrt{3}$ 이라고 주어졌습니다.

    5. 픅수한 삼각형인 $30-60-90$ 삼각형의 비율에 따라서, $30^{\circ}$ 의 대변의 길이는 $\frac{\pi}{2}, 60^{\circ}$ 의 대변인 높이는 $k \sqrt{3}$, 그리고 빗변(삼각형의 한 변) 길이는 $s$ 입니다.
    
    $$
    \frac{s}{2}=k
    $$
    
    6. 이를 통해 $k$ 를 구하면 다음과 같습니다.
    
    $$
    \begin{gathered}
    s=2 k \Rightarrow 208=2 k \\
    k=\frac{208}{2}=104
    \end{gathered}
    $$
    
    
    따라서, $k$ 의 값은 104 입니다.
  
\end{GPTA}

