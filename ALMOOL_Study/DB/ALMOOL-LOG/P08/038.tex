\begin{GPTQ}
    <quotedChat>
    
    **내각의 합 이용**: 
    
    삼각형 $M N Q$ 에서 $\angle M N Q$를 구하기 위해 내각의 합 $180^{\circ}$ 를 사용했습니다.
    
    <quotedChat> 
    
    이 문제를 해결하는 더 간단한 방법은 없어? 
\end{GPTQ}

\begin{GPTA}
 확실히 더 간단하게 문제에 접근할 수 있는 방법을 생각해 볼 수 있습니다. 이 문제에서 우리는 기본적으로 삼각헝의 내각의 합과 삼각함수의 관계를 이용하여 각도를 찾고 있었습니다. 간단한 접근법을 제시해드리겠습니다.
    
    문제 해결을 위한 간단한 접근법:  

    1. 삼각형의 각도 부분:

    - 직각 삼각형 $\triangle M N R$ 에서 $\angle N M R=x$ 로서, $\angle N M Q=90^{\circ}$ 입니다.

    - 주어진 $\angle M Q N=72^{\circ}$ 을 이용해 빠르게 계산할 수 있습니다.

    2. 직각의 관계 활용:

    - $\angle M N Q=18^{\circ}$ 는 삼각형 의 직각 관계에서 직접적으로 구할 수 있습니다. 이는 삼각 형 $M N Q$ 에서의 $\angle N M Q=90^{\circ}$ 와 $\angle M Q N=72^{\circ}$ 를 알고 있어 바로 구할 수 있는 것과 동일합니다.

    - 따라서, $180^{\circ}-72^{\circ}-90^{\circ}=18^{\circ}$ 로 $\angle M N Q$ 가 바로 구해집니다.

    3. 각도 $\theta$ 활용 없이 직각삼각형 이용:

    - $\theta=\angle M R N$ 도 굳이 계산하지 않고, 전체 내각 관계와 직각으로 인해 $x=12^{\circ}$ 를 얻을 수 있습니다.
    
    이러한 접근법은 이미 우리가 직각 삼각형의 성질을 충분히 이해하고 있다면 서로의 관계를 통해 쉽게 찾을 수 있습니다. 이처럼 문제의 기본 성질을 활용하면 조금 더 직관적으로 접근할 수 있습니다. 도움이 되었기를 바랍니다! 추가로 더 설명이 필요하다면 언제든지 알려주세요.
   \end{GPTA}

