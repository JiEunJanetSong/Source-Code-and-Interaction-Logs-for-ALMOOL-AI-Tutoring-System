\begin{GPTQ}
    간단한 해결법에 대한 자세한 설명   
\end{GPTQ}

\begin{GPTA}
물론입니다. 문제에 대한 간단한 해결법을 자세허 설명하졌습니다.

문제 이해하기

주어진 문제는 직각 삼가형 $\triangle M N R$ 에 관한 것입니다. 측면 $M N=15, \mathrm{~m}, M R=10 \sqrt{3}, \mathrm{~m}$ 이고, 추가 각도 $\angle M Q N=72^{\circ}$ 가 주어졌습니다. $\angle N M R$ 을 구하는 것이 목적입니다.

1. 대략적인 내각의 합:

직각 삼각형에서 한 각은 항상 $90^{\circ}$ 입니다.

2. 각각의 각 $\angle M N Q$ 구하기:

- 먼저, 각 $\angle M N Q$ 는 삼각형 $M N Q$ 에서 $\angle N M Q=90^{\circ}$ 를 이용하여 구할 수 있습니다.

- 삼각형의 내각의 합은 $180^{\circ}$ 라는 사실을 이용합니다. 따라서:

$$
\begin{gathered}
\angle M N Q+\angle M Q N+\angle N M Q=180^{\circ} \\
\angle M N Q+72^{\circ}+90^{\circ}=180^{\circ} \\
\angle M N Q=180^{\circ}-72^{\circ}-90^{\circ}=18^{\circ}
\end{gathered}
$$

3. 직각 삼각형의 내각 활용:

- 삼각형 $\triangle M N R$ 예서 $\angle N M R=x$ 를 찾고자 합니다.

- $\angle M R N$ 은 주어진 변의 비율로 인해 $\sin \left(60^{\circ}\right)=\frac{\sqrt{3}}{2}$ 로 쉽게 계산될 수 있습니다.

- 하지만 이 경우 $\angle M R N$ 및 $x$ 는 위에서 계산한 다른 각도를 활용하였습니다:

$$
\angle N M R+60^{\circ}+18^{\circ}=90^{\circ}
$$


$$
x+78^{\circ}=90^{\circ}
$$


따라서

$$
x=90^{\circ}-78^{\circ}=12^{\circ}
$$


요약:

이 간단한 접근법으로 삼각형의 기본 내각과 직각삼각형의 성질을 활용하여 계산을 줄이고 각도를 쉽게 구할 수 있었습니다.

이해가 안 되는 부분이 있거나 추가 설명이 필요하다면 언제든지 말씀해 주세요!
\end{GPTA}

