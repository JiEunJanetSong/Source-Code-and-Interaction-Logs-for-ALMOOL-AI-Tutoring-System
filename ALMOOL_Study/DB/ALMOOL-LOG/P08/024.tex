\begin{GPTQ}
    이제 다시 하나의 풀이로 적어줘 
\end{GPTQ}

\begin{GPTA}
물른입니다! 원뿔의 부피로부터 반지름을 구하는 전체 풀이 과정을 하나로 정리해드리젰습니다
    
    문졔 : 전체 원뿔의 부피가 $39,600 \mathrm{~cm}^3$ 일 때, 반지름을 소수 네 자리까지 구하시오. [ $\pi=\frac{22}{7}$ 사용]

    1. 원뿔의 부피 공식:
    
    $$
    V=\frac{1}{3} \pi r^2 h
    $$
    
    
    여기서 $V$ 는 원뿔의 부피, $r$ 은 밑면의 반지름, $h$ 는 높이입니다.

    2. 주어진 값 대입:
    
    문제에서는 $V=39,600 \mathrm{~cm}^3, h=18 \mathrm{~cm}, \pi=\frac{22}{7}$ 가 주어졌습니다. 그러므로 공식을 다음과 같이 설정할 수 있습니다:
    
    $$
    39,600=\frac{1}{3} \times \frac{22}{7} \times r^2 \times 18
    $$
    
    3. 공식을 정리하여 $r^2$ 구하기:  
   
    $$
39,600=\frac{396}{21} \times r^2
$$


이 식을 풀어서 $r^2$ 을 찾습니다:

$$
r^2=\frac{39,600 \times 21}{396}
$$

4. $r^2$ 계산:

$$
r^2=2100
$$

5. 반지름 $r$ 을 구하기 위해 제곱근을 취하기:

$$
r=\sqrt{2100}=10 \sqrt{21} \approx 45.82575632
$$


소수 네 자릿수까지 반을림하여:

$$
r \approx 45.83 \mathrm{~cm}
$$


따라서, 원뿔의 반지름은 약 45.83 cm 입니다.
이와 같이 계산하여 반지름을 구할 수 있습니다. 추가로 궁금한 점이 있으면 언제든지 말씀해 주세요!

\end{GPTA}

