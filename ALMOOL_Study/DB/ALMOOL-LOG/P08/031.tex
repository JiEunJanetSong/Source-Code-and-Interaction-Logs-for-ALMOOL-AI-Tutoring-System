\begin{GPTQ}
    2 번 과정을 설명해줘
\end{GPTQ}

\begin{GPTA}
물론입니다! 각도 $\theta=\angle M R N$ 을 구하는 과정을 자세히 설명드리겠습니다.

    2. 각도 $\theta=\angle M R N$ 구하기
    
    문맥 속 삼각형:

    - 우리는 직각삼각형 $\triangle M N R$ 에서 $\theta=\angle M R N$ 을 찾고 있습니다.

    - $\triangle M N R$ 에서 주어진 값은 $|\overline{M N}|=15 \mathrm{~m}$ 그리고 $|\overline{M R}|=10 \sqrt{3} \mathrm{~m}$ 입니다.   
  
    삼각비 사용하기:

    - 직각삼각형 에서 사인(sine) 함수는 각 $\theta$ 에 대응하여 다음과 같이 정의 됩니다:
    
    $$
    \sin \theta=\frac{\text { 대변 (opposite) }}{\text { 빗변 (hypotenuse) }}
    $$
    
    - 여기서 대변은 $M N$ 이고, 빗변은 $M R$ 입니다. 따라서,
    
    $$
    \sin \theta=\frac{M N}{M R}=\frac{15}{10 \sqrt{3}}
    $$
    
    
    계산 과정:

    1. $\sin \theta=\frac{15}{10 \sqrt{3}}$

    2. 이 값을 간단히 하면:
    
    $$
    \sin \theta=\frac{15}{10 \sqrt{3}}=\frac{15}{10 \sqrt{3}} \times \frac{\sqrt{3}}{\sqrt{3}}=\frac{15 \sqrt{3}}{30}=\frac{\sqrt{3}}{2}
    $$
    
    3. $\theta=\sin ^{-1}\left(\frac{\sqrt{3}}{2}\right)$ 인 $\theta$ 값을 찾습니다.

    4. 사인 함수로 다음 값을 알고 있습니다:
    
    $$
    \sin 60^{\circ}=\frac{\sqrt{3}}{2}
    $$
    
    
    결과적으로, $\theta=60^{\circ}$ 입니다.

    이 과정은 삼각형의 길이를 가지고 각도를 구할 때 매우 유용하며, 특히 삼각비의 관계를 사용하여 정확한 각도를 찾는 데 도움을 줍니다. 추가 질문이 있거나 더 알고 싶은 부분이 있다면 언제든지 문의하세요!
    \end{GPTA}

