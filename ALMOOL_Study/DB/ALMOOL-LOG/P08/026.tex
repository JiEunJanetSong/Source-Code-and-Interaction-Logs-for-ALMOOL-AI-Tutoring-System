\begin{GPTQ}
    각도 구하는 방법을 자세히 알려줘
\end{GPTQ}

\begin{GPTA}
각도를 구하는 데는 다양한 삼각형과 삼각 함수의 성질을 이용합니다. 주어진 문제에서 각도를 구하는 방법을 단계별로 설명드리겠습니다.
    
    주어진 삼각형과 정보들:

    - $\triangle M N R$ 은 직각삼각형으로서 $\angle M N Q$ 와 $\angle M N Q\left(=90^{\circ}\right)$ 이 주어졌습니다.

    - $M N=15 \mathrm{~m}, M R=10 \sqrt{3} \mathrm{~m}$, 그리고 $\angle M Q N=72^{\circ}$ 입니다.
    
    삼각형 $N M Q$ 에서 $\angle N M Q$ 를 구하기:

    1. 삼각형의 내각 합 구하기:
    
    $$
    \angle M N Q+\angle M Q N+\angle N M Q=180^{\circ}
    $$
    
    
    여기서 $\angle M N Q=90^{\circ}$ (직각).
    
    $$
    \angle N M Q+72^{\circ}+90^{\circ}=180^{\circ}
    $$
    
    
    $$
    \angle N M Q+162^{\circ}=180^{\circ}
    $$
    
    
    따라서,
    
    $$
    \angle N M Q=180^{\circ}-162^{\circ}=18^{\circ}
    $$
    
    
    삼각형 $N M R$ 에서 $\theta(\angle M R N)$ 를 구하기:

    1. 삼각비를 사용하여 $\theta$ 구하기:
    
    $$
    \begin{gathered}
    \sin \theta==\frac{M N}{M R} \\
    \sin \theta=\frac{15}{10 \sqrt{3}}
    \end{gathered}
    $$
    
    
    여기서, 계산하면:
    
    $$
    \theta=\sin ^{-1}\left(\frac{15}{10 \sqrt{3}}\right)=60^{\circ}
    $$
    
    
    마지막으로 $\angle N M R$ 에서 $x$ 구하기:

    1. 각도의 합으로 $x$ 계산하기:  

    $$
\triangle M N R \text { 에서는 } \angle  R M N=x
$$


$$
\begin{aligned}
& 18^{\circ}+x+60^{\circ}=90^{\circ} \\
& x=90^{\circ}-78^{\circ}=12^{\circ}
\end{aligned}
$$


각도 $x$ 는 $12^{\circ}$ 입니다.

이러한 과정을 퉁해 각도를 구하는 방법을 배울 수 있습니다. 필요하다면 다른 삼각함수를 사용 하거나, 삼각형의 성질을 사용하여 각도를 구하는 방법도 고려할 수 있습니다. 추가로 질문이 있다면 언제든지 알려 주세요!
\end{GPTA}

