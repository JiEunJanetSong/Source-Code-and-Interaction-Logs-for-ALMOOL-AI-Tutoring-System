\begin{GPTQ}
    원뿔의 반지름을 어펗게 구해?    
\end{GPTQ}

\begin{GPTA}
    줗은 질문입니다! 전체 원뿔의 부피를 이용해 원뿔의 반지름을 구하는 과정을 단계별로 설명드 리젰습니다.
    
    1. 원뽈의 부피 공식:
    
    원뿔의 부피 $V$ 는 다음과 같은 공식으로 주어집니다:
    
    $$
    V=\frac{1}{3} \pi r^2 h
    $$
    
    
    여기서 $r$ 은 원뽈의 밑면 반지름이고, $h$ 는 원뿔의 높이입니다.

    2. 문제에 주어진 값 대입:
    
    문제에서 원뿔의 부피 $V$ 가 $39,600 \mathrm{~cm}^3$ 이고, 원뽈의 높이 $h$ 가 18 cm 라고 주어졌습니다. $\pi$ 는 $\frac{22}{7}$ 로 근사합니다. 따라서 공식을 아래처럼 쓸 수 있습니다:
    
    $$
    39,600=\frac{1}{3} \times \frac{22}{7} \times r^2 \times 18
    $$
    
    3. 공식을 정리하여 $r^2$ 를 구하기:
    
    $$
    39,600=\frac{396}{21} \times r^2
    $$
    
    
    이를 풀면:
    
    $$
    r^2=39,600 \times \frac{21}{396}
    $$
    
    4. $r^2$ 계산:
    
    $$
    r^2=2100
    $$
    
    5. 반지름 $r$ 를 구하기 위혜 졔곱근을 춰하기:
    
    $$
    r=\sqrt{2100}=10 \sqrt{21}
    $$
    
    
    이를 소수로 계산하면 약 45.83 cm 입니다.

    따라서, 원뿔의 반지름은 약 45.83 cm 입니다.
    
\end{GPTA}

