\begin{GPTQ}
    문제를 순서대로 풀어줘     
\end{GPTQ}

\begin{GPTA}
    좋습니다! 주어진 문제를 순서대로 풀어드리겠습니다. 
    문제에서는 ABC 대학이 1 명의 헤드 코치, 9 명의 어시스턴트 코치, 그리고 2 명의 대학원 조교 코치를 허 용하고 있습니다. 따라서 총 코치 의 수는 최대 12 명이 뒬 수 있습니다. 그리고 최소한 1 명의 코치가 4 명의 선수를 책임져야 한다고 했으므로, 코치와 선수의 비율은 최대 $1: 4$ 입니다.

    그래서 아래와 같은 불평등을 사용하여 시스템을 설정할 수 있습니다:

    1. 충 코치의 수는 12 명 이하이어야 합니다.
    
    $$
    C \leq 12
    $$
    
    2. 한 코치가 최대 4 명의 선수를 책입질 수 있으므로, 선수가 코치 수의 4 배를 초과해서는 안 됩니다.

    - $P \leq 4 C$
    
    위의 부등식을 결합하여 시스템을 구성하면, $(C \leq 12)$ 및 $(P \leq 4 C)$ 의 형태로 됩니다.

    정답은 선택지 (B)입니다. 이 선택지는 위의 두 부등식을 모두 만족합니다. 따라서 ABC 대학은 이러한 제한을 통해 코치와 선수의 적절한 비율을 유지할 수 있습니다. 
\end{GPTA}

