\begin{GPTQ}
    equilateral triangle에도 $30-60-90$ 삼각형의 특성을 사용할 수 있는거야?   
\end{GPTQ}

\begin{GPTA}
네, 이론적으로 30-60-90 삼각형의 특성을 사용할 수 있습니다. 등변삼각형에서는 각 변의 길이가 같고, 모든 내각이 $60^{\circ}$ 입니다. 높이를 그리면 두 개의 합동 $30-60-90$ 삼각형으로 삼각형을 나눌 수 있습니다. 이때, 원래의 등변삼각형의 밑변을 중점에서 수직으로 나누기 때문에 각 $30^{\circ}$ 삼각형에는 다음의 특성이 적용됩니다.  

- 나뉘어진 두 변은 각각 30-60-90 삼각형을 이룹니다.

- 30-60-90 삼각형의 변 수학적 대수는 다음과 같습니다:

- $30^{\circ}$ 대변: $\frac{x}{2}$, 삼각형의 밑변을 반으로 나눈 값

- $60^{\circ}$ 대변 (높이): $k \sqrt{3}$ 로 표현된 높이

- 빗변: 등변삼각형의 한 변의 길이 $s$

이 관계로 인해, 30-60-90 삼각형의 특성은 등변삼각형에 적용되어, 특히 높이를 구하거나 변의 길이를 계산할 때 유용하게 사용될 수 있습니다.
\end{GPTA}

