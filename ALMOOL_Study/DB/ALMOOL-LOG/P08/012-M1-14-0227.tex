\begin{GPTQ}
    문제를 순서대로 풀어줘   
\end{GPTQ}

\begin{GPTA}
물론입니다! 문제를 합께 단계별로 해결해 보겠습니다.
    우선 주어진 문제를 다시 살펴보겠습니다:

    문제 : 어떤 원통의 부피가 $x$ 세제곱 인치입니다. 원퉁의 반지름이 변하지 않고 원퉁의 높이가 세 배가 되면, 새 원통의 부피는 $x$ 를 기준으로 얼마가 될까요?
    문제 해결을 위해 다음과 같이 진행할 수 있습니다:

    1. 원래 부피 공식 확인: 원퉁의 부피 $V$ 는 원주율 $\pi$, 반지름 $r$, 높이 $h$ 를 사용하여 다음과 같이 계산합니다:
    
    $$
    V=\pi r^2 h
    $$
    
    
    주어진 부피 $V=x$ 이므로,
    
    $$
    x=\pi r^2 h
    $$
    
    2. 새로운 부피 계산: 높이가 세 배가 된다면 새 원퉁의 높이는 $3 h$ 가 됩니다. 새로운 부피 $V_{\mathrm{new}}$ 는 다음과 같습니다:
    
    $$
    V_{\mathrm{new}}=\pi r^2(3 h)=3\left(\pi r^2 h\right)
    $$
    
    3. 기존 부피에 대한 세로운 부피 비교: 기존 부피가 $x=\pi r^2 h$ 였으므로,
    
    $$
    V_{\mathrm{Dew}}=3 x
    $$
    
    
    따라서, 원통의 새로운 부피는 기존 부피의 세 배인 $3 x$ 가 됩니다. 이리하여 정답은 (B) $3 x$ 입니다.
      
\end{GPTA}

