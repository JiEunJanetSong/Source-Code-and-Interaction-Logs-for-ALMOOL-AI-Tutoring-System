\begin{GPTQ}
    원뿔의 부피 공식에서 사용된 부호들이 각각 어떤 것을 말하는지 알려줘   
\end{GPTQ}

\begin{GPTA}
물론입니다! 원뿔의 부패 공식에서 사용된 각 부호가 나타내는 의미를 설명해드리겠습니다:
    
    $$
    V=\frac{1}{3} \pi r^2 h
    $$

    - $V$ : 원뿔의 부피를 나타냅니다. 따라서 이 공식의 오른쪽 계산 결과가 원뿔의 부피가 됩니다.

    - $\frac{1}{3}$ : 원뿔은 원기둥의 부피의 3 분의 1 에 해당하므로, 이 부분이 원뿔의 부피 공식의 시작 부 분입니다.

    - $\pi$ : 원주율로, 약 3.14159 입니다. 문제에서는 근사값으로 $\frac{22}{7}$ 을 사용하라고 했습니다. 이는 밑면의 원주와 면적을 계산할 때 사용됩니다.

    - $r$ : 원뿔의 밑면의 반지름을 나타냅니다. 밑면은 원의 형태를 가지고 있으므로, 반지름이 필요합니다.

    - $r^2$ : 밑면의 원의 면적을 계산하기 위한 제곱입니다. 즉, $\pi r^2$ 은 원의 면적을 의미합니다.

    - $h$ : 원뿔의 높이를 나타냅니다. 이는 원의 중심에서 원뿔 꼭대기까지의 수직 거리를 말합니다.
    
    이 공식은 원뿔의 기하학적 톡성을 기반으로 부피를 계산하는데 사용뒵니다. 원퉁과 비교하면 높이와 밑면의 크기로 부피를 정하는 방식은 유사하나, 원뿔 모양이 점점 좁아지는 구조로 인해 $\frac{1}{3}$ 이라는 계수가 포합되어 있습니다.
    \end{GPTA}

