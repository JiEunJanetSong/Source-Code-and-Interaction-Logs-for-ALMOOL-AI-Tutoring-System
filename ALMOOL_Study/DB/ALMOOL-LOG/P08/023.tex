\begin{GPTQ}
    위에 2 번 풀이과정을 자세히 설명해줘     
\end{GPTQ}

\begin{GPTA}
    알겠습니다! 2 번 풀이 과정을 좀 더 자세히 설명드리겠습니다. 이 과정에서는 원뽈의 부피 표현 식을 구체화하여 주어진 값을 대입합니다.

    1. 문제에 주어진 값 대입: 우리는 원뿔의 부펴 공식을 사용하고 있으며, 공식은 다음과 같습니다:
    
    $$
    V=\frac{1}{3} \pi r^2 h
    $$
    
    
    여기서:

    - $V$ 는 원뿔의 전체 부피: $39,600 \mathrm{~cm}^3$

    - $\pi$ 는 원주율로 주어진 값: $\frac{22}{7}$

    - $h$ 는 원뽈의 전체 높이: 18 cm
    
    이 주어진 값들을 부피 공식에 대입하여 식을 세웁니다.
    
    $$
    39,600=\frac{1}{3} \times \frac{22}{7} \times r^2 \times 18
    $$
    
    
    이제 공식을 순서대로 풀어 봅시다:

    - 첫 번째로 $\frac{22}{7}$ 는 대략적으로 3.14 에 해당하는 원주율의 근사값입니다.

    - 전체 높이 $h$ 를 18 로 대체합니다.

    - 따라서, 부피 식은 $\frac{396}{21} \times r^2$ 로 변형됩니다.
    
    이 식에서 $r^2$ 을 쉅게 결정하도록, 나머지 상수를 별도로 묶어서 정리할 준비가 되는 단계입니다. 이는 다음 단계에서 $r^2$ 을 산출하기 쉽도록 정리해줍니다.
    
    이 단계에서는 부피 및 높이와 관련된 값을 이미 알고 있으며, 이를 통해 원뿔 밑면의 반지름의 제곱 $r^2$ 을 구할 수 있도록 공식의 형태를 조정하는 과정입니다.

    추가로 궁금한 사항이 있으면 언제든지 말씀하세요!
   \end{GPTA}

