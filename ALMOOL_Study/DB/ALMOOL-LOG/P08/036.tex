\begin{GPTQ}
    아크 사인에 대해 알려줘   
\end{GPTQ}

\begin{GPTA}
 아크 사인 함수, 또는 역 사인 함수는 사인 함수의 반대 역할을 합니다. 주어진 사인 값을 입력으 로 받으면, 그 값에 해당하는 각도를 찾아주는 함수입니다. 이를 통해 원래의 각도를 구할 수 있습니다.
    
    기본 개념

    - 표기법: $\sin ^{-1}(x)$ 또는 $\operatorname{asin}(x)$ 로 표기합니다.

    - 입력과 출력:

    - 입력 값 $x$ 는 -1 에서 1 사이의 실수여야 합니다. 
    이는 사인 함수의 범위이기 때문입니다.

    - 출력 값은 각도 $\theta$ 로, 주로 $\left[-90^{\circ}, 90^{\circ}\right]$ 또는 $\left[-\frac{\pi}{2}, \frac{\pi}{2}\right]$ 범위에 있습니다.
    
    예시로 이해하기

    - 주어진 $\sin ^{-1}(0.5)$ 는 `어떤 각도의 사인 값이 0.5 인가' 라는 질문을 의미합니다.

    - 답은 $30^{\circ}$ 또는 $\pi / 6$ 입니다, 왜냐하면 $\sin 30^{\circ}=0.5$ 이기 때문입니다.
    
    사용 사례

    - 삼각형 문졔: 특정 사인 값을 알고 있을 때, 그 값과 관련된 삼각형의 각도를 찾는 데 유용 합니다.

    - 공학: 주기적인 신호나 파동의 위상을 계산할 때 사용됩니다.
    
    특징

    - 아크 사인 함수는 사인 함수의 역함수로, 원래의 각도를 복원하는 기능을 합니다.

    - 여러 해를 가질 수 있는 삼각 함수와는 달리, 아크 사인 합수는 기본 값 범위 내에서 단일 해를 제공합니다.
    
    시각화

    - 아크 사인 함수의 그래프는 사인 함수 그래프의 뒤집힌 형태로, $y=x$ 선에 대해 대칭입니다.
    
    아크 사인 함수는 삼각형과 삼각 함수에서 각도를 찾는 역산의 도구로 매우 유용합니다. 이와 관련해 더 알고 싶은 내용이나 질문이 있으면 언제든지 말씀해 주세요!
    
\end{GPTA}

