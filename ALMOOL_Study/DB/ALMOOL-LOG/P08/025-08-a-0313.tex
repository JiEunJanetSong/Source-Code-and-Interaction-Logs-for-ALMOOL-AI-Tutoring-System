\begin{GPTQ}
    해결방법을 자세히 설명해줘  
\end{GPTQ}

\begin{GPTA}
물론입니다! 각 단계별로 자세히 설명드리겠습니다.

문제를 이해하기 위해 다이어그램을 해석합니다. 삼각형 $M N R$ 은 직각삼각형이며, $\angle M N R$ 의 각도가 $x$ 로 표시되어 있습니다. 측면 $M N$ 의 길이는 15 m 이고, 대각선 $M R$ 의 길이는 $10 \sqrt{3}$ 입니 다. 삼각형 안쪽의 $\angle M Q N$ 은 $72^{\circ}$ 라고 주어졌습니다.

단계별 해결방법:

1. 삼각형의 내각 합:

삼각형 $M N Q$ 의 각도 합은 $180^{\circ}$ 가 되어야 합니다. 따라서, $\angle M N Q+\angle M Q N+\angle N M Q=180^{\circ}$ 입니다. 여기서 $\angle N M Q=90^{\circ}$ (직각)이므로,

$$
\angle M N Q+72^{\circ}+90^{\circ}=180^{\circ}
$$


식을 전개하면,

$$
\angle M N Q+162^{\circ}=180^{\circ}
$$


따라서,

$$
\angle M N Q=180^{\circ}-162^{\circ}=18^{\circ}
$$

2. 각도 $\theta$ 구하기:

삼각형 $N M R$ 에서도, 우리는 $\theta=\angle M R N$ 이라고 가정 합니다. 삼각형 $N M R$ 에서 $\theta$ 를 구하기 위해, 삼각함수를 사용합니다.

우리는 $\sin \theta=\frac{M N}{M R}$ 를 사용합니다:

$$
\sin \theta=\frac{15}{10 \sqrt{3}}
$$


이는 계산하면,

$$
\theta=\sin ^{-1}\left(\frac{15}{10 \sqrt{3}}\right)
$$


이때 계산하면,

$$
\theta=60^{\circ}
$$

3. 각도 $x$ 구하기:

모든 내각의 합이 $180^{\circ}$ 인 삼각형 $M N R$ 에서,

$$
18^{\circ}+x+60^{\circ}=90^{\circ}
$$


여기서, langle $N R M=x$

$$
x+18^{\circ}+60^{\circ}=90^{\circ}
$$

$x=12^{\circ}$

\end{GPTA}

