\begin{GPTQ}
    807킬로그램을 1세제곱미터로 어떻게 나누는지 알려줘
\end{GPTQ}

\begin{GPTA}
물론입니다! 밀도 문제에서 사용하는 기본적인 계산을 설명해 드리겠습니다.
    밀도는 물질의 질량을 부피로 나눈 값입니다. 여기서는 밀도가 주어졌고, 우리는 주어진 부피에 따른 질량을 계산하려고 합니다. 주어진 밀도는 세제곱미터당 807 킬로그랩입니다. 이것은 1 세 제곱미터의 부패를 가지는 물질의 질량이 807 킬로그켬이라는 것을 의미합니다. 우리가 갖고 있는 샘플의 부피는 0.729 세제곱미터입니다. 샘플의 질량을 계산하기 위해 부피에 밀도를 곱합니다. 1 세제곱미터당 질량은 이미 주어졌기 때문에, 전체 부피에 대혜 아래와 같이 곱셈을 수행합니다:
    
    $$
    \text { 샘플의 질량 }=\text { 밀도 } \times \text { 부피 }= \dfrac{807\text { 킬로그램 }}{\text { 세제곱미터 }} \times 0.729 \text { 세제곱미터 }
    $$
    
    
    계산을 수행하면,
    
    $$
    807 \times 0.729=588.303 \text { 킬로그램 }
    $$
    
    
    이 결과를 가장 가까운 정수로 반올림하면 588 킬로그램이 됩니다. 따라서 이는 샘플의 질량입니다. 
\end{GPTA}

