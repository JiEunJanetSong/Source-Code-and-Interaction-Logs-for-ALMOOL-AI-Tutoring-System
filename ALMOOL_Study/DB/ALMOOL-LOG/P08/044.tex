\begin{GPTQ}
    사인, 탄젠트, 코사인에 대한 근거 
\end{GPTQ}

\begin{GPTA}
    사인, 코사인, 그리고 탄젠트는 삼각형의 기하학적 관계에서 비롯되며, 주로 직각삼각형에서 각 변의 비율을 나타내는 삼각함수입니다. 이 함수들이 어뗳게 정의되고 어디에서 유래했는지를 자세히 설명하겠습니다.

    1. 사인(Sine)
    
    근거:

    - 직각삼각형: 사인은 직각삼각형에서 각도 $\theta$ 의 대변과 빗변의 비율로 정의됩니다. 이 비율은 다양한 삼각형에서 동일한 각도를 가질 때 일정하게 유지됩니다.

    - 단위원: 단위원에서 각도를 기준으로 하여 원 위의 점을 정의 할 수 있습니다. 이때 $(x, y)$ 좌표에서 $y$ 값이 $\sin \theta$ 와 동일합니다. 이는 각도를 기준으로 반지름 1 인 원의 수직 좌표로 볼 수 있습니다.

    2. 코사인(Cosine)
    
    근거:

    - 직각삼각형: 코사인은 직각삼각형에서 각도 $\theta$ 의 인접변과 빗변의 비율로 정 의됩니다. 마찬가지로 동일한 각도에 대해 이 비율은 일정하게 유지됩니다.

    - 단위 원: 단위원에서 점 $(x, y)$ 에서 $x$ 값이 $\cos \theta$ 와 동일합니다. 이는 각도를 기준으로 원의 수평 좌표를 나타냅니다.

    3. 탄젠트(Tangent)
    
    근거:

    - 직각삼각형: 탄젠트는 직각삼각형에서 각도 $\theta$ 의 대변과 인접변의 비율로 정의됩니다. 또는 사인과 코사인의 비율로 불 수 있습니다: $\tan \theta=\frac{\sin \theta}{\cos \theta}$.

    - 단위 원과 접선: $\tan \theta$ 는 단위원에서 각도 $\theta$ 와 연졀된 점이 무한히 확장된 접선 $(y$ 축 방향과 평행)의 $y$ 좌표와 비례합니다.
    
    요약

    이러한 삼각 함수들은 우선적으로 각도와 삼각형의 변 사이의 관계를 설명하는 데 사용됩니다. 단위원에 기반한 정의는 이러한 함수들이 갖는 주기성과 대칭성을 설명하는 데 유용합니다. 이 는 실생활의 다양한 문제를 푸는 데 필수적인 기초 수학 개념입니다.

    삼각 함수의 기원은 기하학과 천문학적 관측에서 비롯되었으며, 시간이 지나면서 수학적 분석과 물리학에 광범위하게 적용되었습니다. 더 알고 싶거나 궁금한 점이 있다면 언제든지 질문해 주세요! 
\end{GPTA}

