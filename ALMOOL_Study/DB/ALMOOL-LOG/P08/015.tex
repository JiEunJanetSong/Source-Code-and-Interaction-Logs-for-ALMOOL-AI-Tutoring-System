\begin{GPTQ}
    순서대로 풀어줘 
\end{GPTQ}

\begin{GPTA}
물론입니다! 물의 어는점과 끓는점을 섭씨에서 켈빈으로 변환하는 과정을 단계별로 풀어드리겠습니다.

    주어진 관계식은 다음과 같습니다:
    
    $$
    \mathrm{K}=273+\mathrm{C}
    $$
    
    
    여기서 C 는 섭씨 온도, K 는 켈빈 온도를 나타냅니다.

    1. 어는점 $\left(0^{\circ} \mathrm{C}\right)$ 변환하기:
    
    - 섭씨 0 도를 대입합니다:
    
    $$
    \mathrm{K}=273+0=273
    $$
    
    
    따라서, 물의 어는점은 273 켈빈입니다.

    2. 끓는점 $\left(100^{\circ} \mathrm{C}\right)$ 변 환하기:

    - 섭씨 100 도를 대입합니다:
    
    $$
    \mathrm{K}=273+100=373
    $$
    
    
    따라서, 물의 끓는점은 373 켈빈입니다.

    결론적으로, 물의 어는점은 273 K , 꿇는점은 373 K 입니다. 질문이 있으면 언제든지 문의 하세요!
    \end{GPTA}

