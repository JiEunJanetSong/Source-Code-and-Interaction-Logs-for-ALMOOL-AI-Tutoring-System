\begin{GPTQ}
    frustrum of a cone을 구하는 방법   
\end{GPTQ}

\begin{GPTA}
 프러스텀(Frustum) 또는 원뿔대(conical frustum)는 원뿔의 꼭짓점 절단 면과 밑면 사이에 남아 있는 부분을 의미합니다.
 
 원뿔대의 부피를 구하는 공식과 관련된 내용은 다음과 같습니다:
    
    원뿔대의 부피

    원뿔대의 부피 $V$ 는 다음 공식으로 계산할 수 있습니다:
    
    $$
    V=\frac{1}{3} \pi h\left(R^2+R r+r^2\right)
    $$
    
    
    여기서,

    - $h$ 는 원뿔대의 높이입니다.

    - $R$ 는 원뿔대의 웟면의 반지름입니다.

    - $r$ 는 원뿔대의 밑면의 반지름입니다.
    
    예시

    만약 원뿔대의 밑면 반지름이 9 cm, 윗면 반지름이 3 cm 이며 높이가 12 cm 라면 부패는 다음과 같이 계산됩니다:
    
    $$
    \begin{gathered}
    V=\frac{1}{3} \pi \times 12 \times\left(9^2+9 \times 3+3^2\right) \\
    V=\frac{1}{3} \pi \times 12 \times(81+27+9) \\
    V=\frac{1}{3} \pi \times 12 \times 117 \\
    V=468 \pi, \mathrm{~cm}^3
    \end{gathered}
    $$
    
    
    위와 같이 공식에 값을 대입하여 계산할 수 있습니다. 이 공식은 원뿔대의 구조를 이해하고 문제를 푸는 데 유용합니다.

    
    
\end{GPTA}

