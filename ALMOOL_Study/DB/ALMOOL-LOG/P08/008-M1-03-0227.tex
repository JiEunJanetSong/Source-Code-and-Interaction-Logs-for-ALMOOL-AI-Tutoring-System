\begin{GPTQ}
    이 문제를 쉽게 푸는 방법으로는 어떤 게 있어?  
\end{GPTQ}

\begin{GPTA}
이 문제를 쉽게 풀기 위해 다음 단계로 접근할 수 있습니다: 

문제:
    
    $$
    7+2(x-3)=-3(x+1)-4
    $$
    
    1. 분배법칙 사용하기: 각각의 괄호를 풀어줍니다.

    - 왼쪽: $2(x-3)$ 를 계산하면 $2 x-6$ 이 됩니다.

    - 오른쪽: $-3(x+1)$ 를 계산하면 $-3 x-3$ 이 됩니다.
    
    이제 식은 다음과 같이 변형됩니다:
    
    $$
    7+2 x-6=-3 x-3-4
    $$
    
    2. 정리하기: 양쪽의 값을 간단히 정리합니다.

    - 왼쪽: $7-6$ 은 1 이 됩니다. 따라서 $1+2 x$ 입니다.

    - 오른쪽: $-3-4$ 은 -7 이 됩니다. 따라서 $-3 x-7$ 입니다.
    
    이제 식은 다음과 같습니다:
    
    $$
    1+2 x=-3 x-7
    $$
    
    3. 변수를 한쪽으로 모으기: $x$ 용어를 한쪽으로 모읍니다.

    - $3 x$ 를 양쪽에 더하면:
    
    $$
    1+5 x=-7
    $$
    
    - 1 을 양쪽에서 때면:
    
    $$
    5 x=-8
    $$
    
    4. $x$ 에 대혜 풀기: 양쪽을 5 로 나눕니다.
    
    $$
    x=\frac{-8}{5}
    $$
    
    
    이제 $x$ 의 값이 $-\frac{8}{5}$ 인 것을 알 수 있습니다. 이 단계들을 따름으로써 문제를 쉽게 풀 수 있습니다.
    
\end{GPTA}

