\begin{GPTQ}
    설명해줘  
\end{GPTQ}

\begin{GPTA}
물론입니다!

    주어진 문제는 오른쪽에 $\triangle M N R$ 이라는 직각삼각형을 포함하고 있는 다이어그램에 관한 것입 니다. 삼각형의 변 $M N$ 의 길이는 15 m 이고, 빗변 $M R$ 의 길이는 $10 \sqrt{3} \mathrm{~m}$ 입니다. $\angle M Q N$ 은 $72^{\circ}$ 로 주어져 있으며, $Q$ 는 변 $M R$ 위의 한 점입니다.

    문제에서 $|\overline{Q R}|$ 의 길이를 계산합니다. 
    
    여기서는 삼각합수와 피타고라스의 정리를 사용하여 문제 를 해결합니다.

    1. 먼저 $\triangle M Q N$ 에서 다음 관계식을 세울 수 있습니다:
    
    $$
    \tan 72^{\circ}=\frac{15}{|N Q|}
    $$
    
    
    이로부터 $|\overline{N Q}|$ 를 구할 수 있습니다:
    
    $$
    |\overline{N Q}|=\frac{15}{\tan 72^{\circ}} \approx 4.8738 \mathrm{~m}
    $$
    
    2. 피타고라스의 정리를 사용하여 $|\overline{M R}|$ 의 나머지 부분을 계산합니다. 주어진 빗변 $M R$ 에 대 한 일부 거리 $5 \sqrt{3}$ 를 다른 두 변의 길이로 나누어 구합니다.
    
    $$
    (10 \sqrt{3})^2=15^2+|\overline{R M}|^2 300 \quad=225+|\overline{R M}|^2|\overline{R M}|^2=75|\overline{R M}| \quad=\sqrt{75}=5 \sqrt{3} \mathrm{~m}
    $$
    
    3.
    
    $$
    \begin{aligned}
    & |\overline{N R}|=|\overline{N Q}|+|\overline{Q R}| \text { 에서, } \\
    & 5 \sqrt{3}=4.8738+|\overline{Q R}| \text { 를 만족하도록 }|\overline{Q R}| \text { 를 계산합니다: } \\
    & \quad|\overline{Q R}|=5 \sqrt{3}-4.8738 \quad=3.7865 \mathrm{~m} \approx 4 \mathrm{~m} \text { (가장 가까운 정수로 반올림) }
    \end{aligned}
    $$
    
    
    그래서 $|\overline{Q R}|$ 의 길이는 4 미터입니다. 더 궁금한 점이 있으면 언제든지 질문해 주세요!
    \end{GPTA}

