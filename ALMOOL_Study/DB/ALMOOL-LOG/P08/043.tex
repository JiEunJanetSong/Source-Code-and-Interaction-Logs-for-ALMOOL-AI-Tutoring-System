\begin{GPTQ}
사인, 탄젠트, 코사인에 대한 정의와 이것들의 근거  
\end{GPTQ}

\begin{GPTA}
삼각 함수는 주로 직각삼각형과 관련하여 정의되며, 각도와 변의 관계를 설명합니다. 아래는 사인, 탄젠트, 코사인 합수의 정의와 그 이론적 근거를 설명합니다.

1. 사인(Sine) 함수

정의 :

- 직각삼각형에서 각 $\theta$ 에 대한 사인은 대변(각의 반대편 변)의 길이를 빗변(직각의 반대편 변)의 길이로 나눈 값입니다.    
 
$$
\sin \theta=\frac{\text { 대변의 길이 }}{\text { 빗변의 길이 }}
$$


근거:

- 단위원(circle with radius $=1$ )에서 각 $\theta$ 는 원 위의 한 점을 정의하는 데 사용됩니다. 이때, 그 점의 $y$ 좌표가 사인 값과 동일합니다.

2. 코사인(Cosine) 함수

정의:

- 직각삼각형에서 각 $\theta$ 에 대한 코사인은 인접변(각의 이웃 변)의 길이를 빗변의 길이로 나눈 값입니다.

$$
\cos \theta=\frac{\text { 인접변의 길이 }}{\text { 빗변의 길이 }}
$$


근거:

- 단위원 위에서, 각 $\theta$ 에 대응되는 점의 $x$ 좌표가 바로 이 각의 코사인 값입니다.

3. 탄젠트(Tangent) 함수

정의:

- 직각삼각형에서 각 $\theta$ 에 대한 탄젠트는 대변의 길이를 인접변의 길이로 나눈 값입니다. 이는 사인과 코사인 간의 비율로도 표현됩니다.

$$
\tan \theta=\frac{\text { 대변의 길이 }}{\text { 인접변의 길이 }}=\frac{\sin \theta}{\cos \theta}
$$


근거:

- 단위원에서, 각 $\theta$ 는 무한대로 연장된 $x$-축에 수직인 접선의 $y$ 좌표와도 관계가 있습니다.

요약

- 이러한 삼각 함수는 삼각형 문제, 파동 운동, 진동 분석 등 다양한 물리적인 문제를 푸는 데 필수적입니다.

- 특히, 단위원 중심에서 각의 회전으로 정의된 $(x, y)$ 위치를 통해 삼각 함수들의 값이 유도 됩니다.

각 함수의 정의와 근거는 삼각형의 기하학적 성질로부터 자연스럽게 나오며, 수학의 여러 분야에서 필수적으로 활용됩니다. 더 알고 싶은 부분이 있거나 질문이 있다면 언제든지 말씀해 주세요!

\end{GPTA}

