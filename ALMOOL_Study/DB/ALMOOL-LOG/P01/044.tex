\begin{GPTQ}
    What is the steps? and what formulas should I use?   
\end{GPTQ}

\begin{GPTA}
To solve the problem of finding the radius of the cone given the volume, we'll need to use the formula fo $r$ the volume of a cone:
    
    $$
    V=\frac{1}{3} \pi r^2 h
    $$
    
    where:

    - $V$ is the volume of the cone,

    - $r$ is the radius of the base of the cone,

    - $h$ is the height of the cone. 
  
 For this problem:

- You are given the volume of the cone, $V=39,600, \mathrm{~cm}^3$.

- The height of the cone is indirectly provided through the height of the frustum, but we'll need som e geometric insight to find the height of the actual cone. If a frustum is created by removing the sm aller cone from the top of a larger cone, then the height of the larger cone is the sum of the height of the frustum and the height of the smaller cone.

Steps:

1. Set Up the Volume Formula:

Substitute the given volume and the known values into the formula for the volume of a cone:

$$
39,600=\frac{1}{3} \times \frac{22}{7} \times r^2 \times h
$$

2. Resolve for $r^2$ :

First, simplify the equation to isolate $r^2$ :

$$
r^2=\frac{39,600 \times 3 \times 7}{22 \times h}
$$


Since the problem statement directly resolves this as $r^2=2100$, and $r=\sqrt{2100}$, we know that it is derived assuming a certain height $h$. In the problem, the solutions provided assumed $h=18, \mathrm{~cm}$ for the entire cone.

3. Calculate $r$ :

Calculate $r$ using:

$$
r=\sqrt{2100} \approx 45.83, \mathrm{~cm}
$$


Given $h=18, \mathrm{~cm}$ (as inferred from the solution), substitute that to find $r$.

4. Confirm the Height Correlation:

Based on the problem's image and description, ensure that if the proportions and dimensions are c onsidered, the calculations fit the entire cone height.

In conclusion, you utilize the volume formula for the entire cone, and by using the volume $39,600, \mathrm{~cm}^3$ and height, you solve for the radius $r$.
   
\end{GPTA}

