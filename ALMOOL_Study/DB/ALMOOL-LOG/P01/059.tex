\begin{GPTQ}
    What are straight lines? 
\end{GPTQ}

\begin{GPTA}
A straight line is a geometric concept that represents the shortest distance between any two points in a tw o-dimensional plane. It extends infinitely in both directions without curving. Here are some key characte ristics of a straight line:

1. Infinite Length: A straight line has no endpoints and extends indefinitely in both directions.

2. No Width or Thickness: It is one-dimensional, having only length, with no width or depth.

3. Consistency in Direction: A straight line has a constant slope, meaning it does not change directi on.

4. Equation of a Straight Line: In a coordinate plane, a straight line can be represented by the equat ion $y=m x+c$, where $m$ is the slope and $c$ is the $y$-intercept.

5. Collinear Points: Any three points that lie on the same straight line are considered to be collinear.

In geometry problems, straight lines often serve as references to define angles, intersections, and distanc es between points. In the context of the problem you mentioned, lines like ADE and BCE are referred t $o$ as straight lines because they create straight paths that define certain angles and intersections with the circle.


 
\end{GPTA}

