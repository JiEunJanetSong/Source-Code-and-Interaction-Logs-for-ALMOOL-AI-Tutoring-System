\begin{GPTQ}
    이게 무슨 뜻이야?
\end{GPTQ}

\begin{GPTA}
    이것은 삼각형에 대한 문제로, 주어진 삼각형 $R S T$ 와 그것과 닯은 삼각형 $U V W$ 에 대혜 세 변의 길이를 알고 있는 상황입니다. 이 문제는 주어진 삼각형에서 각도를 이용하여 닮은 삼각형에서 픅정 각의 $\tan$ (탄젠트) 값을 계산하는 것입니다.
    
    여기서 중요한 점은 주어진 삼각형의 변의 각 길이와 각자의 관계를 퐈악하여 $\tan W$ 값을 찾는
    데 있습니다. key는 $\triangle R S T$ 와 $\triangle U V W$ 가 닮은 삼각형이므로 각 $T$ 와 각 $W$ 가 같다는 것입니다.

    - $R S=20$ (변 $R S$ 는 각 $T$ 의 반대편에 위치한 변)
    
    - $S T=48$ (변 $S T$ 는 각 $T$ 의 인접한 변)

    - 이 정보를 이용하여 $\tan T$ 를 구하면, $\tan T=\frac{R S}{S T}=\frac{20}{48}=\frac{5}{12}$ 입니다.
    
    각 $W$ 는 각 $T$ 와 동일 한 크기를 가지고 있으므로 $\tan W=\tan T=\frac{5}{12}$ 가 됩니다. 즉, $\tan W$ 의 값은 $\frac{5}{12}$ 입니다.
 
\end{GPTA}

