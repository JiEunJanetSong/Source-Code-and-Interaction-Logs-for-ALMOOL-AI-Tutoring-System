\begin{GPTQ}
    어떻게 알수있어?   
\end{GPTQ}

\begin{GPTA}
문제를 해결하기 위해 풜요한 단계는 다음과 같습니다:

    1. 부펴 계산: 주어진 정육면체의 한 변의 길이는 0.9 미터입니다. 정육면채의 부피는 $V=a^3$ 로 계산됩니다. 
    
    따라서:
    
    $$
    V=0.9^3=0.729 \text {, 세제곱미터 }
    $$
    
    2. 질 량 계산: 밀도가 주어졌습니다, 807 , 킬로그램/세제곱미터. 질량은 밀도에 부피를 곱한 값입니다:
    
    $$
    \text { 질량 }=0.729, \text { 세제곱미터 } \times 807 \text {, 킬로그램 } / \text { 세제곱미터 }=588.303 \text {, 킬로그램 }
    $$
    
    3. 반올림: 질량을 가장 가까운 정수로 반올림하면 588 킬로그램이 됩니다.
    
    따라서, 이 샘플의 질량은 588 킬로그램입니다.
    \end{GPTA}

