\begin{GPTQ}
    what is the formula to get the answer?   
\end{GPTQ}

\begin{GPTA}
    To solve for $\angle A E B$, we use the property of angles subtended by intersecting lines from an external point to a circle. In this problem, $\angle A E B, \angle D C E$, and $\angle C D E$ form a linear pair that sum to $180^{\circ}$. Here's the step-by-s tep approach:

1. Identify the Relationship: Since $\angle A E B, \angle D C E$, and $\angle C D E$ are exterior angles around point $E$, the y add up to $180^{\circ}$. Therefore, we can set up the equation:

$$
\angle A E B+\angle D C E+\angle C D E=180^{\circ}
$$

2. Given Values: We are given that $\angle D C E=56^{\circ}$. We also know from the ratio given ( $\angle D C E: \angle C D E=7: 8$ ), we can find $\angle C D E$.

3. Calculate $\angle C D E$ :

Given the ratio, $\angle C D E=64^{\circ}$, derived as follows:

$$
\frac{\angle D C E}{\angle C D E}=\frac{7}{8} \angle C D E=\frac{8}{7} \times 56^{\circ}=64^{\circ}
$$

4. Substitute and Solve for $\angle A E B$ :

$$
\angle A E B+56^{\circ}+64^{\circ}=180^{\circ} \angle A E B+120^{\circ} \quad=180^{\circ} \angle A E B=180^{\circ}-120^{\circ} \angle A E B \quad=60^{\circ}
$$


This gives us the answer $\angle A E B=60^{\circ}$.
\end{GPTA}

