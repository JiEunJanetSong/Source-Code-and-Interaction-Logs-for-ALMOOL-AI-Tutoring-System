\begin{GPTQ}
    Can you only show the process of solving this problem?   
\end{GPTQ}

\begin{GPTA}
 Certainly! Let's break down the solution step by step.

    Given Information:

    - The circle has center $O$, with a circumference of $144 \pi$.

    - There are diameters $\overline{P R}$ and $\overline{Q S}$.

    - The length of arc $P S$ is twice the length of arc $P Q$.
    
    Step-by-Step Solution:

    1. Understanding Diameters and Radii:

    - Both $\overline{P R}$ and $\overline{Q S}$ are diameters, which means each is a straight line through the circle's cent er $O$. They divide the circle into two equal semicircles.

    2. Recognizing Vertical Angles:

    - Since $\angle S O P$ and $\angle R O Q$ are vertical angles, they have equal measures. Therefore, arc $P S$ and $\operatorname{arc} Q R$ have the same angle measure and are congruent arcs.

    3. Identifying Congruent Arcs:

    - Similarly, $\angle S O R$ and $\angle P O Q$ are also vertical angles and have the same measure, which m akes arcs $S R$ and $P Q$ congruent.

    4. Setting Variables for Arcs:

    - Let $x$ be the length of the arc $S R$. Since arc $S R$ and arc $P Q$ are congruent, arc $P Q$ is also $x$

    5. Express Arc Lengths:

    - Since are $P S$ is twice the length of arc $P Q$, arc $P S$ is $2 x$.

    - Since arc $P S$ and $\operatorname{arc} Q R$ are congruent, arc $Q R$ is also $2 x$.

    6. Sum of Arc Lengths:

    - The total length of these arcs around the circle is:
    
    $$
    x+x+2 x+2 x=6 x
    $$
    
    7. Using the Circumference:

    - The circumference of the circle is given as $144 \pi$. So, set up the equation:
    
    $$
    6 x=144 \pi
    $$
    
    8. Solving for $x$ :

    - Divide both sides by 6 to solve for $x$ :
    
    $$
    x=24 \pi
    $$
    
    9. Determine the Length of Arc $Q R$ :

    - The length of arc $Q R$ is $2 x$, which is:
    
    $$
    2(24 \pi)=48 \pi
    $$
    
    
    Thus, the length of arc $Q R$ is $48 \pi$.
    
\end{GPTA}

