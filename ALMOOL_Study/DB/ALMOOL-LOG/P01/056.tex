\begin{GPTQ}
    what is pythagorean theory?   
\end{GPTQ}

\begin{GPTA}
The Pythagorean Theorem is a fundamental principle in geometry that establishes a relationship between the sides of a right triangle. It states that in a right triangle, the square of the length of the hypotenuse (th e side opposite the right angle) is equal to the sum of the squares of the lengths of the other two sides.

In mathematical terms, if we have a right triangle with:

- $c$ as the hypotenuse,

- $a$ and $b$ as the other two sides (legs),
then the Pythagorean Theorem is expressed as:

$$
c^2=a^2+b^2
$$


Key Points:

1. Right Triangle: The theorem specifically applies to right triangles (triangles with one $90^{\circ}$ angle).

2. Hypotenuse: The hypotenuse is always the longest side, opposite the right angle.

3. Applications:

- It's used to determine the third side of a triangle when two sides are known.

- It's foundational in trigonometry, helping to define sine, cosine, and tangent based on triangl e sides.

4. Examples:

- If a right triangle has legs of length 3 and 4 , the hypotenuse can be found using the formula:

$$
c^2=3^2+4^2=9+16=25 \quad \Rightarrow \quad c=\sqrt{25}=5
$$


The Pythagorean Theorem is extensively used in various fields, such as physics, engineering, architecture, and even in solving everyday problems. If you have more questions or need further explanation, feel free to ask!
  
\end{GPTA}

