\begin{GPTQ}
   여기서 각 cde 는 이미 구했다 치고, 그 이후를 삼단계로 알려줘
 
\end{GPTQ}

\begin{GPTA}
    물론 가능합니다! $\angle C D E=64^{\circ}$ 를 이미 구했다고 가정할 때, $x$ 를 찾기 위한 후속 삼단계는 다음과 같습니다:

1. 원주각 사각형의 성질 적용하기:

- 원에 내접하는 사각형에서, 서로 대각선에 위치한 각의 합은 $180^{\circ}$ 입니다.

- 따라서, $\angle A B E+\angle A D C=180^{\circ}$ 입니다.

- $\angle A D C=116^{\circ}$ 이기 때문에, 다음과 같은 관계가 성 립합 니다:

$$
5 x-16^{\circ}+116^{\circ}=180^{\circ}
$$

2. 방정식 정리 하기:

。위 식을 정리하여 $5 x$ 에 관한 식을 만듭니다:

$$
5 x+100^{\circ}=180^{\circ}
$$

3. $x$ 구하기:

- 식을 풀어 $x$ 의 값을 구합니다:

$$
\begin{gathered}
5 x=80^{\circ} \\
x=\frac{80^{\circ}}{5}=16^{\circ}
\end{gathered}
$$

이렇게 $x=16^{\circ}$ 를 구할 수 있습니다.


\end{GPTA}

