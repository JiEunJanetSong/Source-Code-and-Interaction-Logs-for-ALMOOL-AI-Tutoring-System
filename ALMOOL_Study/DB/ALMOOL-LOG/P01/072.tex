\begin{GPTQ}
    그럼 y는?   
\end{GPTQ}

\begin{GPTA}
    이미 $x=16^{\circ}$ 를 구했으므로, 이를 활용하여 $y$ 를 구해보젰습니다. 다음은 그 과정을 나타냅니다:

1. 주어진 방정식을 이용하기:

이전에 우리는 $\angle B A E$ 에 대한 다음 방정식을 세웠습니다:

$$
\frac{3}{2} x+4 y+124^{\circ}=180^{\circ}
$$

- 여기에서 $\angle B A E=\frac{3}{2} x+4 y$ 입니다.

2. $x$ 의 값을 대입하기:

- 이미 구한 $x=16^{\circ}$ 를 위의 식에 대입합니다:

$$
\frac{3}{2}\left(16^{\circ}\right)+4 y+124^{\circ}=180^{\circ}
$$

$-\frac{3}{2} \times 16^{\circ}=24^{\circ}$ 이므로, 대 입하면:

$$
24^{\circ}+4 y+124^{\circ}=180^{\circ}
$$

3. $y$ 구하기:

- 방정식을 정리하여 $y$ 의 값을 구합니다:

$$
4 y=180^{\circ}-24^{\circ}-124^{\circ}
$$


$$
\begin{gathered}
4 y=32^{\circ} \\
y=\frac{32^{\circ}}{4}=8^{\circ}
\end{gathered}
$$

따라서 $y=8^{\circ}$ 입니다.

\end{GPTA}

