\begin{GPTQ}
    what are the way of getting $x$? 
\end{GPTQ}

\begin{GPTA}
To find the value of $x$, we need to use the properties of the cyclic quadrilateral and the given angles. Here's a step-by-step approach:

1. Understand the Relationships:

- Given: $\angle A B E=(5 x-16)^{\circ}$ and the understanding that for cyclic quadrilaterals, opposite angles sum to $180^{\circ}$, i.e., $\angle A B E+\angle A D C=180^{\circ}$.

- From the problem: $\angle D C E=56^{\circ}$.

2. Calculate $\angle C D E$ :

- Ratio provided: $\angle D C E: \angle C D E=7: 8$.

- Therefore, $\frac{56^{\circ}}{\angle C D E}=\frac{7}{8}$.

- Solving for $\angle C D E$, we get:

$$
\angle C D E=\frac{56^{\circ} \times 8}{7}=64^{\circ}
$$

3. Use Relationship in Cyclic Quadrilateral:

- Recall $\angle A B E+\angle A D C=180^{\circ}$.

- From prior calculations, $\angle A D C=180^{\circ}-\angle C D E=180^{\circ}-64^{\circ}=116^{\circ}$.

- Set up the equation:

$$
5 x-16^{\circ}+116^{\circ}=180^{\circ}
$$

4. Solve for $x$ :

- Rearrange the equation:

$$
5 x+100^{\circ}=180^{\circ}
$$

$$
\begin{gathered}
5 x=80^{\circ} \\
x=\frac{80^{\circ}}{5}=16^{\circ}
\end{gathered}
$$


Thus, the value of $x$ is 16 degrees. This approach uses the property of opposite angles in a cyclic quadril ateral and the given ratio of angles.
\end{GPTA}

