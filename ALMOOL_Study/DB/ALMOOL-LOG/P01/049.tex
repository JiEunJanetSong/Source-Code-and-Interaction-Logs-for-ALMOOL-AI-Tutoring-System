\begin{GPTQ}
    Is there any easier version to get the answer? 
\end{GPTQ}

\begin{GPTA}
 Certainly! Let's simplify the explanation by focusing on the essential steps:

1. Sum of Angles Property in a Triangle:

- Recall that the sum of angles in any triangle is $180^{\circ}$.

2. Find $\angle M N Q$ :

- Since $\triangle M N Q$ is a triangle with $\angle M Q N=72^{\circ}$ and a right angle at $N$, you can immediate ly find $\angle M N Q$ :

$$
\angle M N Q=180^{\circ}-90^{\circ}-72^{\circ}=18^{\circ}
$$

3. Right Triangle Pythagorean Relation:

- For $\triangle M N R$, knowing that it is a right triangle, the side ratios can help you quickly identify angles. The relation between the sides tells us:

- $\angle M N R$ (by trigonometrical identity or known angle values for sides) is $60^{\circ}$ (since this is derived directly via sine).

4. Determine Angle $x$ :

- Now, knowing:

- $\angle M N Q=18^{\circ}$,

- Angle at $R=90^{\circ}$,

- $\theta=60^{\circ}$,

- We use the triangle's property directly for the missing angle $x$ in $\triangle M N R$ :

$$
\begin{gathered}
18^{\circ}+60^{\circ}+90^{\circ}+x=180^{\circ} \\
x=180^{\circ}-168^{\circ}=12^{\circ}
\end{gathered}
$$


This direct application of angle sum property and known side angle values gives $x=12^{\circ}$ directly and efficiently.
   
\end{GPTA}

