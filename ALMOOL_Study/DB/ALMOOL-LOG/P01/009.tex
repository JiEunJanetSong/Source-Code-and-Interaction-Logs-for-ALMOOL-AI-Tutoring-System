\begin{GPTQ}
    How can I know that they have same angle through intersection?   
\end{GPTQ}

\begin{GPTA}
     Great question! When two parallel lines are intersected by a transversal, several angle relationships are e stablished. Specifically, the angles related to your query are the corresponding angles, alternate interior angles, and alternate exterior angles.

    In this scenario, let's focus on the exterior and supplementary angles because that is applicable to your pr oblem. When a transversal crosses parallel lines, the exterior angles on the same side of the transversal a re supplementary, meaning their measures add up to $180^{\circ}$. 
    
    Let's break down the logic:

    1. Parallel Lines and Transversal: Here, line $m$ is parallel to line $n$, and line $t$ is the transversal that intersects both lines.

    2. Supplementary Angles: An exterior angle on one parallel line and the corresponding exterior ang le on the other parallel line (on the same side of the transversal) add up to $180^{\circ}$.

    3. Equation Formation: Given that one of the exterior angles on line $m$ is $33^{\circ}$, the corresponding ex terior angle on line $n$ will be supplementary to it. Hence, $x+33=180$.

    4. Solve for $x$ : Subtract 33 from 180 to find $x$ :
    
    $$
    x=180-33=147
    $$
    
    
    Therefore, the angle $x$ is $147^{\circ}$. This concept allows you to find unknown angles when dealing with parallel lines and a transversal.
    \end{GPTA}

