\begin{GPTQ}
    top cone is $6 ?$  
\end{GPTQ}

\begin{GPTA}
In the given problem, we determined that the height of the portion of the cone that corresponds to the top of the frustum is $6, \mathrm{~cm}$. This was found from the equation:
    
    $$
    x=6, \mathrm{~cm}
    $$
    
    
    Here, $x$ represents the height of that smaller portion (the top part of the cone that was removed to form $t$ he frustum).

    If you need further clarification or have other questions, feel free to ask!
    \end{GPTA}

