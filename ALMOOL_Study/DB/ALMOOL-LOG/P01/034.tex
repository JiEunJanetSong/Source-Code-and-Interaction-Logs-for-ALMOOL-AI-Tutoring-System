\begin{GPTQ}
    사인 값은 두 삼각형이 왜 똑같은거야?  
\end{GPTQ}

\begin{GPTA}
삼각형 $F G H$ 와 삼각형 $J K L$ 이 닮은 꼴이라고 주어졌습니다. 닮은 삼각형에서는 대응하는 각들 이 모두 서로 같습니다. 문졔에 따르면 각 $F$ 와 각 $J$ 가 대응하는 각이고, 따라서 $\angle F=\angle J$ 입니다. 사인 값은 각의 크기에 의존하기 때문에, 만약 두 각이 같다면 그 두 각의 사인 값도 같습니다. 즉, $\sin (F)=\sin (J)$ 입니다. 문제에서 $\sin (F)=\frac{308}{317}$ 이라고 주어졌으므로, $\sin (J)$ 역시 $\frac{308}{317}$ 입니다. 이러한 이유로 두 각의 사인 값이 동일하게 유지됩니다.

\end{GPTA}

