\begin{GPTQ}
    can you give me simple formula to get the answer?
\end{GPTQ}

\begin{GPTA}
    Certainly! To find $\angle R S T$, you can use the properties of an isosceles triangle and the fact that the angles around a point sum up to $180^{\circ}$.
    In $\triangle R S T$, since $|\overline{R S}|=|\overline{S T}|$, it is an isosceles triangle, meaning that $\angle T R S=\angle S T R$. Here's the sim ple formula to find $\angle R S T$ :

    1. Identify that $\angle T R S=\angle S T R=33^{\circ}$ (given $\overline{Q T} \| \overline{R S}$, they are alternate angles).

    2. Use the angle sum property of a triangle:
    
    $$
    \angle T R S+\angle S T R+\angle R S T=180^{\circ}
    $$
    
    3. Substitute the known angles:
    
    $$
    33^{\circ}+33^{\circ}+\angle R S T=180^{\circ}
    $$
    
    4. Simplify to find $\angle R S T$ :
    
    $$
    \begin{aligned}
    & 66^{\circ}+\angle R S T=180^{\circ} \\
    & \angle R S T=180^{\circ}-66^{\circ}
    \end{aligned}
    $$
    
    5. Therefore,
    
    $$
    \angle R S T=114^{\circ}
    $$
    
    
    This is your formulaic approach to solving for $\angle R S T$.
    \end{GPTA}

