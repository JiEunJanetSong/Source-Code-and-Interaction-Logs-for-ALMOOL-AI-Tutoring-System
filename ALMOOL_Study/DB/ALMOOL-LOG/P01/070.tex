\begin{GPTQ}
    한국어로 설명 가능해?  
\end{GPTQ}

\begin{GPTA}
물론입니다! 주어진 문제에서 $x$ 를 구하기 위한 세 단계의 간단한 설명을 드리겠습니다. 아래는 한국어로 된 설명입니다:

1. 알 수 없는 각도 구하기:

- 비율로 주어진 $\angle D C E: \angle C D E=7: 8$ 관계를 사용합니다.

- $\angle D C E=56^{\circ}$ 가 주어졌으므로, $\angle C D E$ 를 구합니다:

$$
\angle C D E=\frac{56^{\circ} \times 8}{7}=64^{\circ}
$$

2. 원주각 사각형의 성질 적용하기:

- 원에 내접하는 사각형의 성질을 사용하여 대각의 합이 $180^{\circ}$ 임을 이용합니다: $\angle A B E+\angle A D C=180^{\circ}$.

- $\angle C D E=64^{\circ}$ 이므로 $\angle A D C$ 를 구합니다:

$$
\angle A D C=180^{\circ}-64^{\circ}=116^{\circ}
$$

3. $x$ 구하기:

- $\angle A B E=(5 x-16)^{\circ}$ 를 이용해서 방정식을 세웁니다:

$$
5 x-16^{\circ}+116^{\circ}=180^{\circ}
$$

- $x$ 를 구합니다:

$$
\begin{gathered}
5 x+100^{\circ}=180^{\circ} \\
5 x=80^{\circ} \\
x=\frac{80^{\circ}}{5}=16^{\circ}
\end{gathered}
$$


이러한 단계들을 통해 주어진 문제에서 $x$ 를 효과적으로 구할 수 있습니다.

\end{GPTA}

