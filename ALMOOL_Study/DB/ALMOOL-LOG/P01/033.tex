\begin{GPTQ}
    두 기본값 말고 펼요한건 없는거야?
\end{GPTQ}

\begin{GPTA}
네, 이 문졔를 풀기 위혜서는 주어진 두 기본 정보가 충분합니다.

    1. 직선 $m$ 과 $n$ 이 평행하다는 정보

    2. 한 각의 크기가 $26^{\circ}$ 라는 정보
    
    이 두 정보를 통해 두 평챙선 사이의 동측 내각의 합이 $180^{\circ}$ 라는 사실을 이용하여 $x+26=180^{\circ}$ 이 라는 식을 세울 수 있습니다. 여기서 간단한 산술 연산을 통해 $x=154$ 라는 결과를 얻을 수 있습니다.
    추가적인 정보 없이도, 주어진 정보만으로 문제를 해결할 수 있습니다.
       
\end{GPTA}

