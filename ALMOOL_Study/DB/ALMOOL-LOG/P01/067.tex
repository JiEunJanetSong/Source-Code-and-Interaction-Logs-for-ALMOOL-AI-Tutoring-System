\begin{GPTQ}
    then can you give me a direction to find answer?    
\end{GPTQ}

\begin{GPTA}
    Certainly! To find the answer for the angle $\angle A E B$ using the properties of triangle $\triangle A B E$, let's follow t hese steps:

1. Identify Known Angles:

- From the problem, we know that $\angle A E B=60^{\circ}$ based on our previous calculations.

- We have $\angle A B E=(5 x-16)^{\circ}$ and $\angle B A E=\left(\frac{3}{2} x+4 y\right)^{\circ}$.

2. Sum of Angles in a Triangle:

- Since $\triangle A B E$ is a triangle, the sum of its interior angles is $180^{\circ}$.

$$
\angle A E B+\angle A B E+\angle B A E=180^{\circ}
$$

3. Substitute Known Values:

- From our previous calculation: $\angle A E B=60^{\circ}$.
- Therefore:

$$
60^{\circ}+(5 x-16)^{\circ}+\left(\frac{3}{2} x+4 y\right)^{\circ}=180^{\circ}
$$

4. Solve for Variables (if needed):

- If necessary, you'll need additional information or equations to solve for $x$ and $y$, such as giv en ratios or constraints from the problem.

5. Substitute Solved Values:

- Once you have values for $x$ and $y$ (if required), substitute back to find specific angle measur es.

By setting up and solving the equation above with any additional constraints, you can find unknown angl es or variables within $\triangle A B E$.
\end{GPTA}

