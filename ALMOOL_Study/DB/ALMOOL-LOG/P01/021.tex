\begin{GPTQ}
    How are parallel means same angle?   
\end{GPTQ}

\begin{GPTA}
 When we talk about parallel lines and the angles formed by a transversal intersecting them, we're referrin g to special angle relationships that occur due to their geometric properties. Here's how parallel lines lea d to certain angles being equal:

    1. Corresponding Angles: When two parallel lines are cut by a transversal, each pair of correspondi ng angles is equal. Corresponding angles are ones that are in similar positions relative to the parallel lines and the transversal.

    2. Alternate Interior Angles: These are pairs of angles that lie between the parallel lines on opposit e sides of the transversal. Alternate interior angles are equal as well.

    3. Alternate Exterior Angles: These are pairs of angles that lie outside the parallel lines on opposite sides of the transversal. Alternate exterior angles are also equal.
    
    The equal measures of these angles arise from the fact that the parallel lines maintain a constant distance from each other and the angles formed by a transversal at one line have the same relative position at the other line.

    In summary, the reason certain angles are equal when lines are parallel is due to the geometrical property that parallel lines remain equidistant and never meet, which results in these predictable angle relationships.

    \end{GPTA}

