\begin{GPTQ}
    what steps should I take?  
\end{GPTQ}

\begin{GPTA}
    Let's go through the steps needed to calculate the area of $\triangle M Q R$ :

    1. Calculate the Area of $\triangle M N R$ :

    - Since $\triangle M N R$ is a right triangle, the area can be calculated using the formula:
    
    $$
    \text { Area of } \triangle M N R=\frac{1}{2} \times \text { base } \times \text { height }
    $$
    
    - Here, the base $M N=15$ meters and the height $N R=5 \sqrt{3}$ meters.

    - Therefore, the area is:
    
    $$
    \frac{1}{2} \times 15 \times 5 \sqrt{3}=\frac{75 \sqrt{3}}{2} \mathrm{~m}^2
    $$
    
    - Calculate this to get an approximate value (squared meters).

    2. Calculate the Area of $\triangle M N Q$ :

    - Use trigonometry to determine the length $N Q$ from angle $\angle M Q N=72^{\circ}$.

    - With $M N=15$ meters being the opposite side to angle $\angle M Q N$, use the sine function:
    
    $$
    \sin \left(72^{\circ}\right)=\frac{M N}{M Q} \Rightarrow M Q=\frac{15}{\sin \left(72^{\circ}\right)}
    $$
    
    - Use the cosine rule or trigonometric identities to find $N Q$.

    - Once you find $N Q$, calculate the area of $\triangle M N Q$ using:
    
    $$
    \text { Area of } \triangle M N Q=\frac{1}{2} \times M N \times N Q
    $$
    
    3. Subtract to Find the Area of $\triangle M Q R$ :
    
    - With both areas determined, you can subtract to find the area of $\triangle M Q R$ :
    
    $$
    \text { Area of } \triangle M Q R=\text { Area of } \triangle M N R-\text { Area of } \triangle M N Q
    $$
    
    - Use the calculated values and approximate to the nearest whole number.

    4. Round Off to the Nearest Whole Number:

    - Round the resulting area of $\triangle M Q R$ to obtain the final answer in whole meters.
    
    Follow these steps to arrive at the solution. If you have any specific questions or need to go through mor e detail on a particular step, feel free to ask!
   
   
\end{GPTA}

