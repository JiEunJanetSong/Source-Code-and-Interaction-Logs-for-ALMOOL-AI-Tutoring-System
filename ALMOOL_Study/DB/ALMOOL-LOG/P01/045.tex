\begin{GPTQ}
    so can you simplify the steps?  
\end{GPTQ}

\begin{GPTA}
  Certainly! Here's a simplified breakdown of the steps needed:

1. Understand the Problem:

- You have a cone with a given volume of $39,600, \mathrm{~cm}^3$.

- You need to calculate the radius, given that the formula for volume is $V=\frac{1}{3} \pi r^2 h$.

2. Set Up the Formula:

- Substitute the known values into the volume formula:

$$
39,600=\frac{1}{3} \times \frac{22}{7} \times r^2 \times h
$$

3. Calculate $r^2$ :

- Rearrange to solve for $r^2$ :

$$
r^2=\frac{39,600 \times 3 \times 7}{22 \times h}
$$

4. Substitute the Height (Assuming Height $h=18, \mathrm{~cm}$ ):

- Calculate $r^2$ using $h=18 \mathrm{~cm}$ :

$$
r^2=\frac{39,600 \times 3 \times 7}{22 \times 18}
$$

- Simplifying gives $r^2=2100$.

5. Find the Radius $r$ :

- Calculate $r$ :

$$
r=\sqrt{2100} \approx 45.83 \mathrm{~cm}
$$


These are the core steps to find the radius of the cone using the information given.
\end{GPTA}

