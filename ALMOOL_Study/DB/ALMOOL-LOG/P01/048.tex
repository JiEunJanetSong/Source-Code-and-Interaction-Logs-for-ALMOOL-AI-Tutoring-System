\begin{GPTQ}
    can you explain step by step? 
\end{GPTQ}

\begin{GPTA}
Certainly! Let's go through the steps together to find the value of angle $x$ in the right-angled triangle $\triangle M N R$.

    1. Understanding the Angle Sum Property:
    
    In any triangle, the sum of the angles is always $180^{\circ}$.

    2. Analyze Triangle $\triangle M N Q$ :

    - Given: $\angle M Q N=72^{\circ}$ and $\angle M N Q$ is a right angle $\left(90^{\circ}\right)$.

    - Using the angle sum property in $\triangle M N Q$ :
    
    $$
    \angle M N Q+72^{\circ}+90^{\circ}=180^{\circ}
    $$
    
    - Simplify to find $\angle M N Q$ :
    
    $$
    \angle M N Q=180^{\circ}-162^{\circ}=18^{\circ}
    $$
    
    3. Identify Angle $\theta$ in Triangle $\triangle M N R$ :

    - $\angle M R N=\theta$. Use the sine function to express it because it's related to the sides of $\triangle M N R$

    - Given $|\overline{M N}|=15 \mathrm{~m}$ and $|\overline{M R}|=10 \sqrt{3} \mathrm{~m}$, we can use the sine ratio:
    
    $$
    \sin \theta=\frac{|\overline{M N}|}{|\overline{M R}|}=\frac{15}{10 \sqrt{3}}
    $$
        
 - Evaluate the angle (you may use a calculator for this step), which simplifies to:

$$
\theta=60^{\circ}
$$

4. Calculate Angle $x$ :

- Now, we have the angles $\angle M N Q=18^{\circ}, \angle R=90^{\circ}$, and $\angle M R N=60^{\circ}$.

- Using the angle sum property in $\triangle M N R$ :

$$
\angle M R N+\angle N M R+\angle R=180^{\circ}
$$

- Substitute the known values:

$$
18^{\circ}+x+60^{\circ}+90^{\circ}=180^{\circ}
$$

- Simplify to find $x$ :

$$
x+168^{\circ}=180^{\circ}
$$


$$
x=180^{\circ}-168^{\circ}=12^{\circ}
$$


Thus, the value of angle $x$ is $12^{\circ}$.
\end{GPTA}

