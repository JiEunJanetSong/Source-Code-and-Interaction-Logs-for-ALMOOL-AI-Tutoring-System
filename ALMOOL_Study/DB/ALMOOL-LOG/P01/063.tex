\begin{GPTQ}
    what is a property?   
\end{GPTQ}

\begin{GPTA}
In mathematics, especially when dealing with geometric figures or algebraic expressions, a ``property" re fers to an inherent or essential characteristic that helps define or describe the mathematical object or concept. 

Properties are often rules or theorems that hold true for specific figures or operations. Here are a fe w examples to give you a better idea:

1. Properties of Cyclic Quadrilaterals:

- A cyclic quadrilateral is a four-sided figure with all its vertices on a circle.

- One key property is that the opposite angles of a cyclic quadrilateral sum up to $180^{\circ}$. This w as used in your problem to establish the relationship $\angle A B C+\angle A D C=180^{\circ}$.

2. Properties of Triangles:

- The sum of the interior angles of any triangle is $180^{\circ}$.

- In an isosceles triangle, the angles opposite the equal sides are equal.

3. Properties of Parallel Lines:

- When a transversal cuts through parallel lines, alternate interior angles are equal, correspond ing angles are equal, and consecutive interior angles are supplementary.

4. Properties of Numbers:

- Properties such as commutative, associative, and distributive properties define how addition and multiplication operations behave.

5. Algebraic Properties:

- There are properties like the Zero Product Property, which states that if the product of two n umbers is zero, then at least one of the numbers must be zero.

These properties are incredibly useful because they allow us to solve problems more efficiently by applyi ng known truths about the figures or expressions we're dealing with.

\end{GPTA}

