\begin{GPTQ}
  Why choice a is incorrect?     
\end{GPTQ}

\begin{GPTA}
    Let's consider why choice A , which suggests knowing ``the measure of angle $A$,'' is incorrect.

    In triangles $A B C$ and $D E F$, we already know that:

    - Angles $B$ and $E$ each measure $27^{\circ}$.

    - Angles $C$ and $F$ each measure $41^{\circ}$.
    
    The third angles, $A$ and $D$, can be calculated easily because the sum of angles in a triangle is always $180^{\circ}$. For both triangles:
    
    $$
    \text { Angle } A=\text { Angle } D=180^{\circ}-27^{\circ}-41^{\circ}=112^{\circ}
    $$
    
    
    Thus, angles $A$ and $D$ are also equal, which by itself only confirms that triangles $A B C$ and $D E F$ are in deed similar (since all corresponding angles are equal).

    To determine congruency, however, we need one more condition: all corresponding sides must be equal i n length. Knowing just the angle $A$ does not provide us with any information about the lengths of the sid es. We need a comparison of the lengths of the corresponding sides to confirm whether the two triangles are congruent.

    Therefore, choice A is incorrect because knowing the measure of angle $A$ does not provide the necessary information to determine if the triangles are congruent.
    \end{GPTA}

