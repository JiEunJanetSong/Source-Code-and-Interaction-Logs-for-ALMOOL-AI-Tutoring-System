\begin{GPTQ}
    would vou tell me three stens for get an answer? 
\end{GPTQ}

\begin{GPTA}
    Certainly! Here's a simplified three-step guide to finding the value of $x$ in the given problem related to th e cyclic quadrilateral:
    1. Determine the Unknown Angle:
    - Use the ratio property given for $\angle D C E: \angle C D E=7: 8$.
    - Given $\angle D C E=56^{\circ}$, find $\angle C D E$ :
    
    $$
    \angle C D E=\frac{56^{\circ} \times 8}{7}=64^{\circ}
    $$
    
    2. Apply the Cyclic Quadrilateral Property:
    - Use the property that the opposite angles of a cyclic quadrilateral sum to $180^{\circ}$ :
    
    $$
    \angle A B E+\angle A D C=180^{\circ} .
    $$
    
    - Since $\angle C D E=64^{\circ}$, find $\angle A D C$ :
    
    $$
    \angle A D C=180^{\circ}-64^{\circ}=116^{\circ}
    $$
    
    3. Solve for $x$ :
    - Setup the equation using $\angle A B E=(5 x-16)^{\circ}$ :
    
    $$
    5 x-16^{\circ}+116^{\circ}=180^{\circ}
    $$
    
    - Solve for $x$ :
    
    $$
    \begin{gathered}
    5 x+100^{\circ}=180^{\circ} \\
    5 x=80^{\circ} \\
    x=\frac{80^{\circ}}{5}=16^{\circ}
    \end{gathered}
    $$
    
    
    These steps efficiently use given values and properties to solve for the unknown $x$.  
\end{GPTA}

