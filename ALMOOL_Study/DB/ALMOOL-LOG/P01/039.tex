\begin{GPTQ}
    근데 49 곱하기 51 이 50 제곱이랑 같아?  
\end{GPTQ}

\begin{GPTA}
좋은 질문입니다. 사실 $49 \times 51$ 과 $50^2$ 는 같습니다. 식 상태로는 동일하지 않습니다, 그러나 훙미로운 연걸 방법으로 계산할 수 있습니다. 이런 곱셈 방식은 두 수가 중앙값을 기준으로 대칭적인 구조로 있을 때 유용합니다.

    이렇게 계산할 수 있습니다:

    49 와 51 은 50 을 기준으로 -1 과 +1 입니다. 따라서:
    
    $$
    49 \times 51=(50-1) \times(50+1)
    $$
    
    
    위의 곱셈은 다음과 같이 전개할 수 있습니다 (두 항의 차의 제곱을 이용해서):
    
    $$
    (a-b)(a+b)=a^2-b^2
    $$
    
    
    위 'a'는 $50, \mathrm{~b}^{\prime}$ 는 1 이므로:
    
    $$
    49 \times 51=50^2-1^2=2500-1=2499
    $$
    
    
    따라서 $49 \times 51$ 은 $50^2$ 보다 1 만 작습니다.
    이 방식은 대칭적인 숫자의 곱셈에서 효율적으로 사용될 수 있습니다.
    \end{GPTA}

