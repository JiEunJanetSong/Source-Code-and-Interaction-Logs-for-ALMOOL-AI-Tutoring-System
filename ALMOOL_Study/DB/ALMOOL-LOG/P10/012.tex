\begin{GPTQ}
    문제 설명 해줘  
\end{GPTQ}

\begin{GPTA}
물론이죠! 문제를 번역해드리겠습니다:

    함수 $f$ 가 다음과 같이 정의되어 있습니다: $f(x)=x^3-3 x^2-9 x+11$. 다음 값 중 어느 콧에서 $f$ 가 지역 최소값을 가지나요?

    이제 이 문제를 해결하는 방법을 설명해드리겠습니다.

    해결 과정

    1. 미분을 퉁한 극값 구하기: 함수 $f(x)=x^3-3 x^2-9 x+11$ 의 극값을 찾기 위해, 우선 도함 수 $f^{\prime}(x)$ 를 계산합니다.
    
    $$
    f^{\prime}(x)=3 x^2-6 x-9
    $$
    
    2. 도함수를 0 으로 설정하기: 극값을 위해서 도함수를 0 으로 설정하여 $x$ 값을 찾습니다.
    
    $$
    3 x^2-6 x-9=0
    $$
    
    3. 방정식 풀기: 이 이차 방정식을 풀어서 $x$ 값을 구합니다. 인수분해를 사용합니다.
    
    $$
    \begin{aligned}
    & 3\left(x^2-2 x-3\right)=0 \\
    & 3(x-3)(x+1)=0
    \end{aligned}
    $$
    
    
    따라서 $x=3$ 또는 $x=-1$ 입니다.

    4. 함수의 종류 판단하기: 이 두 점에서의 함수 값을 살펴보아, 각 점이 지역 최소값인지 지역 최대값인지 판단해야 합니다. 이를 위해 이차 미분 또는 함수의 증감성을 통해 확인할 수 있 습니다.

    - $f(3)$ 에서 함수가 지역 최소값을 가집니다.

    - $f(-1)$ 에서 함수가 지역 최대값을 가집니다.
    
    따라서 함수 $f$ 는 $x=3$ 에서 지역 최소값을 가집니다.
    
\end{GPTA}

