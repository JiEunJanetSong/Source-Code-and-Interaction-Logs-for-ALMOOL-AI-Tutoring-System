\begin{GPTQ}
    왜 빗변분의 높이야   
\end{GPTQ}

\begin{GPTA}
$\operatorname{sine}(\theta)$ 가 직각삼각형 에서 각 $\theta$ 에 대해 맞은편 변의 길이를 빗변의 길이로 나눈 값이 되는 이유는 삼각비의 기초적인 정의에서 시작합니다. 이를 이해하기 위해 조금 더 자세히 설명하젰습니다.

    1. 삼각형의 기본 구성요소:

    - 모든 직각삼각형은 세 변으로 구성되어 있습니다: 빗변(hypotenuse), 각의 맞은편 변(o pposite side), 그리고 각의 인접 변 (adjacent side)입니다.

    빗변은 삼각형에서 가장 긴 변이며 직각과 마주 보고 있습니다.

    2. 사용성:

    - 직각삼각형에 대한 여러 가지 삼각 함수 비가 존재하고, 각각의 비는 특정변의 비율을 기술합니다.

    - $\operatorname{sine}(\theta)$ 는 각 $\theta$ 에 대하여 맞은편 변의 길이를 빗변의 길이로 나누어 계산툅니다. 그 이 유는 직각삼각형의 변의 길이들 사이의 비가 같다는 성질을 주로 : 대비차 (상대적인 크기)도를 알기 위해서입니다.

    3. 삼각함수의 의미:

    - sine 함수는 직각삼각형과 동일한 너비 대비 높이의 비율에 따라 정의되어 주어진 각 도에서 얼마나 ``높이"가 상승하는지를 나타냅니다.
    
    이러한 정의는 특히 단위원으로 확장되어 직각삼각형의 관계를 더 큰 수학적 개넘으로 확장하는 데에도 유용합니다. 이러한 비의 정의는 수학을 다양한 상황에 적용하는 데 기반이 되는 중요한 개념입니다. 추가적인 질문이 있다면 언제든지 물어보세요!
    
\end{GPTA}

