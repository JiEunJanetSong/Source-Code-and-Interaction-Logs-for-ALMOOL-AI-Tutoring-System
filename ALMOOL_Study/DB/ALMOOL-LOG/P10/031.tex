\begin{GPTQ}
    a 랑 r 이 어떻게 나온거야   
\end{GPTQ}

\begin{GPTA}
무한 등비수열에서는 각 항이 특정 규칙에 따라 나열 됩니다. 주어진 문제를 고려하여 어떻게 $a$ 와 $r$ 을 찾았는지 살펴보젰습니다. 초기 주어진 문제는:
    
    $$
    \sum_{n=1}^{\infty} \frac{e^n}{\pi^n}
    $$
    
    
    첫 번째 항 $a$ 찾기

    수열의 첫 번째 항 $a$ 는 $n=1$ 일을 때의 항입니다.

    따라서 첫 번째 항은:
    
    $$
    a=\frac{e^1}{\pi^1}=\frac{e}{\pi}
    $$
       
    공비 $r$ 찾기
공비 $r$ 은 연속하는 두 항 사이의 비입니다. 즉, 두 번째 항을 컷 번째 항으로 나눈 값입니다. 이 문 제에서는 처음 두 항을 가지고 공비 $r$ 을 정할 수 있습니다. 하지 만 이미 주어진 모든 항이 공비에 따라 형성되였으므로, 이 경우 공비도 단순히 반복되는 비율로 똑같습니다.
두 번째 항은:

$$
\frac{e^2}{\pi^2}=\left(\frac{e}{\pi}\right)^2
$$


두 항의 비율로부터:

$$
r=\frac{\frac{e^2}{\pi^2}}{\frac{e}{\pi}}=\frac{e}{\pi}
$$


따라서 주어진 문제에 대한 일반적인 등비수열의 형태는:

$$
a=\frac{e}{\pi}, \quad r=\frac{e}{\pi}
$$


이런 방법으로 첫 항과 공비를 결정할 수 있으며, 이를 사용하여 수열의 합이나 조건을 결정하는 데 사용할 수 있습니다. 이 접근법을 통해 우리는 무한 등비수열의 합을 정 확히 구할 수 있습니다.

\end{GPTA}

