\begin{GPTQ}
    나눠져있을때 미분공식 알려줘   
\end{GPTQ}

\begin{GPTA}
물론입니다! 나눗셈의 미분 공식은 몫의 법칙이라고 잘 알려져 있습니다. 이 공식은 두 합수 $u(x)$ 와 $v(x)$ 의 몪인 합수 $h(x)=\frac{u(x)}{v(x)}$ 의 미분을 하는데 사용됩니다.

    몫의 법칙에 따르면:
    
    $$
    h^{\prime}(x)=\left(\frac{u(x)}{v(x)}\right)^{\prime}=\frac{v(x) u^{\prime}(x)-u(x) v^{\prime}(x)}{(v(x))^2}
    $$
    
    
    이 때:

    - $u(x)$ 는 분자의 함수

    - $v(x)$ 는 분모의 합수

    - $u^{\prime}(x)$ 와 $v^{\prime}(x)$ 는 각각 $u(x)$ 와 $v(x)$ 의 미분입니다.
    
    위 공식을 사용하기 위해서는 $u(x)$ 와 $v(x)$ 의 미분 값을 알고 있어야 하며, 결과적으로 $v(x)$ 가 0 이 아닌 경우에만 안전하게 사용할 수 있습니다. 이 공식을 통해 몫의 형태로 주어진 합수의 미분을 쉽게 계산할 수 있습니다.
    \end{GPTA}

