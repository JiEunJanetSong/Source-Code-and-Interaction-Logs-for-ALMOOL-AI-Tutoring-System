\begin{GPTQ}
    삼각함수의 의미를 좀 더 쉽게 다시 설명해줘       
\end{GPTQ}

\begin{GPTA}
    물론 쉽게 설명해드릴게요. 삼각함수는 직각삼각형을 다루는 데 있어서 중요한 도구입니다. 기본 적으로 삼각함수는 삼각형의 각도와 변의 길이 간의 관계를 나타내는 함수입니다. 대표적인 삼각 함수로는 sine, cosine, tangent 세 가지가 있습니다. 각각의 의미를 간단히 알아불게요:
    
    1. Sine $(\sin ):$

    - 직각삼각형에서, 각 $\theta$ 의 $\sin e$ 은 각 $\theta$ 의 맞은편 변의 길이를 빗변의 길이로 나눈 값입니 다. 에를 들어, 높은 곳에 있는 물체를 나무 끝에 고정된 줄자로 잴 때 맞은편 변과 빗 변의 길이 간의 비율처럼 생각할 수 있습니다.

    - 공식: $\sin (\theta)=\frac{\text { Opposite side }}{\text { Hypotenuse }}$

    2. Cosine (cos):

    - $\operatorname{cosine}$ 은 직각삼각형에서 각 $\theta$ 의 인접 변의 길이를 빗변의 길이로 나눈 값입니다. 예 를 들어, 계단의 첫 번째 칸에서 계단 꼭대기까지의 수평 거리와 대각선 길이 간의 비율입니다.

    - 공식: $\cos (\theta)=\frac{\text { Adjacent side }}{\text { Hypotenuse }}$

    3. Tangent $(\tan ):$

    - tangent 은 맞은편 변의 길이를 인접 변의 길이로 나눈 값입니다. 이는 주로 경사 정도를 측정할 때 유용합니다.

    - 공식: $\tan (\theta)=\frac{\text { Opposite side }}{\text { Adjacent side }}$
    
    삼각함수는 이런 방식으로 다양한 실생활 및 수학적 문맥에서 각도와 변의 길이 사이의 관계를 이해하는 데에 매우 유용합니다. 삼각함수를 이용하면 건물 높이를 측정 하거나, 위성 각도를 분석하는 등의 다양한 문제를 해결할 수 있승니다. 더 궁금한 점이 있으면 언제든지 물어보세요!    
\end{GPTA}

