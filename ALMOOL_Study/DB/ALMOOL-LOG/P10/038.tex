\begin{GPTQ}
    10 이랑 5 가 왜 사라지는거야   
\end{GPTQ}

\begin{GPTA}
 좋은 질문입니다! 10 과 5 가 왜 극한 계산에서 사라지는지를 설명해 드리겠습니다.

극한에서의 역할:

1. 지배적인 항의 개념:

- 극한을 구할 때, $x \rightarrow \infty$ 와 같은 매우 큰 값의 챙동을 살펴봅니다.

- 이때 지배적인 항(가장 높은 차수나 가장 빡르게 증가하는 항)이 극한의 값을 주도합니다.

- 작은 항들은 극한에 큰 영향을 미치지 않기 때문에 무시할 수 있습니다.  
   
2. 사라지는 이유:

- 분자: $10-6 x^2$ 에서 비교해 보면, $-6 x^2$ 은 10 보다 매우 크기 때문에 $x$ 가 커지면 10 은 상대적으로 너무 작아져서 무시할 수 있습니다.

- 분모: $5+3 e^z$ 에서 비교해 보면, $3 e^x$ 는 5 보다 매우 빠르게 증가하기 때문에 $x$ 가 커지면 5 는 상대적으로 무시할 수 있게 됩니다.

결론:

큰 $x$ 에서 계산할 때, 가장 큰 효과를 미치는 항들만 고려하여 문제를 축소할 수 있습니다. 이는 계산을 휠씬 효율적으로 만들어 주기 때문에 일반적으로 극한 문제에서 사용하는 방법입니다.

그 결과로 우리는 다음과 같은 근사식을 사용하게 됩니다:

$$
\frac{10-6 x^2}{5+3 e^x} \approx \frac{-6 x^2}{3 e^x}
$$


이 과정을 퉁해, 복잡한 식을 의사결정에 큰 영향을 미치지 않는 요소를 제거하여 간단한 문제로 푸는 것입니다.

이 설명이 이해하는 데 도움이 되었기를 바랍니다! 추가적인 궁금증이 있으시면 언제든 질문해 주세요.
 
\end{GPTA}

