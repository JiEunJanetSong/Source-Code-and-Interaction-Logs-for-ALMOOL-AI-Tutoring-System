\begin{GPTQ}
 $\sin$ 의 정의 알려쥐     
\end{GPTQ}

\begin{GPTA}
    sine 함수는 삼각함수의 하나로, 직각삼각형, 단위원, 주기적 함수 패턴에서 정의 됩니다. 여기 몇 가지 정의 방식을 졔공할게요:

    1. 직각삼각형에서의 정의:

    - $\sin e$ 은 직각삼각형에서 각 $\theta$ 에 대한 $\operatorname{sine}(\theta)$ 는 각의 맞은편 변의 길이를 빗변의 길이로 나눈 값입니다. 즉,
    
    $$
    \sin (\theta)=\frac{\text { opposite side }}{\text { hypotenuse }}
    $$
    
    2. 단위원 (Unit Circle) 정의:

    - 단위원에서 각 $\theta$ 는 원점으로부터 반시계 방향(또는 시계 방향으로 각도)으로 회전하여 얻은 각으로 정의할 수 있습니다. 이때 $\theta$ 에 대응하는 단위원의 점 $(x, y)$ 에서 $\operatorname{sine}$ 은 $y$-좌표로 표현됩니다. 즉,
    
    $$
    \sin (\theta)=y
    $$
    
    
    여기서 단위원은 반지름이 1 인 원입니다.

    3. 함수로서의 정의:

    - $\operatorname{sine}$ 함수는 주기 함수로, 그 일반 형태는 $y=\sin (x)$ 이며, 주기는 $2 \pi$ 입니다. 이 함수는 $(-\infty,+\infty)$ 에서 $[-1,1]$ 으로 값을 가지며, 다양한 수학적, 물리적 현상을 모델링하는 데 사용됩니다.
    
    이러한 정의들은 다양한 상황에서 sine 함수가 어떻게 동작하는지를 설명해 줍니다. 삼각함수에 대해 더 알고 싶으시면 언제든지 질문해 주세요!
    \end{GPTA}

