\begin{GPTQ}
    미분 어떻게 하는거야    
\end{GPTQ}

\begin{GPTA}
 미분은 함수의 변화율을 구하는 과정입니다. 측정 점에서의 순간적 변화를 측정할 수 있어요. 여기서는 기본적인 미분의 예제를 설명드릴게요.

    주어진 함수: $f(x)=x^2$

    이 함수를 미분하면 다음과 같은 과정이 필요합니다.

    1. 주어진 함수가 $f(x)=x^2$ 일 때, 미분한 함수 $f^{\prime}(x)$ 를 구하려면 지수법칙을 이용합니다.

    2. 지수법칙에 따라 $x^n$ 의 미분은 $n x^{n-1}$ 입니다.
    
    그래서, $f(x)=x^2$ 의 경우:
    
    $$
    f^{\prime}(x)=2 x^{2-1}=2 x
    $$
    
    
    결과적으로, $f(x)=x^2$ 의 순간 변화율은 $f^{\prime}(x)=2 x$ 입니다. 이 함수의 순간 변화를 알고 싶은 특 정 $x$ 값을 대입하면 해당 점에서의 변화율을 알 수 있습니다.
    \end{GPTA}

