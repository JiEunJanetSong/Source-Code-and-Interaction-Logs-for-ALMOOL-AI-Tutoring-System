\begin{GPTQ}
    좀 더 자세하게 설명해줘   
\end{GPTQ}

\begin{GPTA}
 물론입니다! 리만 합에 의한 적분의 근사와 그 결과의 해석을 좀 더 상세히 설명드리겠습니다.

Step-by-Step Explanation

1. 문제 이헤 및 구간 나누기:

- 주어진 구간은 $[0,12]$ 입니다.

- 표에 주어진 $t$ 값은 $0,3,7,12$ 입니다. 따라서, 구간 $[0,12]$ 는 세 개의 서브 구간으로 나눌 수 있습니다: $[0,3],[3,7],[7,12]$.

2. 좌측 리만 합을 사용한 적분 근사:

- 좌측 리만 합은 각 서브 구간의 왼쪽 끝값에서의 함수 값을 사용하는 방법입니다.

- 서브 구간 $[a, b]$ 에서, 넓이를 $f(a)$ 와 $\Delta x=b-a$ 를 곱하여 구합니다.

3. 각 서브 구간의 계산:

- 구간 $[0,3]$ :

- 왼쪽 꿑값 $C(0)=100$ 을 사용합니다.

- 길이는 $3-0=3$ 입니다.

- 넓이는 $3 \cdot 100=300$.

- 구간 [3,7]:

- 왼쪽 끌값 $C(3)=85$ 를 사용합니다.

- 길이는 $7-3=4$ 입니다.

- 넓이는 $4 \cdot 85=340$.

- 구간 $[7,12]$ :

- 왼쪽 끌값 $C(7)=69$ 를 사용합니다.

- 길이는 $12-7=5$ 입니다.

- 넓이는 $5 \cdot 69=345$.

4. 리만 합의 합산:

$$
\int_0^{12} C(t) d t \approx 3 \cdot 100+4 \cdot 85+5 \cdot 69=300+340+345=985
$$

5. 평균 값 계산의 해석:

- $\frac{1}{12} \int_0^{12} C(t) d t$ 는 주어진 구간에서의 평균 값을 의미합니다.

- 여기서, 985 는 주어진 시간 동안 $C(t)$ 의 총 누적 값이므로, 이를 총 시간으로 나눠주면 평균이 뵙니다.

- 따라서, $\frac{985}{12}$ 는 $t=0$ 에서 $t=12$ 까지의 커피의 평균 온도입니다.

6. 계산:

- $\frac{985}{12} \approx 82.08$   

결론적으로 이 평균 온도는 12 분 동안 커피가 냉각되는 동안의 평균 온도가 약 $82.08^{\circ} \mathrm{C}$ 라는 것을 의미합니다.

리만 합을 사용하면 적분 값을 근사할 수 있으며, 픅히 실헙적 데이터나 함수 식이 복잡할 경우 유용합니다. 추가적으로 질문이나 설명이 필요하시면 언제든지 말씀해 주세요!

\end{GPTA}


