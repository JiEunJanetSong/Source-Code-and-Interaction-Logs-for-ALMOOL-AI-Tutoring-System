\begin{GPTQ}
    차수가 흘수일때와 짝수일때의 차이가 뭐야    
\end{GPTQ}

\begin{GPTA}
 차수가 흘수일 때와 짝수일 때에 대해 설명드릴게요.

인수의 차수에 따른 그래프의 성질:

1. 흘수 차수:

- 부호 변화: 인수가 흘수 차수일 경우, 해당 인수 값을 기준으로 함수의 부호가 변합니 다.

- 그래프의 특징: 예를 들어, $x(x-3)$ 같은 인수에서 $x=3$ 은 차수가 1 인 흘수이므로, $x=3$ 을 지날 때 함수의 부호가 바뀝니다. 이는 그래프가 $x$ 축을 지나거나 서로 다른 방향으로 전환됨을 의미합니다.
 
2. 짝수 차수:

- 부호 불변: 인수가 짝수 차수일 경우, 해당 인수 값을 기준으로 합수의 부호가 변하지 않습니다.

- 그래프의 톡징: 예를 들어, $(x-10)^2$ 같은 경우, $x=10$ 에서 차수가 2 인 짝수이므로 부호가 변하지 않고 같은 방향을 유지합니다. 이때 그래프는 $x$ 축을 건드리지 않으며, 축의 한쪽에서 다른 쪽으로 반사됩니다.

요약:

- 홀수 차수: 부호가 변하므로 함수는 해당 점에서 $x$ 축을 교차합니다. 이는 변곡점 또는 극점일 수 있습니다.

- 짝수 차수: 부호가 변하지 않으므로 함수는 $x$ 축을 건드리지 않고 같은 방향으로 움직입니다. 이는 지역 극소점 또는 극대점과 관련될 수 있습니다.

이 차이는 함수가 점에서 어떻게 반응하는지를 보여주며, 부호 변화는 함수의 그래프에서 중요한 변화가 있는지를 결정합니다. 추가 질문이 있으시면 언제든지 말씀해 주세요!
\end{GPTA}

