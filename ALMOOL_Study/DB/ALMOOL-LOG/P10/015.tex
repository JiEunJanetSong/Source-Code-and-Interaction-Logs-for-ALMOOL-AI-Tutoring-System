\begin{GPTQ}
    외부 미분 내부 미분이 무슨 뜻이야   
\end{GPTQ}

\begin{GPTA}
외부 미분과 내부 미분은 체인 률(chain rule)을 이용해서 합성 함수(composite function)를 미분할 때 사용하는 개념입니다. 함수가 여러 개의 함수로 구성되어 있을 때, 각각의 부분을 별도로 미분 한 후 체인 룰을 적용하여 최종적으로 도함수를 구합니다.

개념 설명

1. 외부 미분:

- 주어진 함수에서 가장 바깥쪽에 있는 함수를 미분하는 것입니다.

- 예를 들어, 함수 $y=2(\sin \sqrt{x})^2$ 에서 외부 합수는 $2 u^2$ 입니다. 여기서 $u=\sin \sqrt{x}$ 입니 다.
- 외부 함수 $2 u^2$ 를 $u$ 에 대해 미분하면 $4 u$ 가 됩니다.

2. 내부 미분:

- 외부 함수 내부에 있는 함수를 미분하는 것입니다.

- 내부 함수는 $u=\sin \sqrt{x}$ 이며, 이를 $x$ 에 대해 미분합니다.

- $u=\sin \sqrt{x}$ 의 미분은 $\cos \sqrt{x} \cdot(\sqrt{x})^{\prime}$ 입니다.

- 다시 말해, $\sqrt{x}$ 의 미분값 $(\sqrt{x})^{\prime}$ 이 $\frac{1}{2 \sqrt{x}}$ 인 체인 룰을 사용합니다.

체인 룰 적용

체인 룰은 다음과 같이 표현됩니다:

$$
\frac{d y}{d x}=\frac{d y}{d u} \cdot \frac{d u}{d x}
$$

이 규칙을 사용하여 절국 전체 도함수를 구하계 됩니다:

$$
\frac{d}{d x} 2(\sin \sqrt{x})^2=4(\sin \sqrt{x})(\cos \sqrt{x}) \frac{1}{2 \sqrt{x}}
$$


그래서 주어진 함수의 최종 도함수는 다음과 같습니다:

$$
\frac{2 \sin \sqrt{x} \cos \sqrt{x}}{\sqrt{x}}
$$


이렇게 외부와 내부의 미분을 통해 전체 함수를 미분할 수 있습니다.
\end{GPTA}

