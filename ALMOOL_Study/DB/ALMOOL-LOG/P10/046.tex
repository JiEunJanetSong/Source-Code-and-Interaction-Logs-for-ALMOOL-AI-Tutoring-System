\begin{GPTQ}
    왜 흘수일땐 변하고 짝수일땐 안변해?   
\end{GPTQ}

\begin{GPTA}
 차수가 흘수일 때는 부호가 변하고, 짝수일 때는 변하지 않는 이유는 인수가 어떻게 작용하는지와 관련이 있습니다. 이를 더 자세히 설명드릴게요.

수학적 관점:

1. 홀수 차수 인수 $(a \neq 0)$ :

- 아래와 같은 간단한 다항식을 고려해 봅시다: $x^n$, 여기서 $n$ 은 홀수입니다.

- 이때, 근 $x=0$ 에서:

$$
x^n=(-x)^n=-x^n \quad(\text { 홀수는 부호가 반전됨 })
$$

- 이는 $x$ 가 0 보다 작거나 0 보다 를 때 부호가 서로 반대입을 나타냅니다. 다시 말해, 그래프가 이 점을 지나면서 방향이 바뀝니다.

2. 짝수 차수 인수 $(a \neq 0)$ :

- 이제 $x^n$, 여기서 $n$ 은 짝수입니다.

- 근 $x=0$ 에서:

$$
x^n=(-x)^n \quad \text { (짝수는 부호가 유지됨) }
$$

- 따라서 $x$ 의 부호와 상관없이 결과는 항상 양수가 되어 부호가 변하지 않습니다. 이는 그래프가 $x$ 축을 건드리지 않고 같은 방향으로 유지됨을 의미합니다.   
   
직관적 이해:

- 홀수 차수:

- 스케일이 작은 점에 대해 말하자면, $(x-a)$ 의 흘수 차수로 인해 $a$ 를 지나면 그래프의 기울기가 바뀌게 됩니다. 이 변화로 인해 함수의 값의 부호가 바뀝니다.

- 짝수 차수:

- $(x-a)$ 의 짝수 차수로 인해 $a$ 를 지나도 그래프의 기울기가 변하지 않고, 함수를 감싸 안은 모양으로 반사해서 다시 같은 쪽으로 진행 하게 됩니다. 따라서 부호가 변경되지 않습니다.

즉. 함수의 인수가 진수함수일 때는 본질적으로 부호가 바펴지만, 반면에 짝수 심도의 인수일 경우 그것은 함수의 값을 늘 변화 없이 유지하게 됩니다.

질문에 대한 답변이 되었길 바랍니다. 추가적으로 더 궁금한 점이 있다면 말씀해 주세요!
\end{GPTA}

