\begin{GPTQ}
    리만 합이 뭐야  
\end{GPTQ}

\begin{GPTA}
리만 합(Riemann sum)은 주어진 구간에서 합수의 적분을 근사하기 위한 방법입니다. 리만 합은 구간을 여러 개의 작은 부분 구간으로 나누고, 각 부분 구간에서 함수 값을 평가하여 그 값을 구간 의 길이와 곱한 다음 모두 더하여 전체 구간의 적분 값을 근사하는 방법입니다.

    리만 합은 사용하는 부분 구간의 위치에 따라 여러 가지 방법으로 나뉩니다. 그 중 가장 일반적인 것들은 다음과 같습니다:

    1. 좌측 리만 합 (Left Riemann sum): 각 부분 구간의 왼쪽 끌에서의 함수 값을 사용합니다.

    2. 우측 리만 합 (Right Riemann sum): 각 부분 구간의 오른쪽 끌에서의 함수 값을 사용합니다.

    3. 중앙 리만 합 (Midpoint Riemann sum): 각 부분 구간의 중간점에서의 함수 값을 사용합니다.
    
    리만 합은 구간을 나누는 부분의 수를 늘리면 (즉, 구간을 더 잘게 쪼개면) 실제 적분 값에 더 가까워집니다. 이는 리만 적분의 정의를 기반으로 한 것으로, 연속 함수의 경우 부분 구간의 수가 무한으로 갈 때 리만 합이 함수의 정확한 적분 값에 수렴하게 됩니다.
    \end{GPTA}

