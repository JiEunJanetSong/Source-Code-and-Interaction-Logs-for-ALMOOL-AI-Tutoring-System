\begin{GPTQ}
    원뿔의 높이를 어떻게 구해?   
\end{GPTQ}

\begin{GPTA}
원뿔의 넓이를 구하려면 표면적을 구해야 합니다. 원뿔의 표면적은 두 부분으로 구성됩니다: 밑 면의 넓이와 옆면의 넓이입니다.

    1. 밑면의 넓이 (Base Area): 원뿔의 밑면은 원이기 때문에, 밑면의 넓이는 $\pi r^2$ 입니다. 여기에 서 $r$ 은 원뿔 밑면의 반지름입니다.

    2. 옆면의 넒이 (Lateral Surface Area): 옆면의 넓이는 원뿔의 모선(원뿔의 꼭지점에서 밑면 원주로 이어지는 선)의 길이를 $l$ 이라고 할 때, $\pi r l$ 입니다.
    
    따라서, 원뿔의 전체 표면적 $A$ 는 다음과 같이 계산할 수 있습니다:
    
    $$
    A=\pi r^2+\pi r l=\pi r(r+l)
    $$
    
    
    여기에서:

    - $r$ 은 밀면의 반지름,

    - $l$ 은 원뿔의 모선의 길이입니다.
    
    이 공식을 사용하여 원뽈의 표면적을 쉽게 구할 수 있습니다.
\end{GPTA}

