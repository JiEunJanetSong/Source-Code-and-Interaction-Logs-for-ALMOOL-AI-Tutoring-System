\begin{GPTQ}
    왜 저맇게 치환을 하는거야?    
\end{GPTQ}

\begin{GPTA}
    적분 문제를 해결할 때 특정 치환을 사용하는 이유는 문제를 보다 더 간단하게 만들고 효율적으로 접근할 수 있기 때문입니다. 이번 문제에서는 $t=\pi-x$ 라는 치환을 사용했는데, 그 이유는 대칭성을 활용하기 위해서입니다.

    치환의 이유

    1. 대칭성 탐색: 주어진 함수 $\frac{e^x-1}{e^{x-x}}$ 느 $[0, \pi]$ 구간에서 특정 대칭 구조를 가질 수 있습니다. $t=\pi-x$ 치환은 이 대칭을 표현하기 위한 수단입니다.

    2. 통합의 단순화: 두 적분을 더한 후에 결과가 단순화될 수 있어 계산이 더 간단해질 수 있습니다. 우리가 얻은 두 적분을 합했을 때, 자연 로그 형태로 간단하게 정리할 수 있었습니다.

    3. 해답의 유도: 대칭적 변화를 통해 어떤 기준점(이 경우에는 $\pi$ )을 중심으로 변환하면서, 둘 의 변형을 합쳐 계산할 수 있습니다. 이는 복잡한 적분의 해답을 보다 직접적으로 이끌어내는 패턴 중 하나입니다.
    
    결국 이러한 치환은 문제를 쉽게 풀기 위한 전략적 도구이며, 주어진 함수의 성질(즉, 대칭 등)을 활용하려는 목적입니다. 추가 질문이 있다면 언제든지 말씀해 주세요!
    \end{GPTA}

