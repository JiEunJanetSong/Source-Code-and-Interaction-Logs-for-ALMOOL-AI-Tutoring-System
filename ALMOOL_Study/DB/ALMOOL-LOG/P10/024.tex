\begin{GPTQ}
    밑이 e일 때만 적용되는거야?   
\end{GPTQ}

\begin{GPTA}
아니요, 지수 함수의 미분 규칙은 밑이 $e$ 인 경우뿐만 아니라 다른 밑을 가지는 경우에도 적용할 수 있습니다. 다만, 밑이 $e$ 일 때는 직접적으로 자연로그의 성질을 활용하여 쉽게 미분할 수 있습니다. 다른 밑을 가지는 경우도 관련된 규칙을 사용하여 미분할 수 있습니다. 그 과정을 살펴보겠습니다.
       
일반적인 지수 함수의 미분

밑이 $a$ 인 지수 함수 $f(t)=a^u$ 의 도함수는:

$$
\frac{d}{d t} a^u=a^u \cdot \ln (a) \cdot \frac{d u}{d t}
$$


여기서 $\ln (a)$ 는 밑 $a$ 의 자연로그입니다.

밑이 $e$ 인 경우

밑이 $e$ 라는 것이 특별한 이유는 $\ln (e)=1$ 이기 때문입니다. 그래서 도함수의 공식이 단순화됩니다:

$$
\frac{d}{d t} e^u=e^{\mathrm{u}} \cdot \frac{d u}{d t}
$$


즉, 일반적인 지수 함수에 비해 밑 $e$ 인 경우는 더 간단하게 미분이 가능합니다. 하지만, 위 공식은 모든 밑의 지수 함수에도 적용되며, 밑이 무엇이든지 관련된 로그를 포함시키면 됩니다.
\end{GPTA}

