\begin{GPTQ}
    잘 이해가 안돼   
\end{GPTQ}

\begin{GPTA}
이해를 돕기 위해 다시 설명하젰습니다.

    핵심 개념:

    - 연속성과 미분 가능성: 어떤 함수 $f(x)$ 가 $x=a$ 에서 미분 가능하려면 $f(x)$ 는 $x=a$ 에서 연속이어야 합니다. 즉, $\lim _{x \rightarrow a} f(x)=f(a)$ 이 성립해야 합니다. 그렇지만 $f(x)$ 가 연속이라고 해서 반드시 미분 가능한 것은 아넙니다. 예를 들어, $f(x)=|x|$ 는 $x=0$ 에서 연속하지만 미분 가능하지 않습니다.

    - 미분의 정의: $f(x)$ 가 $x=a$ 에서 미분 가능하려면 다음의 극한이 존재해야 합니다:
    
    $$
    \lim _{x \rightarrow a} \frac{f(x)-f(a)}{x-a}
    $$
    
    
    이 극한이 존재하지 않으면 $f(x)$ 는 $x=a$ 에서 미분 가능하지 않습니다.
    
    왜 선택지 C가 맞는 답인지:

    - 선백지 C: $\lim _{x \rightarrow 5} \frac{f(x)-f(5)}{x-5}$ 가 존재하지 않는다는 것은 바로 미분 불가능성을 정의하는 조건 입니다. 이 극한은 $f(x)$ 가 주어진 점에서 미분 가능한지를 판단하는 기준입니다.

    - 선택지 $\mathbf{B}: \lim _{z \rightarrow 5} f(x)$ 가 존재하지 않는다는 것만으로는 $f(x)$ 가 $x=5$ 에서 미분 가능하지 않다는 것을 충분히 설명하지 않습니다. $f(x)$ 가 불연속일 수 있지만, 불연속이라는 것만 가지고 미분 가능성의 모든 경우를 커버 할 수는 없습니다.
    
    결론적으로, 미분 가능성의 문제에서 가장 중요한 기준은 선백지 C에 제시된 극한이 존재하는지 여부입니다. 이 때문에 $\mathbf{C}$ 가 유일한 확실한 답변이 됩니다.
    훅시 더 궁금한 부분이 있으면 말씀해 주세요! 추가 설명드리젰습니다.
  \end{GPTA}

