\begin{GPTQ}
    최소값과 최대값 어떻게 구분해   
\end{GPTQ}

\begin{GPTA}
    좋은 질문입니다! 극점을 찾은 후, 해당 점이 지역 최소값인지 최대값인지 구분하기 위해 두 가지 방법을사용할 수 있습니다: 이차 도함수 테스트와 일차 도함수 테스트입니다. 여기에서는 이차 도함수 테스트를 사용하여 설명 하겠습니다.

    이차 도함수 테스트

    이차 도함수 테스트는 극점 $x=c$ 에서 조건을 확인하여 그 점의 성질을 결정하는 방법입니다.

    1. 이차 도함수 구하기: 함수 $f(x)$ 의 두 번째 도함수 $f^{\prime \prime}(x)$ 를 계산합니다.
    
    $$
    \begin{gathered}
    f(x)=x^3-3 x^2-9 x+11 \\
    f^{\prime}(x)=3 x^2-6 x-9
    \end{gathered}
    $$
    
    
    다시 미분하여 이차 도함수를 구하면,
    
    $$
    f^{\prime \prime}(x)=6 x-6
    $$
    
    2. 이차 도함수의 부호에 따라 판단하기:

    - $f^{\prime \prime}(c)>0$ : 이 경우, $x=c$ 에서 함수는 극소값을 가집니다.

    - $f^{\prime \prime}(c)<0$ : 이 경우, $x=c$ 에서 함수는 극대값을 가집니다.

    - $f^{\prime \prime}(c)=0$ : 이 경우, 테스트로는 결정 할 수 없으며, 다른 방식 펼요.

    3. 계산하기:

    - $x=3$ 에 대해서:
    
    $$
    f^{\prime \prime}(3)=6(3)-6=18-6=12>0
    $$
    
    
    따라서, $x=3$ 에서 함수는 극소값을 가집니다.
    - $x=-1$ 에 대해서:
    
    $$
    f^{\prime \prime}(-1)=6(-1)-6=-6-6=-12<0
    $$
    
    
    따라서, $x=-1$ 에서 함수는 극대값을 가집니다.
    이러한 이유로 $x=3$ 에서는 극소값, $x=-1$ 에서는 극대값이 됩니다.
    \end{GPTA}

