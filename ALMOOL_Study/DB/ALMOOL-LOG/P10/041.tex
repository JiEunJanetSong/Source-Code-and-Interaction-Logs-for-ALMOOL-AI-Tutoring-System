\begin{GPTQ}
 $\lim \mathrm{x}->\mathrm{af}(\mathrm{x})$ 가 존재하지 않을때 미분이 안되는거면 B 도 맞는 답 아니야?   
\end{GPTQ}

\begin{GPTA}
좋은 질문입니다! 하지만 여기서 중요한 것은 $f(x)$ 가 $x=5$ 에서 정의되지 않거나 접선이 존재하 지 않는 것과 $\lim _{x \rightarrow 5} f(x)$ 가 존재하지 않는 것이 반드시 동일한 것이 아님을 이해하는 것입니다.

차이를 설명하자면:

- Continuity and Differentiability: $f(x)$ 가 $x=5$ 에서 미분 가능하려면, $f(x)$ 는 연속이어야 하고, 미분계수를 정의 할 수 있는 상태여야 합니다. 그러나 미분 가능하지 않을 경우 반드시 불연속인 것은 아닙니다.

- Statement B: $\lim _{x \rightarrow 5} f(x)$ 가 존재하지 않는다는 말은 $f(x)$ 의 좌극한 및 우극한이 $x=5$ 에서 다르거나 그 자체로 정의되지 않았을 수 있습니다. 이는 불연속성을 의미할 수 있지만, 모든 미분 불가능한 경우에 해 당되지는 않습니다.

결과적으로, Statement C 는 명확하게 $f(x)$ 가 $x=5$ 에서 미분둴 수 없다는 것을 증명합니다. 따라서 미분 불가능성에 대한 직관적으로 확실한 답변은 Option C이고, Option B는 미분 불가능성의 충분조건이 아닙니다.
  
   
\end{GPTA}

