\begin{GPTQ}
    왜 $\cos x$ 가 돼   
\end{GPTQ}

\begin{GPTA}
 삼각함수의 미분 공식은 여러 방법으로 이해 할 수 있지만, 가장 기본적인 이해는 $\sin x$ 와 $\cos x$ 의 그래프와 기하학적 관계를 퉁해 가능합니다.

직관적 이해

1. 단위원 단위 원(circle) 위에 있는 각 $\theta$ 로부터:
- 단위원은 반지름이 1 인 원이고, $(\cos \theta, \sin \theta)$ 가 원 위의 좌표입니다.

- 각 $\theta$ 의 작은 변화률 고려할 때, $\sin \theta$ 의 변화율이 $\cos \theta$ 와 관련이 있게 됩니다.

- 본질적으로 이 변화율은 $\cos \theta$ 의 값으로 수렴합니다.

2. 직각삼각형 및 극한 생각하기:

- $\sin x$ 의 변화를 $\Delta x$ 가 아주 작을 때 고려할 수 있습니다. 예를 들어

$$
\lim _{\Delta x \rightarrow 0} \frac{\sin (x+\Delta x)-\sin x}{\Delta x}
$$


기준으로 극한을 구하면 $\cos x$ 로 극한이 다가오게 됩니다.

이는 곧 $\sin x$ 의 도합수가 $\cos x$ 가 쾬다는 것을 보여줍니다.

3. 그래프적 관점에서 이혜하기:

- 만약 $\sin x$ 의 그래프를 보면, 각 점에서의 기울기가 $\cos x$ 에 의해 나타난다는 것을 직관적으로 불 수 있습니다.

- $\sin x$ 의 변화율이 변화할 때 극대로 가는 점과 동일한 점에서 $\cos x$ 도 0 이 되는 등의 규칙성이 있습니다.

정리하자면, $\sin x$ 를 $x$ 에 대해 미분하면 $\cos x$ 가 빕니다. 그 이유는 $\sin x$ 의 기하학적 성질과 삼각 함수의 본질적인 관계 때문입니다. 이는 일반적인 미적분에서 많은 응용을 갖고 있으며 기본적인 삭각함수 미분법칙 중 하나입니다. 
   
\end{GPTA}

