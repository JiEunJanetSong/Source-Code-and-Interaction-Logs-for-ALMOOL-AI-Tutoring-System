\begin{GPTQ}
    By comparing with the definite integral definition 이거에 대해 자세히 설명해줘   
\end{GPTQ}

\begin{GPTA}
이 한 줄(``By comparing with the definite integral definition: ...")은 리만 합(Riemann sum)의 정의와 현재 주어진 유리식들의 합을 비교하여, 해당 식이 어떤 정적분을 나타내는지를 확인하는 과정입니다. 조금 더 자세히 살펴보겠습니다.

1. 리만 합(Riemann Sum)이란?

함수 $f(x)$ 가 구간 $[a, b]$ 에서 정의되고, 구간 $[a, b]$ 를 $n$ 개의 작은 구간으로 나누었을 때 각 작은 구간의 대표점 $x_k^*$ 에서 함수값을 평가하고, 그 값에 작은 구간의 길이(보통 $\Delta x$ )를 곱해 모두 더한 것을 리만 합이라고 합니다. 즉,

$$
\sum_{k=1}^n f\left(x_k^*\right), \Delta x
$$


이 리만 합입니다. 그리고 $n$ 이 무한히 커져 각 구간의 길이 $\Delta x$ 가 0 에 가까워지면, 이 합이 바로 정적분

$$
\int_a^b f(x), d x
$$

에 가까워집니다. 이를 식으로 쓰면,

$$
\int_a^b f(x), d x=\lim _{n \rightarrow \infty} \sum_{k=1}^n f\left(x_k^*\right), \Delta x
$$

2. 문제에서 주어진 합을 리만 합 형태로 보기

문제에서 다루는 식은

$\lim_{n \to \infty}\dfrac{1}{n}\left( \dfrac{1}{2+\dfrac{1}{n}}+\dfrac{1}{2+\dfrac{2}{n}}+\dfrac{1}{2+\dfrac{3}{n}}+\cdots + \dfrac{1}{2+\dfrac{n}{n}} \right)$

이를 간단히 쓰면,

$$
\lim _{n \rightarrow \infty} \sum_{k=1}^n \frac{1}{n}, \frac{1}{2+\frac{k}{n}}
$$


이때

- $\Delta x$ 역할을 하는 것이 $\frac{1}{n}$ 입니다.

- $\frac{k}{n}$ 가 작은 구간에서의 ``대표점" $x_k^*$ 역할을 합니다.

즉, 아래와 같이 대응시킬 수 있습니다:

1. 작은 구간의 길이: $\Delta x=\frac{1}{n}$.

2. 대표점: $x_k^*=\frac{k}{n}$.

3. 함수: $f(x)=\frac{1}{2+x}$.

4. 구간: $x$ 가 0 부터 1 까지 변한다(예냐하면 $k$ 는 1 부터 $n$ 까지 변하므로, $\frac{k}{n}$ 은 $\frac{1}{n}$ 에서 $\frac{n}{n}=1$ 까지 변합니다. $n \rightarrow \infty$ 가 되면 사실상 $[0,1]$ 전 구간을 촘촘히 채운다고 볼 수 있습니다).

3. 정적분으로 전환

위 대응을 퉁해, 문제의 합은 아래와 같은 리만 합이 뵙니다:

$$
\sum_{k=1}^n f\left(\frac{k}{n}\right), \Delta x=\sum_{k=1}^n \frac{1}{2+\frac{k}{n}} \cdot \frac{1}{n}
$$

$n$ 이 무한히 커질 때 $($ 즉, $n \rightarrow \infty)$ 이 리만 합은 함수 $f(x)=\frac{1}{2+x}$ 를 $[0,1]$ 구간에서 적분한 값과 같 아집니다:

$$
\int_0^1 \frac{1}{2+x}, d x
$$


따라서 문제의 극한값은

$$
\lim _{n \rightarrow \infty} \frac{1}{n} \sum_{k=1}^n \frac{1}{2+\frac{k}{n}}=\int_0^1 \frac{1}{2+x}, d x
$$


와 동일함을 알 수 있습니다.
4. 결론
"By comparing with the definite integral definition:"이라는 문장은 바로 이 리만 합 표현과 정적분의 정의가 동일한 형식입을 확인했다는 의미입니다. 문제의 유리식 합이 리만 합 꼴임을 인식하고, 그에 해당하는 적분이 무엇인지 찾아낸 것이죠.
이를 통해

$$
\lim _{n \rightarrow \infty} \frac{1}{n}\left(\frac{1}{2+\frac{1}{n}}+\frac{1}{2+\frac{2}{n}}+\cdots+\frac{1}{2+\frac{n}{n}}\right) ;=; \int_0^1 \frac{1}{2+x}, d x
$$


라는 결론을 얻습니다.

즉, 리만 합의 정의와 직접 대응시킴으로써, 유한 개의 합이 무한 개로 갈 때 나타나는 극한 값을 정적분으로 해석할 수 있음을 보여주는 과정이었습니다.
\end{GPTA}

