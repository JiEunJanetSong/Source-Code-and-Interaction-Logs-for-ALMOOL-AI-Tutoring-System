\begin{GPTQ}
    왜 저렿게 단순화 되는거야   
\end{GPTQ}

\begin{GPTA}
단순화 과정에 대해 조금 더 자세히 설명해드리겠습니다. 

우리가 사용하는 것은 $x \rightarrow \infty$ 일 때 지 배적인 항을 이용하여 복잡한 식을 대략적으로 표현하는 방법입니다. 이 방법은 다음과 같은 과정으로 이루어집니다:

1. 분자의 지배적인 항 찾기:

- 원래 분자 식은 $10-6 x^2$ 입니다.

- $x$ 가 아주 큰 경 우에는 $-6 x^2$ 가 10 보다 훤씬 크고 지배적이 됩니다.

- 따라서, $10-6 x^2 \approx-6 x^2$ 로 대략화합니다.

2. 분모의 지배적인 항 찾기:

- 원래 분모 식은 $5+3 e^x$ 입니다.

- $x$ 가 아주 큰 경우예는 $3 e^x$ 가 5 보다 훤씬 크고 지배적이 됩니다.

- 따라서, $5+3 e^x \approx 3 e^x$ 로 대략화합니다.

3. 전체 식의 근사화:

- 이제 식은 다음과 같이 근사화할 수 있습니다:

$$
\frac{10-6 x^2}{5+3 e^z} \approx \frac{-6 x^2}{3 e^z}
$$

4. 상수 분리 및 단순화:

- $\frac{-6}{3}$ 은 상수이므로 계산할 수 있습니다: $\frac{-6}{3}=-2$

- 따라서 식은 다음과 같이 단순화할 수 있습니다:

$$
\frac{-6}{3} \cdot \frac{x^2}{e^x}=-2 \cdot \frac{x^2}{e^x}
$$

이 단순화는 $x$ 가 매우 큰 상황에서 주어진 문제의 지배적인 성질을 간단하게 표현할 수 있게 해줍니다. 극한을 구할 때, 작은 항들은 $\infty$ 에서의 성질을 거의 바꾸지 않는다는 점을 이용합니다.

이 과정이 이해되시나요? 추가 설명이 필요하면 언제든지 말씀해 주세요.

\end{GPTA}

