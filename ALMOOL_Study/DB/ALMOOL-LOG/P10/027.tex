\begin{GPTQ}
    정수가 아닐때는?
\end{GPTQ}

\begin{GPTA}
    시그마 표기법에서는 보통 정수를 사용하여 시작 값과 끝 값을 지정합니다. 그러나 합계 표현식 자체는 어떠한 실수나 그 외의 값을 포함할 수 있습니다. 시그마 표기법에서 합을 구하는 변수, 예 를 들어 위의 예에서는 $n$ 이 정수로 변하는 동안, 각 항이 실수나 복소수 등일 수도 있습니다. 
    
    여기에는 정수가 아닌 경우와 관련된 예제입니다:

    $\sum_{n=1}^3 \dfrac{1}{n}$

    여기서 합을 구해야 하는 항들은 분수(실수)입니다:
    
    $$
    \sum_{n=1}^3 \frac{1}{n}=\frac{1}{1}+\frac{1}{2}+\frac{1}{3} \quad=1+0.5+0.333 \ldots \approx 1.833
    $$
    
    
    이처럼 변수 $n$ 이 정수 범위 안에서 변할 때, 각 항은 실수가 됩니다. 그러나 매개 변수가 정수가 아닌 경 우에는 보통 적분을 사용합니다. 영역을 정수로 제한하는 것이 시그마의 일반적인 사용 방식입니다.
    
\end{GPTA}

