\begin{GPTQ}
    x를 어떻게 구해
\end{GPTQ}

\begin{GPTA}
    좋습니다, $\angle x$ 를 계산하는 방법을 단계별로 설명해 드리겠습니다.

    1. 우선 $\angle M N Q$ 를 찾습니다:
    
    $$
    \begin{gathered}
    \angle N M Q+72^{\circ}+90^{\circ}=180^{\circ} \\
    \angle N M Q+162^{\circ}=180^{\circ} \\
    \angle N M Q=180^{\circ}-162^{\circ}=18^{\circ}
    \end{gathered}
    $$
    
    2. $\angle M R N$ 를 구합니다. 이 각은 $\angle x$ 이며, 삼각함수를 사용하여 구할 수 있습니다: 삼각형 $\triangle M N R$ 에서 $\sin \theta=\frac{\text { 대변 }}{\text { 빗변 }}=\frac{M N}{M R}=\frac{15}{10 \sqrt{3}}$
    
    $$
    \theta=\sin ^{-1}\left(\frac{15}{10 \sqrt{3}}\right)
    $$
    
    
    계산을 퉁해:
    
    $$
    \theta=60^{\circ}
    $$
    
    3. $x$ 를 찾기 위혜, 삼각형 $\triangle M N R$ 의 내각의 합을 사용합니다:
    
    $$
    \begin{gathered}
    \angle NMQ +x+\angle M R N + 90^{\circ} =180^{\circ} \\
    x=180^{\circ}-78^{\circ}=12^{\circ}
    \end{gathered}
    $$
    
    
    이맇게 하여 $x$ 는 $12^{\circ}$ 입니다. 추가적인 도움이 푤요하면 언제든지 말씀해 주세요!
    
\end{GPTA}

