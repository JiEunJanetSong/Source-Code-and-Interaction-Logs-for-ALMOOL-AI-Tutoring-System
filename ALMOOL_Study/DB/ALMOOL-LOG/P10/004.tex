\begin{GPTQ}
    문제 풀이 해줘   
\end{GPTQ}

\begin{GPTA}
룰론입니다! 주어진 문제를 해석하고 풀이 과정을 단계별로 설명드리젰습니다.

    문제 풀이 (b)

    주어진 표의 값을 이용하여 좌측 리만 합으로 적분 $\int_0^{12} C(t) d t$ 값을 근사하고, 그 의미를 해석 하겠습니다.
    
    좌측 리만 합을 사용하여 $\int_0^{12} C(t) d t$ 의 근사값을 구합니다.

    1. 구간 나누기: 주어진 구간 $[0,12]$ 를 표에 나와 있는 점들에 따라 서브 구간으로 나눕니다. 서브 구간들은 다음과 같습니다: $[0,3],[3,7],[7,12]$.

    2. 각 구간에서 함수의 값 사용하기: 좌측 리만 합을 사용하므로, 각 서브 구간의 왼쪽 끝값에서의 함수 값을 사용합니다.

    3. 리만 합 계산:
    
    $$
    \begin{gathered}
    \int_0^{12} C(t) d t \approx(3-0) \cdot C(0)+(7-3) \cdot C(3)+(12-7) \cdot C(7) \\
    =3 \cdot 100+4 \cdot 85+5 \cdot 69
    \end{gathered}
    $$
    
    4. 계산하기:
    
    $$
    =3 \cdot 100+4 \cdot 85+5 \cdot 69=300+340+345=985
    $$
    
    
    따라서, $\int_0^{12} C(t) d t \approx 985$ 입니다.   
    
    1. 해석:

$$
\frac{1}{12} \int_0^{12} C(t) d t
$$


이 값은 $t=0$ 에서 $t=12$ 까지의 시간 동안 커피의 평균 온도(섭씨)를 의미합니다.

따라서, 문제의 해석은 ``해당 12 분 동안의 커패의 평균 온도는 $\frac{985}{12}$ 섭씨 도"라는 의미입니다. 이 값은 82.08 섭씨 도로 근사할 수 있습니다.

이 과정을 통해 좌측 리만 합을 사용하여 적분 값을 근사하고, 그 결과를 해석할 수 있었습니다. 추가적인 질문이 있으면 언제든지 말씀해 주세요!

\end{GPTA}

