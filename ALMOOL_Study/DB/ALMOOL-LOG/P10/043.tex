\begin{GPTQ}
    연속성이라는게 무슨 뜻이야  
\end{GPTQ}

\begin{GPTA}
연속성에 대해 설명해드리겠습니다.

    연속성의 의미

    함수 $f(x)$ 가 어떤 점 $x=a$ 에서 연속이라고 할 때, 다음 세 가지 조건이 모두 성립해야 합니다:

    1. 함수가 정의되어 있어야 함: $f(a)$ 는 존재해야 합니다.

    2. 극한이 존재해야 함: $\lim _{x \rightarrow a} f(x)$ 는 존재해야 합니다.

    3. 극한과 함수 값이 일치혜야 함: $\lim _{z \rightarrow a} f(x)=f(a)$.
    
    이 세 조건이 모두 만족될 때, 함수 $f(x)$ 는 $x=a$ 에서 연속이라고 합니다.  
   
    연속성의 예

- 연속인 함수: 예를 들어, $f(x)=2 x+3$ 는 모든 점에서 연속입니다. 즉, 어떤 $x$ 에 대해서도 위의 세 가지 조건이 만족됩니다.

- 연속이지 않은 함수: 함수 $f(x)$ 가 $x=a$ 에서 정의되어 있지 않거나, $\lim _{x \rightarrow a} f(x)$ 가 존재하 지 않거나, $\lim _{z \rightarrow a} f(x) \neq f(a)$ 라면, $f(x)$ 는 $x=a$ 에서 연속이지 않습니다. 예를 들어, $f(x)=\frac{1}{x}$ 는 $x=0$ 에서 정의되지 않아 연속이지 않습니다.

연속성은 함수의 그래프가 끊어짐 없이 매끄럽게 그려질 수 있는지를 의미합니다. 혹시 더 궁금 한 점이 있다면 언제든지 질문해 주세요!
\end{GPTA}

