\begin{GPTQ}
    더 자세히 설명해줘   
\end{GPTQ}

\begin{GPTA}
물론입니다! 주어진 문제 $\sum_{n=1}^{\infty} \frac{e^n}{\pi^n}$ 의 풀이를 한 단계씩 좀 더 자세히 설명해드리겠습니다.  

문제 이해하기

이 수열은 각 항이 $n$ 의 함수로 주어진 무한 등비수열입니다. 여기서 각 항은 $\frac{s^n}{\pi^n}$ 로 주어집니다. 이 수열은 무한대까지 더해지므로 무한 등비수열의 합으로 접근할 수 있습니다.

첫째 단계: 수열의 일반항 찾기

주어진 수열은 다음과 같이 나타낼 수 있습니다:

$$
\sum_{n=1}^{\infty}\left(\frac{e}{\pi}\right)^n
$$


이것은 무한 등비수열의 형태 $\sum_{n=1}^{\infty} a r^n$ 중에서:

- 첫 번째 항 $a=\left(\frac{e}{\pi}\right)$

- 공비 $r=\left(\frac{e}{\pi}\right)$

둘쪠 단계: 수렴 조건

무한 등비수열이 수렵하려면 공비 $r$ 의 절대값이 1 보다 작아야 합니다. 즉, $|r|<1$ 이어야 합니다. 주어진 공비를 살펴보면:

$$
\left|\frac{e}{\pi}\right|<1
$$


주어진 $e \approx 2.718$ 과 $\pi \approx 3.141$ 로 인해 $\frac{e}{\pi}<1$ 임을 확인할 수 있습니다. 따라서 수열은 수렴합니다.

셋째 단계 : 수열의 합 구하기

수렴하는 무한 등비수열의 합은 다음과 같은 공식으로 구할 수 있습니다:

$$
S=\frac{a}{1-r}
$$


이 공식을 주어진 수열에 적용하면:

$$
S=\frac{\frac{e}{\pi}}{1-\frac{e}{\pi}}
$$


분자와 분모에서 $\frac{1}{\pi}$ 를 공통으로 사용하는 형태라서 형태를 조금 더 정리하면:

$$
S=\frac{e}{\pi-e}
$$


결론적으로, 주어진 무한 등비수열의 합은:

$$
\frac{e}{\pi-e}
$$


따라서 이 문제의 정답은 $\frac{c}{\pi-e}$ 이며, 이는 선택지에서 $(\mathrm{B})$ 에 해당합니다.

\end{GPTA}

