\begin{GPTQ}
    u 가 정 확히 어떨 때 쓰이는거야   
\end{GPTQ}

\begin{GPTA}
$u$ 는 주로 합성 함수의 미분에서 사용되는 변수로, 복잡한 함수의 일부를 단순화하기 위해 임시로 정의되는 것입니다. 이를 통해 미분 과정을 더 체계적이고 명확하게 할 수 있습니다. 은밀히 말하면, 이는 연쇄 법칙(Chain Rule)을 활용하기 위한 방법입니다. 연쇄 법칙에 들어가기 전에 $u$ 의 사용 사례를 설명하젰습니다.

$u$ 의 사용 예시

예시 $1: \ln \left(t^2+1\right)$

함수 $x(t)=\ln \left(t^2+1\right)$ 의 도함수를 구하려고 할 때,

- 내부 함수: $u(t)=t^2+1$

- $u(t)$ 에 대한 도함수: $\frac{d u}{d t}=2 t$

연쇄 법칙을 사용하면:

$$
x^{\prime}(t)=\frac{d}{d t} \ln (u)=\frac{1}{u} \cdot \frac{d u}{d t}=\frac{1}{t^2+1} \cdot 2 t=\frac{2 t}{t^2+1}
$$


예시 2: $e^{3-t}$

함수 $y(t)=e^{3-t}$ 의 도함수를 구할 경우,

- 내부 함수: $u(t)=3-t$

- $u(t)$ 에 대한 도합수: $\frac{d u}{d t}=-1$

연쇄 법칙을 적용하면:   
  
$$
y^{\prime}(t)=\frac{d}{d t} e^u=e^u \cdot \frac{d u}{d t}=e^{3-t} \cdot(-1)=-e^{3-t}
$$


연쇄 법칙(Chain Rule)
연쇄 법칙은 합성 함수 $f(g(t))$ 의 도함수를 구할 때 사용되며, $f(g(t))$ 의 도함수는 다음과 같이 나타낼 수 있습니다:

$$
\frac{d}{d t} f(g(t))=f^{\prime}(g(t)) \cdot g^{\prime}(t)
$$


여기에서 $g(t)$ 를 $u(t)$ 로 나타내어서 연쇄 법칙을 쉽게 적용할 수 있습니다.

따라서 $u$ 는 이러한 합수 구분과 미분 과정에서 단순화와 구조화를 돕고자 사용됩니다.
\end{GPTA}

