\begin{GPTQ}
    왜 그렇게 나와       
\end{GPTQ}

\begin{GPTA}
    구의 부피 공식 $V=\frac{4}{3} \pi r^3$ 는 기하학적 적분을 퉁해 유도됩니다. 구는 무한히 많은 얆은 원퐌(laye r)들로 구성되어 있다고 생각할 수 있습니다. 각 원판의 부피를 모두 더하게 되면 전체 구의 부피를 계산할 수 있습니다. 이를 수학적으로 표현하면 다음과 같은 적분을 통해 구할 수 있습니다:

    $$
    V=\int_{-r}^r \pi\left(r^2-x^2\right), d x
    $$
    
    
    위 적분은 구의 중심을 기준으로 하는 얇은 원판들의 부피 합을 계산하는 과정입니다. 이를 풀면 클래식한 구의 부피 공식 $V=\frac{4}{3} \pi r^3$ 을 얻게 됩니다.

    이 과정에 대한 더 많은 기하학적 또는 적분적 설명이 필요하시다면 말씀해 주세요.
    \end{GPTA}

