\begin{GPTQ}
  무한등비수열이 뭐야  
\end{GPTQ}

\begin{GPTA}
    무한 등비수열은 각 항이 일정 한 비율(공비 $r$ )로 이전 항과 연결된 무한히 긴 수열입니다. 등비수 열이라는 이름은 각 항이 일정한 비율을 이용해서 생성된다는 의미입니다. 수열의 일반형은 다음과 같습니다:

    $$
    a, a r, a r^2, a r^3, \ldots
    $$
    
    
    여기서:

    - $a$ 는 첫 번째 항을 나타냅니다.

    - $r$ 은 각 항이 앞의 항과 곱해지는 일정한 비율, 즉 공비입니다.
    
    무한 등비수열에서 중요한 점 중 하나는 공비 $r$ 의 값에 따라 수열이 수렴하는지 혹은 발산하는지 여부입니다. 
    
    기본적인 수렴 조건은 아래와 같습니다:

    - 무한 등비수열 $\sum_{n=0}^{\infty} a r^n$ 은 $|r|<1$ 일 때 수렴합니다.

    - 수열이 수렴하면 그 합은 다음과 같이 주어집니다:
    
    $$
    S=\frac{a}{1-r}
    $$
    
    
    여기서 $S$ 는 수렴한 수열의 합입니다.

    예를 들어, 수열 $a=1, r=\frac{1}{2}$ 인 경우의 무한 등비수열은 다음과 같습니다:
    
    $$
    1, \frac{1}{2}, \frac{1}{4}, \frac{1}{8}, \ldots
    $$
    
    
    이 경우, 공비 $r=\frac{1}{2}$ 로 $|r|<1$ 이므로 수열이 수렴하며, 그 합은
    
    $$
    S=\frac{1}{1-\frac{1}{2}}=2
    $$
    
    
    으로 계산됩니다.

    무한 등비수열은 수학에서 매우 중요한 개념으로, 다양한 수학적, 물리적 문제에서 활용됩니다.
    \end{GPTA}

