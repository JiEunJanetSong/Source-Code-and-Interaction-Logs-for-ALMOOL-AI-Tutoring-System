\begin{GPTQ}
    벡터가 왜 저렇게 돼?   
\end{GPTQ}

\begin{GPTA}
  좋은 질문입니다. 벡터가 어떻게 도출되었는지 단계별로 살펴보겠습니다.

1. 위치 함수 미분: 주어진 매개변수 방정식인 $x(t)=\ln \left(t^2+1\right)$ 과 $y(t)=e^{3-t}$ 의 서로 다른 두 변수를 각각 $t$ 에 대해 미분합니다.

2. $x(t)$ 의 도함수:

- $x(t)=\ln \left(t^2+1\right)$ 의 도함수를 계산합시다.

- 자연 로그 함수의 도함수는 $\frac{d}{d t} \ln (u)=\frac{1}{v}$. $\frac{d u}{d t}$ 이라는 공식과 유사합니다.

- 여기서 $u(t)=t^2+1$ 이므로 $\frac{d u}{d t}=2 t$ 입니다.

- 따라서 $x^{\prime}(t)=\frac{2 t}{t^2+1}$ 입니다.

3. $y(t)$ 의 도함수:

- $y(t)=e^{3-t}$ 의 도함수를 계산합시다.

- 지수 함수의 도합수는 $\frac{d}{d t} e^u=e^u \cdot \frac{d u}{d t}$ 즉각 계산에 적용됩니다.

- 여기서 $u(t)=3-t$ 이므로 $\frac{d u}{d t}=-1$ 입니다.

- 따라서 $y^{\prime}(t)=-e^{3-t}$ 입니다.

4. $t=1$ 일 때 속도 벡터:

- 각각의 도함수 $x^{\prime}(t)$ 와 $y^{\prime}(t)$ 를 $t=1$ 에 대입하여 계산합니다.

- $x^{\prime}(1)=\frac{2 \cdot 1}{1^2+1}=\frac{2}{2}=1$

- $y^{\prime}(1)=-e^{3-1}=-e^2$

결국, 속도 벡터는 $\left\langle x^{\prime}(1), y^{\prime}(1)\right\rangle=\left\langle 1,-e^2\right\rangle$ 가 뒵니다. 이 벡터는 시간 $t=1$ 에서의 위치 변화율을 나타냅니다.
 
\end{GPTA}

