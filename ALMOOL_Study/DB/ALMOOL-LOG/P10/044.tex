\begin{GPTQ}
    문제 설명 해줘   
\end{GPTQ}

\begin{GPTA}
문제를 설명해 드리졌습니다. 곡점을 가지는 $x$ 값들을 찾는 것입니다.

세부 단계는 다음과 같습니다:

1. 변곡점 후보 식별:

- 변곡점은 $f^{\prime \prime}(x)=0$ 인 곳에서 발생하며, 이차 도함수가 부호 변화를 해야 합니다.

- 주어진 이차 도함수 $x(x-3)^5(x-10)^2=0$ 을 풀면 $x=0,3,10$ 의 값을 얻습니다.

2. 부호 변화 확인:

- $x=0$ 에서: 인수 $x$ 는 흘수 차수 $(1$ 이기 때문에 $) \rightarrow$ 부호 변화가 발생합니다.

- $x=3$ 에서: 인수 $(x-3)^5$ 는 휼수 차수 $(5$ 이기 때문에) $\rightarrow$ 부호 변화가 발생합니다.

- $x=10$ 에서: 인수 $(x-10)^2$ 는 짝수 차수 ( 2 이기 때문에 $) \rightarrow$ 부호 변화가 없습니다.

3. 변곡점 식별:

- 변곡점은 $f^{\prime \prime}(x)$ 가 부호 변화를 가지는 곳에서 빨생하므로, $x=0$ 과 $x=3$ 에서 변곡점이 있습니다.

4. 정답 선택:

- 제시된 선택지 중에서 정 답은 0 과 3 만 포합된 옵션 $\mathbf{B}$ 입니다.

이렿게 문제를 해결할 수 있습니다. 추가 질문이나 다른 도움 펼요하시면 말씀해 주세요!
\end{GPTA}

