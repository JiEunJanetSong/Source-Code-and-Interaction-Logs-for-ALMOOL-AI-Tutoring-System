\begin{GPTQ}
    어떻게 구해    
\end{GPTQ}

\begin{GPTA}
구의 표면적 공식 $A=4 \pi r^2$ 는 구의 기하학적 성질을 활용하여 유도됩니다. 이 공식은 구의 반지름 $r$ 에 기반하여 전체 표면적을 계산합니다.

    이 공식의 유도과정은 구의 표면을 무한히 많은 작은 면적 요소들로 나눈 다음, 이를 모두 합산하여 얻을 수 있습니다. 이를 수학적으로 접근하면 다음과 같이 적분을 통해 구할 수 있습니다:

    구의 표면적 요소는 수직 방향으로 놓인 작은 직사작형들 또는 원퉁형 띠들로 생각뒬 수 있습니다. 각 면적 요소의 면적은 $2 \pi r \sin \theta \cdot r, d \theta$ 가 되며, 이를 0 에서 $\pi$ 까지 적분하면 전체 표면적을 얻게 됩니다:
    
    $$
    A=\int_0^\pi 2 \pi r^2 \sin \theta, d \theta=4 \pi r^2
    $$
    
    
    이 적분은 구의 표면을 기준으로 한 각도 $\theta$ 를 사용하여 표면적을 구하는 과정입니다. 이를 퉁해 구의 표면적을 계산할 수 있습니다. 더 펼요하신 설명이 있다면 언제든지 질문해 주세요.

\end{GPTA}

