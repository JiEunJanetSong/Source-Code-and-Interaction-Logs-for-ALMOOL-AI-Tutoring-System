\begin{GPTQ}
  $x(t)$ 미분하는 풀이과정 더 자세히 설명해줘  
\end{GPTQ}

\begin{GPTA}
물론입니다. $x(t)$ 를 미분하는 과정에 대해 더욱 자세히 설명해드릴게요.

함수 $x(t)=\ln \left(t^2+1\right)$ 는 $t$ 에 대한 자연 로그 함수입니다. 이를 미분하려면, 로그 합수의 미분 공 식을 사용해야 합니다:

1. 로그 함수의 미분 공식:

$$
\frac{d}{d t} \ln (u)=\frac{1}{u} \cdot \frac{d u}{d t}
$$
  
2. 우리의 함수에 적용:

- 여기서 $u(t)=t^2+1$ 입니다.

- 따라서 먼저 $u(t)$ 에 대해 미분합니다:

$$
\frac{d u}{d t}=\frac{d}{d t}\left(t^2+1\right)=2 t
$$

3. 로그 함수의 미분 공식을 적용:

- 이제 $x(t)$ 의 미분을 위해 고유의 $u(t)$ 와 오류연결 도함수를 대입 합니다:

$$
x^{\prime}(t)=\frac{d}{d t} \ln \left(t^2+1\right)=\frac{1}{t^2+1} \cdot(2 t)=\frac{2 t}{t^2+1}
$$


따라서, $x(t)$ 의 도합수는 $\frac{2 t}{t^2+1}$ 로 나타낼 수 있습니다. 이러한 과정은 $\ln$ 내부의 표현식을 이해하고, 두 가지를 연속적으로 계산하여 결과적으로 전체 도함수를 이끌어내는 방식으로 이루어집니다. 

\end{GPTA}

