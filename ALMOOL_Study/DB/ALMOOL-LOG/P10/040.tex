\begin{GPTQ}
    어떨때 미분이 불가능해?   
\end{GPTQ}

\begin{GPTA}
미분 불가능성에 대해 자세허 알아보도록 합시다. 합수 $f(x)$ 가 어떤 점 $x=a$ 에서 미분이 불가능 한 경우는 다음과 같습니다:

1. 불연속성: 함수 $f(x)$ 가 $x=a$ 에서 불연속일 때, 즉 $\lim _{x \rightarrow a} f(x)$ 가 존재하지 않거나 $f(a)$ 와 다를 때.

2. 뾰족한 코너: 그래프가 $x=a$ 에서 급격히 방향을 바껴 ``뾰족한 코너"를 가질 때, 예를 들어 절댓값 함수 $f(x)=|x|$ 는 $x=0$ 에서 미분 불가능합니다.

3. 수직 접선: 함수 $f(x)$ 가 $x=a$ 에서 수직 접선을 가지면, 기울기가 무한대가 되어 미분이 불 가능합니다. 예를 들어, $f(x)=\sqrt[3]{x}$ 는 $x=0$ 에서 수직 접선을 가집니다.

4. 급격한 진동: 합수가 $x=a$ 에서 급격한 진동을 보일 때, 정의된 미분계수가 존재하지 않을 수 있습니다.

이러한 조건 중 하나라도 만족하면 함수는 해당 점에서 미분 불가능합니다. 미분 가능성을 알아 불 때 주의 깊게 살펴볼 펼요가 있습니다.
\end{GPTA}

