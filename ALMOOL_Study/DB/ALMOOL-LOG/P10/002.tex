\begin{GPTQ}
    평균 변화율 계산 어떻게해    
\end{GPTQ}

\begin{GPTA}
평균 변화율은 어떤 합수의 주어진 구간에서의 변화량을 구간의 길이로 나누어 계산합니다. 이는 수학적으로 두 점 사이의 기울기로 해석할 수 있습니다. 주어진 함수가 $C(t)$ 이고, 이 합수의 두 점 이 $\left(t_1, C\left(t_1\right)\right)$ 와 $\left(t_2, C\left(t_2\right)\right)$ 라고 할 때, 그 평균 변화율은 다음과 같이 계산됩니다:
    
    $$
    \frac{C\left(t_2\right)-C\left(t_1\right)}{t_2-t_1}
    $$
    
    
    이 개념을 사용하여 $3 \leq t \leq 7$ 구간에서 주어진 함수 $C(t)$ 의 평균 변화율을 계산하면:
    1. $C(3)=85$ 이고 $C(7)=69$ 입니다.
    2. 구간의 양끌 점은 $t_1=3$ 와 $t_2=7$ 입니다.
    3. 따라서 평균 변화율은 다음과 같습니다:
    
    $$
    \frac{C(7)-C(3)}{7-3}=\frac{69-85}{7-3}=\frac{-16}{4}=-4
    $$
    
    
    이 계산은 시간 $t$ 가 3 분에서 7 분으로 변화할 때 온도 $C(t)$ 가 얼마나 평균적으로 변화하는지를 보여 줍니다. -4는 온도가 매 분마다 평균적으로 4도씩 감소함을 의 미합니다.   
\end{GPTA}

