\begin{GPTQ}
    좀 더 쉽게    
\end{GPTQ}

\begin{GPTA}
물론입니다! 이 관계는 삼각형과 각도의 기본적인 성질에서 나옵니다. 직각삼각형을 생각해 봅시다.

    1. 직각삼각형: 삼각형에서는 세 각도의 합이 $180^{\circ}$ 입니다. 직각 $\left(90^{\circ}\right)^{\circ}$ 이 한 각이라면, 나머지 두 각의 합은 $90^{\circ}$ 가 툅니다. 그래서 각도를 $x$ 라고 하면, 다른 각은 $90^{\circ}-x$ 가 됩니다.

    2. 삼각함수:

    - $\sin x$ 는 직각삼각형에서 대변(직각과 붙어있지 않은 변)의 길이를 빗변의 길이로 나눈 것입니다.

    - $\cos x$ 는 직각 삼각형에서 각도의 인접 변을 빗변의 길이로 나눈 것입니다.
    
    따라서, 만약 $\angle A=x$ 라면, 다른 각 $\angle B=90^{\circ}-x$ 이고, 그에 따라 $\sin \left(90^{\circ}-x\right)$ 는 $\cos x$ 의 값과 동일합니다. 이는 직각삼각형의 보완 관계 때문에 발생합니다.

    그렇게 해서:

    - $\sin \left(90^{\circ}-x\right)=\cos x$
    
    이 관계는 흔히 직각삼각형을 다룰 때 자주 사용됩니다. 쉅게 생각하면 같은 삼각형의 다른 각에 대응하는 $\sin$ 과 $\cos$ 값이 서로 대응한다는 것을 의미합니다.
 
\end{GPTA}

