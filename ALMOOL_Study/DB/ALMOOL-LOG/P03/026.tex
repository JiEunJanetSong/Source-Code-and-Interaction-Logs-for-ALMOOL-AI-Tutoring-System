\begin{GPTQ}
    이 문제를 풀기 위해 제일 처음으로 해야하는게 뭐야?    
\end{GPTQ}

\begin{GPTA}
이 문제를 풀기 위한 첫 번째 단계는 주어진 정보에 의해 두 개의 유사한 삼각형을 설정하는 것입 니다. 이는 원뿔의 절단 부분과 전체 원뿔에서 유사성을 통해 도출할 수 있습니다. 

아래의 설명을 참고하세요:

    1. 유사한 삼각형 설정: 전체 원뿔의 높이, 절단 원뿔의 높이(문제에서는 12 cm ), 그리고 상단과 하단 원의 반지름 비율을 사용해 유사한 삼각형의 비율을 설정합니다.

    2. 반지름 비율 이혜 하기: 문제에 따르면 하단 원의 직경이 상단 원의 직경의 세 배이므로, 하단 반지름 $\left(r_2\right)$ 은 상단 반지름 $\left(r_1\right)$ 의 세 배입니다. 즉, $r_2=3 r_1$ 입니다.
    
    이 정보를 그림 또는 방정식으로 나타내면 문제 해결의 기반을 마련할 수 있습니다. 이를 퉁해 유사 삼각형 공식을 사용하여 전체 원뿔의 높이를 찾아갈 것입니다.

    - 주어진 반지름 비율과 높이로부터 수립될 수 있는 비례 공식:
    
    $$
    \frac{\text { 전체 원뿔의 높이 }}{\text { 하단 반지름 }}=\frac{\text { 상단의 원뿔 높이 }}{\text { 상단 반지름 }}
    $$
  
\end{GPTA}

