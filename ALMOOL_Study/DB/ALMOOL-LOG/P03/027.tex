\begin{GPTQ}
    반지름의 비율을 이용해서 어떻게 구하는데   
\end{GPTQ}

\begin{GPTA}
반지름의 비율을 이용해서 전체 원뿔의 높이를 찾는 방법은 유사 삼각형의 원리를 사용하는 것입니다. 여기서 사용되는 개념은 두 원뿔, 즉 절단된 원뿔과 원래 원뿔이 동일한 비율 관계를 가진다는 것입니다.

    1. 반지름 설정:

    - 상단 원의 반지름 $r_1$.

    - 하단 원의 반지름 $r_2=3 r_1$ (문제에서 주어진 조건에 따라).

    2. 아래의 비례식을 사용하여 높이를 구할 수 있습니다:
    
    원 뿔의 높이 $h$ 와 꼲어진 부분의 높이 $x$ 는 다음과 같은 비례식을 만족합니다 (유사 삼각형의 특성에서):
    
    $$
    \frac{h}{3 r_1}=\frac{x}{r_1}
    $$  
    
    3. 비례식을 통하여 $\mathbf{x}$ (상단 원뿔의 높이)를 찾아내기:

$$
\begin{gathered}
h-12=\frac{x}{r_1} \times 3 r_1=3 x \\
\Rightarrow h-12=3 x
\end{gathered}
$$

4. $h$ 의 식 정리:

이 때, $x=\frac{h-12}{3}$.

절단된 원뿔의 높이는 12 cm 라고 주어졌으므로:

$$
h=x+12
$$

5. 높이 $h$ 를 찾기 위한 식 세우기: 위 식을 퉁해:

$$
h=\frac{h-12}{3}+12
$$

6. 정리하여 $h$ 를 구하기:

$$
\begin{gathered}
3 h=h-12+36 \\
3 h=h+24 \\
3 h-h=24 \\
2 h=24 \\
h=12 \times 2 \\
h=18 \mathrm{~cm}
\end{gathered}
$$


따라서, 전체 원뿔의 높이는 18 cm 입니다. 이처럼 비율과 유사 삼각형을 이용해 계산할 수 있습니다.
\end{GPTA}

