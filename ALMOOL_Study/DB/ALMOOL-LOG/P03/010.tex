\begin{GPTQ}
why did you choose 1990 and 2000 instead of other years    
\end{GPTQ}

\begin{GPTA}
    Choosing the years 1990 and 2000 to calculate the slope of the best-fit line is based on the following con siderations:

    1. Clear Data Points: These years have clear and identifiable data points, which makes it easier to re ad the approximate number of carp from the graph.

    2. Span: The span from 1990 to 2000 covers a decade ( 10 years), which allows for a straightforward calculation because the time interval simplifies the division in the slope calculation.

    3. Trend Visibility: These points provide a good representation of the overall trend observed in the $g$ raph, as they span the entire time period shown and reflect the general increase in carp numbers.
    
    4. Accuracy: Selecting points at the beginning and end of the timeline can help capture the overall tr end more effectively, reducing the impact of potential fluctuations or anomalies in between.
    
    While you could use other years, using endpoints from the dataset's time range often gives the most relia ble estimate for a trend or pattern. This approach simplifies calculations and provides a quick insight into the rate of change over the entire period.
 
\end{GPTA}

