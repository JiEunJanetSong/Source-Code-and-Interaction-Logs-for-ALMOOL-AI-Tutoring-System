\begin{GPTQ}
    is there a formula to solve this problem   
\end{GPTQ}

\begin{GPTA}
    Yes, there is a formula for compound interest that can be used to solve this problem. The formula for the future value $A$ of an investment or loan is given by:

    $$
    A=P\left(1+\frac{r}{n}\right)^{n t}
    $$
    
    where:

    - $A$ is the amount of money accumulated after n years, including interest.

    - $P$ is the principal amount (initial amount of money).

    - $r$ is the annual nominal interest rate (as a decimal).

    - $n$ is the number of times that interest is compounded per year.

    - $t$ is the time in years.
    
    In this particular problem:

    - The principal amount, $P$, is $x$.

    - The annual interest rate, $r$, is 0.03 (since $3 \%$ is equivalent to 0.03 ).

    - Since the interest is compounded annually, $n$ is 1.

    - The time period, $t$, is from January 1, 2015, to January 1, 2017, which is 2 years.
    
    Substituting these values into the formula gives:
    
    $$
    A=x\left(1+\frac{0.03}{1}\right)^{1 \cdot 2}=x(1.03)^2=x \cdot 1.0609
    $$
    
    
    So, the amount in the account on January 1, 2017, will be $1.0609 x$. This matches the solution choice (C)
    
\end{GPTA}

