\begin{GPTQ}
    이거 말고 피타고라스 쓰면 안 돼?   
\end{GPTQ}

\begin{GPTA}
물론 가능합니다! $\sin p=\frac{5}{13}$ 를 사용하여 직각삼각형에서 피타고라스 정리를 통해 $\cos p$ 를 구할 수 있습니다.

먼저, $\sin p=\frac{5}{13}$ 은 직각삼각형에서 반대변의 길이가 5, 빗변의 길이가 13 이라는 것을 의미합니다. 펴타고라스 정리에 따르면:

$$
(\text { 빗변 })^2=(\text { 반대변 })^2+(\text { 인접변 })^2
$$


즉, 빗변인 13, 반대변이 5 라면 인접변의 길이는 $x$ 로 표현할 수 있고 식은 다음과 같이 됩니다:

$$
13^2=5^2+x^2
$$


이 식을 풀면:

$$
\begin{gathered}
169=25+x^2 \\
x^2=169-25=144 \\
x=\sqrt{144}=12
\end{gathered}
$$


따라서, $\cos p=\frac{x}{\text { 빗변}}=\frac{12}{13}$ 입니다.

이를 통해 피타고라스 정리를 사용하여 $\cos p$ 를 구할 수 있습니다. 이렇게 하면 삼각형의 변의 길이를 직접적으로 이해할 수 있는 장점이 있습니다.
\end{GPTA}

