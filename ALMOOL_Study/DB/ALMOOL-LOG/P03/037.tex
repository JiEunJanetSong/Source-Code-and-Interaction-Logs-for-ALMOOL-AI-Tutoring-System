\begin{GPTQ}
    그럼 이 문제를 풀기 위해 풜요한 첫번째 단계가 뮈야   
\end{GPTQ}

\begin{GPTA}
이 문제를 풀기 위한 첫 번째 단계는 삼각형 $\triangle M N Q$ 의 내각의 합을 사용하는 것입니다. 주어진 $\angle M Q N=72^{\circ}$ 와 직각 $\angle M N Q=90^{\circ}$ 를 통해 $\angle M N Q$ 를 구할 수 있습니다.
    
    $$
    \angle M N Q+\angle M Q N+\angle N Q M=180^{\circ}
    $$
    
    
    여기서,

    - $\angle M Q N=72^{\circ}$

    - $\angle M N Q=90^{\circ}$ (직각)
    
    따라서, 먼저 $\angle  Q M N$ 를 구해야 합니다:
    
    $$
    \angle M N Q+72^{\circ}+90^{\circ}=180^{\circ}
    $$
    
    
    이를 계산하면,
    
    $$
    \angle Q M N =180^{\circ}-162^{\circ}=18^{\circ}
    $$
    
    
    위 계산을 퉁해 $\angle Q M N$ 의 크기를 알 수 있습니다. 이 과정에서 얻은 $\angle Q M N $ 는 이후에 다른 내각 이나 변을 구하는 데 필요한 기본 정보가 됩니다. 추가적인 설명이나 다른 단계에 대한 안내가 필요하면 언제든지 말씀해 주세요!
    
\end{GPTA}

