\begin{GPTQ}
    a simple way 
\end{GPTQ}

\begin{GPTA}
Certainly! Let's simplify the approach:

1. Understand the Expressions: We have four expressions:

- (A) $y=n x+3$ : A linear expression.

- (B) $y=-n x+3$ : A linear expression with a negative slope.

- (C) $y=x^n+3:$ An exponential expression.

- (D) $y=x^{-n}+3$ : A reciprocal expression.

2. Identify Behavior:

- Linear Expressions (A and B): The change in $y$ depends linearly on $x$, meaning they won't increase as rapidly for large $x$.

- Exponential Expression (C): The term $x^n$ increases very rapidly as $x$ increases, especially when $n>1$. This makes (C) grow much faster than any linear expression.

- Reciprocal Expression (D): The term $x^{-n}$ becomes very small as $x$ increases, leading to mi nimal change in $y$.

3. Conclusion: Expression (C) $y=x^n+3$ exhibits the fastest growth as $x$ increases because $x^n$ rapi dly increases for large $x$. Therefore, answer lextbf $\{(\mathrm{C})\}$ is the correct choice for the greatest over all increase in $y$.
  
This simplified insight helps you quickly determine that an exponential expression will produce a much greater increase compared to linear or reciprocal ones as $x$ grows.

\end{GPTA}

