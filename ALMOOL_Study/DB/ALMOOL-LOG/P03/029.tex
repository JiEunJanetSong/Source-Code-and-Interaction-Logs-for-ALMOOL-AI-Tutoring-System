\begin{GPTQ}
    왜 이런 공식이 나오는거야      
\end{GPTQ}

\begin{GPTA}
    원뿔의 부피 공식이 어떻게 도출되는지 설명드리겼습니다. 이는 기하학적 고려 와 적분을 퉁해 얻 을 수 있습니다.
    원뿔의 부피는 원기둥의 부피의 $\frac{1}{3}$ 입니다. 이 사실은 원뽈을 원기둥에 확 차게 접합하여 직관적으 로 나타납니다. 이제 그 과정을 설명하겠습니다:

    1. 원기둥의 부펴: 반지름이 $r$ 이고 높이가 $h$ 인 원기둥의 부피는 밑면의 넓이에 높이를 곱한 값입니다. 따라서 원기둥의 부피 $V_c$ 는 다음과 같습니다:
    
    $$
    V_c=\pi r^2 h
    $$
    
    2. 원풀의 부패: 원뿔을 원기둥과 비교해보면 같은 밑면과 같은 높이를 가진 원기둥 부피의 $\frac{1}{3}$ 입니다. 이를 수학적으로 증명하기 위해 미적분학의 기법을 사용합니다. 이는 다음과 같은 이유에서입니다:

    - 원뿔을 원기둥으로 수많은 가상의 수평 원푠으로 나눈다고 가정합니다.

    - 원기둥과 달리 원뿔에서 원판의 반지름은 높이에 따라 선형적으로 감소합니다. 적분 과정을 통해 계산하면 원뿔의 각 원퐌 부피는 동일한 높이에서 원기둥 전체 부피의 $\frac{1}{3}$ 이라는 결과를 얻습니다.

    3. 위의 과정율 통해 원뿔의 부피는 다음과 같이 도출됩니다:
    
    $$
    V=\frac{1}{3} \pi r^2 h
    $$
    
    
    이렇게 해서 원뿔의 부피 공식이 유도됩니다.
    
\end{GPTA}

