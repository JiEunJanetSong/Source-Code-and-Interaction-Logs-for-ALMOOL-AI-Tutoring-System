\begin{GPTQ}
    그럼 높이는 어떻게 구해?     
\end{GPTQ}

\begin{GPTA}
    원뿔의 높이는 원뿔의 꼭대기(정점)에서 밑면 중심까지 수직으로 내려오는 거리입니다. 높이를 구하기 위해서는 일반적으로 직접 측정 방식이나 다른 주어진 요소들을 이용할 수 있습니다. 다음은 원뿔 높이를 구하는 몇 가지 방법입니다:

    1. 직접 측정: 만약 원뿔을 실제로 가지고 있다면, 자 등으로 직접 높이를 측정할 수 있습니다.

    2. 기하학적인 방법:

    - 원뿔의 반지름과 부피가 주어진 상황에서 높이를 구할 수 있습니다. 부피 공식에서 식을 변형하면 높이를 다음과 같이 구할 수 있습니다:
    
    $$
    V=\frac{1}{3} \pi r^2 h
    $$
    
    
    $$
    h=\frac{3 V}{\pi r^2}
    $$
    
    3. 펴타고라스 정리 사용 (경우에 따라):

    - 만약 원뿔의 측면(모선)의 길이 $l$ 이 주어지고 반지름 $r$ 이 주어진 경우, 펴타고라스 정리를 사용하여 높이를 구할 수 있습니다. 원뽈 높이 $h$, 반지름 $r$, 그리고 모선 $l$ 의 관계는 다음과 같습니다:
    
    $$
    \begin{aligned}
    & l^2=r^2+h^2 \\
    & h=\sqrt{l^2-r^2}
    \end{aligned}
    $$
    
    
    이러한 방법을 퉁해 원뿔의 높이를 구할 수 있습니다.   
\end{GPTA}

