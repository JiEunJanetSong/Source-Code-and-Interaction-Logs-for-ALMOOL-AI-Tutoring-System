\begin{GPTQ}
    보각이 뭔데   
\end{GPTQ}

\begin{GPTA}
각의 ``보각"은 두 각의 합이 $180^{\circ}$ 가 되는 관계를 말합니다. 즉, 만약 두 각이 보각 관계에 있다면, 한 각을 알고 있을 때 다른 각은 $180^{\circ}$ 에서 그 각의 크기를 뼈서 구할 수 있습니다.

    예를 들어, 만약 각 $\alpha$ 가 있다면 그의 보각 $\beta$ 는 다음과 같이 구할 수 있습니다:
    
    $$
    \beta=180^{\circ}-\alpha
    $$
    
    
    이러한 보각 개념은 삼각형에서 나머지 각을 구할 때 유용합니다. 직각 삼각형의 경우 직각이 $90^{\circ}$ 이므로 다른 두 각의 합도 $90^{\circ}$ 가 되어야 합니다. 직각이 아닌 하나의 각을 보각으로 쉅게 구할 수 있습니다.

    궁금한 점이 더 있거나 추가적인 설명이 풜요하면 말씀혜 주세요!
    \end{GPTA}

