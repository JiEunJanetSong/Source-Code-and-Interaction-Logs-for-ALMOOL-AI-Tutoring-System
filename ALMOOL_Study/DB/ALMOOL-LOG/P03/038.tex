\begin{GPTQ}
    그럼 다음 단계에서 해야하는건 뭐야     
\end{GPTQ}

\begin{GPTA}
    다음 단계에서는 삼각형 $\triangle N M R$ 의 내각을 분석하여 각 $\angle Q M R$ 의 크기를 구하는 것입니다. 앞서 구한 $\angle N M Q=18^{\circ}$ 는 $\triangle N M R$ 의 내각 중 하나입니다. $\triangle N M R$ 내에는 또한 $\angle Q M R$ 이 있습니다. 이 각도를 구하기 위해 삼각함수와 주어진 변의 길이를 사용할 것입니다.

    단계 2: 삼각 함수 사용하기

    1. $\sin \theta=\frac{\text {높이}}{\text{빗변}}$ 공식을 사용하여 $\angle N R M$, 즉 $\theta$를 구합니다. 주어진 값에 따라:
    
    $$
    \sin \theta=\frac{M N}{M R}=\frac{15}{10 \sqrt{3}}
    $$
    
    2. 따라서 $\theta$ 를 구하면:
    
    $$
    \theta=\sin ^{-1}\left(\frac{15}{10 \sqrt{3}}\right)
    $$
    
    3. 계산하면 $\theta=60^{\circ}$ 입니다.
    
    단계 3: 각 $x$ 구하기

    여기서 각 $x$ 는 다음처럼 구할 수 있습니다:
    
    $$
    x+18^{\circ}+60^{\circ}+90^{\circ}=180^{\circ}
    $$
    
    
    각도 관계식에서 각 $x$ 를 퉁해:
    
    $$
    x+168^{\circ}=180^{\circ}
    $$
    
    
    따라서,
    
    $$
    x=180^{\circ}-168^{\circ}=12^{\circ}
    $$
    
    
    위 단계를 통해 각 $x$ 의 크기를 구할 수 있습니다. 추가적으로 궁금한 점이나 더 설명이 필요한 부 분이 있으면 언제든 말씀해 주세요!
     
\end{GPTA}

