\begin{GPTQ}
더 간단한 방법은?
     
\end{GPTQ}

\begin{GPTA}
더 간단한 방법으로는 주어진 두 방정식의 관계를 이용하는 것입니다. 

두 방정식은:

    1. $2 x+3 y=7$

    2. $10 x+15 y=35$  
    
    위 두 번째 방정식을 보면, 이것은 첫 번째 방정식에 5 를 곱한 것임을 알 수 있습니다. 
    
    즉, 두 방정식은 본질적으로 같은 직선을 나타냅니다.

    이제, 직선의 방정식에서 점을 찾으려면 두 번째 방정식을 굳이 사용하지 않아도 됩니다. 그래서 첫 번째 방정식만 생각하면 툅니다.

    일반적인 형태의 점을 $x=r$ 로 두고 $y$ 를 구해본다면:
    
    $$
    2 x+3 y=7 \Longrightarrow 3 y=7-2 x \Longrightarrow y=\frac{7-2 x}{3}
    $$
    
    
    여기서 $x=r$ 라고 두면,
    
    $$
    y=\frac{7-2 r}{3}
    $$
    
    
    따라서 선택지를 보면서, 선택지 $\mathrm{C}:\left(r, \frac{2 r}{3}+\frac{7}{3}\right)$ 인 경우에만 $y$ 값이 $\frac{7-2 r}{3}$ 형태와 일치하고 있을 가능성을 직관적으로 배제할 수 있습니다.

    하지만, 만약 선택지의 구조가 바로 위의 형태와 맞지 않다고 보여 진다면, 선택지 B가 본질적으로 다른 구조로 나타낼 수 있는지 확인을 통해 접근하면 더욱 간단할 수 있습니다. 선택지들은 사실 $x, y$ 를 각각의 변수 형식으로 적용해 놓은 것이기 때문에, 수학적 구조만을 붕해 땅르게 푸는 전략도 함께 검토할 수 있습니다.

    이러한 간단한 관찰 및 대칭적 구조의 인식을 통해, 주어진 고정된 대입 식을 활용하여 크게 노력을 들이지 않고 문제를 풀 수 있습니다.
    \end{GPTA}

