\begin{GPTQ}
    왜 $\cos (60-\mathrm{x})$ 야    
\end{GPTQ}

\begin{GPTA}
이 부분을 명확하계 설명해드리겠습니다. 처음 주어진 식을 변형하는 과정에서, 다음과 같은 관계를 사용혔습니다:
    
    성질 활용하기

    주어진 식은 $\cos \left(2 x+15^{\circ}\right)=\sin \left(x-30^{\circ}\right)$ 였습니다.

    1. $\sin$ 을 $\cos$ 로 변환:
    
    우리는 $\sin \left(x-30^{\circ}\right)$ 를 $\cos$ 으로 변환하려고 했습니다. 이 때 다음 삼각함수의 성질을 사용했습니다:
    
    $$
    \sin (\theta)=\cos \left(90^{\circ}-\theta\right)
    $$
    
    
    이 성질을 사용하면 다음과 같이 변환할 수 있습니다:
    
    $$
    \sin \left(x-30^{\circ}\right)=\cos \left(90^{\circ}-\left(x-30^{\circ}\right)\right)
    $$
    
    2. 각의 단순화:

    $\cos \left(90^{\circ}-\left(x-30^{\circ}\right)\right)$ 를 계산해 보면,
    
    $$
    \cos \left(90^{\circ}-x+30^{\circ}\right)=\cos \left(60^{\circ}-x\right)
    $$
    
    
    여기에서 $90^{\circ}-\left(x-30^{\circ}\right)$ 를 간단히 풀면 $90^{\circ}-x+30^{\circ}$ 가 되고, $90^{\circ}+30^{\circ}$ 는 $120^{\circ}$ 이므로 결국 $120^{\circ}-x$ 로 나타낼 수 있습니다.
    
    아까 설명이 틀렸던 것 같습니다. 실제로는 $\cos \left(60^{\circ}-x\right)$ 가 아니라 $\cos \left(120^{\circ}-x\right)$ 가 되어야 합니다.

    이해에 도움이 되셨기를 바랍니다. 더 궁금한 부분이 있다면 언제든지 말씀해 주세요!
        
\end{GPTA}

