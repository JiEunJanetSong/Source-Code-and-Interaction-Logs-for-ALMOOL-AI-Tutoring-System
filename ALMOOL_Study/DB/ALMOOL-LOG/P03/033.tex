\begin{GPTQ}
    비율과 유사 삼각형이 뭔데    
\end{GPTQ}

\begin{GPTA}
    비율과 유사 삼각형은 기하학의 근본적인 개념 중 하나로, 도형을 분석하고 문제를 해결할 때 자주 사용됩니다. 
    아래에 각각에 대한 설명을 드리겠습니다.

    비율 (Ratio)

    - 비율은 두 수나 양을 비교할 때 사용합니다. 비율을 통해 두 양이 서로 어떻게 상관되는지를 알 수 있습니다.

    - 예를 들어, 반지름이 $r_1$ 인 원과 반지름이 $r_2$ 인 원이 있다면, 이들의 비율은 $\frac{r_2}{r_1}$ 로 표현할 수 있습니다.

    - 문제에서 하단의 반지름이 상단의 세 배라고 하였으므로, 비율은 $\frac{3 r_1}{r_1}=3$ 입니다.
    
    유사 삼각형 (Similar Triangles)

    - 유사 삼각형은 모양은 같지만 크기가 서로 다른 삼각형입니다. 두 삼각형이 유사하려면 각각의 대응하는 각들이 모두 같고, 대응하는 변의 길이의 비율이 같아야 합니다.

    - 유사 삼각형의 속성을 이용하면, 한 삼각형의 변의 길이를 다른 삼각형의 변의 길이로 쉽게 구할 수 있습니다. 즉, 비율이 동일하다면, 삼각형을 구성하는 변들은 일정한 비율로 관련되어 있습니다.
    
    예시

    - 문제에서 다룬 절단 원뿔의 경우, 전체 원뿔과 절단 없이 남은 상단의 작은 원뿔은 유사 삼 각형입니다.

    - 반복적으로 같은 각을 이루므로 대응하는 변의 비율이 같습니다. 이를 활용하여, 윈래의 원 뿔의 높이와 절단된 원뿔의 높이 사이의 관계를 유사 삼각형의 비율로 풀어낼 수 있었습니다.
    
    이러한 비윤과 유사 삼각형의 개념은 기하학뿐만 아니라 다양한 수학 문제를 해결하는 데 매우 유용한 도구입니다.   
\end{GPTA}



