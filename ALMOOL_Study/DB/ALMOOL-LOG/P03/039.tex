\begin{GPTQ}
    그럼 이 방법 말고 쓸 수 있는 또 다른 방법도 있어?     
\end{GPTQ}

\begin{GPTA}
    물론, 이 문제를 햬결하는 또 다른 방법이 있습니다. 
    
    $\triangle N M R$ 의 내각과 삼각 함수를 사용하지 않 고, 주어진 각도를 통해 직접적으로 문제를 접근하는 방법을 소개하겠습니다.

    대체 방법: 보각 및 여각 활용

    $\triangle M N Q$ 에서 각도 활용

    1. $\triangle M N Q$ 는 $\angle M N Q=18^{\circ}$ 와 $\angle M Q N=72^{\circ}$ 로 구성되어 있습니다. 
    
    $\angle M N R=90^{\circ}$ 는 주어진 직각입니다.

    $\triangle N M R$ 의 이해
    
    $\triangle N M R$ 은 직각 삼각형이므로, $\angle N M R$ 은 $\angle N M R$ 과 $\angle Q M R$ 로 나뉘어집니다.

    1. 주어진 변의 길이를 통해 직접 각도를 계산하였습니다만, 삼각 함수 대신 내각의 합과 $\triangle N M Q$ 에서 차례로 구해도 되었습니다. 이 방법은 각도 사이의 관계를 통해 더 간단하게 다른 내각을 도출할 수 있습니다.

    2. $\angle M N R=90^{\circ}$, 즉 빗변과 밑변을 포함한 각도로 문제를 다시 분석해야 했습니다.
    
    결국, 이러한 수학적 퐌계를 바탕으로 여전히 같은 결과를 얻으며, 문제가 덜 복잡하게 보일 수 있습니다. 특수 삼각 합수의 관계식을 활용하지 않는 경우 내각의 합과 보각의 개념을 최대한 활용 하여 쉽게 문제를 해결할 수 있습니다. 기타 궁금한 사항이나 설명이 펼요하면 언제든 말씀해 주세요!
    \end{GPTA}

