\begin{GPTQ}
    그러면 이 문제를 풀기 위한 첫번째 단계는 뭐야      
\end{GPTQ}

\begin{GPTA}
    이 문제를 풀기 위한 첫 번째 단계는 주어진 식을 변형하여 동일한 삼각 함수로 만드는 것입니다. 이렇게 하면 방정식을 쉅게 풀 수 있습니다. 구체적으로 아래와 같은 방식으로 시작할 수 있습니다:

    첫 번째 단계

    1. 동일하계 변형하기:

    - 주어진 문제는 $\cos \left(2 x+15^{\circ}\right)=\sin \left(x-30^{\circ}\right)$ 입니다.

    - 삼각 함수의 보수성(identity)을 사용하여 $\sin \left(x-30^{\circ}\right)$ 를 $\cos$ 형태로 변형할 수 있습니다.

    2. 식의 정리:

    - $\sin \left(x-30^{\circ}\right)$ 를 $\cos$ 에 대한 표현으로 나타내기 위해 다음 관계를 사용합니다:
    
    $$
    \sin \left(x-30^{\circ}\right)=\cos \left(90^{\circ}-\left(x-30^{\circ}\right)\right)
    $$
    
    - 이 관계를 그대로 대입하면:
    
    $$
    \cos \left(2 x+15^{\circ}\right)=\cos \left(90^{\circ}-x+30^{\circ}\right)
    $$
    
    
    이렇게 식을 정리하면 두 $\cos$ 식이 동일한 것을 알 수 있습니다. 다음 단계에서는 이 두 $\cos$ 식을 비교하여 $x$ 의 값을 찾을 수 있습니다.

    이 단계가 이해되셨으면, 계속해서 방정식을 풀어나가시면 됩니다. 추가적인 설명이 필요하면 언제든지 말씀해 주세요!
            
\end{GPTA}

