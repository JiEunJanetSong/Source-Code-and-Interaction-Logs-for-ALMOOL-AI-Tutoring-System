\begin{GPTQ}
    다음 단계   
\end{GPTQ}

\begin{GPTA}
먼저 $\cos p$ 를 구한 후 다음 단계는 $\tan p$ 를 계산하는 것입니다. $\tan p$ 는 $\sin p$ 와 $\cos p$ 를 통해 쉽게 계산할 수 있습니다.

    1. $\tan p$ 계산:

    - $\tan p$ 는 $\sin p$ 를 $\cos p$ 로 나눈 값입니다. 즉,
    
    $$
    \tan p=\frac{\sin p}{\cos p}=\frac{\dfrac{5}{13}}{\dfrac{12}{13}}=\frac{5}{12}
    $$
    
    2. 최종 계산: $\cos p-\tan p$ :

    - 이제 $\cos p$ 와 $\tan p$ 의 값을 가지고 $\cos p-\tan p$ 를 계산할 수 있습니다.
    
    $$
    \cos p-\tan p=\frac{12}{13}-\frac{5}{12}
    $$
    
    
    이 식을 계산하기 위해 공통 분모를 찾아야 합니다:
    - 공통 분모는 156 (이는 $13 \times 12$)입니다.
       
    따라서,

    $$
    \cos p-\tan p=\frac{12 \times 12}{156}-\frac{5 \times 13}{156}=\frac{144}{156}-\frac{65}{156}=\frac{79}{156}.
    $$
    
    
    그래서 최종적으로, $\cos p-\tan p=\frac{79}{156}$ 입니다.
    이로써 문제를 해결할 수 있습니다.

    \end{GPTA}

