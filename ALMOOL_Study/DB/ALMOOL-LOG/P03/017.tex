\begin{GPTQ}
    more specifically 
\end{GPTQ}

\begin{GPTA}
Certainly! Let's take a closer look at each step:
    Step 1: Choose Specific Values
    - We will use $x=2, x=100$, and $n=3$. These are just example values to help compare how each expression behaves over this range.
    
    Step 2: Evaluate Each Expression
    
    Expression (A): $y=n x+3$
    - Substitute $n=3$ and $x=2$ :
    
    $$
    y=3 \times 2+3=6+3=9
    $$
    
    - Substitute $n=3$ and $x=100$ :
    
    $$
    y=3 \times 100+3=300+3=303
    $$
    
    - Change in $y: 303-9=294$
    
    Expression (B): $y=-n x+3$
    - Substitute $n=3$ and $x=2$ :
    
    $$
    y=-3 \times 2+3=-6+3=-3
    $$
    
    - Substitute $n=3$ and $x=100$;
    
    $$
    y=-3 \times 100+3=-300+3=-297
    $$
    
    - Change in $y$ : $-297-(-3)=-294$ 

    Expression (C): $y=x^n+3$
- Substitute $n=3$ and $x=2$ :

$$
y=2^3+3=8+3=11
$$

- Substitute $n=3$ and $x=100$ :

$$
y=100^3+3=1,000,000+3=1,000,003
$$

- Change in $y: 1,000,003-11=999,992$

Expression (D): $y=x^{-n}+3$
- Substitute $n=3$ and $x=2$ :

$$
y=2^{-3}+3=\frac{1}{8}+3=3.125
$$

- Substitute $n=3$ and $x=100$ :

$$
y=100^{-3}+3 \approx 0.000001+3=3.000001
$$

- Change in $y: 3.000001-3.125=-0.124999$

Step 3: Determine the Greatest Increase
Upon calculation, expression (C) $y=x^n+3$ shows the greatest increase from 11 to $1,000,003$, making i t the correct answer, as it shows the largest numerical growth with changing $x$.
\end{GPTA}

   


