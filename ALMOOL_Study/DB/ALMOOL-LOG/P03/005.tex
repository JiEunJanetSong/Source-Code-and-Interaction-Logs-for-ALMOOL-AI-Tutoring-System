\begin{GPTQ}
    대입하지 않고 풀 수 있는 다른 방법은 없어?   
\end{GPTQ}

\begin{GPTA}
네, 대입 없이 문제를 푸는 접근 방식에 대해 설명드리젰습니다.

    이 문제에서 주어진 두 식은 사실 동일한 직선을 나타내고 있습니다. 이를 발견할 수 있다면,
    $2 x+3 y=7$ 의 헤가 $10 x+15 y=35$ 의 해가 된다는 것을 바로 알 수 있습니다. 따라서 두 식 가운 데 하나의 식만을 검토해도 됩니다.

    이 문제에서는 선택지를 분석하여 접근할 수 있습니다. 각 선백지의 형태에서 $x$ 와 $y$ 가 서로를 어떻게 포함하는지 살펴보면, 일종의 패턴을 발견할 수 있습니다.

    1. 선택지 $\mathrm{A}:\left(\frac{r}{5}+7,-\frac{r}{5}+35\right)$

    2. 선택지 B: $\left(-\frac{3 r}{2}+\frac{7}{2}, r\right)$

    3. 선택지 $\mathrm{C}:\left(r, \frac{2}{3}+\frac{7}{3}\right)$

    4. 선백지 $\mathrm{D}:\left(r,-\frac{3 r}{2}+\frac{7}{2}\right)$
    
    네 가지 선택지를 살펴보면, 선택지 B 와 D 가 $y$ 가 $r$ 에 대해 직접적으로 표현되는 방식이기 때문에, $x$ 의 값이 $y$ 와 연관되어 있는지, 그리고 두 식이 비슷한 백터 공간 구조를 가질지를 유추할 수 있습니다.

    이 점에서 선택지 B 는 방정식의 계수의 조합과 잘 맞아뗠어지며, 연립 방정식이 보는 서로의 구 조적 특성을 잘 반영합니다. 
    Coefficient term 구조를 비교해보면 선택지 B가 부합 하는 것을 직관적으로 알 수 있습니다.

    이러한 이유로, $x$ 와 $y$ 가 방정식의 구조와 잘 맞아떨어지는 선벽지를 전략적으로 선백하는 방법이 대안이 될 수 있습니다.
   
\end{GPTA}

