\begin{GPTQ}
    what should I first do to solve this problem?   
\end{GPTQ}

\begin{GPTA}
To solve the problem of finding the approximate slope of the best-fit line for the data in the graph, you should begin by identifying key points from the graph that represents the number of Asian carp over the years. Here are the steps you should follow:
    
    Step 1: Identify Key Points

    \begin{itemize}
    \item Examine the graph and find two points that lie on or close to the best-fit line.

    \item Specifically, note the year and the corresponding number of carp for these points.

    - For example, based on the provided information, in:
    
    - 1990, the number of carp is approximately 0.

    - 2000, the number of carp is approximately 38.  
    \end{itemize}    

    Step 2: Calculate the Slope

- Use the slope formula to calculate the change in carp numbers over the change in years:

$$
\text { slope }=\frac{\text { change in carp }}{\text { change in years }}=\frac{\text { carp in } 2000-\text { carp in } 1990}{2000-1990}
$$

- Substitute the values you identified from the graph:

$$
\text { slope }=\frac{38-0}{2000-1990}=\frac{38}{10}=3.8
$$


Step 3: Select the Closest Answer

- Compare your calculated slope with the provided answer choices.

- Choose the value closest to your calculation. In this case, the closest value to 3.8 is 3.3.

By following these steps, you can determine that the correct answer is A) $\mathbf{3 . 3}$.

\end{GPTA}

