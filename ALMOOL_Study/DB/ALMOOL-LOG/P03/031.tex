\begin{GPTQ}
    그럼 저 문제에서의 높이는 어떻게 구해  
\end{GPTQ}

\begin{GPTA}
문제에서 실제로 언급하고 있는 것은 전체 원뿔이 아니라 원뿔의 일부인 원뽈대입니다. 하지만 문제에서 요구하는 것은 전채 원뿔의 부피이며, 이를 통해 원뿔대의 높이 이외에 추가로 원뿔의 총 높이를 구해야 합니다.

    주어진 조건을 통해 원뿔의 전쳬 높이를 구할 수 있습니다:

    - 원뿔대의 높이는 $12, \mathrm{~cm}$ 로 주어졌습니다.

    - 원뿔대의 아래쪽 원의 지름은 위쪽 원의 세 배입니다.

    - 우리는 파이 값을 $\pi=\frac{22}{7}$ 으로 사용하여 반지름 $r$ 를 이미 구했습니다: $r=45.83, \mathrm{~cm}$.
    
    원뿔대의 부족한 부분(전체 원뿔의 나머지 부분)의 높이를 구하기 위해 비례 관계를 이용합니다. 원뿔 전체와 원뽈대의 구조가 상사 삼각형 관계라는 것을 알 수 있습니다:

    1. 원례 전제 원폴:

    - 아래쪽 부분의 반지름이 3 배 크다고 하였으므로, 원래 위쪽 원의 반지름을 $r$ 이라 하면 , 아래는 $3 r$ 가 뒵니다.

    2. 전체 원뿔의 높이:

    - 상사 삼각형 에서 높이의 비례는 반지름의 비례와 같습니다. 원래 전체 원뿔의 높이를 $H$ 라 하면, 아래쪽의 나머지 높이 $h^{\prime}$ 는:
    
    $$
    \frac{H}{H-h^{\prime}}=\frac{3 r}{r}=3
    $$
    
    
    문제에서 주어진 원뿔대의 높이는 12 cm 이고, 이 높이는 $H-h^{\prime}$ 입니다. 따라서 이 방정식을 새로 정리하면:
    
    $$
    H=3(H-12)
    $$
    
    - 이를 계산하면 $H=3 H-36$, 즉 $2 H=36$ 이 뒵니다.
    
    $$
    H=18
    $$
    
    
    따라서 전체 원뿔의 높이는 $18 \mathrm{~cm}$ 입니다.   
\end{GPTA}

