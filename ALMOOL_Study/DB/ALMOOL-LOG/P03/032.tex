\begin{GPTQ}
    그럼 이제 문제를 풀려면 숫자를 다 대입하면 되는건가?    
\end{GPTQ}

\begin{GPTA}
네, 맞습니다! 이제 문제의 주어진 조건을 이용하여 원래의 원뿔에 대해 부피 공식을 사용하여 해 를 구할 수 있습니다. 우리가 확인한 것처럼 전체 원뿔의 높이 $H=18, \mathrm{~cm}$ 이고, 원뿔 부패 공식 $V=\frac{1}{3} \pi r^2 h$ 에 값을 대입하면 됩니다.
    
    예제 문제 풀이:

    문제에서 전체 원뿔의 부피를 $39,600, \mathrm{~cm}^3$ 로 주었습니다. 따라서, 대입해 보면:
    
    $$
    39,600=\frac{1}{3} \times \frac{22}{7} \times r^2 \times 18
    $$
    
    1. 위 식을 정리합니다:
    
    $$
    39,600=\frac{396}{21} r^2
    $$
    
    2. $r^2$ 를 구하기 위해 양변을 $\frac{396}{21}$ 로 나누어 즙니다:
    
    $$
    r^2=39,600 \div \frac{396}{21}
    $$
    
    3. 이 값을 계산하면:
    
    $$
    r^2=2100
    $$
    
    4. 반지름 $r$ 을 구하기 위해 제곱근을 취합니다:
    
    $$
    r=\sqrt{2100}=10 \sqrt{21} \approx 45.83, \mathrm{~cm}
    $$
    
    
    위의 계산을 통해 원뽈의 밑면 반지름을 정확히 소수점 네 번째 자리까지 구했습니다.

    이제 모든 정보를 기반으로 올바른 결과를 확인했습니다! 퓰요하다면 각 단계를 검증하거나 추가적인 질문을 해주세요.   
\end{GPTA}

