\begin{GPTQ}
    how do we identify how many real solutions the system of equation have   
\end{GPTQ}

\begin{GPTA}
 To find the number of real solutions for a system of equations, particularly when dealing with two equati ons in two variables, you consider the intersection points of the graphs represented by these equations. Let's analyze the system of equations you've presented:
    
    $$
    y=2 x-5 y \quad=2 x^2+4
    $$
    
    Steps to Identify Real Solutions:

    1. Set the Equations Equal to Each Other:
    
    Since both equations equal $y$, you can set them equal to each other to find the $x$-coordinates of the intersection points:
    
    $$
    2 x-5=2 x^2+4
    $$
    
    2. Rearrange to Form a Quadratic Equation:
    
    Bring all terms to one side to form a standard quadratic equation:
    
    $$
    2 x^2-2 x+9=0
    $$
    
    3. Calculate the Discriminant:
    
    The discriminant $(\Delta)$ of a quadratic equation $a x^2+b x+c=0$ is given by:
    
    $$
    \Delta=b^2-4 a c
    $$
    
    
    For our equation $2 x^2-2 x+9=0$, we have $a=2, b=-2$, and $c=9$. Plug these values into th e discriminant:
    
    $$
    \Delta=(-2)^2-4 \cdot 2 \cdot 9=4-72=-68
    $$
    
    4. Analyze the Discriminant:

    - If $\Delta>0$, there are two distinct real solutions.

    - If $\Delta=0$, there is one real solution (a repeated root).

    - If $\Delta<0$, there are no real solutions, only complex solutions.
    
    Since $\Delta=-68<0$, the quadratic equation has no real solutions. This means the two curves, a line and a parabola, do not intersect at any real points.
    Therefore, the system of equations has zero real solutions.

\end{GPTA}

