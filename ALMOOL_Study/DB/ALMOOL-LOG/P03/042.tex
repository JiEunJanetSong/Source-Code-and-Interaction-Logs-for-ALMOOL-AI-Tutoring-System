\begin{GPTQ}
$\sin (90-x)=\cos x$ 왜 이렇게 되는거야    
\end{GPTQ}

\begin{GPTA}
이것은 삼각 함수의 동등성과 관련된 기본 개념 중 하나입니다. 이는 삼각형의 성질과 각의 관계
에서 비롯됩니다. 다음은 이 관계를 이해하는 데 도움이 될 설명입니다:   

개념 이해

1. 삼각형과 단위원(circle):

- 단위원을 고려할 때, 특정 각도 $x$ 에 대한 $\sin$ 와 $\cos$ 값은 단위원의 $x$-좌표와 $y$-좌표에 해당합니다.
- 각도 $90^{\circ}-x$ 는 원점 대칭에 의해 $\cos x$ 의 값과 동일한 길이를 가집니다.

2. 직각삼각형:

- 직각삼각형에서 $\sin \theta$ 는 대변의 길이를 경사면에 대 한 길이로 표현하고, $\cos \theta$ 는 인접 변의 길이를 경사면에 대한 길이로 표현합니다.

- 직각이 있는 삼각형에서, 각 $\theta$ 와 남은 각 $\left(90^{\circ}-\theta\right)$ 는 보완각입니다.

따라서 $\sin \left(90^{\circ}-x\right)$ 는 단위원의 성질과 직각 삼각형의 관계에 의해 $\cos x$ 와 동일하게 됩니다.
이를 수식으로 표현하면 다음과 같습니다:

$$
\sin \left(90^{\circ}-x\right)=\cos x
$$


이 관계는 삼각 함수의 중요한 특성 중 하나이며, 문제 해결 시 자주 사용뒵니다.
\end{GPTA}

