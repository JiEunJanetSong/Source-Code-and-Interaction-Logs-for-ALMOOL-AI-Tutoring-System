\begin{GPTQ}
    이 풀이에서 왜 C는 안돼?    
\end{GPTQ}

\begin{GPTA}
줗은 질문입니다! 선백지 C를 살펴보고 첫 번쨰 방정식 $2 x+3 y=7$ 을 만족하는지 확인해 보젰습니다.

    선택지 C 는 $\left(r, \frac{2 r}{3}+\frac{7}{3}\right)$ 입니다. 이 점이 주어진 방정식을 만족하는지 알아보도록 하젰습니다. 먼저, $x=r$ 이고 $y=\frac{2 r}{3}+\frac{7}{3}$ 을 첫 번째 방정식에 대입해 봅시다:
    
    $$
    2 x+3 y=7 \Longrightarrow 2 r+3\left(\frac{2 r}{3}+\frac{7}{3}\right)=7
    $$

    이 식을 계산해 보면,
    
    $$
    \begin{gathered}
    2 r+3 \times \frac{2 r}{3}+3 \times \frac{7}{3}=7 \\
    2 r+2 r+7=7 \\
    4 r+7=7
    \end{gathered}
    $$
    
    
    위 식에서 $4 r=0$ 이라는 결과를 얻게 되는데, 이는 $r=0$ 일 때만 가능하다는 것을 의미합니다. 따라서, 선택지 C 는 모든 실수 $r$ 에 대해 만족하지 않으므로 적합하지 않습니다.
    하지만 선택지 B 에서는, 대입 후 항상 성립하는 등식을 얻을 수 있으므로 모든 $r$ 에 대해 만족합니다. 이 점에서 선택지 B 가 적합한 선택지입니다.    
    
\end{GPTA}

