\begin{GPTQ}
    각 MRN을 구하는 과정을 좀더 세밀하게 알려쥐  
\end{GPTQ}

\begin{GPTA}
$\angle M R N$ 을 구하기 위한 과정을 세밀하게 설명드리겠습니다. 
 
주어진 정보:

- $|\overline{M N}|=15, \mathrm{~m}$

- $|\overline{M R}|=10 \sqrt{3}, \mathrm{~m}$

- 삼각형 $\triangle M N R$ 은 직각 삼각형으로 $\angle M N R=90^{\circ}$.

단계별 $\angle M R N$ 구하기:

1. 삼각비식 활용:

$$
\sin (\angle M R N)=\frac{\text { 대변 } \mathrm{MN}}{\text { 빗변 } \mathrm{MR}}
$$


주어진 $|\overline{M N}|=15, \mathrm{~m}$ 그리고 $|\overline{M R}|=10 \sqrt{3}, \mathrm{~m}$ 을 대입합니다:

$$
\sin (\angle M R N)=\frac{15}{10 \sqrt{3}}
$$

2. 비율 단순화:

분모의 $\sqrt{3}$ 율 제거하여 개선할 수 있습니다. 다음과 같이 나타낼 수 있습니다:

$$
\frac{15}{10 \sqrt{3}}=\frac{15 \cdot \sqrt{3}}{10 \cdot 3}=\frac{15 \sqrt{3}}{30}=\frac{\sqrt{3}}{2}
$$

3. $\angle M R N$ 촺기:
$\sin (\theta)=\frac{\sqrt{3}}{2}$ 인 각도 $\theta$ 는 흔히 알려져 있습니다. 이는 표준 삼각비에서 찾을 수 있으며:

$$
\theta=\angle M R N=60^{\circ}
$$

$\sin (\theta)=\frac{\sqrt{3}}{2}$ 이라는 값은 종종 60 도 각도에서 발생하는 값입니다. 따라서 $\angle M R N$ 는 $60^{\circ}$ 입니다. 질문이 더 있으면 말씀해 주세요!

\end{GPTA}

