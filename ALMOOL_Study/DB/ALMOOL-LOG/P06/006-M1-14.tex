\begin{GPTQ}
    어떤식으로 식을 세워야 할까  
\end{GPTQ}

\begin{GPTA}
 식 설정에 대해 설명드리겠습니다. 주어진 문제를 해결하기 위해서는 두 가지의 변수와 두 개의 부등식을 사용할 수 있습니다.

    1. 변수 설정:

    - 작은 촛불의 개수를 $s$ 라고 하고,

    - 큰 촛불의 개수를 $\ell$ 이라고 합니다.

    2. 부등식 설정:

    - 첫 번째 부등식은 예산 관련 부등식입니다. 작은 촛불과 큰 촛불의 가격 그리고 예산 한도에 대한 부등식을 다음과 같이 설정할 수 있습니다:
    
    $$
    4.90 s+11.60 \ell \leq 2,200
    $$
    
    
    이는 작은 촛불 $s$ 개와 큰 촛불 $\ell$ 개를 구매하는 비용의 합이 2,200 달라 이하여야 한다는 것을 나타냅니다.

    - 두 번째 부등식은 최소 구매 수량 조건과 관련된 부등식입니다:
    
    $$
    s+\ell \geq 200
    $$
    
    
    이는 촛불을 총 200 개 이상 구매해야 함을 나타냅니다.

    이 두 부등식을 조합하여 문제를 해결할 수 있습니다. 이 설정을 통해 문제를 더 쉽게 풀 수 있기를 바랍니다. 추가 질문이 있다면 언제든지 물어보세요!
    \end{GPTA}

