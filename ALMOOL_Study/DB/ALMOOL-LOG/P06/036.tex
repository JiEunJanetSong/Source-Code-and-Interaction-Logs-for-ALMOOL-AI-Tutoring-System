\begin{GPTQ}
 $(\mathrm{X}+12) \mathrm{d}$ 를 풀어줄때는 분배법칙을 이용한거지?   
\end{GPTQ}

\begin{GPTA}
네, 정확합니다! $(x+12) d$ 을 풀 때 분배법칙을 사용합니다. 분배법칙은 다음과 같이 적습니다:

$$
(a+b) c=a c+b c
$$


여기서 $(x+12) d$ 는 분배법칙을 적용하여 다음과 같이 펼칠 수 있습니다:

$$
(x+12) d=x d+12 d
$$


위의 과정은 각 항에 $d$ 를 곱해 주는 형태입니다. 이렁게 해서 식을 좀 더 정리하거나 단순화할 수 있습니다. 이 법칙은 대수학에서 매우 유용하게 사용되고 있습니다. 질문이 더 있다면 언제든지 물어보세요!
\end{GPTA}

