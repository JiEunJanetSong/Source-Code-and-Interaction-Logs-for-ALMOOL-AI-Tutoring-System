\begin{GPTQ}
    풀이는 어떤식으로 진행해?  
\end{GPTQ}

\begin{GPTA}
  문제를 해결하기 위한 단계별 풀이를 설명드리겠습니다.

    1. 주어진 부등식 설정
    
    먼저, 두 가지 부등식을 다시 한 번 정리해 보겠습니다.

    - 비용 관련 부등식:
    
    $$
    4.90 s+11.60 \ell \leq 2,200
    $$
    
    - 최소 구매 수량 관련 부등식:
    
    $$
    s+\ell \geq 200
    $$
       
  2. 부등식 결합 및 단순화

먼저, 두 번째 부등식을 변형하여 첫 번째 부등식과 연결시킬 수 있도록 할 수 있습니다. 이를 위 해 두 번째 부등식의 $s$ 를 구하면:

$$
s \geq 200-\ell
$$


이를 첫 번째 부등식에 대입하면:

$$
4.90(200-\ell)+11.60 \ell \leq 2,200
$$

3. 단일 부등식으로 변형

위 부등식을 풀어보면:

$$
\begin{gathered}
980-4.90 \ell+11.60 \ell \leq 2,200 \\
980+6.70 \ell \leq 2,200
\end{gathered}
$$

4. 최대 큰 촛불 수 구하기

이 부등식을 $\ell$ 에 대해 풀면:

$$
\ell \leq \frac{1,220}{6.70} \approx 182.09
$$


따라서, 큰 촛불의 최대 개수는 $\ell=182$ 입니다.

결론

큰 촛불을 최대 182 개까지 구입할 수 있으며, 이는 소수점 이하를 내림한 값입니다. 추가적으로 질문이나 궁금한 점이 있다면 언제든지 물어보세요!
\end{GPTA}

