\begin{GPTQ}
    호의 길이 구하는 방법 알려줘   
\end{GPTQ}

\begin{GPTA}
호의 길이를 구하기 위혜서는 원의 반지름과 중심각을 알고 있어야 합니다. 아래는 호의 길이를 구하는 방법입니다.

호의 길이 공식

호의 길이는 중심각의 크기에 따라 결정되며, 공식은 다음과 같습니다:
  
$$
L=2 \pi r \cdot \frac{\theta}{360^{\circ}}
$$


여기서,

- $L$ 은 호의 길이입니다.

- $r$ 은 원의 반지름입니다.

- $\theta$ 는 호의 중심각(도 단위)입니다.

만약 라디안 단위를 사용할 경우

중심각이 라디안 단위일 경우, 호의 길이는 다음과 같은 공식으로 계산됩니다:

$$
L=r \theta
$$


여기서 $\theta$ 는 라디안 단위의 중심각입니다.

예제

1. 반지름이 $10^{\circ}$ 이고 중심각이 $60^{\circ}$ 인 호의 길이를 구해봅시다.

- 호의 길이:

$$
L=2 \pi \cdot 10 \cdot \frac{60}{360}=\frac{1}{6} \cdot 20 \pi=\frac{20 \pi}{6} \approx 10.47
$$

2. 만약 중심각이 $\frac{\pi}{3}$ 라디안이라면:

- 호의 길이:

$$
L=10 \cdot \frac{\pi}{3}=\frac{10 \pi}{3} \approx 10.47
$$


이 방법이 호의 길이를 계산하는데 도움이 되었기를 바랍니다. 추가 질문이 있으면 언제든지 물어보세요!
\end{GPTA}

