\begin{GPTQ}
    각 MNQ 가 90 도라는 것을 기반해서 문제 다시 줄어줘   
\end{GPTQ}

\begin{GPTA}
  물른입니다: $\angle M N Q=90^{\circ}$ 라는 정보를 기반으로 문제를 다시 풀어보겠습니다. 이로 인해 각도 계산이 수정될 수 있습니다.

    1. 주어진 조건과 추가 정보:

    - $|\overline{M N}|=15, \mathrm{~m}$

    - $|\overline{M R}|=10 \sqrt{3}, \mathrm{~m}$

    - $\angle M Q N=72^{\circ}$

    - $\angle M N Q=90^{\circ}$

    2. $\triangle N M Q$ 각 계산:

    - 삼각형의 내각 합으로 $\angle M N Q=90^{\circ}$ 로 주어졌습니다.

    3. $\angle M R N$ 계산:
    
    - 직각삼각형 $\triangle M N R$ 에서 $\angle M N R=90^{\circ}$ 에 대해 나머지 각은 $\angle M R N$ 로 계산합니다:
    
    $$
    \sin (\angle M R N)=\frac{M N}{M R}=\frac{15}{10 \sqrt{3}}
    $$
    
    - 이를 계산해 봅시다:
    
    $$
    \angle M R N=\sin ^{-1}\left(\frac{15}{10 \sqrt{3}}\right) \approx 30^{\circ}
    $$
    
    4. 각 $x$ 계산:

    - $\angle N M R$ 은 $x$ 이며, 다음과 같이 계산합니다:
    
    $$
    x=180^{\circ}-\left(90^{\circ}+30^{\circ}\right)=60^{\circ}
    $$
    
    
    따라서, $x$ 는 $60^{\circ}$ 입니다.

    질문이나 더 필요한 부분이 있다면 말씀해 주세요!
 
\end{GPTA}

