\begin{GPTQ}
    삼각형 그려서 문제 해석 가능해?
\end{GPTQ}

\begin{GPTA}
물론입니다. 30-60-90 삼각형을 그려서 설명 하겠습니다.

    삼각형 설명

    30-60-90 삼각형은 각의 크기에 따라 특정한 비율로 변의 길이가 절정되는 직각삼각형입니다. 이 삼각형에서는 다음과 같은 비율이 적용됩니다:

    - 30 도의 각을 가지는 변: $x$

    - 60 도의 각을 가지는 변: $\sqrt{3} x$

    - 90 도의 각을 가지는 빗변 (hypotenuse): $2 x$
    
    문제에서는 빗변의 길이가 8 이라고 주어져 있습니다. 따라서, 이 식을 사용하여 다른 변의 길이를 계산할 수 있습니다:
    
    $$
    2 x=8 x \quad=4
    $$
    
    
    따라서, 가장 짧은 변의 길이 $x$ 는 4 입니다.

    위의 설명에 기초하여 삼각형을 시각화한다면, 다음과 같을 것입니다:

    - 빗변 (8): 삼각형 의 가장 긴 변

    - 짧은 변 (4): $30^{\circ}$ 와 마주보는 변
    - 다른 변 $(4 \sqrt{3}): 60^{\circ}$와 마주보는 변
    
    이 시각화를 통해 문제를 이해하는 데 도움이 되었기를 바랍니다. 추가적인 질문이 있으면 언제든지 말씀해 주세요!  
\end{GPTA}

