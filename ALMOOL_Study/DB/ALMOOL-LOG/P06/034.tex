\begin{GPTQ}
    지름을 d 로 하지 않고 바로 반지름을 d 로 설정 하는건 불가능해?   
\end{GPTQ}

\begin{GPTA}
   물론 가능합니다! 지름 대신 반지름을 직접 $d$ 로 설정 하고 풀 수도 있습니다. 단계별로 설명혜드리 젰습니다.

1. 변수 설정:

- 위쪽 원의 반지름을 $d$ 라고 설정합니다. 따라서 반지름은 $r=d$ 입니다.

- 바닥의 반지름은 세 배이므로 $3 d$ 입니다.

2. 비례식을 셰우기:

- 이제 원뿔의 유사한 삼각형에서 비례식을 사용합니다.

$$
\frac{x+12}{3 d}=\frac{x}{d}
$$

3. 비례식을 풀어 $x$ 구하기:

- 양변에 $3 d$ 를 곱하여 식을 정리 합니다.

$$
\frac{x+12}{3 d}=\frac{x}{d} \quad \Rightarrow \quad(x+12) d=3 d x
$$

- 식을 풀어주면,

$$
x d+12 d=3 d x
$$

- 이를 정리하면,

$$
12 d=3 d x-x d
$$

- 그러면,

$$
12 d=2 d x \quad \Rightarrow \quad x=\frac{12 d}{2 d}=6
$$

4. 전체 원뿔의 높이 구하기:

- $x=6$, cm

- 따라서 전체 원뽈의 높이는 $x+12=18, \mathrm{~cm}$.

따라서 반지름을 $d$ 라고 설정해도 같은 결과인 전체 원뿔의 높이는 $18 \mathrm{~cm}$ 입니다.


\end{GPTA}

