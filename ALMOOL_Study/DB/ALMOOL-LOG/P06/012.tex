\begin{GPTQ}
    Circumference가 뭔지 해석해줘 
\end{GPTQ}

\begin{GPTA}
``Circumference"는 한국어로 ``원의 둘레" 또는 간단히 ``둘레"라고 해석할 수 있습니다. 원의 둘레 는 원주를 따라 있는 전체 길이를 의미합니다.

둘레 계산 공식

원의 반지름 $r$ 를 사용하여 둘례를 다음과 같이 구할 수 있습니다:

$$
C=2 \pi r
$$


또는 지름 $d$ 를 사용하여 다음과 같이 표현할 수도 있습니다:

$$
C=\pi d
$$


여기서

- $C$ 는 원의 둘레입니다.

- $r$ 는 원의 반지름입니다.

- $d$ 는 원의 지름이며, $d=2 r$ 입니다.

- $\pi$ 는 수학 상수로, 약 3.14159 입니다.

이 정보가 도움이 되었길 바랍니다. 추가 질문이 있으면 언제든지 물어보세요!
   
\end{GPTA}

