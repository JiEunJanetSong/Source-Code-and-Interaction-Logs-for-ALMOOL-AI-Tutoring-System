\begin{GPTQ}
    X 의 값은 60 이 아니고 12 야
\end{GPTQ}

\begin{GPTA}
 그렇군요, 문제의 컨텍스트를 다시 확인해보고 $\angle M R N$ 를 $60^{\circ}$ 로 계산하여, 이전에 약간의 착오가 있었던 것 같습니다.

    올바르게 계산해봅시다:

    정보:

    - $\angle M N R=90^{\circ}$ (직각)

    - $\angle M R N \approx 60^{\circ}$

    - $\angle M N Q=90^{\circ}$

    - $\angle M Q N=72^{\circ}$

    $\angle N M R$ 의 올바른 계산:

    1. $\triangle M N R$ 을 고려합니다. 이 때 내각의 합은 $180^{\circ}$ 입니다:
    
    $$
    x+\angle M R N+\angle M N R=180^{\circ}
    $$
    
    2. 각도 대입:

    - $\angle M R N \approx 60^{\circ}$ 이므로,
    
    $$
    x+60^{\circ}+90^{\circ}=180^{\circ}
    $$
    
    3. 계산:
    
    $$
    x+150^{\circ}=180^{\circ}
    $$
    
    4. 따라서,
    
    $$
    x=180^{\circ}-150^{\circ}=30^{\circ}
    $$
    
    
    결과 이전에 60 도의 계산을 기반으로 진행했기 때문에,
    
    이 부분에서 흔동이 있었던 것 같았습니 다. $x$ 의 값은 실제로 위에서 약간 다르게 설명되었네요. 잘못된 내용을 확인해주셔서 고맙습니다. 추가로 할 질문이 있다면 언제든 물어보세요!
  
   
\end{GPTA}

