\begin{GPTQ}
    그렿다면 이 문제는 잘못된 문제 아닐까?   
\end{GPTQ}

\begin{GPTA}
    이 문제는 주어진 정보가 충분하지 않아서 여러 해를 가질 수 있는 구조입니다. 실질적으로 해결 하기 위해서는 추가 정보가 필요하거나, 특정한 전제 조건이 추가되어야 합니다.

    잘 정의된 문제라면 펼요한 모든 변수(에: 높이 등)가 주어지거나, 추가 조건(예: 반지름과 높이의 관계)이 명확해야 합니다.

    이 문제의 경우:

    - 높이가 주어져야 하거나,

    - 반지름과 높이 사이의 관계가 명확하게 주어져야 합니다.
    
    문제의 정의와 정확성을 위해 이런 조건들이 충족되지 않으면 다양한 가능성으로 인한 모호성이 발생할 수 있습니다. 따라서, 교육이나 시험 환경에서 이런 문제는 모호성을 줄이기 위해 재조정 하거나 필요한 정보를 추가해야 합니다.
    문제를 명확히 하기 위해선 추가적으로 누락된 정보를 졔공하거나, 주어진 조건을 명시적으로 밝히는 것이 중요합니다.

  \end{GPTA}

