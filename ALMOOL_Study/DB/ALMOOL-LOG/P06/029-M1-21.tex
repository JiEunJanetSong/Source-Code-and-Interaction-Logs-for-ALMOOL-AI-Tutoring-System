\begin{GPTQ}
    원의 방정식은 어떻게 풀어?   
\end{GPTQ}

\begin{GPTA}
 원의 방정식을 푸는 방법에 대해 설명드리겠습니다.

    주어진 방정식은 $(x-4)^2+(y-3)^2=36$ 입니다. 이 식은 다음의 일반적인 원의 방정식 형태로 되어 있습니다:
    
    $$
    (x-h)^2+(y-k)^2=r^2
    $$
    
    
    여기서 $(h, k)$ 는 원의 중심 좌표이고 $r$ 은 반지름입니다. 주어진 방정식 분석:

    - $(x-4)^2$ : 이는 $x$ 의 좌표에서 4 만큼 오른쪽으로 이동한 것을 의미합니다. 따라서 $h=4$.

    - $(y-3)^2$ : 이는 $y$ 의 좌표에서 3 만큼 위로 이동한 것을 의미합니다. 따라서 $k=3$.

    - $r^2=36$ : 반지롬 $r$ 의 제곱입니다. 따라서 $r=\sqrt{36}=6$.
    
    따라서, 주어진 원의 정보는 다음과 같이 요약할 수 있습니다:

    - 중심: $(4,3)$

    - 반지름: 6
    
    원을 이해하고 중심과 반지름을 구했으니, 원하는 다른 계산이나 추가적인 질문이 있다면 알려주세요!
      
\end{GPTA}

