\begin{GPTQ}
    그래프에서 한 선에 점이 왜 하나가 아니야? 
\end{GPTQ}

\begin{GPTA}
그래프에서 한 선에 여러 점이 배치되어 있는이유는, 각 점들이 서로 다른 연도에 해당하는 잉어 개체군 수치를 나타내기 때문입니다. 이 점들은 개별적으로 특정 연도에 해당하는 잉어 개체 수 를 보여줍니다. 이러한 데이터 포인트들은 시간이 지남에 따라 잉어 개체수가 어뗳게 변 했는지를 시각화하는 데 사용됩니다.

그래프의 선은 이 점들을 가장 잘 나타내는 경향선을 나타내며, 이러한 경향선을 기준으로 기울기를 계산할 수 있습니다. 기울기는 시간이 지남에 따라 잉어 개체수가 증가하는 비율을 나타내며, 이것이 문제에서 요구하는 ``기울기"입니다. 어떤 질문이든 더 알려주세요!
\end{GPTA}

