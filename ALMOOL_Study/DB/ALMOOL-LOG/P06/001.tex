\begin{GPTQ}
    왜 c가 답인지 한국어로 설명해줘 
\end{GPTQ}

\begin{GPTA}
    물론, 문제에서 주어진 표현식을 살펴보겠습니다. $6 x^8 y^2+12 x^2 y^2$ 라는 표현식이 있습니다.
    이 표현식은 각 항의 공퉁 인수를 묶어낼 수 있습니다. 두 항 모두 $6 x^2 y^2$ 라는 공통 인수를 가지고 있습니다. 이를 묶어내면 다음과 같이 쓸 수 있습니다:
    
    $$
    6 x^2 y^2\left(x^6\right)+6 x^2 y^2(2)
    $$
    
    
    이 식을 좀 더 간단하게 표현하면:
    
    $$
    6 x^2 y^2\left(x^6+2\right)
    $$
    
    
    따라서 위의 과정을 통혜 문제에서 주어진 선택지 중 $6 x^2 y^2\left(x^6+2\right)$ 에 해당하는 (C) 선택지가 정 답이 됩니다. 공퉁 인수를 묶어념으로써 이렇게 간단하게 표현할 수 있다는 점이 중요합니다. 
\end{GPTA}

