\begin{GPTQ}
    현의 길이 구하는 공식 알려줘 
\end{GPTQ}

\begin{GPTA}
현의 길이를 구하기 위해서는 원의 반지름과 중심각(또는 중심각에 기반한 각도)을 사용할 수 있습니다. 여기서는 중심각을 이용한 방법을 설명하젰습니다.

현의 길이 공식

원의 현의 길이는 중심각을 알고 있을 때 다음 공식을 사용할 수 있습니다:

$$
C=2 r \sin \left(\frac{\theta}{2}\right)
$$


여기서:

- $C$ 는 현의 길이입니다.

- $r$ 는 원의 반지름입니다.

- $\theta$ 는 호의 중심각(라디안 단위)입니다.
  
중심각이 도일 경우

중심각이 도 단위라면 먼저 라디안으로 변환한 후 사용합니다:

- $\theta_{\text {radians }}=\theta_{\text {degrees }} \times \frac{\pi}{150}$

예제

반지름이 $10^{\circ}$ 이고 중심각이 $60^{\circ}$ 일 경우의 현의 길이를 구해보겠습니다.

1. 중심각을 라디안으로 변환:

$$
\theta_{\text {radians }}=60 \times \frac{\pi}{180}=\frac{\pi}{3}
$$

2. 현의 길이:

$$
C=2 \times 10 \times \sin \left(\frac{\frac{\pi}{3}}{2} \right)=20 \times \sin \left(\frac{\pi}{6}\right)
$$

3. $\sin \left(\frac{\pi}{6}\right)=\frac{1}{2}$ 이므로:

$$
C=20 \times \frac{1}{2}=10
$$


이 방법을 퉁해 현의 길이를 구할 수 있습니다. 추가적인 질문이 있다면 언제든지 물어보세요!
\end{GPTA}

