\begin{GPTQ}
    14 가 나오는 이유가 뭐야?   
\end{GPTQ}

\begin{GPTA}
최대 데이터 값은 주어진 데이터 집합에서 가장 큰 값입니다.
 
주어진 빈도표를 살펴보면, 각 데이터 값이 얼마나 자주 발생하는지를 알 수 있습니다:

    - 6 은 3 번 발생

    - 7은 3 번 발생

    - 8 은 8 번 발생

    - 9 는 8 번 발생

    - 10 은 9 번 발생

    - 11 은 11 번 발생

    - 12 는 9 번 발생

    - 13 은 0 번 발생
    
    - 14 는 6 번 발생
    
    따라서, 최대 데이터 값은 14 입니다. 픅히, 13 이라는 값은 전혀 발생하지 않기 때문에 최대값으로 고려되지 않습니다. 14 는 해당 데이터 집합에서 실제로 발생하며 빈도는 6 입니다. 따라서, 최대 데이터 값은 14 입니다.
    
\end{GPTA}

