\begin{GPTQ}
    그럼 이 문제는 어떻게 풀어야 하는거야?
\end{GPTQ}

\begin{GPTA}
주어진 문제는 원뿔의 전체 부피와 원뿔의 대퇴 부분(프러스텀)의 정보를 사용하여 반지름을 구하는 문제입니다. 문제 해결에는 다음 단계가 포함될 수 있습니다:

- 프러스텀은 원뿔의 윗부분이 잘린 형태입니다. 윗 지름 $d_1$, 아랫 지름 $d_2$, 아래쪽 지름이 위쪽 지름의 3 배인 경우, $d_2=3 d_1$ 입니다.

2. 원뿔의 전체 부피 조건 사용:

- 원뿔의 전체 높이는 전체 높이에서 프러스텀의 높이를 합산하여 파악해야, 문제의 주 어진 높이 10 cm 를 이용해서 문제 해걸 시 조건에 맞춰 설정한 후 실질적인 계산을 위해 추가 정보를 가정해야 합니다.

3. 주어진 부피 공식을 사용하여 $r$ 를 찾기 위한 식을 설정:
$\circ V=\frac{1}{3} \pi r^2 h$ 식에서 부피를 알고 높이를 대체해 대입하는 과정을 통해 $r^2$ 를 구합니다.

4. 계산을 통해 $r$ 을 구하는 접근법:

$$
\begin{gathered}
r^2=\frac{39,600 \times 21}{396} \\
r^2=2100 \quad \rightarrow \quad r=\sqrt{2100}
\end{gathered}
$$

5. 최종 결과:

- 반지름 $r=10 \sqrt{21} \approx 45.83 \mathrm{~cm}$ 입니다.

제공된 정보와 변수를 걸합한 종합적인 계산을 통해 원뿔의 반지름을 찾습니다. 이 접근법으로 계획적이고 단계별로 원뿔의 대퇴 부분과 연절하여 최종 값을 완성할 수 있습니다.

디자인되는 해결법예서 변질 정보 제공:

주어진 문제는 실제로 대퇴와 하우징한 건축물 등 원하는 방법으로 모든 변수를 종합하여 반지름을 식별하는 정보를 제공해야 할 수 있습니다.
\end{GPTA}

