\begin{GPTQ}
    어떤식으로 문제를 풀어야 할까?   
\end{GPTQ}

\begin{GPTA}
문제를 푸는 방법을 단계별로 설명드리겠습니다.

    우리는 두 개의 방정식을 갖고 있으며 이 방정식둘이 같다고 가정하여 문제를 풀 것입니다:
    
    $$
    y=2 x-5 y \quad=2 x^2+4
    $$
    
    1. 두 식을 동일하계 두기: 각 식은 y로 정의되어 있으므로 두 식을 서로 같게 눟을 수 있습니다.
    
    $$
    2 x-5=2 x^2+4
    $$
    
    2. 방정식을 정리 하기: 방정식을 한쪽으로 모아서 정리합니다.
    
    $$
    2 x^2-2 x+9=0
    $$
    
    3. 근의 공식을 사용하기: 이 차수 2 의 이차 방정식의 해를 찾기 위해, 근의 공식을 사용할 수 있습니다.
    
    $$
    x=\frac{-b \pm \sqrt{b^2-4 a c}}{2 a}
    $$
    
    
    $$
    \text { 여기서 } a=2, b=-2, c=9 \text { 입니다. }
    $$
    
    4. 판별식 계산하기: 해의 개수를 판별식 $b^2-4 a c$ 을 탕해 알 수 있습니다.
    
    $$
    b^2-4 a c=(-2)^2-4 \times 2 \times 9=4-72=-68
    $$
    
    
    이 경우 판별식이 음수이므로, 실제 해가 존재하지 않습니다.

    5. 실수 해의 개수 결정하기: 따라서 이 시스템은 실수 해를 가지지 않으며, 정답은 0개입니 다.
    
    이 과정이 방정식 시스템의 실수 해를 찾는 방법을 잘 보여줍니다. 추가적인 질문이 있으시면 언제든지 말씀하세요!

    \end{GPTA}

