\begin{GPTQ}
    어떤식으로 문제를 풀어야할까?    
\end{GPTQ}

\begin{GPTA}
이 문제를 푸는 방법을 단계별로 안내해드리겠습니다.

Step 1: 그래프에서 주요 점 식별하기

먼저, 그래프에서 두 개의 주요 점을 식별합니다. 예를 들면:

- 1990 년에 잉어 수가 약 0 마리 입니다.

- 2000년에 잉어 수가 약 38 마리입니다.

이 두 점은 기울기를 계산하는 데 사용됩니다.

Step 2: 기울기 계산하기

기울기(Slope)는 두 점 사이의 변화율을 나타냅니다. 계산하는 공식은 다음과 같습니다:

$$
\text { slope }=\frac{\text { change in carp }}{\text { change in years }}
$$


이를 주어진 점들에 적용하면:

$$
\text { slope }=\frac{38-0}{2000-1990}=\frac{38}{10}=3.8
$$

Step 3: 가장 가까운 답 선텩하기

계산된 기울기 값 3.8 에 가까운 답을 선택합니다. 졔공된 선택지 중 가장 가까운 것은 3.3 입니다. 따라서, 최종 답은:
(A) 3.3

이 방법을 통해 문제를 풀 수 있습니다. 추가 질문이 있다면 언제든지 말씀해 주세요!
  
\end{GPTA}

