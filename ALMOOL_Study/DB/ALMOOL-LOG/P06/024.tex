\begin{GPTQ}
    판별식이 뭐야?  
\end{GPTQ}

\begin{GPTA}
   판별식은 이차 방정식의 해의 개수와 해의 성질을 결정하는데 중요한 역할을 하는 값입니다. 일반적인 이차 방정식은 다음과 같은 형태로 이루어져 있습니다:
    
    $$
    a x^2+b x+c=0
    $$
    
    
    여기서 $a, b, c$ 는 상수이며 $a \neq 0$ 입니다. 이 방정식의 판별식은 다음과 같이 정의됩니다:
    
    $$
    \Delta=b^2-4 a c
    $$
       
  $$
\Delta=b^2-4 a c
$$


판별식의 값에 따라 해의 개수가 결정됩니다:

- 만약 $\Delta>0$ 이면, 서로 다른 두 개의 실수 해가 존재합니다.

- 만약 $\Delta=0^{\circ}$ 이면, 중복된 하나의 실수 해가 존재합니다. (즉, 중근)

- 만약 $\Delta<0$ 이면, 실수 해는 없고 서로 다른 두 개의 복소수 해가 존재합니다.

판별식은 이차 방정식의 성질을 분석하고 어떠한 종류의 해가 존재하는지를 판단하는데 유용한 도구입니다. 추가적인 궁금증이 있으시면 언졔든지 물어보세요!
\end{GPTA}

