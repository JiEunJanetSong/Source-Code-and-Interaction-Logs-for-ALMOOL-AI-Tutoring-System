\begin{GPTQ}
    지금 문제에서 주어진게 pi의 값과 부피뿐인데 이걸로 어떻게 반지름을 구해야할까 
\end{GPTQ}

\begin{GPTA}
주어진 부피와 $\pi$ 값을 이용하여 원뿔의 반지름을 구할 수 있습니다. 문제에서 우리는 이미 전체 원뿔의 높이를 18 cm 로 가정하고 풀이를 시작할 수 있습니다.

    전체 원뿔의 부피 공식을 이용하여 반지름을 구하는 방법은 다음과 같습니다:

    1. 원뿔의 부피 공식:
    
    $$
    V=\frac{1}{3} \pi r^2 h
    $$
    
    2. 주어진 값들을 대입합니다:
    
    $$
    39,600=\frac{1}{3} \times \frac{22}{7} \times r^2 \times 18
    $$
    
    3. 위 식을 $r^2$ 에 대해 정리합니다:
    
    $$
    39,600=\frac{396}{21} r^2
    $$
        
  $$
\begin{aligned}
& 39,600=\frac{396}{21} r^2 \\
& r^2=\frac{39,600 \times 21}{396}
\end{aligned}
$$

4. 계산을 퉁해 $r^2$ 를 구한 후, 제곱근을 계산하여 $r$ 을 구합니다:

$$
r^2=2100
$$


$$
r=\sqrt{2100}=10 \sqrt{21} \approx 45.83 \mathrm{~cm}
$$


이 과정을 통해 반지름 $r$ 을 구할 수 있습니다. 주어진 값들을 정확하게 대입하면 최종적으로 원뿔 의 반지름을 얻을 수 있습니다.
 
\end{GPTA}

