\begin{GPTQ}
    11이 답이 아닌 이유가 뭐야?
\end{GPTQ}

\begin{GPTA}
11 은 가장 자주 발생하는 데이터 값이지만, 우리는 데이터 집합의 최대값을 찾고 있습니다. 최대 값이란 수치적으로 가장 큰 값을 의미합니다. 데이터 값의 크기를 비교해 보면:

    - 11 은 숫자 상으로는 클 수 있지만,

    - 14 가 11 보다 크기 때문에, 최대값은 14 가 됩니다.
    
    따라서 14 는 이 데이터 집합에서 실제로 나타나며(6번), 데이터 값 중 가장 큰 값이기도 하기 때문에 최 대값입니다. 11 은 빈도는 제일 높지만, 최대값은 아닙니다. 따라서 최대값은 바로 14 입니다. 
\end{GPTA}

