\begin{GPTQ}
    연립으로 풀 수 있어?
\end{GPTQ}

\begin{GPTA}
물론입니다! 연립방정식을 사용해서도 이 문제를 해결할 수 있습니다. 다음은 그 과정을 설명한 것입니다.

1. 변수 및 방정식 설정

우리는 두 개의 방정식을 가지고 있습니다:

1.1 예산 방정식:

$$
4.90 s+11.60 \ell=2,200
$$

1.2 최소 구매 수량 방정식:
  
$$
s+\ell=200
$$

2. 대입법을 사용하여 연립 방정식 풀기

첫 번째 단계는 $s+\ell=2 0 0$ 방정식에서 $s$ 를 구하는 것입니다:

$$
s=200-\ell
$$


이 식을 예산 방정식에 대입합니다:

$$
4.90(200-\ell)+11.60 \ell=2,200
$$


이것을 풉니다:

$$
980-4.90 \ell+11.60 \ell=2,200
$$


$$
980+6.70 \ell=2,200
$$

3. $\ell$ 에 대해 풀기

위의 수식을 정리하여 $\ell$ 를 구하면:

$$
6.70 \ell=2,200-980
$$


$$
6.70 \ell=1,220
$$


$$
\ell=\frac{1,220}{6.70} \approx 182.09
$$

4. 최댓값 선택

$\ell$ 의 값은 정수여야 하므로, $\ell=182$ 로 내림합니다.

결론

이 과정을 통해 큰 춧불은 최대 182 개 구입 가능하고, 이는 예산과 최소 구매 수량 요구를 모두 만족합니다.

이 방법이 문제를 푸는데 도움이 되었길 바랍니다. 추가 질문이 있으면 알려주세요!

\end{GPTA}

