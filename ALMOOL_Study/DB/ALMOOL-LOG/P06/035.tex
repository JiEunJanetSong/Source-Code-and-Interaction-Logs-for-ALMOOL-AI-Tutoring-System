\begin{GPTQ}
    비례식은 대부분 3:3,  4:3 처럼 비율로 나오는데 예 여기선 $=$ 을 통해 표현될까    
\end{GPTQ}

\begin{GPTA}
    좋은 질문입니다! 비례식을 이용할 때 특정한 상 황에서는 비율을 $\frac{a}{b}=\frac{c}{d}$ 형태로 나타내는 것이 유용할 때가 있습니다. 이는 각기 대응하는 부분의 길이의 비윤이 같은 것을 의미합니다. 이러한 비례식은 두 개의 유사한 삼각형에서 대응하는 변의 비율이 같다는 것을 사용하기 때문입니다. 이 문제에서 비례식을 $\frac{x+12}{3 d}=\frac{x}{d}$ 와 같은 $=\frac{\text { 비}}{\text { 비 }}$ 형태로 나타내는 이유는 다음과 같습니다.

1. 유사한 삼각형: 주어진 문제의 구조는 두 유사한 삼각형을 만들어주기 때문입니다. 전체 원뿔과 frustum 위 부분의 삼각형이 유사하므로 대웅 변의 비율이 서로 같다는 관계를 세울 수 있습니다.

2. 타당한 비교: 유사한 삼각형의 대응 변의 비율과 같은 확장된 비를 사용하여 대응하는 길이 간의 수학적 절차를 보장합니다.

3. 문제의 요구: 문제는 두 점(예: 높이의 길이와 전체 원뿔의 비율)을 함께 고려해야 하며, 대응변에 관한 정보를 가져오는 데 비례식이 유용합니다.

비례식에서 $=$ 을 사용하여 표현하는 것이 이러한 문제를 푸는 데 더욱 명확한 구조와 연산의 단순화를 제공합니다. 

비례식의 서로 다른 형태는 상호 간의 변수를 해석하고 겹치지 않도록 쉽게 도와줍니다.

추가적인 질문이 있다면 언제든지 물어보세요!
\end{GPTA}