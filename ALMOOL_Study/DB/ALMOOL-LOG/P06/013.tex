\begin{GPTQ}
    반지름이 72 이고 중심각을 모를때 현의 길이는 어떤식으로 구할 수 있을까?   
\end{GPTQ}

\begin{GPTA}
중심각을 알지 못하더라도 원의 반지름과 현의 길이를 통해 삼각형의 일부 데이터를 제공받는다 면, 현의 길이를 유도할 수 있는 경우가 있습니다. 하지 만 중심각이나 다른 정보를 제공받지 않고 현의 길이를 바로 구하는 것은 어렵습니다. 다음의 상황을 고려할 수 있습니다:

1. 중심에서 현까지의 거리 제공 시:

- 만약 왼의 중심에서 현까지의 수직 거리(현의 높이)가 주어지면, 이를 활용해 현의 길이를 계산할 수 있습니다. 구체적인 공식은 앞서 설명한 대로 사용하는 것입니다.

$$
C=2 \sqrt{r^2-d^2}
$$


여기서

- $C$ 는 현의 길이,

- $r$ 는 반지름(여기서는 72),

- $d$ 는 원의 중심에서 현까지의 높이입니다.

2. 다른 추가 정보 제공 시:

- 문제에서 더 제공되는 추가적인 데이터가 없다면 중심각 없이 현의 길이를 직접 구할 수 있는 공식은 존재 하지 않습니다. 따라서 중심각 관련 정보, 또는 원의 중심에서 현 까지의 떨어진 위치를 추가적으로 받아야 합니다.

\end{GPTA}

