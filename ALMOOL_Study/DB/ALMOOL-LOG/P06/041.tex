\begin{GPTQ}
    근데 높이를 가정하면 정확한 답이 나오지 않는거 아니야? 
\end{GPTQ}

\begin{GPTA}
맞습니다. 높이를 가정 하는 경우 정확한 답을 보장할 수는 없습니다. 하지만 문제에서 부패와 지름의 비율이 주어졌으므로, 적절한 앞선 설명대로 계산을 진행한 것에 의미가 있습니다.

만약 문제에서 전체 원뿔의 높이가 주어지지 않았다면, 문제의 추가적인 조건이나 다른 지시 사항이 필요합니다. 주어져 있는 정보로 반지름을 정확히 구하려면 원뿔의 높이가 주어진 것으로 가정 하는 것이 아니라 다른 추가 정보가 제공되어야 합니다.

이를 염두에 두고 주어진 요소들을 활용하여 주어진 문제를 그렇게 풀어야 한다는 것을 가정 하고 설명을 드린 것입니다. 주어진 자료를 정확히 고려하는 것이 중요하다는 것을 기억해 주세요.
\end{GPTA}

