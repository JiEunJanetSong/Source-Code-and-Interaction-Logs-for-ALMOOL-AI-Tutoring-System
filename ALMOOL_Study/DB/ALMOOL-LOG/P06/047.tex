\begin{GPTQ}
    그랬을때 답이 달라져?  
\end{GPTQ}

\begin{GPTA}
 네, 원뿔의 높이가 달라짐에 따라 반지름을 계산한 답도 달라집니다. 부피 $V$ 는 고정되어 있고, 높 이 $h$ 가 달라지면 $r^2$ 도 변경됩니다. 공식적으로는 다음과 같이 진행됩니다:
    
    $$
    V=\frac{1}{3} \pi r^2 h
    $$
    
    
    같은 부피에서 다른 높이를 대입하면 부피를 유지하기 위해 반지름이 조정되어야 하기 때문입니다.
    
    예:
    1. 예시: $h=10 \mathrm{~cm}$
    
    $$
    39,600=\frac{1}{3} \times \frac{22}{7} \times r^2 \times 10
    $$
    
    
    계산에서:
    
    $$
    r^2=3,780
    $$
     
    2. 높이가 $h=12 \mathrm{~cm}$ 라면:

    $$
    39,600=\frac{1}{3} \times \frac{22}{7} \times r^2 \times 12
    $$
    
    
    계산에서:
    
    $$
    \begin{gathered}
    r^2=3,150 \\
    r=\sqrt{3,150} \approx 56.12 \mathrm{~cm}
    \end{gathered}
    $$
    
    
    따라서, 높이를 다른 값으로 변경 하면 반지름은 반드시 다르게 계산됩니다. 원뿔의 부피를 일정 하게 유지하기 위해 반지름이나 다른 치수가 조정되는 것이므로 값이 달라진다고 볼 수 있습니다.

    높이가 달라지면 자연스럽게 반지름이 변하게 되고, 이로 인해 결과값이 달라집니다.
    \end{GPTA}

