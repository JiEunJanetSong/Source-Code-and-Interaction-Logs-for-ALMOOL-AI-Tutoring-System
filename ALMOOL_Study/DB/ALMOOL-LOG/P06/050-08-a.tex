\begin{GPTQ}
    이 문제를 푸는 순서는 어떻게 될까?
\end{GPTQ}

\begin{GPTA}
문제를 푸는 순서는 다음과 같습니다:

    1. 삼각형 $\triangle  N M Q$ 각 계산:

    - 삼각형의 내각의 합은 $180^{\circ}$ 이므로:
    
    $$
    \angle N M Q+72^{\circ}+90^{\circ}=180^{\circ}
    $$
    
    
    $$
    \Rightarrow \angle N M Q=18^{\circ} .
    $$
    
    2. 각 $\angle M R N$ 계산:

    - $\triangle M N R$ 에서 $\angle M R N=\theta$ 이고, $\sin \theta=\frac{15}{10 \sqrt{3}}$.

    - 따라서, $\theta=\sin ^{-1}\left(\frac{15}{10 \sqrt{3}}\right)$ 이고 $\theta=60^{\circ}$.

    3. 각 $x$ 계산:
    
    - 전체 각의 합을 이용해 $x$ 를 계산합니다:
    
    $$
    18^{\circ}+x+60^{\circ}+90^{\circ}=180^{\circ}
    $$
    
    
    $$
    \begin{aligned}
    & \Rightarrow x+168^{\circ}=180^{\circ} \\
    & \Rightarrow x=180^{\circ}-168^{\circ}=12^{\circ} .
    \end{aligned}
    $$
    
    
    각 단계를 퉁해 $\triangle N M R$ 의 각 $x$ 를 $12^{\circ}$ 로 계산할 수 있습니다.   
  
\end{GPTA}

