\begin{GPTQ}
    뭐가 발생하는건지 알 수 있을까?      
\end{GPTQ}

\begin{GPTA}
    물론입니다! 주어진 데이터에서 ``빈도"는 픅정 데이터 값이 얼마나 자주 등장하는지를 말합니다. 따라서, 아래에 각 데이터 값이 얼마나 자주 나타나는지 정리했습니다:

    - 데이터 값 6 은 3 번 나타납니다.

    - 데이터 값 7 은 3 번 나타납니다.

    - 데이터 값 8 은 8 번 나타납니다.

    - 데이터 값 9 는 8 번 나타납니다.

    - 데이터 값 10 은 9 번 나타납니다.

    - 데이터 값 11 은 11 번 나타납니다.

    - 데이터 값 12 는 9 번 나타납니다.

    - 데이터 값 13 은 나타나지 않습니다 (빈도가 0 입니다).

    - 데이터 값 14 는 6 번 나타납니다.
    
    이 표에서 알 수 있듯이, 빈도가 높은 수일수록 더 자주 그 데이터 값이 나타난 것입니다. 예를 들 어, 11 이 가장 자주 발생하여 11 번 데이터 집합에 포합됩니다. 쳐대값을 찾기 위해서는 단순히 가장 큰 숫자를 찾으면 되며, 데이터 집합 내 ``발생 여부" 즉 빈도가 0 이상인 값을 찾아야 합니다. 이 경우 가장 큰 수가 14 이며, 발생 빈도는 6 입니다. 14 는 데이터 집합에서 최대값입니다.   
\end{GPTA}

