\begin{GPTQ}
    인수분해에 대해 조금 더 자세히 알려줘 
\end{GPTQ}

\begin{GPTA}
인수분해는 다항식을 더 간단한 곱의 형태로 변형하는 과정을 말합니다. 주로 이차 방정식 형태 의 다항식에서 자주 사용뵙니다. 여기서는 방정식 $x^2+6 x-55$ 의 인수분해 과정을 자세히 설명 하젰습니다.
    
    주어진 다항식
    
    $$
    x^2+6 x-55
    $$
    
    
    이 다항식은 기본적인 형태가 $a x^2+b x+c$ 로, $a=1, b=6, c=-55$ 입니다.

    인수분해 방법은

    1. 곱셈과 합계 찾기:

    - 목표는 다음 조건을 만족하는 두 수를 찾는 것입니다:

    - 두 수의 곱이 $a \times c$ 즉, -55 가 되어야 합니다.

    - 두 수의 합이 $b$, 즉 6 이 되어야 합니다.

    2. 올바른 수를 찾기:

    - -55 의 약수 쌍 중에서 합이 6 이 되는 두 수를 찾아야 합니다.

    - 이 조건을 만족하는 수는 11 과 -5 입니다.

    - $11 \times(-5)=-55$

    - $11+(-5)=6$

    3. 인수분해 작성:

    - 이제 다항식을 다음과 같이 인수분해할 수 있습니다:
    
    $$
    x^2+6 x-55=(x+11)(x-5)
    $$
    
    
    인수분해 확인

    인수분해가 올바른지 확인하기 위해, 다시 전개를 해봅시다:

    1. 전개:
    
    $$
    \begin{gathered}
    (x+11)(x-5)=x(x-5)+11(x-5) \\
    =x^2-5 x+11 x-55 \\
    =x^2+6 x-55
    \end{gathered}
    $$
    
    
    이로써 인수분해가 을바르게 되었다는 것을 확인할 수 있습니다. 인수분해는 방정식의 해를 구하거나 다항식의 근을 분석할 때 많이 사용됩니다.   
\end{GPTA}

