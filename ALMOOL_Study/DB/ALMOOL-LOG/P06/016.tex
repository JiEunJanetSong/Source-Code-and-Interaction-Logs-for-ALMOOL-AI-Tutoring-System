\begin{GPTQ}
    한 년도에 여러 점이 그래프에 나열되는데 이건 뭐야?  
\end{GPTQ}

\begin{GPTA}
그래프에 한 해 동안 여러 점이 나열되어 있다면, 이는 각 해에 여러 번의 샘플링이 이루어졌다는 것을 의미할 수 있습니다. 이러한 샘플링 데이터는 특정 연도에 걸쳐 잉어 개체 수의 변동성을 보여주기 위해 수집되었을 가능성이 있습니다.

각 점은 특정 시점에서의 잉어 개체 수를 나타내며, 연구자는 여러 지점에서 데이터를 수집하여 전체적인 추세를 더 잘 이해하고 다양한 환경적 또는 생물학적 요인이 개체 수에 미치는 영향을 분석할 수 있습니다.

이런 방식으로 데이터가 수집되면 더 많은 정보를 통해 연구 결과의 신뢰성을 높일 수 있게 됩니다. 다른 질문이 있으면 언제든지 말씀해 주세요!
\end{GPTA}

