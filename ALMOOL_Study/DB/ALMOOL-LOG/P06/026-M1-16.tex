\begin{GPTQ}
    어떻게 풀어야할까?  
\end{GPTQ}

\begin{GPTA}
 이 문제를 푸는 방법은 각 식에 대해 $x$ 에 대한 값을 대입해보고 결과를 비교하는 것입니다. 예를 들어, $x=2$ 와 $x=100$ 일 때의 $y$ 값을 비교하여 $y$ 값의 변 화를 판단할 수 있습니다. $n$ 을 3 과 같은 상수로 설정 했을 때 각 식을 대입해볼 수 있습니다.

    각 식에 대해 다음과 같이 계산해볼 수 있습니다:

    1. (A) $y=n x+3$ :

    - $x=2, y=3 \times 2+3=9$

    - $x=100, y=3 \times 100+3=303$

    - 증가량: $303-9=294$

    2. (B) $y=-n x+3$ :

    - $x=2, y=-3 \times 2+3=-3$

    - $x=100, y=-3 \times 100+3=-297$

    - 증가량: $-297-(-3)=-294$

    3. (C) $y=x^n+3$ :

    - $x=2, y=2^3+3=11$

    - $x=100, y=100^3+3=1,000,003$

    - 증가량: $1,000,003-11=999,992$

    4. (D) $y=x^{-n}+3$ :

    - $x=2, y=2^{-3}+3=3 \frac{1}{8}=3.125$

    - $x=100, y=100^{-3}+3=3.000001$

    - 증가량: $3.000001-3.125 \approx-0.125$
    
    따라서 (C) $y=x^n+3$ 는 $x$ 가 2 에서 100 으로 변화할 때 $y$ 값의 증가가 가장 큽니다. 따라서 가장 큰 증가량을 가지는 식은 (C)입니다.  
\end{GPTA}

