\begin{GPTQ}
    이 문제를 푸는 방법 알려줘  
\end{GPTQ}

\begin{GPTA}
물론입니다. 전체 원뿔의 높이를 계산하는 방법을 단계별로 설명드리젰습니다.

    1. 지름과 반지름 설정:

    - 위쪽 원의 지름을 $d$ 라고 설정합니다.

    - 그러면 위쪽 원의 반지름은 $r=\frac{d}{2}$ 입니다.

    2. 변수 설정:

    - frustum의 높이가 $12, \mathrm{~cm}$ 이므로, 전체 원뿔의 높이를 $h$ 라고 하고, frustum의 위쪽 꼭짓 점에서 바닥까지의 높이를 $x$ 라고 합시다.

    3. 비례식을 셔우기:

    - 원뿔에서 두 삼각형이 유사한 도형임을 알 수 있습니다. 이는 다음과 같은 비례식을 이용해서 풀 수 있습니다:
    
    $$
    \frac{x+12}{\frac{34}{2}}=\frac{x}{\frac{d}{2}}
    $$
    
    4. 비례식을 풀어 $x$ 구하기:

    - 비례식을 정리해주면,
    
    $$
    \frac{2(x+12)}{3 d}=\frac{2 x}{d}
    $$
    
    - 양변에 $3 d$ 를 곱한 뒤 식을 풀면,
    
    $$
    2 x+24=6 x
    $$
    
    - 이 식을 다시 정리하면,
    
    $$
    6 x-2 x=24
    $$
    
    - 따라서,
    
    $$
    4 x=24 \quad \Rightarrow \quad x=6
    $$
    
    5. 전체 원뽈의 높이 구하기:

    - frustum의 위쪽에서 바닥까지의 높이 $x=6, \mathrm{~cm}$.
    
    - 전체 원뿔의 높이는 $x+12=18, \mathrm{~cm}$.
    
    따라서 전체 원뽈의 높이는 $18, \mathrm{~cm}$ 입니다.  
\end{GPTA}

