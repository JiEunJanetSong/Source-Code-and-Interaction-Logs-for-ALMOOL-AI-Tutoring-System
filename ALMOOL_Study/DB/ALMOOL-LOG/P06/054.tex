\begin{GPTQ}
    저 삼각비율은 어떤 공식이야?   
\end{GPTQ}

\begin{GPTA}
삼각비율은 삼각형의 각도와 그에 대응하는 변의 길이 사이의 관계를 나타냅니다. 특히 직각삼각형에서 세 가지 주요 삼각비를 사용할 수 있습니다: 사인(sin), 코사인(cos), 탄젠트(tan).

    각각의 정의는 다음과 같습니다:

    1. Sine (사인): 주어진 각도의 대변(높이)과 빗변의 비율 $\sin (\theta)=\frac{\text { 대변 }}{\text {빗변}}$

    2. Cosine (코사인): 주어진 각도의 인접변(밑변)과 빗변의 비율 $\cos (\theta)=\frac{\text { 인접변 }}{\text { 빗변 }}$.

    3. Tangent (탄젠트): 주어진 각도의 대변과 인접변의 비율 $\tan (\theta)=\frac{\text {대변 }}{\text{인접변}}$.
    
    이러한 삼각비들은 직각삼각형의 각종 문제를 해결하는데 매우 유용합니다. 
 
    위 문제에서 $\sin (\theta)=\frac{\text {대변 }}{\text{빗변}}$를 활용하여 각도를 계산했습니다. 질문이 더 있다면 자유롭게 물어보세요!
\end{GPTA}

