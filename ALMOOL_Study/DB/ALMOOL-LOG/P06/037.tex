\begin{GPTQ}
  $\mathrm{xd}+12 \mathrm{~d}=3 \mathrm{dx}$ 에서 $12 \mathrm{~d}=3 \mathrm{dx}-\mathrm{xd}$ 가 될 때 왜 xd 는 + 에서 - 로 변했어?  
\end{GPTQ}

\begin{GPTA}
  식의 양변에서 동일한 항을 제거하거나 이항할 때, 부호의 변화가 일어납니다. 이는 대수학적인 연산의 기본 법칙 중 하나입니다. 여기서 어떤 과정이 진향되는지 설명해 드리겠습니다. 
  
  주어진 식은:

$$
x d+12 d=3 d x
$$


이 식을 정리하여 $x d$ 항을 $3 d x$ 쪽으로 옮기려고 합니다. 이때, $x d$ 를 식의 오른쪽으로 옮기면 다음과 같이 됩니다:

$$
12 d=3 d x-x d
$$


이 상황에서 $x d$ 를 이항할 때 부호가 바뀌는 이유는 $-x d$ 로 표현할 때 수평적으로 왼쪽에서 오른쪽으로 항을 이동시키면서 발생합니다. 이와 같이, 항을 한쪽에서 다른 쪽으로 옮길 때는 방정식의 균형을 유지하기 위해 부호를 바꿉니다.
추가적인 설명이 펼요하다면 언제든지 말씀해 주세요!
  
\end{GPTA}

