\begin{GPTQ}
    중심각을 모르면 풀 수 없을까?   
\end{GPTQ}

\begin{GPTA}
중심각을 알지 못하더라도 원의 현의 길이를 구할 수 있는 방법이 있습니다. 주어진 원의 반지름과 현의 길이를 활용하여 관련된 삼각형의 높이(즉, 현의 중심에서 원의 중심까지의 거리)를 구할 수 있을 때 현의 길이를 간접적으로 계산할 수 있습니다.

현의 길이를 구하는 공식 (현의 높이가 주어질 경우)
만약 현의 길이를 알지 못하고 대신 원의 반지름 $r$ 과 헌의 중심에서 원의 중심까지의 거리 $d$ 를 알 고 있다면, 피타고라스 정리를 활용하여 현의 절반 길이 $\frac{C}{2}$ 를 구할 수 있습니다. 다음 공식이 유도될 수 있습니다:

$$
\left(\frac{C}{2}\right)^2=r^2-d^2
$$


이를 통해 현의 길이 $C$ 를 구할 수 있습니다:

$$
C=2 \sqrt{r^2-d^2}
$$


예제

예를 들어, 원의 반지름이 $10^{\circ}$ 이고 현의 중심에서 원의 중심까지의 거리가 8 이라면:

현의 길이를 구하는 공식에 대입하면:

$$
C=2 \sqrt{10^2-8^2}=2 \sqrt{100-64}=2 \sqrt{36}=2 \times 6=12
$$

이렇게 하면 중심각이 없이도 원내 현의 길이를 구할 수 있습니다. 이런 방법이 이해되었길 바랍니다. 추가 질문이 있으면 언제든지 물어봐 주세요!
\end{GPTA}

