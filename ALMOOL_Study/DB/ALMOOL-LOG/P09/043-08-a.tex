\begin{GPTQ}
    이 문제에서 원래 쓰는 공식이 뭐야?  
\end{GPTQ}

\begin{GPTA}
 주어진 문제는 삼각형의 내각의 합과 삼각함수를 이용한 문제입니다. 이를 풀기 위해 다음 두 가 지 공식을 사용합니다:
 
1. 삼각형 의 내각의 합: 삼각형의 모든 내각의 합은 $180^{\circ}$ 입니다. 따라서, $\triangle M N Q$ 와 같은 삼 각형에서 내각의 합은:

$$
\angle M N Q+\angle M Q N+\angle N M Q=180^{\circ}
$$

2. 삼각함수의 정의: 직각삼각형에서는 삼각함수를 사용하여 변과 각의 관계를 구할 수 있습 니다. 예를 들어, $\angle M N R$ 에서 사인 함수를 사용하여:

$$
\sin \theta=\frac{\text { 대변 길이 }}{\text { 빗변 길이 }}
$$


여기서 대변은 $\overline{M N}$ 이고, 빗변은 $\overline{M R}$ 입니다.   
   
이 두 공식을 활용하여 각 $x$ 의 값을 구하는 단계는 주어진 해설에 나와 있습니다.
   
   
\end{GPTA}

