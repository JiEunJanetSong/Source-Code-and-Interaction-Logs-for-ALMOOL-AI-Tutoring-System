\begin{GPTQ}
    양변 정리가 이해가 안 돼  
\end{GPTQ}

\begin{GPTA}
 네, 양변 정리 부분을 조금 더 천천히 설명드릴게요.

우리가 현재 가진 식은 다음과 같습니다:

$$
100(30-x)+50 x=2,400
$$


여기서 괄호를 풀어주는 것이 첫 번째 단계입니다. 

괄호는 분배법칙을 이용해 풀 수 있습니다:

$$
100 \times 30-100 \times x+50 x=2,400
$$


이 식을 계산하면:

- $100 \times 30=3,000$

- $-100 \times x=-100 x$

그래서 식은 다음과 같이 됩니다:  
   
$$
3,000-100 x+50 x=2,400
$$


여기서 $-100 x$ 와 $50 x$ 를 합쳐줍니다. 이 두 항은 같은 변수 $(x)$ 에 대한 항이므로 서로 합칠 수 있습니다:

$$
3,000-50 x=2,400
$$


다음으로, 양쪽에서 3,000 을 빼 보겠습니다. 이렇게 하면 $-50 x$ 를 단독으로 남길 수 있습니다:

$$
3,000-50 x-3,000=2,400-3,000
$$


계산하면:

$$
-50 x=-600
$$


여기까지 따라오셨다면 마지막 단계는 -50 으로 양변을 나누어 $x$ 를 구하는 것입니다:

$$
x=\frac{-600}{-50}=12
$$


결론적으로, 어린이의 수는 12 입니다. 도움이 되었기를 바랍니다!
\end{GPTA}

