\begin{GPTQ}
    이거를 좀더 이해하기 쉽게 설명해줘     
\end{GPTQ}

\begin{GPTA}
    물론이죠! 문제를 더 쉽게 이해하도록 설명드리겠습니다.
    함수 $f(w)=6 w^2$ 는 직사각형의 넓이를 나타내는데, 이때 폭은 $w$ 피트이고 길이는 폭의 6 배입니 다. 
    
    문제에서 주어진 $f(14)=1,176$ 은 $w=14$ 일 때 함수의 값을 의미합니다.

    1. 문제의 뜻: $f(14)$ 가 1,176 이라는 것은 폭이 14 피트일 때의 직사각형의 넓이가 $1,176 \mathrm{ft}^2$ 이 라는 것입니다. 이게 정답입니다.

    2. 함수의 역 할: 함수 $f(w)=6 w^2$ 는 주어진 폭 $w$ 에 따라 직사각형의 넓이를 구하는 역할을 합니다.

    3. 문제를 풀기:

    - 만약 폭 $w=14$ 피트라면, $f(14)=6 \times 14^2=1,176$ 이 되어 넓이가 $1,176 \mathrm{ft}^2$ 입니다.

    - 따라서 정답은 ``폭이 14 피트일 때, 넓이는 $1,176 \mathrm{ft}^2$ 이다"입니다.
    
    다른 선택지들은 이 문제를 잘못 해석했을 때 나올 수 있는 오류입니다. 예를 들어, $f(w)$ 를 길이로 착각하거나, 폭과 넓이의 역할을 뒤바꾸었을 수 있습니다.
    \end{GPTA}

