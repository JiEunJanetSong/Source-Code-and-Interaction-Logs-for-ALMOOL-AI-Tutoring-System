\begin{GPTQ}
    이해하기 너무 어려워    
\end{GPTQ}

\begin{GPTA}
 걱정하지 마세요! 좀 더 간단히 설명드리겠습니다.

    1. 함수 설명: $f(w)=6 w^2$ 이란 함수가 있어요. 여기서 $w$ 는 직사각형의 폭이에요. 이 함수는 넓이를 계산합니다.

    2. 주어진 값: $f(14)=1,176$ 이라는 것은, 폭이 14 피트일 때, 직사각형의 넓이가 $1,176, \mathrm{ft}^2$ 라 는 뜻이에요.

    3. 이해하기:

    - 폭 $w=14$ 피트라고 설정했어요.

    - 그러면 $f(14)=6 \times(14)^2=1,176$, 즉 넓이입니다.
    
    결론: 폭이 14 피트일 때, 직사각형의 넓이는 $1,176, \mathrm{ft}^2$. 이게 그 값의 의미에요.

    나머지 선택지는 이 해석을 잘못한 것이에요. 필요한 부분이 더 있으면 언제든지 말씀해 주세요!
  
\end{GPTA}

