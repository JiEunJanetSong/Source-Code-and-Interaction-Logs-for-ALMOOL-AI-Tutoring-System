\begin{GPTQ}
    최종적 공식   
\end{GPTQ}

\begin{GPTA}
문제를 직접 풀기 위한 공식은 결국 내부 과정에서 사용한 여러 개념들로 구성되어 있습니다. 최 종적으로 얻고자 하는 것은 각 $x$ 의 값을 구하는 것이며, 필요한 공식과 계산은 다음과 같습니다:

    1. 내각의 합 공식을 이용하여 $\angle M N Q$ 계산:
    
    $$
    \angle N M Q=180^{\circ}-(\angle M Q N+\angle M N Q)
    $$
    
    
    여기서 $\angle M Q N=72^{\circ}$ 이므로:
    
    $$
    \angle N M Q=180^{\circ}-72^{\circ}-90^{\circ}=18^{\circ}
    $$
    
    2. 삼각함수를 이용하여 $\angle M R N$ (표기한 $\theta$ ) 계산:
    
    $$
    \sin \theta=\frac{\overline{M N}}{\overline{M R}}=\frac{15}{10 \sqrt{3}}
    $$
       
    따라서,

    $$
    \theta=\sin ^{-1}\left(\frac{15}{10 \sqrt{3}}\right)
    =60^{\circ}$$
    
    3. 내각의 합을 이용하여 최종적으로 $\angle M N R$ 내 $x$ 계산:
    
    $$
    x=180^{\circ}-(\angle N M Q+\theta)
    $$
    
    
    이 과정을 통해 최종 결과는 $x=12^{\circ}$ 로 계산할 수 있습니다. 이 공식은 단계별로 문제를 해결하는 데 필요한 사전 지식과 함께 사용됩니다.

    \end{GPTA}

