\begin{GPTQ}
    아니다 저 위에 공식을 요약해줘   
\end{GPTQ}

\begin{GPTA}
 물론입니다. 원뿔의 부피 공식을 사용하여 주어진 식을 유도해보겠습니다.

    주어진 부피 $V=39,600, \mathrm{~cm}^3$ 이고, 높이 $h=18, \mathrm{~cm}$ 일 때, 공식을 통해 반지름 $r$ 를 구해야 합니다. 원뿔의 부피 공식은 다음과 같습니다:
    
    $$
    V=\frac{1}{3} \pi r^2 h
    $$
    
    이 식에 주어진 값을 대입하면:

$$
39,600=\frac{1}{3} \times \frac{22}{7} \times r^2 \times 18
$$


이 식을 정리하면:

1. $\frac{1}{3} \times \frac{22}{7} \times 18$ 은 상수로 묶을 수 있습니다.

2. 이 상수는 $\frac{396}{21}$ 이 되고, 문제의 마무리 때 나누어지게 됩니다.

따라서, 위의 식을 변형하여 $r^2$ 를 구하면:

$$
r^2=39,600 \times {\dfrac{21}{396}}
$$


이런 식으로 반지름의 제곱을 풀어 구할 수 있습니다. 도움이 더 필요하시면 알려주세요!
\end{GPTA}

