\begin{GPTQ}
    혹시 이런 문제를 풀때 쓰는 단어들과 정의를 알려줘 
\end{GPTQ}

\begin{GPTA}
물론입니다! 이런 유형의 기하 문제를 풀 때 자주 사용되는 핵심 용어와 정의를 알려드리젰습니 다.

    1. Parallel Lines (평행선): 두 직선이 동일한 평면에서 서로 만나지 않으면 평행하다고 합니다

    2. Transversal (횡단선): 두 개 이상의 다른 선을 교차하는 선입니다.

    3. Interior Angles (내각): 두 선이 평행하고 다른 선에 의해 교차될 때, 그 안쪽에 형성되는 각 입니다.

    4. Same Side Interior Angles (동측 내각): 평행선 사이에 있는 같은 쪽의 두 내각. 이 각들은 보각 관계를 가집니다, 즉 합이 $180^{\circ}$ 가 뒵니다.

    5. Alternate Interior Angles (엇각): 횡단선의 반대편에 위치한 이내각은 서로 크기가 같습니다.

    6. Supplementary Angles (보각): 두 각의 합이 $180^{\circ}$ 가 되는 각들입니다.

    7. Vertical Angles (맞꼭지각): 두 교차하는 직선이 이루는 서로 마주보는 각들은 같다고 합니다.
    
    이러한 용어와 개념을 이용하면 문제를 체계적으로 풀 수 있습니다. 도움이 되었기를 바랍니다! 다른 질문이 있으면 언제든지 물어보세요.
       
\end{GPTA}

