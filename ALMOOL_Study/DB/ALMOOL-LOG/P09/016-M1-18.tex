\begin{GPTQ}
    난 이거를 풀때 60 나누기 1440 한다음에 24나와서 나누기 2 해서 정답이 나왔어   
\end{GPTQ}

\begin{GPTA}
당신이 어떻게 문제를 풀었는지 설명해 준 것에 감사드립니다. 그러나 한 가지 점검하고 넘어가 야 할 것이 있습니다. 문제에서 요구하는 것은 그룹에 있는 어린이의 수를 찾는 것입니다. 이 문제 는 두 개의 연립방정식을 사용해야 하는 문제입니다. 문제에서는 다음과 같은 두 가지 조건을 제시합니다:

1. 성인과 어린이로 구성된 총 인원이 21 명이다: $x+y=21$ ( $x$ 는 어린이의 수, $y$ 는 성인의 수)

2. 총 수익이 1,440 달러이다: $60 x+80 y=1,440$

이 두 방정식을 해결하여 $x$ 의 값을 구하는 것이 중요합니다. 해결 과정은 다음과 같습니다: 

먼저 방정식 $x+y=21$ 를 이용하여 $y$ 에 대한 식을 세웁니다:

$$
y=21-x
$$


이 식을 두 번째 방정식에 대입합니다:

$$
\begin{gathered}
60 x+80(21-x)=1,440 \\
60 x+1,680-80 x=1,440 \\
-20 x+1,680=1,440
\end{gathered}
$$


양변에서 1,680 을 뺍니다:

$$
-20 x=-240
$$


양변을 -20 으로 나누어, $x$ 를 구합니다:

$$
x=12
$$


따라서, 그룹에 있는 어린이의 수는 12 명입니다. 60 을 1,440 으로 나누는 것은 이 문제에서는 사용 되지 않는 부분이므로, 복잡하지 않게 연립방정식을 푸는 것이 올바릅니다. 선택지 중에서 $C$ 번이 정답입니다.
\end{GPTA}

