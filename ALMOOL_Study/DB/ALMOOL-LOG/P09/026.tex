\begin{GPTQ}
    근데 $x^{\wedge} n$ 을 4 랑 곱해서 $4 x^{\wedge} n$ 이 나온거는 이해가 되는데 그러면 $x^{\wedge} n$ 곱하기 $x^{\wedge} 3 n$ 이 왜 $x^{\wedge} n$ 이 나와?   
\end{GPTQ}

\begin{GPTA}
좋은 질문입니다! $x^n$ 을 묶어내는 과정에서 혼동이 있을 수 있어요. 제가 차근차근 설명해드릴게요. 주어진 식을 살펴보면:
    
    $$
    4 x^n+x^{4 n}
    $$
    
    
    이 식을 $x^n$ 으로 묶어낼 수 있습니다. 이를 통해 다음과 같은 식을 얻습니다:
    
    $$
    4 x^n+x^{4 n}=x^n \cdot 4+x^n \cdot x^{3 n}
    $$
    
    
    위의 표현에서, $x^n \cdot x^{3 n}$ 은 지수 법칙에 따라 간단하게 만들 수 있습니다. $x^a \cdot x^b=x^{a+b}$ 라는 법칙을 사용하면,
    
    $$
    x^n \cdot x^{3 n}=x^{n+3 n}=x^{4 n}
    $$
    
    
    이제, $x^n$ 과 곱해진 부분들을 다시 하나의 식으로 묶으면:
    
    $$
    x^n\left(4+x^{3 n}\right)
    $$
    
    
    위 식은 $4 x^n+x^{4 n}$ 을 $x^n$ 으로 묶어낸 결과입니다. 따라서 $x^n$ 을 서로 곱하는 과정에서 $x^n$ 이 되는 게 아니라, $x^n$ 과 $x^{3 n}$ 을 곱해서 $x^{4 n}$ 이 되는 것입니다. 이 점을 기억하시면 좋습니다!
      
  
\end{GPTA}

