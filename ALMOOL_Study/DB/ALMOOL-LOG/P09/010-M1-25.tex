\begin{GPTQ}
    이거 이해하는 공식을 알려줘       
\end{GPTQ}

\begin{GPTA}
 문제를 이해하는 데 도움이 되는 공식을 단계별로 설명해드리겠습니다.

    주어진 문제는 두 삼각형 $R S T$ 와 $U V W$ 가 유사한 상황을 다루고 있습니다. 유사한 삼각형에서는 대응하는 각의 크기가 같고, 대응하는 변의 비율이 같습니다.

    1. 삼각형 $R S T$ 의 정보:

    - $R S=20, S T=48, T R=52$

    - $S$ 는 직각입니다.

    - 따라서 $T R$ 은 빗변이 됩니다.

    2. 삼각형 $R S T$ 와 유사한 삼각형 $U V W$ 에서 $S$ 는 $V$ 에 해당하고, $T$ 는 $W$ 에 해당합니다. 따라서 $\angle T=\angle W$ 가 됩니다.

    3. $\tan T$ 를 계산합니다:

    - $\tan T$ 는 각 $T$ 에 대해 대변과 인접변의 비율입니다.

    - 대변: $R S=20$

    - 인접변: $S T=48$

    - 따라서 $\tan T=\frac{R S}{S T}=\frac{20}{48}=\frac{5}{12}$

    4. $\tan W$ 의 값을 찾습니다:

    - 유사한 삼각형에서 대응하는 각의 탄젠트 값도 같습니다.

    - 따라서 $\tan W=\tan T=\frac{5}{12}$ 입니다.
    
    이제 $\tan W$ 의 값이 $\frac{5}{12}$ 라는 것을 알 수 있습니다. 선택지들 중에서 정답은 $\mathbf{B}, \frac{5}{12}$ 입니다.
    
\end{GPTA}

