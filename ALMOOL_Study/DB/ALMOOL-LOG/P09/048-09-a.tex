\begin{GPTQ}
    여기서 쓰는 공식들
\end{GPTQ}

\begin{GPTA}
이 문제에서 사용된 공식들은 삼각형의 내각의 합 성질과 관련이 있습니다. 자세히 살펴보면:

1. 삼각형의 내각의 합:

- 어떤 삼각형이든 세 내각의 합은 항상 $180^{\circ}$ 입니다.

- 따라서 삼각형 $A E B$ 에서 $\angle A E B+\angle D C E+\angle C D E=180^{\circ}$ 를 사용합니다.

위의 공식을 사용하여 $\angle A E B$ 를 계산할 수 있습니다. 주어진 정보에 따라 $\angle D C E=56^{\circ}$ 이고, $\angle C D E=64^{\circ}$ 이므로:

$\angle A E B+56^{\circ}+64^{\circ}=180^{\circ}$

$\angle A E B+120^{\circ} \quad=180^{\circ}$

$\angle A E B=180^{\circ}-120^{\circ}$

$\angle A E B=60^{\circ}$


위 계산 결과, $\angle A E B=60^{\circ}$ 라는 결론에 도달합니다. 이를 통해 삼각형의 성질을 이해하는데 도움을 받을 수 있습니다. 다른 관련 공식이나 개념이 필요하시다면 말씀해 주세요!
\end{GPTA}

