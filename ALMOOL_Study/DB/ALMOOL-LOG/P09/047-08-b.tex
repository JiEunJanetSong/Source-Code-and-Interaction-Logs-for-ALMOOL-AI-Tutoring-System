\begin{GPTQ}
    여기서는 무슨 방정식을 써야해?    
\end{GPTQ}

\begin{GPTA}
    이 문제에서는 주어진 삼각형을 이해하기 위해 몇 가지 방정식을 사용할 수 있습니다. 문제에서 주어진 정보를 바탕으로 아래와 같은 방법으로 접근할 수 있습니다:

    1. 삼각형의 구성 요소 이해하기:

    - 주어진 직각삼각형 $M N R$ 에서 $\angle M N R=90^{\circ}$ 입니다.

    - $\overline{M N}=15 \mathrm{~m}, \overline{M R}=10 \sqrt{3} \mathrm{~m}$ 입니다.

    2. 삼각형에서 다양한 삼각함수 관계 사용하기:

    - $\triangle M Q N$ 에서 $\tan 72^{\circ}=\frac{\overline{M N}}{\overline{N Q}}$ 를 통해 $\overline{N Q}$ 를 찾을 수 있습니다.
    
    $$
    \overline{N Q}=\frac{15}{\tan 72^{\circ}}
    $$
    
    3. 피타고라스 정리 사용하기:

    - 삼각형 $M R N$ 에서는 피타고라스 정리를 사용하여 $\overline{N R}$ 를 구할 수 있습니다.
    
    $$
    M R^2=\overline{M N}^2+\overline{N R}^2
    $$
    
    - 그리고 $N R=N Q+Q R$ 관계를 사용하여 $\overline{Q R}$ 도 구할 수 있습니다.
    
    이러한 방정식들이 문제를 해결하는 데 필요합니다. 위의 방법을 따라가며 주어진 조건을 대입하면 최종적으로 원하는 값을 찾을 수 있습니다.    
\end{GPTA}

