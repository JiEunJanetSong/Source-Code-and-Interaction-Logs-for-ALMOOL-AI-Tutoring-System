\begin{GPTQ}
    그 외에 더 있어?  
\end{GPTQ}

\begin{GPTA}
물론입니다. 기하학 문제를 풀 때 자주 쓰이는 추가 용어와 개념 몇 가지를 더 소개하겠습니다:

    1. Corresponding Angles (동위각): 두 평행선을 가로지르는 형단선에 의해 형성된, 같은 위치 에 있는 각. 이 각들은 서로 동일합니다.

    2. Adjacent Angles (인접 각): 공퉁의 변을 공유하는 두 각. 이 각들은 서로 옆에 위치합니다.

    3. Linear Pair (일직선 각쌍): 두 인접 각이 일직선을 이루면, 그 합이 $180^{\circ}$ 입니다.

    4. Congruent Angles (합동각): 크기가 동일한 각.

    5. Angle Bisector (각의 이등분선): 각을 두 개의 동일한 각으로 나누는 선분.

    6. Right Angle (직각): 크기가 $90^{\circ}$ 인 각.

    7. Acute Angle (여각): $90^{\circ}$ 보다 작은 각.

    8. Obtuse Angle (둔각): $90^{\circ}$ 보다 크고 $180^{\circ}$ 보다 작은 각.
    
    이 용어들은 다양한 기하학적 상황과 문제에 적용되어, 문장을 작성하고 해석하는 데 큰 도움이 됩니다. 도움이 되었기를 바랍니다! 더 궁금한 것이 있으면 언제든지 질문하세요.

    
\end{GPTA}

