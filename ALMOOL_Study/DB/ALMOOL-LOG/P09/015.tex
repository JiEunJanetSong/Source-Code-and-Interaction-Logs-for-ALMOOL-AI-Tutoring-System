\begin{GPTQ}
    이거를 풀때의 풀이 과정   
\end{GPTQ}

\begin{GPTA}
 문제를 단계별로 해결해 보겠습니다.

    1. 변수 설정:

    - $s$ : 작은 양초의 개수

    - $\ell$ : 큰 양초의 개수

    2. 조건 수립:
    
    - 비용 조건: 작은 양초는 한 개당 $4.90 \$$, 큰양초는 한개당 $11.60$ 입니다. 
    
    따라서, 전체비용은 $4.90s +11.60\ell$ 이고, 이는 $2,200$ 이하여야 합니다.
    
    $$
    4.90 s+11.60 \ell \leq 2,200
    $$
    
    - 최소 구매 조건: 총 양초의 개수는 최소 200 개여야 합니다.
    
    $$
    s+\ell \geq 200
    $$
    
    3. 조건 전개 및 결합:

    - 두 번째 조건을 조정하기 위해 $\ell$ 에 대해 정리합니다:
    
    $$
    s=200-\ell
    $$
    
    - 이것을 이용해 $s$ 를 대체하면:
    
    $$
    4.90(200-\ell)+11.60 \ell \leq 2,200
    $$
    
    - 괄호를 풀어 정리합니다:
    
    $$
    980-4.90 \ell+11.60 \ell \leq 2,200
    $$
    
    - $\ell$ 에 관련된 항들을 합산합니다:
    
    $$
    980+6.70 \ell \leq 2,200
    $$
    
    4. 계산:

    - 양변에서 980 을 뺍니다:
    
    $$
    6.70 \ell \leq 1,220
    $$
    
    - $\ell$ 에 대해 나누어 정리합니다:   

    $$
\ell \leq \frac{1,220}{6.70} \approx 182.09
$$

5. 결론:
- $\ell$ 은 양의 정수가 되어야 하므로, 가장 큰 정수 값을 선택합니다:

$$
\ell=182
$$


따라서, 예산 내에서 최대한 구매할 수 있는 큰 양초의 개수는 182 개입니다. 이 풀이 과정을 통해 문제를 해결할 수 있습니다. 추가 궁금한 점이 있으면 언제든지 질문하세요!
\end{GPTA}

