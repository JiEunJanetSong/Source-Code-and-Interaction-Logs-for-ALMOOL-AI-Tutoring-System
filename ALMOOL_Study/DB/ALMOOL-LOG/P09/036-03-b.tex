\begin{GPTQ}
    그러면 이 문제에서 쓰이는 공식    
\end{GPTQ}

\begin{GPTA}
 이 문제에서는 원뿔의 부피 공식을 사용합니다. 전체 원뿔의 부피는 다음과 같은 공식으로 계산 됩니다:
    
    $$
    V=\frac{1}{3} \pi r^2 h
    $$
    
    
    여기서:

    - $V$ 는 원뿔의 부피입니다.

    - $r$ 는 원뿔 밑면의 반지름입니다.

    - $h$ 는 원뿔의 높이입니다.

    - $\pi$ 는 원주율로, 이 문제에서는 $\pi=\frac{22}{7}$ 로 사용합니다.
    
    이 문제에서는 주어진 부피 $V=39,600, \mathrm{~cm}^3, h=18, \mathrm{~cm}, \pi=\frac{22}{7}$ 인 경우를 고려하여 방정식을 풀어야 합니다.

    위 정보를 사용하여 방정식을 푼 다음, 반지름 $r$ 를 구하게 됩니다. 추가적으로 설명이 필요하시면 알려주세요!
   \end{GPTA}

