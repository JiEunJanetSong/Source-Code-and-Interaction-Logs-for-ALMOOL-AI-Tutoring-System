\begin{GPTQ}
    이런 예시 문제를 또 줄수있어?   
\end{GPTQ}

\begin{GPTA}
물론입니다! 여기 직각 삼각형과 관련된 또 다른 예시 문제를 드릴게요.

문제:

한 직각 삼각형의 두 변의 길이가 각각 9 cm 와 12 cm 입니다. 이 삼각형의 빗변의 길이 $h$ 를 구하고, 만약 이 $h$ 가 $k \sqrt{d}$ 의 형태로 나타낼 수 있다면, $k$ 와 $d$ 의 값을 구하세요.

해결과정:

1. 주어진 직각 삼각형의 두 변의 길이는 9 cm 와 12 cm 입니다.

2. 피타고라스 정리를 사용하여 빗변의 길이를 구합니다:

$$
h^2=9^2+12^2=81+144=225
$$

3. 따라서, $h=\sqrt{225}$.

4. 225 는 $15^2$ 이므로, $h=15$ 입니다.

5. 이 $h$ 는 이미 자연수입니다. 그러나 문제의 형식에 맞추어 $15=3 \sqrt{25}$ 라고 쓸 수 있습니다.

따라서 $k=3$ 이고 $d=25$ 입니다.

이런 방식으로 다양한 문제를 연습하면 개념을 더 잘 이해하는 데 도움이 될 거예요. 다른 질문이나 추가 문제가 필요하시면 언제든지 말씀해 주세요!

 
\end{GPTA}

