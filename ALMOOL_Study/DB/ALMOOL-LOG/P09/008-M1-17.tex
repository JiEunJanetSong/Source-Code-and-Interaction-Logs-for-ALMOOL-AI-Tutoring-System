\begin{GPTQ}
    만약에 여기서 파이를 모르면은 어떻게 해?     
\end{GPTQ}

\begin{GPTA}
    파이 $(\pi)$ 는 수학에서 매우 중요한 상수입니다. 대략 3.14159 라는 값을 가지고 있으며, 이 값은 원의 지름에 대한 원의 둘레의 비율입니다. 대부분의 수학 및 과학 문제에서는 이 값을 그대로 사용 하거나 3.14 또는 $\frac{22}{7}$ 로 근사하여 계산을 수행할 수 있습니다.
    만약 위의 문제에서 파이 값을 모르지만, 대략적인 값을 사용해야 한다면, 다음과 같은 방법으로 접근할 수 있습니다:

    1. 문제에서 주어진 원의 둘레 $144 \pi$ 는 실제로 $144 \times 3.14$ 로 계산할 수 있습니다.

    2. $x+x+2 x+2 x=144 \pi$ 가 주어졌을 때, 파이를 대략 3.14 로 두고, 식을 풀어 나가면 $6 x=144 \times 3.14$ 가 됩니다.

    3. 이 식을 통해 $x$ 값을 구할 수 있으며, 그 값을 통해 궁극적으로 아크 $Q R$ 의 길이를 계산할 수 있습니다.
    
    그러나 대부분의 경우, 소수점 근사는 계산의 정확성을 떨어뜨릴 수 있으므로, 가능한 경우에는 $\pi$ 의 기호 그대로 계산하는 것이 좋습니다.
    \end{GPTA}

