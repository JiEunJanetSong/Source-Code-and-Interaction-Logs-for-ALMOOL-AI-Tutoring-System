\begin{GPTQ}
    호의 중심각 계산법만 알려줘 
\end{GPTQ}

\begin{GPTA}
호의 중심각 계산법을 설명해드리겠습니다.

    주어진 원의 반지름 $r$ 와 호의 길이 $L$ 이 있을 때, 중심에서 이 호가 만드는 각 $\theta$ (도 단위)는 다음 공식을 이용하여 계산할 수 있습니다:
    
    $$
    L=\frac{\theta}{360^{\circ}} \times 2 \pi r
    $$
    
    
    여기서:

    - $L$ 은 호의 길이,

    - $r$ 은 반지름,

    - $\pi$ 는 원주율입니다.
    
    이 공식을 이용해 $\theta$ 를 구하는 방법은 다음과 같습니다:
    
    1. 위의 공식을 $\theta$ 에 대해 풀어줍니다:
    
    $$
    \theta=\frac{L \times 360^{\circ}}{2 \pi r}
    $$
    
    2. 주어진 값을 대입합니다:
    
    $$
    L=14, \mathrm{~cm},, r=7, \mathrm{~cm}, \pi=\frac{22}{7}
    $$
    
    3. 위 공식을 통해 $\theta$ 를 구합니다.    
\end{GPTA}

