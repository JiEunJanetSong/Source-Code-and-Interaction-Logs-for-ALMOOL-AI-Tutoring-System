\begin{GPTQ}
    왜 이게 정답이야?
\end{GPTQ}

\begin{GPTA}
문제를 풀어 설명드리겠습니다.

    로켓은 발사 전에는 $467,000 \mathrm{~kg}$ 의 추진제를 가지고 있었고, 발사 21 초 후에는 $362,105 \mathrm{~kg}$ 의 추진제 가 남아있었습니다. 이는 로켓이 21 초 동안 연소한 추진제 양을 다음과 같이 계산할 수 있다는 뜻 입니다:
    
    $$
    467,000-362,105=104,895
    $$
    
    
    따라서 21 초 동안 $104,895 \mathrm{~kg}$ 의 추진제가 연소되었습니다. 이 양을 21 초로 나누면, 초당 평균 연소 양을 구할 수 있습니다:
    
    $$
    \frac{104,895}{21}=4,995
    $$
    
    
    따라서, 로켓은 평균적으로 매초 $4,995 \mathrm{~kg}$ 의 추진제를 연소했습니다. 이는 선택지 중에서 $\mathbf{A}$ 입니다.

    \end{GPTA}

