\begin{GPTQ}
    what do you have to do to get the value of $x$? 
\end{GPTQ}

\begin{GPTA}
    To find the value of $x$, which represents the height of the upper portion of the cone that was removed to create the frustum, follow these steps:

    Steps to Find $x$ :

    1. Establish Proportional Relationships:

    - Use similar triangles formed by the cross-section of the cone and the frustum. The top porti on of the cone that was removed and the whole cone share a similar triangle relationship.

    2. Define Dimensions:

    - Let the diameter of the top circle of the frustum be $d$. Hence, its radius is $\frac{d}{2}$.

    - The diameter of the bottom circle is given as $3 d$, so its radius is $\frac{3 d}{2}$.

    3. Set Up a Proportion Using Similar Triangles:

    - The proportionality relates the entire cone and the smaller cone (from which the top was re moved):
    
    $$
    \frac{x+12}{\frac{3 d}{2}}=\frac{x}{\frac{d}{2}}
    $$
    
    4. Cross Multiply:

    - This helps eliminate the fractions:
    
    $$
    (x+12) \cdot \frac{d}{2}=x \cdot \frac{3 d}{2}
    $$
    
    5. Solve the Equation:

    - Distribute and simplify:
    
    $$
    d x+12 d=3 d x
    $$
    
    - Rearrange to isolate terms involving $x$ :
    
    $$
    2 d x=12 d \quad \Rightarrow \quad x=\frac{12 d}{2 d}=6
    $$
    
    
    So, $x=6, \mathrm{~cm}$.

    Through these steps, the value of the height of the removed cone (the upper segment) is determined, wh ich is then used to find the original total height of the cone.

    If you have any further questions or need additional clarification, feel free to ask!

\end{GPTA}

