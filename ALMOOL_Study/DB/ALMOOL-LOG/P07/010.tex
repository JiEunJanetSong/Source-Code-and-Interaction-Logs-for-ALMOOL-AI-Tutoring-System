\begin{GPTQ}
    how did you even come up with $1 / 2 \mathrm{~d}$ ?   
\end{GPTQ}

\begin{GPTA}
The $\frac{1}{2} d$ comes from converting the diameter of a circle to its radius, which is necessary for many geom etric calculations.  

Explanation:

1. Basic Circle Geometry:

- The radius $r$ of a circle is half of its diameter $d$.

2. Calculating the Radius:

- Therefore, if you have a diameter $d$, the radius $r$ is given by:

$$
r=\frac{d}{2}
$$


Application to the Problem:

- In the context of this problem, we have two circles related to the frustum:

1. Top Circle (smaller):

- Diameter $=d$

- Radius $=\frac{d}{2}$

2. Bottom Circle (larger):

- Diameter $=3 d$ (as given, it's three times the top diameter)

- Radius $=\frac{3 \mathrm{~d}}{2}$

By using these radii, we can apply the properties of similar triangles formed by the different parts of the truncated cone to solve for the height $x$ and then find the total height of the original cone. If there's any other part of the solution you'd like more details about, just let me know!

\end{GPTA}

