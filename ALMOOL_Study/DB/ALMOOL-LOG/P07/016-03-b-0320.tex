\begin{GPTQ}
    what are two ways to solve this problem 
\end{GPTQ}

\begin{GPTA}
Let's explore two different methods to solve this problem of finding the radius of a cone when you are g iven its volume.

Method 1: Using the Volume of the Cone Formula

1. Start with the volume formula for a cone:

$$
V=\frac{1}{3} \pi r^2 h
$$

2. Rearrange to solve for $r^2$ :

$$
r^2=\frac{3 V}{\pi h}
$$

3. Substitute the known values:

Given $V=39,600 \mathrm{~cm}^3, h=18 \mathrm{~cm}$, and $\pi=\frac{22}{7}$,

$$
r^2=\frac{3 \times 39,600}{\frac{22}{7} \times 18}
$$

4. Simplify and solve for $r$ :

$$
\begin{aligned}
& r^2=\frac{831,600}{396}=2,100 \\
& r=\sqrt{2,100} \approx 45.83 \mathrm{~cm}
\end{aligned}
$$


Method 2: Using Proportions and Scaling

Given the constraints of the problem, the diameter of the frustum's base (bottom) is three times that of $t$ he top, creating a larger imaginary cone. However, here we focus on the given cone's volume.
1. Recognize that if another similar shape problem was asked (which it isn't in this case), scali ng factors could be used. The problem already provides the key formula for direct calculati on of $r$.

By using the formula directly, you get the radius as calculated. This second specific methodology emph asizes potential workarounds for imaginary problems but your problem gives direct volume already set $f$ or solving.

Comparison and Conclusion
While Method 1 is straightforward and directly uses the cone's volume formula, Method 2 demonstrates the awareness of scaling within geometry problems. For this exact problem, Method 1 is the way to app roach it due to the known values and formula setup. If new scenarios with less direct information or mul tiple cones had been involved, Method 2 could potentially be helpful with geometrical insights.
Let me know if this clarifies your question or if you'd like assistance with another scenario!
\end{GPTA}

