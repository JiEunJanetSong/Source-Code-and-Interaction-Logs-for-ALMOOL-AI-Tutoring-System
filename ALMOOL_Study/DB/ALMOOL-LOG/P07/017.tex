\begin{GPTQ}
    show method 2 in numbers  
\end{GPTQ}

\begin{GPTA}
Sure! Let's delve into Method 2 using numerical calculations. We'll explore how understanding similarit $y$ and scaling can be utilized, even if the direct approach is more suitable. However, this method general ly applies to problems involving similar-sized objects or derived properties from them, such as changin g dimensions while keeping the volume in mind.

Method 2: Understanding Scaling and Proportions

In this method, we'll conceptually derive using scaling techniques if there were a need to adjust sizes ba sed on known dimensions rather than using direct volume calculation.

Conceptual Background:

Without changing the character of the given problem, let's consider a hypothetical situation where you might need to consider, for example, the full cone reenacting from the frustum, identifying the propertie $s$ based on proportional changes. The math below indicates how proportionally related changes can be $s$ ensed from given parameters, though direct use won't change here:

1. Ratio Recognition:

Suppose the cone was scaled by a known factor. Ratio can help separate derivation:

- A given frustum might have top circle proportional to new geometry.

Here, knowledge helps perceive similar shapes by trends, yet direct number switching won't adjust:

2. Volume Relation and Changes for such Scale Hypothetical:

If dimensions had similar property adjustments, finding relationships, such as doubling radius $r \mathrm{c}$ hanges, needs practical calc:

$$
V \propto r^2 h
$$

3. Derive Finding Details:

None explicitly alter original text yet, but ratio thinking prepares future complex geometric adjustment.

4. Use Ratio Help Understanding

Zero conversion here for each specified parameter derives as:

$$
r_{\text {ideal stories }}=\sqrt{\text { number theory uses }}
$$

This serves conceptually in generic advanced mathematical concept preps to solve potential new scale-b ased projection, while basic volume formula remained true.
The essence of using proportions in geometry is understanding the potential for multiple dimensions ma nipulating volume properties, but direct details rely on pre-given calculation spelled out as last seen. This method leans into recognizing relations, but original solutions validate. Use this mode when dealin g with transformative problems beyond this scope as a learning preparation exercise!

\end{GPTA}

