\begin{GPTQ}
    give me the first step on how to solve this 
\end{GPTQ}

\begin{GPTA}
Certainly! The first step to solve this problem involves setting up the relationship between the sections of the frustum and the whole cone using similar triangles.

    1. Identify Known Ratios:

    - Let $d$ be the diameter of the top circle of the frustum. Then, the diameter of the bottom circl e is $3 d$.

    - Therefore, the radius of the top circle is $\frac{d}{2}$ and the radius of the bottom circle is $\frac{3 d}{2}$.

    2. Set Up a Proportion Using Similar Triangles:

    - Consider the full height of the original cone as $h$.

    - The height of the frustum is given as $12, \mathrm{~cm}$.
    
    You can set up a proportion between two similar triangles: the triangle representing the entire cone and t he triangle representing the missing top portion of the cone.
    
    $$
    \frac{x+12}{\frac{3 d}{2}}=\frac{x}{\frac{d}{2}}
    $$
    
    
    Here, $x$ represents the height of the missing top portion of the cone. Solving this equation will help find $x$ and subsequently the height $h$ of the entire cone. Let me know if you have any more questions!

\end{GPTA}

