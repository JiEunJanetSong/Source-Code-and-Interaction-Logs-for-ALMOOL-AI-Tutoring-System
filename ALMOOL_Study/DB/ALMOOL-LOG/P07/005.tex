\begin{GPTQ}
    what is used to solve this? dont tell me the asnwer tho 
\end{GPTQ}

\begin{GPTA}
 To solve this problem, you can use the concept of similar triangles.

    When dealing with the dimensions of a frustum derived from a cone, the triangles formed by the height of the entire cone and the height of the frustum with corresponding radii on both ends are similar.

    Here's a brief outline of the approach:

    1. Set up the ratio of the radii of the top and bottom circles in the frustum.

    2. Use the ratio of these radii to set up an equation involving the unknown height of the whole cone.

    3. Solve this equation to find the height of the entire cone.
    
    If you need more explanation on any specific steps or concepts, feel free to ask!
    \end{GPTA}

