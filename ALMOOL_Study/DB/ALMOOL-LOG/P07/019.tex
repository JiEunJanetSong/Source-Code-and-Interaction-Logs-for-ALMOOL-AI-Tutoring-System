\begin{GPTQ}
    show me two ways to solve this 
\end{GPTQ}

\begin{GPTA}
 Certainly! Let's go through two different methods to find $\angle R S T$.

    Method 1: Using Alternate Angles and Isosceles Triangle Properties

    1. Since $\overline{Q T} \| \overline{R S}$, the angles $\angle R T Q$ and $\angle T R S$ are equal because they are alternate interior angles formed by the transversal $\overline{T R}$ intersecting the parallel lines. Therefore, $\angle T R S=33^{\circ}$.

    2. In $\triangle R S T$, we are given that $|\overline{R S}|=|\overline{S T}|$, which means $\triangle R S T$ is isosceles with $\angle T R S=\angle S T R$.

    3. Since $\angle T R S=\angle S T R=33^{\circ}$, we can write:
    
    $$
    \angle T R S+\angle S T R+\angle R S T=180^{\circ}
    $$
    
    
    Substituting the known angles:
    
    $$
    33^{\circ}+33^{\circ}+\angle R S T=180^{\circ}
    $$
    
    4. Solving for $\angle R S T$ :
    
    $$
    \begin{gathered}
    66^{\circ}+\angle R S T=180^{\circ} \\
    \angle R S T=180^{\circ}-66^{\circ}=114^{\circ}
    \end{gathered}
    $$
    
    
    Method 2: Using Exterior Angle Theorem
    
     1. Observe that $Q T \| R S$ and $\angle R T Q$ is given as $33^{\circ}$. This makes the angle $\angle T R S$ also $33^{\circ}$ due to alternate angles.
2. According to the Exterior Angle Theorem, the exterior angle $\angle R S T$ of $\triangle R S T$ is equal to the su m of the two opposite interior angles (which are $\angle T R S$ and $\angle S T R$ ).
3. Since $\triangle R S T$ is isosceles with $|\overline{R S}|=|\overline{S T}|$ :

$$
\angle T R S=\angle S T R=33^{\circ}
$$

4. Therefore, $\angle R S T$ is:

$$
\angle R S T=\angle T R S+\angle S T R=33^{\circ}+33^{\circ}=66^{\circ}
$$

5. Checks:

The sum of angles in the triangle must be $180^{\circ}$, since $\angle R S T=114^{\circ}$ provides a consistent soluti on from both methods, showing steps lead to $\angle R S T=114^{\circ}$ by appropriate means as expected.

Both methods allow us to conclude that $\angle R S T=114^{\circ}$.
\end{GPTA}

