\begin{GPTQ}
    give me each step on how to solve this problem 
\end{GPTQ}

\begin{GPTA}
Certainly! Let me guide you through each step to solve the problem.

    Problem:

    We have a frustum of a cone where the bottom diameter is three times the top diameter. The height of th e frustum is given as $12, \mathrm{~cm}$, and we need to calculate the height of the original whole cone.
    
    Steps:

    1. Identify the Variables and Given Information:

    - Let the diameter of the top circle (small circle) be $d$.

    - Then, the radius of the top circle is $\frac{d}{2}$.

    - The diameter of the bottom circle (large circle) is $3 d$, so the radius is $\frac{3 d}{2}$.

    - The height of the frustum is $12, \mathrm{~cm}$.

    2. Set Up the Equation Using Similar Triangles:

    - Let $x$ be the height of the cone above the frustum (i.e., the height of the portion of the cone cut off at the top).

    - Thus, the total height of the original cone is $h=x+12$.
    
    Using similar triangles (proportional radii and heights), we set up the equation as:
    
    $$
    \frac{x+12}{\frac{3 d}{2}}=\frac{x}{\frac{d}{2}}
    $$
    
    3. Solve the Equation for $x$ :

    - Cross-multiply to get:
    
    $$
    (x+12) \cdot \frac{d}{2}=x \cdot \frac{3 d}{2}
    $$
       
- Simplify the equation:

$$
2 x+24=6 x
$$

- Rearrange terms and solve for $x$ :

$$
6 x-2 x=24 \Longrightarrow 4 x=24 \Longrightarrow x=6
$$

4. Calculate the Total Height of the Cone:

- The total height of the original cone $h$ is:

$$
h=x+12=6+12=18, \mathrm{~cm}
$$


Thus, the height of the whole cone is $18, \mathrm{~cm}$.
If you need any more clarification, feel free to ask!

\end{GPTA}

