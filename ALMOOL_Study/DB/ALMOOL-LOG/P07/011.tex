\begin{GPTQ}
    frustrum meaning  
\end{GPTQ}

\begin{GPTA}
  A frustum is a geometric shape that results when a cone or pyramid is sliced by a plane parallel to its b ase, effectively removing the upper part. The remaining shape still possesses the same base, but with a $t$ runcated top, forming a new, smaller, parallel base.

Key Characteristics of a Frustum:

1. Generated from a Cone or Pyramid:

- The original three-dimensional shape is either a cone (with a circular base) or a pyramid (w ith a polygonal base).

2. Parallel Bases:

- A frustum's top and bottom are parallel to each other. In the case of a cone, these bases are circles, while in a pyramid, the bases are similar polygons.

3. Truncated Section:

- The frustum is the part of the cone or pyramid that remains after the top is removed.

4. Applications:

- Frustums are common in architectural and engineering designs, like lampshades, cups, and chimneys.

A visual representation can be helpful to fully grasp the concept. Let me know if you would like a detailed diagram or additional examples!

\end{GPTA}

