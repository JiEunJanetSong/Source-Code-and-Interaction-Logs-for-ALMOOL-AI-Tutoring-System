\begin{GPTQ}
    how did you find the x value?  
\end{GPTQ}

\begin{GPTA}
    To find the value of $x$, we used the concept of similar triangles. Here's a breakdown of that process:

    1. Set Up the Proportion:

    - We established the similarity between triangles as:
    
    $$
    \frac{x+12}{\frac{3 d}{2}}=\frac{x}{\frac{d}{2}}
    $$
    
    
    This equation is derived from the fact that the triangles formed by the height of the frustum and the remainder of the cone (from the top) are similar, due to equivalent angles and propo rtional sides.

    2. Cross-Multiply:

    - Cross-multiplying the equation to eliminate the fractions:
    
    $$
    (x+12) \cdot \frac{d}{2}=x \cdot \frac{3 d}{2}
    $$
    
    
    Simplifying brings us to:
    
    $$
    d(x+12)=3 d x
    $$
    
    3. Distribute and Simplify:

    - Distribute the $d$ on the left-hand side:
    
    $$
    d x+12 d=3 d x
    $$
    
    - Subtract $d x$ from both sides:
    
    $$
    12 d=3 d x-d x=2 d x
    $$
    
    4. Solve for $x$ :

    - Divide both sides of the equation by $2 d$ to solve for $x$ :
    
    $$
    \frac{12 d}{2 d}=x \Rightarrow x=6
    $$
    
    
    This leads us to the conclusion that the height of the portion of the cone that was cut off (above the frust um ) is $6, \mathrm{~cm}$.

    If you have further questions on these steps, feel free to let me know!

\end{GPTA}

