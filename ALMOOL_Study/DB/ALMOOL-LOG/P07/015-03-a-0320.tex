\begin{GPTQ}
    what are two ways to solve this problem?
\end{GPTQ}

\begin{GPTA}
    To solve this problem of finding the original height of a cone with a frustum where the bottom diameter is three times the top diameter, you can use two different approaches: similar triangles and direct algebr aic manipulation.

Method 1: Similar Triangles

1. Identify Proportions:

- Let the diameter of the top circle of the frustum be $d$, so its radius is $\frac{d}{2}$.

- The bottom circle has a diameter $3 d$ and radius $\frac{3 d}{2}$.

- The original full height of the cone is assumed to be $x+12$ after partitioning it, with $x$ bei ng the height of the portion above the frustum.

2. Set Up Similar Triangles:

- The triangles formed by the heights and corresponding radii of the frustum and the entire c one are similar:

$$
\frac{x+12}{\frac{3 d}{2}}=\frac{x}{\frac{d}{2}}
$$

3. Cross-Multiply and Simplify:

- Solve the proportion by cross-multiplying:

$$
(x+12) \cdot \frac{d}{2}=x \cdot \frac{3 d}{2}
$$

- Simplifying:

$$
d x+12 d=3 d x \quad \Rightarrow \quad 12 d=2 d x
$$

- Solve for $\boldsymbol{x}$ :

$$
x=\frac{12 d}{2 d}=6
$$

4. Find Full Cone Height:

- Total height becomes $h=x+12=6+12=18, \mathrm{~cm}$.

Method 2: Algebraic Manipulation

1. Establish Relationship Directly:

- Given that the frustum's bottom diameter is three times the top, the segments' heights relate strongly:

$$
\frac{x+12}{x}=\frac{3}{1}
$$

2. Manipulate Algebraically:

- From the proportion, express the equation:

$$
x+12=3 x
$$

- Solve by rearranging:

$$
3 x-x=12 \quad \Rightarrow \quad 2 x=12 \quad \Rightarrow \quad x=6
$$

3. Determine Total Height:

- Calculate the full height:

$$
h=x+12=6+12=18, \mathrm{~cm}
$$


Both methods lead to the original cone height of $18, \mathrm{~cm}$, just approaching the proportions slightly differ ently with geometric similarity or algebraic manipulation.

If there are particular parts of these methods that require more explanation, please let me know!
\end{GPTA}

