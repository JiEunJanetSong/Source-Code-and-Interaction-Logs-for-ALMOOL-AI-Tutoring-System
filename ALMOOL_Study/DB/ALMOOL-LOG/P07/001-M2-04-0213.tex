\begin{GPTQ}
    how do you find the value of $f(0)$ ?  
\end{GPTQ}

\begin{GPTA}
    To find the value of $f(0)$, we need to look at the graph of $y=f(x)$ and find the $y$-coordinate when $x=0$. This process involves the following steps:

    1. Locate $x=0$ on the $x$-axis: On the horizontal axis of the graph, find the point where $x=0$. This is typically where the axis labels are marked or where the axis crosses vertically.

    2. Find the $y$-coordinate at $x=0$ : From this point, move vertically along the graph to find where t he curve intersects the line $x=0$.

    3. Read the corresponding $y$-value: The $y$-coordinate at this point of intersection is the value of $f(0)$.
    
    For the graph provided, when $x=0$, the corresponding $y$-value is 3. Therefore, $f(0)=3$. This means c hoice $D$ is correct.
    \end{GPTA}

