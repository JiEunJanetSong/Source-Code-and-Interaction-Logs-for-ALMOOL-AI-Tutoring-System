\begin{GPTQ}
3번에서 체인를 적용을 보여줄래?
\end{GPTQ}

\begin{GPTA}
물론입니다. 체인 룰을 자세히 사용하여 미분하는 과정을 보여드리겠습니다.

우리는 $f(x)=2(\sin (\sqrt{x}))^2$ 의 미분을 찾고 있습니다. 이 함수는 합성 함수이므로 체인 룰을 적용합니다.

전체 과정으

1. 외부 함수 식별: 외부 함수는 $v=2 u^2$ 이며, 여기서 $u=\sin (\sqrt{x})$ 입니다.

2. 외부 함수의 미분: 외부 함수 $v=2 u^2$ 를 $u$ 에 대해 미분하면:

$$
\frac{d v}{d u}=4 u
$$

3. 내부 함수의 미분: 내부 함수 $u=\sin (\sqrt{x})$ 의 $x$ 에 대한 미분은 전에 계산했던 대로:

$$
\frac{d u}{d x}=\frac{\cos (\sqrt{x})}{2 \sqrt{x}}
$$

4. 체인 룰 적용: 체인 를은 다음과 같이 표현됩니다:

$$
\frac{d v}{d x}=\frac{d v}{d u} \cdot \frac{d u}{d x}
$$


외부 함수의 미분과 내부 함수의 미분을 체인 룰로 결합하면:

$$
\frac{d v}{d x}=4 u \cdot \frac{\cos (\sqrt{x})}{2 \sqrt{x}}
$$

5. $u$ 를 대입: $u=\sin (\sqrt{x})$ 를 다시 대입하여 정리하면:

$$
\begin{gathered}
\frac{d}{d x}\left(2(\sin (\sqrt{x}))^2\right)=4 \sin (\sqrt{x}) \cdot \frac{\cos (\sqrt{x})}{2 \sqrt{x}} \\
=\frac{2 \sin (\sqrt{x}) \cos (\sqrt{x})}{\sqrt{x}}
\end{gathered}
$$


이맇게 해서 체인 룰을 사용하여 미분을 구할 수 있습니다.
\end{GPTA}

