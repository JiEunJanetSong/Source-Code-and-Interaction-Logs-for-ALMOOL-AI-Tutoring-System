\begin{GPTQ}
frustrum 이랑 cone 의 높이는 달라?  
\end{GPTQ}

\begin{GPTA}
맞습니다. 프러스텁과 원뿔의 높이는 다를 수 있습니다. 원뿔의 전체 높이 $h$ 는 원뿔의 밑면에서 꼭대기까지의 직선 높이를 의미합니다. 반면, 프러스텀(절두체)의 높이는 잘려진 원뿔의 아래쪽 면에서 위쪽 면까지의 직선 높이를 의미합니다.
문제에서는 프러스텀이 있는 원뿔의 전체 높이를 18 cm 로 설정 하고, 프러스텀의 높이를 12 cm 로 주어졌습니다.
만약 더 구체적인 도움이 필요하면 말씀해 주세요.
\end{GPTA}

