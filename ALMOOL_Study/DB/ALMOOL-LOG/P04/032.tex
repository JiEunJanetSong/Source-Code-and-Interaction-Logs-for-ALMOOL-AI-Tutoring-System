\begin{GPTQ}
정답이 10 은 왜 안돼?    
\end{GPTQ}

\begin{GPTA}
변곡점은 $f^{\prime \prime}(x)=0$ 이면서 부호가 바뀌어야 합니다. 주어진 두 번째 미분 $f^{\prime \prime}(x)=x(x-3)^5(x-10)^2$ 에서는 각각의 값에서 다음과 같은 결과 를 확인할 수 있습니다:
1. $x=0: x(x-3)^5(x-10)^2$ 에서 $x$ 의 계수가 1 로 흘수입니다. 이는 부호가 바폅을 의미하므로 변곡점입니다.
2. $x=3:(x-3)^5$ 의 계수가 5 로 휼수입니다. 이것 역시 부호가 바뀌는 것을 의미하므로 변곡점입니다.
3. $x=10:(x-10)^2$ 의 계수가 2 로 짝수입니다. 짝수 차수는 부호가 변하지 않음을 의미하므로 변곡점이 아닙니다.

따라서 $x=10$ 에서는 부호가 바껴지 않고, 이는 변곡점의 조건이 만족되지 않아서 변곡점이 아닙니다. 변곡점은 $x=0$ 과 $x=3$ 에서만 발생합니다.
\end{GPTA}

