\begin{GPTQ}
B 보기랑 C 보기랑 다른게 뭐야? 
\end{GPTQ}

\begin{GPTA}
보기 B 와 보기 C 의 차이를 설명해 드리겠습니다.

보기 B: $\lim _{x \rightarrow 5} f(x)$ 가 존재 하지 않는다.

- 이 문장은 $x$ 가 5 에 가까워질 때 함수 $f(x)$ 의 값이 특정한 유한한 값에 수렴하지 않는다는 의미입니다.

- 이 한계(limit)가 존재하지 않는다면, 함수의 그래프는 $x=5$ 에서 불연속(incontinuous)일 가능성이 큽니다.

- 하지만, 함수가 $x=5$ 에서 불연속적이긴 해도, 미분 가능성과 직접적인 관계가 없기 때문에 반드시 참이거나 거짓이라고 할 수 없습니다.

보기 C: $\lim _{x \rightarrow 5} \frac{f(x)-f(5)}{x-5}$ 가 존재하지 않는다.

- 이 문장은 함수 $f(x)$ 가 $x=5$ 에서 미분 가능하지 않다는 것을 직접적으로 나타내는 표현입니다.

- 이 극한은 미분의 정의로, 함수가 특정 점에서 미분 가능하려면 이 극한이 유한한 값으로 존재해야 합니다.

- 따라서 이 극한이 존재하지 않는다는 것은 합수가 그 점에서 미분 가능하지 않음을 의미합니다.

\textbf{차이점 정리}

- 보기 B 는 함수 값 그 자체의 연속성에 관한 것입니다. 함수가 연속적인지 아닌지를 말해주지만, 미분 가능성과는 직접적인 관련이 없습니다.

- 보기 C 는 미분 가능성 자체를 의미하며, 미분 정의에서 나온 극한이 존재하지 않음을 뜻합니다. 따라서 함수가 미분 가능하지 않다는 정 보를 직접적으로 제공합니다.

따라서, 함수가 미분 가능하지 않다는 조건에서는 보기 C가 반드시 참이 됩니다.
\end{GPTA}

