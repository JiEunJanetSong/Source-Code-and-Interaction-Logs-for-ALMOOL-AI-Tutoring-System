\begin{GPTQ}
보기 D는 뭐야 해석해줘   
\end{GPTQ}

\begin{GPTA}
확인해 보겠습니다. 보기 D 의 설명은 다음과 같습니다:

``D: $\int_0^5 f(x), d x$ does not exist."
이 문장은 $f(x)$ 라는 함수가 0 에서 5 까지의 구간에서 적분이 존재하지 않는다는 의미입니다. 적분이 존재하려면 함수가 구간 내에서 적어도 구 간별 연속이어야 하고, 함수 값이 무한하거나 이상하지 않아야 합니다.

적분은 함수 아래의 넓이를 계산하는 과정이기 때문에 함수가 특정 점에서 미분 가능하지 않더라도, 전체 구간 내에서 녋이를 계산할 수 있으면 적분이 잘 정의될 수 있습니다. 

예를 들어, $f(x)=|x|$ 는 $x=0$ 에서 미분 가능하지 않지만, 해당 구간의 적분 $\int_0^5|x|, d x$ 는 잘 정의되어 있습니다. 

따라서, 적분이 존재하지 않는다는 문장은 함수가 구간 내에서 연속되지 않거나, 함수 값이 특정 조건으로 인해 적분이 불가능한 경우를 포함합니다.
\end{GPTA}

