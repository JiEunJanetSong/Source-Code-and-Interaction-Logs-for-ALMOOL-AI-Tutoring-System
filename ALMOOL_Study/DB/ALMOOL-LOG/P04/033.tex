\begin{GPTQ}
계수와 짝수 휼수랑 무슨 상관이야?   
\end{GPTQ}

\begin{GPTA}
계수라는 용어 대신 ``지수" 또는 ``차수"라는 말을 사용하는 것이 더 적절할 것 같습니다. 여기서 중요한 것은 각 항의 지수(차수)와 부호 변화의 관계입니다.
1. 짹수 차수: 짝수 차수 $(n)$ 는 해당 항의 부호가 바뀌지 않음을 의미합니다. 예를 들어 $(x-10)^2$ 는 $x=10$ 에서의 차수가 2 입니다. 해당 항이 제곱이므로, $x$ 를 조금씩 증가 또는 감소시켜도 전체 식의 부호는 변하지 않습니다. 이는 $(a)^2$ 과 $(-a)^2$ 이 항상 양수인 것과 같습니다.
2. 흘수 차수: 흘수 차수 $(n)$ 는 해당 항의 부호가 바펍을 의미합니다. 예를 들어 $(x-3)^5$ 는 $x=3$ 에서 차수가 5 로 홀수입니다. 해당 항이 흘수 차수이기 때문에 $x$ 를 기준으로 양쪽에서 값을 넣었을 때 부호가 반대가 됩니다. 이는 $(a)^3$ 과 $(-a)^3$ 이 각각 양수와 음수가 되는 것과 같습 니다.
따라서 변곡점에서는 $f^{\prime \prime}(x)$ 가 0 이 되는 점을 찾고, 그 주변에서 부호가 바뀌는지를 확인하기 위해 해당 항의 차수가 중요한 역할을 합니다.
\end{GPTA}

