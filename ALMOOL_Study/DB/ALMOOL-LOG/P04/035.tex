\begin{GPTQ}
단계별로 풀어줄래?
\end{GPTQ}

\begin{GPTA}
물론입니다. 주어진 문제를 단계별로 풀어보겠습니다.

문제는 다음과 같습니다:

$$
\int_0^\pi \frac{e^x-1}{e^x-x}, d x
$$


단계 1: 대칭성을 고려 한 치환

우선, 다음과 같은 치환을 사용합니다:

$$
t=\pi-x
$$


그러면 미분은 다음과 같습니다:

$$
d t=-d x
$$


적분의 한계도 바뀝니다:

- $x=0$ 일 때, $t=\pi$.

- $x=\pi$ 일 매, $t=0$.

이를 이용해 적분의 변수를 바꾸고 새롭게 적분을 정의합니다:

$$
I=\int_0^\pi \frac{e^{\pi-x}-1}{e^{\pi-x}-(\pi-x)}, d x
$$


이제, $e^{\pi-x}$ 를 $\frac{e^\pi}{e^x}$ 로 바꿉니다:

$$
I=\int_0^\pi \frac{\dfrac{e^\pi}{e^x}-1}{\dfrac{e^\pi}{e^x}-(\pi-x)}, d x
$$


분자와 분모에 $e^z$ 를 곱하면 다음과 같이 됩니다:

$$
I=\int_0^\pi \frac{e^\pi-e^x}{e^\pi-e^x(\pi-x)}, d x
$$


단계 2: 두 적분을 더하기

원래의 적분과 변환된 적분을 더해 보겠습니다:

$$
2 I=\int_0^\pi \frac{e^x-1}{e^x-x}, d x+\int_0^\pi \frac{e^\pi-e^x}{e^\pi-e^x(\pi-x)}, d x
$$


두 적분을 계산해보면, 더한 결과가 다음과 같이 단순화툅니다:

$$
2 I=\ln \left(e^\pi-\pi\right)
$$


이로부터 $I$ 는 다음과 같습니다:

$$
I=\frac{1}{2} \ln \left(e^\pi-\pi\right)
$$

단계 3: 정답 확인
계산 결과는 $\ln \left(e^\pi-\pi\right)$ 이므로, 다음과 같으며 정답은:

$$
\ln \left(e^\pi-\pi\right)
$$


위 단계별 과정을 퉁해 문제를 해결했습니다. 도움이 더 필요하면 언제든지 이야기해 주세요!
\end{GPTA}

