\begin{GPTQ}
r 은 왜 positive constant 야?
\end{GPTQ}

\begin{GPTA}
반지름 $r$ 은 항상 양수 (positive)여야 하는 이유는 반지름의 정의에 있습니다. 원의 반지름은 중심에서 원주까지의 가장 짧은 거리이며, 이 거리는 물리적으로 유효한 길이를 나타내므로 항상 0 보다 커야 합니다. 수학적으로도, 반지름이 음수가 되면 원의 크기나 모양을 잘못 나타낼 수 있게 되고, 기하학적으로 의미가 없어집니다.

따라서, 반지름 $r$ 을 다룰 때는 항상 양수로 간주하는 것이 일반적입니다.
\end{GPTA}
