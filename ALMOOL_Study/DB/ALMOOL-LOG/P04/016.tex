\begin{GPTQ}
왜 기울기를 계산해야하지?
\end{GPTQ}

\begin{GPTA}
그 질문에 답변해 드리겠습니다. 기울기를 계산하는 이유는 시간에 따른 아시아 잉어 개체 수의 증가율을 파악하기 위함입니다. 기울기는 두 변 수 사이의 관계를 나타내는 중요한 수치인데, 여기에서는 아시아 잉어의 수와 시간이 그 두 변수입니다.

기울기를 찾음으로써 우리는 데이터가 나타내는 경향을 이해할 수 있고, 이는 미래의 변화를 예측하는 데 도움을 줍니다. 

예를 들어, 아시아 잉어의 개체 수가 매년 얼마나 증가하고 있는지를 알아내면, 이 정보를 퉁해 추가적인 조치를 취할 때 도와 줄 수 있습니다. 기울기는 다음과 같은 수식으로 계산됩니다:

$$
\text { slope }=\frac{\text { 변화한 잉어 수 }}{\text { 변화한 시간(년도) }}
$$


이 수식을 이용해 아시아 잉어의 개체 수가 매년 얼만큼 변화하는지를 알 수 있습니다.
\end{GPTA}
