\begin{GPTQ}
어떤식으로 접근해야하는지 알려줄래?
\end{GPTQ}

\begin{GPTA}
물론입니다! 주어진 문제는 정적분 문제입니다. 우리가 주어진 함수 $4 x^3-x$ 의 정적분을 찾고, 그 결과를 주어진 범위 에서 계산해야 합니다. 이렇게 하면 면적 또는 총 변화량을 구할 수 있습니다.

접근 방법:

- 기본 함수의 적분 계산: 함수 $4 x^3-x$ 의 부정적분을 계산합니다. 부정적분을 통해 기본 함수의 원시함수를 찾을 수 있습니다.

- 부분별 적분 계산: 정적분의 기본정리를 적용하여, 다음과 같이 계산합니다.

$$
\int\left(4 x^3-x\right), d x=\int 4 x^3, d x-\int x, d x \quad=x^4-\frac{1}{2} x^2+C
$$

- 정적분 계산: 발견한 원시함수를 사용하여 $x=1$ 부터 $x=2$ 까지의 함수 값을 평가합니다.

$$
\int_1^2\left(4 x^3-x\right), d x=\left[x^4-\frac{1}{2} x^2\right]_1^2=\left(2^4-\frac{1}{2} \cdot 2^2\right)-\left(1^4-\frac{1}{2} \cdot 1^2\right)=(16-2)-\left(1-\frac{1}{2}\right) \quad=14-\frac{1}{2}=\frac{27}{2}
$$


이를 통해 정적분 값은 $\frac{27}{2}$ 임을 알 수 있습니다. 이 과정을 퉁해 정적분에 접근하는 방법을 배울 수 있습니다. 문제가 있으시면 언졔든지 질문하세요!
\end{GPTA}
