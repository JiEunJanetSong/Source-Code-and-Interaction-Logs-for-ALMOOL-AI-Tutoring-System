\begin{GPTQ}
풀의 과정 맨 마지막에 인수분해 하고 3 이랑 -1 이 나온 후에 대입하는거 맞아?
\end{GPTQ}

\begin{GPTA}
맞습니다. $f^{\prime}(x)=3(x-3)(x+1)$ 을 0 으로 설정하고 인수분해를 통해 $x=3$ 과 $x=-1$ 을 구합니다. 그 후, $x=3$ 과 $x=-1$ 을 $f(x)$ 에 대입하여 각각의 함수 값을 구하고, 이는 극값의 픅성율 분석하는 데 사용됩니다. 분석 길과, $x=3$ 에서 함수는 지역 최소값을 가지며, $x=-1$ 에서 함수 는 지역 최대값을 가집니다.

이를 계산 과정으로 정리해보면 다음과 같습니다:

$$
f^{\prime}(x)=3 x^2-6 x-9 \quad=3\left(x^2-2 x-3\right)=3(x-3)(x+1)
$$


여기에서 $3(x-3)(x+1)=0$ 이 될 때, $x=3$ 또는 $x=-1$ 입니다. 이후 $f(x)$ 에서의 함수값을 구해서 각 $x$ 값이 최소값 또는 최대값인지 확인하는 것이 맞습니다.
\end{GPTA}

