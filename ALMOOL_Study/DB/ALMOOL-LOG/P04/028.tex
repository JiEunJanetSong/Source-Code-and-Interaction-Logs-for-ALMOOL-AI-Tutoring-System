\begin{GPTQ}
순간 변화율이 무한대가 된다는것이 잘 이해가 안가  
\end{GPTQ}

\begin{GPTA}
순간 변화율이 무한대가 된다는 것은 어떤 점에서 그 함수의 그래프의 직선 접선이 수직이 되는 상황을 의미합니다. 이는 일반적으로 접선의 기 울기가 무한대 또는 음의 무한대가 되는 경우를 나타냅니다. 이 상황을 더 이해하기 쉽게 예를 들어 설명하겼습니다.

예시: 함수 $f(x)=\sqrt[3]{x}$
이 합수는 $x=0$ 에서 수직 접선을 가지므로 그 점에서 미분 가능하지 않습니다. 이를 살펴보면:
- 함수 $f(x)=\sqrt[3]{x}$ 는 여러 점에서 매끄럽게 보이지만, $x=0$ 에서 완전히 수직인 방향으로 접하는 선을 형성합니다.

수식적으로 보면, $f^{\prime}(x)$ 를 극한을 통해 구하면:

$$
f^{\prime}(x)=\lim _{h \rightarrow 0} \frac{\sqrt[3]{x+h}-\sqrt[3]{x}}{h}
$$


이 식을 $x=0$ 에 대해 계산하면 접선의 기울기를 얻으려면 미분을 해야 하는데, 그 기울기가 순간적으로 무한대가 뷥니다. 그래프 상에서 이러 한 점은 수직으로 보이는 접선을 가지며, 미분 계수를 구할 수 없기 때문에 미분 가능하지 않습니다.

시각적 이해
그래프 관점에서 보면, 수직 접선이 있는 점에서는 그래프가 매우 가퐈르게 변화해 접선의 기울기를 정의할 수 없습니다. 이 때문에, 순간 변화 율이 무한대가 된다는 것은 기울기가 수직으로 치솟는다는 의미가 됩니다.
이해하시기에 도움이 되었기를 바랍니다. 추가 질문이 있다면 언제든지 말쏨해 주세요!
\end{GPTA}

