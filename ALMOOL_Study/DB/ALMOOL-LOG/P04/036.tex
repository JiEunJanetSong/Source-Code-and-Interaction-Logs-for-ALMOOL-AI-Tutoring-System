\begin{GPTQ}
대칭성을 고려한 치환이라는계 무슨 말이야?  
\end{GPTQ}

\begin{GPTA}
``대칭성을 고려한 치환"은 적분 범위 내에서 함수가 대칭적인 성질을 가지고 있는지를 퐌단하여, 이를 활용해 적분을 간단하게 풀기 위한 기법 입니다.

주어진 적분 문제에서 이와 같은 방법을 사용했습니다:

$$
I=\int_0^\pi \frac{e^x-1}{e^x-x}, d x
$$


대칭성의 의미

함수의 대칭성을 보려면 적분 구간의 중점(여기서는 $\pi / 2$ )을 기준으로 함수의 두 부분이 서로 비슷한 특성을 갖는지를 확인하는 것입니다.

치환

픅정한 치환(여기서는 $t=\pi-x)$ 을 사용하여 대칭 성질을 이용합니다.

이 방식은 다음과 같이 작동합니다:

- 치환: $t=\pi-x, d t=-d x$

- 새로운 범위: 원래 범위 와 반대로 $x=0$ 일 때, $t=\pi$ 이고 $x=\pi$ 일 때, $t=0$.

이런 치환은 본질적으로 같은 함수의 거울상 같은 역할을 하여, 전체 적분을 계산하는 데 도움이 됩니다.

결과

이 치환을 사용해 변형된 적분을 계산하여 원래 적분과 결합함으로써 간단하게 계산할 수 있습니다.

이와 같은 대칭적 접근법은 특수한 경우에 매우 유용하며, 많은 상황에서 어렵지 않게 적분을 계산할 수 있계 해 줍니다.
 
추가적인 질문이나 더 깊은 이해가 펼요하시면 알려주세요!
\end{GPTA}

