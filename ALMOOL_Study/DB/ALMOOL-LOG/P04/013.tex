\begin{GPTQ}
아니 정답을 대입을 왜해
\end{GPTQ}

\begin{GPTA}
이 문제에서는 주어진 각 점이 두 방정식의 그래프 위에 놓이는지를 확인하기 위해 하나 이상의 점을 방정식에 대입하여 그 점이 실제로 방정식을 만족하는지 살펴 볼 필요가 있습니다.

특히, 주어진 두 방정식은 서로 동일한 관계이므로(두 번째 방정식은 첫 번째 방정식의 스칼라 배수), 두 방정식 중 한 방정식에 대해서만 점을 대입하고 확인하면 충분합니다.

다시 정리하자면:

1. 방정식이 하나만 있으면, 주어진 점이 방정식을 만족하는지를 확인합니다.

2. 우리는 $2 x+3 y=7$ 에 대해 점 $\left(-\frac{3 r}{2}+\frac{7}{2}, r\right)$ 을 대입합니다.

3. 이 과정을 통해 동일한 결과를 얻으므로, 이 점이 두 방정식의 그래프 모두에 놓이는 점이라는 것을 확인할 수 있습니다.

정확하고 일관된 검증 절차를 통해 결과를 확신할 수 있도록 하는 방법입니다.
\end{GPTA}