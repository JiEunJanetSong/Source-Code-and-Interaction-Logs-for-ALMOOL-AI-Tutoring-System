\begin{GPTQ}
$\sin (\sqrt x)$ 미분을 자세히 알려줄래?  
\end{GPTQ}

\begin{GPTA}
물론입니다! $\sin (\sqrt{x})$ 의 미분을 계산하는 데 체인 룰을 사용합니다. 먼저 체인 룰(Chain Rule)이 무엇인지 설명하겠습니다:

체인 룰은 복합 함수의 미분을 다룰 때 사용하며, 함수가 합성되어 있을 때 각 함수의 미분을 계산하여 전체의 미분을 구합니다. 수식으로 체인 룰은 다음과 같이 표현됩니다:

$$
(f(g(x)))^{\prime}=f^{\prime}(g(x)) \cdot g^{\prime}(x)
$$


이제 $\sin (\sqrt{x})$ 를 단계별로 미분해 봅시다.

1. 외부 함수 식별: 외부 함수는 $\sin (u)$ 입니다. 여기서 $u=\sqrt{x}$ 로 설정합니다.

2. 외부 함수의 미분: $\sin (u)$ 의 $u$ 에 대한 미분은 $\cos (u)$ 입니다.

3. 내부 함수 식별: 내부 함수는 $u=\sqrt{x}$ 입니다.

4. 내부 함수의 미분: $\sqrt{x}$ 의 $x$ 에 대한 미분은 $\frac{1}{2} x^{-\frac{1}{2}}$ 입니다. 왜냐하면 $\sqrt{x}$ 은 $x^{1 / 2}$ 로 표현할 수 있기 때문입니다. 따라서

$$
\frac{d}{d x} \sqrt{x}=\frac{d}{d x} x^{1 / 2}=\frac{1}{2} x^{-1 / 2}
$$

5. 쳬인 룰 적용: 이제 체인 룰을 적용하여 전체 미분을 구합니다.

$$
\frac{d}{d x} \sin (\sqrt{x})=\cos (\sqrt{x}) \cdot \frac{1}{2} x^{-1 / 2}
$$


따라서, $\sin (\sqrt{x})$ 의 미분은 다음과 같습니다:

$$
\frac{d}{d x} \sin (\sqrt{x})=\frac{\cos (\sqrt{x})}{2 \sqrt{x}}
$$


이 방법으로 $\sin (\sqrt{x})$ 의 미분을 계산할 수 있습니다.
\end{GPTA}

