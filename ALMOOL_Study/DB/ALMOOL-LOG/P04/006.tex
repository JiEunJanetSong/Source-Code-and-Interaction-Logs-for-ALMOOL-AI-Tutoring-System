\begin{GPTQ}
아니 왜 $15 / 3$ 이냐고  
\end{GPTQ}

\begin{GPTA}
좋은 질문입니다. $15 / 3$ 는 15 시간 동안 3 시간마다 세포가 두 배로 증가하는 헷수를 계산하기 위한 것입니다

설명:

- 세포 수는 3 시간마다 두 배로 증가합니다.

- 15 시간 동안 증가한 횟수를 구하려면 15 을 3 으로 나늡니다.

$$
\frac{15}{3}=5
$$


따라서 15 시간 동안 총 5 번 두 배로 증가합니다. 이것이 왜 (2) $\frac{15}{3}$ 로 지수에 사용되는지를 설명합니다. 이를 통해 15 시간 동안 세포 수는 5 번 두 배로 증가한다는 것을 나타냅니다. 이 개념이 문제 해결의 핵심입니다. 
\end{GPTA}
