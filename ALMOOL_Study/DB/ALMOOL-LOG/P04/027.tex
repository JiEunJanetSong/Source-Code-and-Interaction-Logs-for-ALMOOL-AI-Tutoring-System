\begin{GPTQ}
미분 가능하지 않다는 말이 무슨말이야 
\end{GPTQ}

\begin{GPTA}
함수가 특정 점에서 미분 가능하지 않다는 것은 그 점에서 함수의 순간 변화율을 구할 수 없다는 것을 의미합니다. 보다 구체적으로, 함수 $f$ 가 점 $x=c$ 에서 미분 가능하지 않다는 것은 다음과 같은 이유들로 인해 발생할 수 있습니다:
1. 불연속성: 함수 $f$ 가 $x=c$ 에서 불연속이면 미분 가능하지 않습니다. 즉, 그 점에서 함수가 끊기거나 갑작스러운 점프를 하는 것입니다.
2. 샤프 코녀(sharp corner) 또는 커스프(cusp): 합수가 $x=c$ 에서 뾰족한 모서리를 가지면, 그 점에서의 접선 기울기를 정의할 수 없으므로 미분 가능하지 않습니다. 예를 들어 $f(x)=|x|$ 는 $x=0$ 에서 미분 가능하지 않습니다.
3. 수직 접선(vertical tangent): 함수의 순간 변화율이 무한대가 되면 미분 가능하지 않습니다. 즉, 기울기가 $\pm \infty$ 이 되는 경우입니다.
4. 미분계수의 존졔 하지 않음: 미분계수를 정의하는 극한이 존재하지 않거나 무한대일 때, 함수는 미분 가능하지 않습니다.

위의 몇 가지 이유들로 인해, 합수가 특정 점에서 미분 가능하지 않을 수 있습니다. 미분 가능성은 함수의 매끄러움과 관련이 있으며, 접선의 기 울기를 계산할 수 있을 때 미분 가능하다고 말합니다.
\end{GPTA}

