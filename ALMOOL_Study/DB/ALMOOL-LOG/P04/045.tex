\begin{GPTQ}
문제 풀어줘
\end{GPTQ}

\begin{GPTA}
문제를 풀어드리겠습니다. 주어진 정보에 따라 원뿔의 부피를 사용하여 밑면의 반지름을 찾아보겠습니다.

주어진 사항:

- 원뿔의 전체 부펴는 $V=39,600 \mathrm{~cm}^3$ 입니다.

- 원뿔의 부피 공식:

$$
V=\frac{1}{3} \pi r^2 h
$$

- $\pi=\frac{22}{7}$

- 주어진 식에 의해 원뿔의 높이 $h=18 \mathrm{~cm}$ 로 사용되어 있음을 확인하려고 합니다.

2. 방정식을 정리합니다:

$$
39,600=\frac{396}{21} r^2
$$

3. $r^2$ 를 풀어야 하므로, 다음과 같이 계속 계산합니다:

$$
r^2=39,600 \div \frac{396}{21}
$$

4. 위의 계산을 수행합니다:

$$
r^2=2,100
$$

5. 양변에 루트를 취해 $r$ 을 찾습니다:

$$
r=\sqrt{2,100}=10 \sqrt{21} \approx 45.83 \mathrm{~cm}
$$


위의 계산에서 확인된 수학적 과정을 통해 원뿔의 밑면 반지름인 $r$ 은 약 45.83 cm 입니다. 계산 과정이나 다른 질문이 있을 경우 언제든지 말씀해 주세요.
\end{GPTA}
