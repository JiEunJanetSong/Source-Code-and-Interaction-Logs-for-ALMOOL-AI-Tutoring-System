\begin{GPTQ}
다 이해되었고 $\sin (r o o t x)$ 까지 왔어 그 다음은?
\end{GPTQ}

\begin{GPTA}
잘하셨습니다! 이제 $\sin (\sqrt{x})$ 의 미분을 활용하여 전제 함수를 미분해 봅시다.

주어진 함수는 $2(\sin (\sqrt{x}))^2$ 입니다. 우리는 이미 $\sin (\sqrt{x})$ 의 미분인 $\frac{\cos (\sqrt{x})}{2 \sqrt{x}}$ 을 구했습니다.

목표는 주어진 문제 $\frac{d}{d x}\left(2(\sin (\sqrt{x}))^2\right)$ 의 미분을 찾는 것입니다. 접근 방법은 다음과 같습니다:

1. 외부 함수의 미분: 외부 함수는 $2 u^2$ 의 형태입니다, 여기서 $u=\sin (\sqrt{x})$ 라고 할 수 있습니다. 이 함수에 대한 미분은:

$$
\frac{d}{d u} 2 u^2=4 u
$$

2. 내부 함수의 미분: 내부 함수인 $u=\sin (\sqrt{x})$ 의 미분은 앞서 구한대로 $\frac{\cos (\sqrt{x})}{2 \sqrt{x}}$ 입니다.

3. 채인 룰 적용: 두 미분을 체인 룰을 사용하여 결합합니다:

$$
\frac{d}{d x}\left(2(\sin (\sqrt{x}))^2\right)=4(\sin (\sqrt{x})) \cdot \frac{\cos (\sqrt{x})}{2 \sqrt{x}}
$$

4. 결과 정리: 최종적으로 식을 정리하면,

$$
=\frac{2 \sin (\sqrt{x}) \cos (\sqrt{x})}{\sqrt{x}}
$$


따라서 전체 함수를 미분한 값은 $\frac{2 \sin (\sqrt{x}) \cos (\sqrt{x})}{\sqrt{x}}$ 입니다.
\end{GPTA}

