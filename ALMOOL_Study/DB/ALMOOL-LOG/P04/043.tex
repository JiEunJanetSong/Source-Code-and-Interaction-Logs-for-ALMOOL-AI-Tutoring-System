\begin{GPTQ}
근데 문제에서 높이가 안주어졌는데
\end{GPTQ}

\begin{GPTA}
이해했습니다. 문제에서 직접적으로 높이가 주어지지 않았다면, 수학적 접근이 포함된 과정이 출제자에 의해 전제되어 있다거나 이전의 정보 나 문제들에서 유추할 수 있는 경우일 수 있습니다.
다음은 주어진 문제에 대한 과정입니다. 프러스텀의 부분만 고려한 것이 아니라 전체 원뿔의 부피에 대한 정보만 공유되었습니다.
프러스텀의 완벽한 해석이 필요하거나 푸는 과정의 다른 보충설명이 필요하다면 더 구체적으로 부탁드립니다. 추가 정보를 퉁해 더욱 명확히 설명드리젰습니다.
\end{GPTA}

