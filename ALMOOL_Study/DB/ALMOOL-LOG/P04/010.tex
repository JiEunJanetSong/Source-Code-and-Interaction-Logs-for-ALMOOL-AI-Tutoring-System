\begin{GPTQ}
    좌표평면에서 시각적으로 보여줄래?
\end{GPTQ}

\begin{GPTA}
    물론입니다! 주어진 원의 정보를 좌표평면에 시각화해보겠습니다.

    1. 중심 좌표와 반지름:

    - 중심은 $(2,9)$ 입니다.

    - 반지름은 $r=5$ 입니다.

    2. 원의 원방 방정식:
    
    방정식은 $(x-2)^2+(y-9)^2=5^2$ 입니다.

    3. 원의 그래프: 중심을 기준으로 반지름 만큼 떨어진 모든 점들을 연결하면 원을 그릴 수 있습니다.

    4. 지름의 양 끝점:

    - 지름의 끝점은 $(2,4)$ 와 $(2,14)$ 입니다. 이를 통해 원의 크기와 위치를 쉽게 파악할 수 있습니다.
    
    - $O$ 표시는 지름의 끝점을 나타내며, 중심에서 지름 방향으로 위아래로 5 만큼 떨어져 있는 위치입니다.
    - 중간의 둘은 원의 일부를 나타내며, 실제로는 모든 방향으로 뺃어 나갑니다.
    
    이렇게 좌표평면에 표시된다면, 원의 위치와 크기를 시각적으로 쉽게 이해할 수 있습니다.    
\end{GPTA}
