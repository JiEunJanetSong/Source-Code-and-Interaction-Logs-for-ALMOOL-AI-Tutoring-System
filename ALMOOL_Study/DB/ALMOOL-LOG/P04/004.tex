\begin{GPTQ}
아니 왜 $y=a(b)^{\tfrac{x}{k}}$ 냐고    
\end{GPTQ}

\begin{GPTA}
줗은 질문입니다. 이 식은 지수 성장(exponential growth)을 설명하기 위한 수학적 모델입니다. 세포가 일정 한 비율로 일정 한 주기마다 증가하는 경우에 이러한 식을 사용합니다.

식의 의미

- $y=a(b)^{\tfrac{x}{k}}$ 에서,

- $y$ 는 시간 $x$ 가 지난 후의 세포 수입니다.

- $a$ 는 초기 세포 수입니다. 이 경우 초기 세포 수는 300,000 입니다.

- $b$ 는 증가하는 배수입니다. 이 예에서는 세포 수가 두 배로 증가하므로 $b=2$ 입니다.

- $k$ 는 주기 시간으로, 세포 수가 배수로 증가하기까지 걸리는 시간(이 경우 3 시간)입니다.

- $x$ 는 초기 관측 이후 경과한 시간입니다.

왜 지수 함수인가?

세포 수는 각 주기마다 동일한 비율로 곱해지며 증가합니다. 초기 세포 수가 배수로 곱해질 때마다 같은 비 율로 증가하므로, 이러한 상황을 수학적으로 설명하기에 지수 함수가 적합합니다. 예를 들어, 3 시간마다 두 배가 된다면, 6 시간 후에는 네 배, 9 시간 후에는 여덟 배가 됩니다.
이러한 종류의 성장에서는 시간이 흐름에 따라 매우 빠른 속도로 양이 증가하게 됩니다. 그래서 지수 함수를 사용하여 변화율을 설명할 수 있습니다.
이해가 되었기를 바라며, 다른 질문이 있다면 언제든지 말씀해 주세요.
\end{GPTA}

