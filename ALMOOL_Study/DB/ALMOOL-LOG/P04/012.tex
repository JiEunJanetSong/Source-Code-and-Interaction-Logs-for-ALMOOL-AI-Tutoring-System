\begin{GPTQ}
설명좀
\end{GPTQ}

\begin{GPTA}
물론, 단계별로 설명드리젰습니다.

주어진 문제에서는 선 $h$ 와 선 $k$ 가 주어졌고, $h$ 의 일정 데이터를 바탕으로 $k$ 의 $x$-절편을 찾는 문제입니다.

1. 데이터 확인:

- 주어진 데이터 테이블에서 선 $h$ 는 다음 세 점을 포함합니다: $(18,130),(23,160),(26,178)$.

2. 기울기 $(m)$ 계산:

- 두 점을 사용하여 기울기를 구할 수 있습니다. 기울기의 공식은 $m=\frac{y_2-y_1}{x_2-x_1}$ 입니다.

- 예를 들어 $(18,130)$ 와 $(23,160)$ 를 사용하면:

$$
m=\frac{160-130}{23-18}=\frac{30}{5}=6
$$

- 따라서 선 $h$ 의 기울기는 6 입니다.

3. $y$-절편 $(b)$ 계산:

- 선의 방정식은 $y=m x+b$ 형태입니다.

- 점 $(18,130)$ 를 사용하여 $b$ 를 찾습니다:

$$
130=6 \times 18+b
$$

- 계산을 완료하면:

$$
130=108+b \Longrightarrow b=22
$$

- 따라서 선 $h$ 의 방정식은 $y=6 x+22$ 가 됩니다.

4. 선 $k$ 의 방정식 찾기:

- 선 $k$ 는 선 $h$ 를 $y$ 축 방향으로 5 단위 내려간 것이므로:

- $y$ 절편이 5 만큼 줄어듭니다: $b=22-5=17$.

따라서 선 $k$ 의 방정식은 $y=6 x+17$ 입니다.

5. $x$-절편 구하기:

- $x$-절편은 $y=0$ 일 때의 $x$ 값입니다. 즉:

$$
0=6 x+17
$$

- $x$ 에 대해 방정식을 풀면:

$$
6 x=-17 \Longrightarrow x=-\frac{17}{6}
$$


결론적으로, 선 $k$ 의 $x$-절편은 $\left(-\frac{17}{6}, 0\right)$ 입니다. 이는 선택지 중에서 $\mathbf{D}$ 입니다.
\end{GPTA}

