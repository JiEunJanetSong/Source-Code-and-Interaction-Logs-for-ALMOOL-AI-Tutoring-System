\begin{GPTQ}
이 문제를 풀기위한 모든 공식을 알려줄래?  
\end{GPTQ}

\begin{GPTA}
프러스텀 문제를 해결하기 위해 풜요한 공식들을 소개하겠습니다.

1. 전제 원뿔 높이 관련 공식:

프러스텁의 원뿔을 생각할 때, 전체 원뿔의 높이 $h$ 와 프러스텀의 높이 $H$ 사이에는 다음 관계가 있습니다.

$$
\circ=x+H
$$


여기서, $x=x+H$ 는 프러스텀의 상단 원뿔 부분의 높이입니다.

2. 비례 관계를 이용하여 원뿔 부분의 높이를 찾는 공식: 

주어진 조건에서, 아래쪽 지름 $\left(D_1\right)$ 이 윗쪽 지름 $\left(D_2\right)$ 의 세 배입니다. 반지름을사용해 비례식을 적용할 수 있습니다.

$$
\frac{x+H}{R_1}=\frac{x}{R_2}
$$


위 식에서 $R_1=\frac{D_1}{2}, R_2=\frac{D_2}{2}$ 입니다. $D_1=3 D_2$ 이므로, $R_1=\frac{3 D_2}{2}$ 이고 $R_2=\frac{D_2}{2}$ 가 뒵니다.

3. 프러스텀의 쳬적 공식:

프러스텀의 체적 $V$ 는 다음과 같은 공식으로 구할 수 있습니다.

$$
V=\frac{1}{3} \pi H\left(R_1^2+R_1 R_2+R_2^2\right)
$$


프러스텀 문제를 풀 때 이러한 공식을 활용하여 각 변수를 대입하여 해를 구할 수 있습니다. 이 경우에는 전체 원뿔의 높이를 구하기 위해 비례 식을 사용하는 것이 주요합니다.
\end{GPTA}

