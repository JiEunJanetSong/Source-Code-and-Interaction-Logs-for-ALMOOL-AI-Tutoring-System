\SetValue{Module}{1}\SetValue{SectionAB}{A}\SetValue{MainChapter}{}\SetValue{SubChapter}{}\SetValue{Contents}{%%
    
The figure below shows a large white circle with a number of smaller white and shaded circles in its interior. What fraction of the interior of the large white circle is shaded?

\begin{center}		
    \includegraphics[scale=0.5]{AMC-8-pics/2023-12.png}
	\end{center}
	
\ansFIVEs{%start
		\frac{1}{4} }{%<--- (A)
		\frac{11}{36} }{%<--- (B)
		\frac{1}{3} }{%<--- (C)
		\frac{19}{36} }{%<---- (D)
		\frac{5}{9} }%<---- (D)
}\SetValue{Concept}{%

아래 그림은 큰 흰색 원과 그 안쪽에 여러 개의 작은 흰색 및 음영 처리된 원이 있는 것을 보여줍니다. 큰 흰색 원 내부의 몇 퍼센트가 음영 처리되어 있나요?

}\SetValue{AltText}{%



}\SetValue{Solution}{%

First, the total area of the radius 3 circle is simply just $9 \cdot \pi$ when using our area of a circle formula.

Now from here, we have to find our shaded area. This can be done by adding the areas of the $\frac{1}{2}$-radius circles and add; then, take the area of the 1 radius circles and subtract that from the area of the 2 radius circle to get our resulting complex shape area. Adding these up, we will get $3 \cdot \frac{1}{4} \pi+4 \pi-\pi-\pi=\frac{3}{4} \pi+2 \pi=\frac{11 \cdot \pi}{4}$.
So, our answer is $\frac{\frac{11}{4} \pi}{9 \pi}=$ (B) $\frac{11}{36}$.
}\SetValue{Rubric}{%Markdown



}\SetValue{Hint}{%
Solution Goes Here
}\SetValue{Answer}{%

}
\ProcessDATA



