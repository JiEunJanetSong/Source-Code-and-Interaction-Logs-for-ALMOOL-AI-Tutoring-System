\SetValue{Module}{1}\SetValue{SectionAB}{A}\SetValue{MainChapter}{}\SetValue{SubChapter}{}\SetValue{Contents}{%%
    
Isosceles triangle $A B C$ has equal side lengths $A B$ and $B C$. In the figures below, segments are drawn parallel to $\overline{A C}$ so that the shaded portions of $\triangle A B C$ have the same area. The heights of the two unshaded portions are 11 and 5 units, respectively. What is the height $h$ of $\triangle A B C ?$

\begin{center}		
    \includegraphics[scale=0.4]{AMC-8-pics/2023-24.png}
	\end{center}

\ansFIVEs{%start
		14.6 }{%<--- (A)
		14.8 }{%<--- (B)
		15 }{%<--- (C)
		15.2 }{%<---- (D)
		15.4 }%<---- (D)
}\SetValue{Concept}{%



}\SetValue{AltText}{%



}\SetValue{Solution}{%

First, we notice that the smaller isosceles triangles are similar to the larger isosceles triangles. We can find that the area of the gray area in the first triangle is $[A B C] \cdot\left(1-\left(\frac{11}{h}\right)^2\right)$. Similarly, we can find that the area of the gray part in the second triangle is $[A B C] \cdot\left(\frac{h-5}{h}\right)^2$. These areas are equal, so $1-\left(\frac{11}{h}\right)^2=\left(\frac{h-5}{h}\right)^2$. Simplifying yields $10 h=146$ so $h=$ $\square$ (A) 14.6.
}\SetValue{Rubric}{%Markdown



}\SetValue{Hint}{%
Solution Goes Here
}\SetValue{Answer}{%

}
\ProcessDATA



