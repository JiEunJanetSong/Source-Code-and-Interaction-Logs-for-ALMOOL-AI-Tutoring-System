\SetValue{Module}{1}\SetValue{SectionAB}{A}\SetValue{MainChapter}{}\SetValue{SubChapter}{}\SetValue{Contents}{%%
    
The digits $2,0,2$, and 3 are placed in the expression below, one digit per box. What is the maximum possible value of the expression?

\begin{center}		
    \includegraphics[scale=0.6]{AMC-8-pics/2023-06.png}
	\end{center}

\ansFIVEs{%start
		0 }{%<--- (A)
		8 }{%<--- (B)
		9 }{%<--- (C)
		16 }{%<---- (D)
		18}%<---- (D)
}\SetValue{Concept}{%



}\SetValue{AltText}{%



}\SetValue{Solution}{%

Step-by-step analysis

We want to maximize:

$$
a^b \times c^d
$$

using the digits: 0, 2, 2, 3 (each used once).

Observations:

- Anything to the power of 0 is 1 (except $0^0$, which we avoid).

- $0^x=0$ for $x>0 \rightarrow$ this will zero out the whole product.

- So we must avoid using 0 as a base.

Hence, 0 should be used as an exponent.

Try strong combinations

Try: $3^2 \times 2^0=9 \times 1=9$

Other possibilities:

- $2^3 \times 2^0=8$

- $2^2 \times 3^0=4$

- $2^0 \times 3^2=1 \times 9=9$

- $3^0 \times 2^2=1 \times 4=4$

Maximum value is $\square$ 9

Answer: (C) 9
}\SetValue{Rubric}{%Markdown



}\SetValue{Hint}{%
Solution Goes Here
}\SetValue{Answer}{%

}
\ProcessDATA



