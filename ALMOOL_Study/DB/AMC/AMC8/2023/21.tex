\SetValue{Module}{1}\SetValue{SectionAB}{A}\SetValue{MainChapter}{}\SetValue{SubChapter}{}\SetValue{Contents}{%%
    
Alina writes the numbers $1,2, \ldots, 9$ on separate cards, one number per card. She wishes to divide the cards into 3 groups of 3 cards so that the sum of the numbers in each group will be the same. In how many ways can this be done?

\ansFIVEs{%start
		0 }{%<--- (A)
		1 }{%<--- (B)
		2 }{%<--- (C)
		3 }{%<---- (D)
		4 }%<---- (D)
}\SetValue{Concept}{%



}\SetValue{AltText}{%



}\SetValue{Solution}{%

First, we need to find the sum of each group when split. This is the total sum of all the elements divided by the \# of groups. $1+2 \cdots+9=\frac{9(10)}{2}=45$. Then, dividing by 3 , we have $\frac{45}{3}=15$, so each group of 3 must have a sum of 15 . To make the counting easier, we will just see the possible groups 9 can be with. The possible groups 9 can be with 2 distinct numbers are $(9,2,4)$ and $(9,1,5)$. Going down both of these avenues, we will repeat the same process for 8 using the remaining elements in the list. Where there is only 1 set of elements getting the sum of 7,8 needs in both cases. After 8 is decided, the remaining 3 elements are forced in a group, yielding us an answer of (C) 2 as our sets are $(9,1,5)(8,3,4)(7,2,6)$ and $(9,2,4)(8,1,6)(7,3,5)$.
}\SetValue{Rubric}{%Markdown



}\SetValue{Hint}{%
Solution Goes Here
}\SetValue{Answer}{%

}
\ProcessDATA



