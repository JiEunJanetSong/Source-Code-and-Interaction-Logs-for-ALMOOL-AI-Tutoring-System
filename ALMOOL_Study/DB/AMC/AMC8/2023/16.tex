\SetValue{Module}{1}\SetValue{SectionAB}{A}\SetValue{MainChapter}{}\SetValue{SubChapter}{}\SetValue{Contents}{%%
    
The letters $\mathrm{P}, \mathrm{Q}$, and R are entered into a $20 \times 20$ table according to the pattern shown below. How many Ps, Qs, and Rs will appear in the completed table?

\begin{center}		
    \includegraphics[scale=0.5]{AMC-8-pics/2023-16.png}
	\end{center}
	
\ansFIVEs{%start
		132 \mathrm{Ps}, 134 \mathrm{Qs}, 134 \mathrm{Rs} }{%<--- (A)
		133 \mathrm{Ps}, 133 \mathrm{Qs}, 134 \mathrm{Rs} }{%<--- (B)
		133 \mathrm{Ps}, 134 \mathrm{Qs}, 133 \mathrm{Rs} }{%<--- (C)
		134 \mathrm{Ps}, 132 \mathrm{Qs}, 134 \mathrm{Rs} }{%<---- (D)
		134 \mathrm{Ps}, 133 \mathrm{Qs}, 133 \mathrm{Rs} }%<---- (D)
}\SetValue{Concept}{%



}\SetValue{AltText}{%



}\SetValue{Solution}{%

\textbf{Solution 1}

In our $5 \times 5$ grid, there are 8,9 and 8 of the letters $\mathrm{P}, \mathrm{Q}$, and R, respectively, and in a $2 \times 2$ grid, there are 1,2 and 1 of the letters $\mathrm{P}, \mathrm{Q}$, and R , respectively. We see that in both grids, there are $x, x+1$, and $x$ of the $\mathrm{P}, \mathrm{Q}$, and R , respectively. This is because in any $n \times n$ grid with $n \equiv 2(\bmod 3)$, there are $x, x+1$, and $x$ of the $\mathrm{P}, \mathrm{Q}$, and R , respectively. We can see that the only answer choice which satisfies this condition is
(C) 133 Ps, $134 \mathrm{Qs}, 133 \mathrm{Rs}$.

\textbf{Solution 2}

Since $20 \equiv 2(\bmod 3)$ and $Q$ is in the 2 nd diagonal, it is also in the 20 th diagonal, and so we find that there are $2(2+5+8+11+14+17)+20=134 Q s$. Since all the P's and R's are symmetric, the answer is (C) $133 \mathrm{Ps}, 134 \mathrm{Qs}, 133 \mathrm{Rs}$.
}\SetValue{Rubric}{%Markdown



}\SetValue{Hint}{%
Solution Goes Here
}\SetValue{Answer}{%

}
\ProcessDATA



