\SetValue{Module}{1}\SetValue{SectionAB}{A}\SetValue{MainChapter}{}\SetValue{SubChapter}{}\SetValue{Contents}{%%
    
On Monday, Taye has $\$ 2$. Every day, he either gains $\$ 3$ or doubles the amount of money he had on the previous day. How many different dollar amounts could Taye have on Thursday, 3 days later?

\ansFIVEs{%start
		3 }{%<--- (A)
		4 }{%<--- (B)
		5 }{%<--- (C)
		6 }{%<---- (D)
		7 }%<---- (D)
}\SetValue{Concept}{%



}\SetValue{AltText}{%



}\SetValue{Solution}{%

Solution 1

How many dollar values could be on the first day? Only 2 dollars. The second day, you can either add 3 dollars, or double, so you can have 5 dollars, or 4 . For each of these values, you have 2 values for each. For 5 dollars, you have 10 dollars or 8 , and for 4 dollars, you have 8 dollars or 7 dollars. Now, you have 2 values for each of these. For 10 dollars, you have 13 dollars or 20 , for 8 dollars, you have 16 dollars or 11 , for 8 dollars, you have 16 dollars or 11 , and for 7 dollars, you have 14 dollars or 10 .

On the final day, there are $11,11,16$, and 16 repeating, leaving you with $8-2=$
(D) 6 different values.

Solution 2

Continue as in Solution 1 to get 7, 8, or 10 dollars by the 2nd day. The only way to get the same dollar amount occurring twice by branching (multiply by 2 or adding 3 ) from here is if $7+3=10 \cdot 2$ or $7+3=8 \cdot 2$ which both aren't true. Hence our answer is

$$
3 \cdot 2=(\mathrm{D}) 6
$$

}\SetValue{Rubric}{%Markdown



}\SetValue{Hint}{%
Solution Goes Here
}\SetValue{Answer}{%

}
\ProcessDATA



