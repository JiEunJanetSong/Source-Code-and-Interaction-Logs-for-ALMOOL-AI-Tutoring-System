\SetValue{Module}{1}\SetValue{SectionAB}{A}\SetValue{MainChapter}{}\SetValue{SubChapter}{}\SetValue{Contents}{%%
    
Rohan keeps 90 guppies in 4 fish tanks.

- There is 1 more guppy in the 2 nd tank than in the 1 st tank.

- There are 2 more guppies in the 3rd tank than in the 2nd tank.

- There are 3 more guppies in the 4 th tank than in the 3rd tank.

How many guppies are in the 4th tank?

\ansFIVEs{%start
		20 }{%<--- (A)
		21 }{%<--- (B)
		23 }{%<--- (C)
		24 }{%<---- (D)
		26 }%<---- (D)
}\SetValue{Concept}{%



}\SetValue{AltText}{%



}\SetValue{Solution}{%

Solution 1
Let $x$ denote the number of guppies in the first tank.
Then, we have the following for the number of guppies in the rest of the tanks:
- The number of guppies in the second tank is $x+1$
- The number of guppies in the third tank is $x+1+2$
- The number of guppies in the fourth tank is $x+1+2+3$

The number of guppies in all of the tanks combined is 90 , so we can write the equation

$$
(x)+(x+1)+(x+1+2)+(x+1+2+3)=90 .
$$


Simplifying the equation gives

$$
4 x+10=90 .
$$


Solving the resulting equation gives $x=20$, so the number of guppies in the fourth tank is

$$
20+1+2+3=(\mathbf{E}) 26 \text {. }
$$

Solution 2
Suppose there are no guppies in the first tank. Then, the number of guppies in the other tanks are 1,3 , and 6 , or 10 guppies in total. We need to add $90-10=80$ guppies into 4 tanks or 20 guppies in each tank, so the number of guppies in the fourth tank is $20+6=$ $\square$ (E) 26 .
}\SetValue{Rubric}{%Markdown



}\SetValue{Hint}{%
Solution Goes Here
}\SetValue{Answer}{%

}
\ProcessDATA



