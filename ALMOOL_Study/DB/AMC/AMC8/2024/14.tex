\SetValue{Module}{1}\SetValue{SectionAB}{A}\SetValue{MainChapter}{}\SetValue{SubChapter}{}\SetValue{Contents}{%%
    
The one-way routes connecting towns $A, M, C, X, Y$, and $Z$ are shown in the figure below(not drawn to scale). The distances in kilometers along each route are marked. Traveling along these routes, what is the shortest distance from $A$ to $Z$ in kilometers?

\begin{center}		
    \includegraphics[scale=0.6]{AMC-8-pics/2024-14.png}
	\end{center}

\ansFIVEs{%start
		28 }{%<--- (A)
		29 }{%<--- (B)
		30 }{%<--- (C)
		31 }{%<---- (D)
		32 }%<---- (D)
}\SetValue{Concept}{%



}\SetValue{AltText}{%



}\SetValue{Solution}{%

Note / Warning
As tempting as it may seem, these diagrams are not drawn to scale. What may visually look like the shortest distance may not be the shortest in terms of the distance shown. I know it may be obvious after some time, but many people are at first lured to path $A \rightarrow M \rightarrow C \rightarrow Z$ just because visually it looks like the shortest distance. This could potentially lead to the incorrect answer (E) 32 .
$\sim$ s.khunti
Solution 1
We can simply see that path $A \rightarrow X \rightarrow M \rightarrow Y \rightarrow C \rightarrow Z$ will give us the smallest value. Adding, $5+2+6+5+10=28$. This is nice as it's also the smallest value, solidifying our answer.

You can also simply brute-force it or sort of think ahead - for example, getting from A to M can be done 2 ways; $A \rightarrow X \rightarrow M(5+2)$ or $A \rightarrow M(8)$, so you should take the shorter route $(5+2)$. Another example is M to C , two ways - one is $6+5$ and the other is 14 . Take the shorter route. After this, you need to consider a few more times - consider if $5+10$ ( $Y \rightarrow C \rightarrow Z)$ is greater than $17(Y \rightarrow Z)$, which it is not, and consider if $25(M \rightarrow Z)$ is greater than $14+10(M \rightarrow C \rightarrow Z)$ or $6+5+10(M \rightarrow Y \rightarrow C \rightarrow Z)$ which it is not. TLDR: $5+2+6+5+10=28$. [Note: This is probably just the thinking behind the solution.] \{Double-note: As MaxyMoosy said, since this answer is the smallest one, it has to be the right answer.\}
}\SetValue{Rubric}{%Markdown



}\SetValue{Hint}{%
Solution Goes Here
}\SetValue{Answer}{%

}
\ProcessDATA



