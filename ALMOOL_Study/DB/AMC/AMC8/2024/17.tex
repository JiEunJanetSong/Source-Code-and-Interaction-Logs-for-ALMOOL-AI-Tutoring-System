\SetValue{Module}{1}\SetValue{SectionAB}{A}\SetValue{MainChapter}{}\SetValue{SubChapter}{}\SetValue{Contents}{%%
    
A chess king is said to attack all the squares one step away from it, horizontally, vertically, or diagonally. For instance, a king on the center square of a $3 \times 3$ grid attacks all 8 other squares, as shown below. Suppose a white king and a black king are placed on different squares of a $3 \times 3$ grid so that they do not attack each other (in other words, not right next to each other). In how many ways can this be done?

\begin{center}		
    \includegraphics[scale=0.6]{AMC-8-pics/2024-17.png}
	\end{center}

\ansFIVEs{%start
		20 }{%<--- (A)
		24 }{%<--- (B)
		27 }{%<--- (C)
		28 }{%<---- (D)
		32 }%<---- (D)
}\SetValue{Concept}{%



}\SetValue{AltText}{%



}\SetValue{Solution}{%

Solution 1
If you place a king in any of the 4 corners, the other king will have 5 spots to go and there are 4 corners, so $5 \times 4=20$. If you place a king in any of the 4 edges, the other king will have 3 spots to go and there are 4 edges so $3 \times 4=12$. That gives us $20+12=32$ spots for the other king to go into in total. So (E)32 is the answer. 

Solution 2
We see that the center is not a viable spot for either of the kings to be in, as it would attack all nearby squares.
This gives three combinations:
Corner-corner: There are 4 corners, and none of them are touching orthogonally or diagonally, so it's $\binom{4}{2}=6$
Corner-edge: For each corner, there are two edges that don't border it, $4 \cdot 2=8$
Edge-edge: The only possible combinations of this that work are top-bottom and left-right edges, so 2 for this type

$$
6+8+2=16
$$


Multiply by two to account for arrangements of colors to get (E) 32 
}\SetValue{Rubric}{%Markdown



}\SetValue{Hint}{%
Solution Goes Here
}\SetValue{Answer}{%

}
\ProcessDATA



