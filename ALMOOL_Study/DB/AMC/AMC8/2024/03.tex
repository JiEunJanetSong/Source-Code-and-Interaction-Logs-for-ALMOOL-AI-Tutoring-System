\SetValue{Module}{1}\SetValue{SectionAB}{A}\SetValue{MainChapter}{}\SetValue{SubChapter}{}\SetValue{Contents}{%%
    
Four squares of side length $4,7,9$, and 10 are arranged in increasing size order so that their left edges and bottom edges align. The squares alternate in color white-gray-white-gray, respectively, as shown in the figure. What is the area of the visible gray region in square units?

\begin{center}		
    \includegraphics[scale=0.6]{AMC-8-pics/2024-03.png}
	\end{center}

\ansFIVEs{%start
		42 }{%<--- (A)
		45 }{%<--- (B)
		49 }{%<--- (C)
		50 }{%<---- (D)
		52 }%<---- (D)
}\SetValue{Concept}{%



}\SetValue{AltText}{%



}\SetValue{Solution}{%

We work inwards. The area of the outer shaded square is the area of the whole square minus the area of the second largest square. The area of the inner shaded region is the area of the third largest square minus the area of the smallest square. The sum of these areas is

$$
10^2-9^2+7^2-4^2=100-81+49-16=19+33=(\mathbf{E}) 52
$$

}\SetValue{Rubric}{%Markdown



}\SetValue{Hint}{%
Solution Goes Here
}\SetValue{Answer}{%

}
\ProcessDATA



