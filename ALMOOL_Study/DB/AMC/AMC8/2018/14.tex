\SetValue{Module}{1}\SetValue{SectionAB}{A}\SetValue{MainChapter}{}\SetValue{SubChapter}{}\SetValue{Contents}{%%
    
Let $N$ be the greatest five-digit number whose digits have a product of 120. What is the sum of the digits of $N$?

\ansFIVEs{%start
		15 }{%<--- (A)
		16 }{%<--- (B)
		17 }{%<--- (C)
		18 }{%<---- (D)
		20 }%<---- (D)
}\SetValue{Concept}{%



}\SetValue{AltText}{%



}\SetValue{Solution}{%

Answer (D): The leftmost digit of $N$ must be the greatest single-digit factor of $120=2^3 \cdot 3 \cdot 5$. Note that $2^3=8$ is a factor of 120, but 9 is not, hence the leftmost digit of $N$ must be 8. Because $5 \cdot 3=15$ cannot be a digit, it follows that 5 must be immediately to the right of the 8, and 3 must be immediately to the right of the 5. The remaining rightmost two digits of $N$ must be 1, so that $N=85311$. Thus the sum of the digits of $N$ is 18.
}\SetValue{Rubric}{%Markdown



}\SetValue{Hint}{%
Solution Goes Here
}\SetValue{Answer}{%

}
\ProcessDATA



