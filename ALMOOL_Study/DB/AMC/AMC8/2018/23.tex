\SetValue{Module}{1}\SetValue{SectionAB}{A}\SetValue{MainChapter}{}\SetValue{SubChapter}{}\SetValue{Contents}{%%
   
From a regular octagon, a triangle is formed by connecting three randomly chosen vertices of the octagon. What is the probability that at least one of the sides of the triangle is also a side of the octagon?

\begin{center}		
    \includegraphics[scale=0.6]{AMC-8-pics/2018-23.png}
	\end{center}

\ansFIVEs{%start
		\frac{2}{7} }{%<--- (A)
		\frac{5}{42} }{%<--- (B)
		\frac{11}{14} }{%<--- (C)
		\frac{5}{7} }{%<---- (D)
		\frac{6}{7} }%<---- (D)
}\SetValue{Concept}{%



}\SetValue{AltText}{%



}\SetValue{Solution}{%

Answer (D): For each side of the octagon, there are 6 triangles containing that side. Because the 8 triangles containing two adjacent sides of the octagon are counted twice, there are a total of $8 \cdot 6-8=40$ triangles sharing a side with the octagon. The total number of triangles that can be formed from the eight vertices is $\frac{8 \cdot 7 \cdot 6}{3!}=56$, so the probability is $\frac{40}{56}=\frac{5}{7}$.

}\SetValue{Rubric}{%Markdown



}\SetValue{Hint}{%
Solution Goes Here
}\SetValue{Answer}{%

}
\ProcessDATA



