\SetValue{Module}{1}\SetValue{SectionAB}{A}\SetValue{MainChapter}{}\SetValue{SubChapter}{}\SetValue{Contents}{%%
    
Laila took five math tests, each worth a maximum of 100 points. Laila's score on each test was an integer between 0 and 100, inclusive. Laila received the same score on the first four tests, and she received a higher score on the last test. Her average score on the five tests was 82. How many values are possible for Laila's score on the last test?

\ansFIVEs{%start
		4 }{%<--- (A)
		5 }{%<--- (B)
		9 }{%<--- (C)
		10 }{%<---- (D)
		18 }%<---- (D)
}\SetValue{Concept}{%



}\SetValue{AltText}{%



}\SetValue{Solution}{%

Answer (A): Because Laila's score on the last test was higher than her other test scores and all of her scores were integers, for each point below 82 that she scored on each of the first four tests, she had to score four points above 82 on the last test. Therefore, her possible scores on the last test were $86,90,94$, and 98 (with scores on the first four tests of $81,80,79$, and 78, respectively). Thus there are four possible values for her score on the last test.
}\SetValue{Rubric}{%Markdown



}\SetValue{Hint}{%
Solution Goes Here
}\SetValue{Answer}{%

}
\ProcessDATA



