\SetValue{Module}{1}\SetValue{SectionAB}{A}\SetValue{MainChapter}{}\SetValue{SubChapter}{}\SetValue{Contents}{%%
    
Abby, Bridget, and four of their classmates will be seated in two rows of three for a group picture, as shown.

\begin{center}		
    \includegraphics[scale=0.6]{AMC-8-pics/2018-11.png}
	\end{center}

If the seating positions are assigned randomly, what is the probability that Abby and Bridget are adjacent to each other in the same row or the same column?

\ansFIVEs{%start
		\frac{1}{3} }{%<--- (A)
		\frac{2}{5} }{%<--- (B)
		\frac{7}{15} }{%<--- (C)
		\frac{1}{2} }{%<---- (D)
		\frac{2}{3} }%<---- (D)
}\SetValue{Concept}{%



}\SetValue{AltText}{%



}\SetValue{Solution}{%

Answer (C): There are 6 possible positions for Abby, and this leaves 5 possible positions for Bridget. Because their order doesn't matter, the two girls can be placed in any of $\frac{6 \cdot 5}{2}=15$ pairs of positions. There are 2 pairs of positions that are adjacent in the top row, 2 pairs that are adjacent in the bottom row, and 3 pairs that are adjacent in the same column. So the probability that they occupy adjacent positions is $\frac{2+2+3}{15}=\frac{7}{15}$.

OR

There is a $\frac{2}{3}$ chance Abby is assigned to a corner position and a $\frac{1}{3}$ chance that Abby is assigned to a middle position. If Abby is assigned to a corner position, there is a $\frac{2}{5}$ chance that Bridget is adjacent to Abby. If Abby is assigned to a middle position, there is a $\frac{3}{5}$ chance that Bridget is adjacent to Abby. Thus, the probability that Abby and Bridget are adjacent to each other in the same row or column is $\frac{2}{3} \cdot \frac{2}{5}+\frac{1}{3} \cdot \frac{3}{5}=\frac{7}{15}$.
}\SetValue{Rubric}{%Markdown



}\SetValue{Hint}{%
Solution Goes Here
}\SetValue{Answer}{%

}
\ProcessDATA



