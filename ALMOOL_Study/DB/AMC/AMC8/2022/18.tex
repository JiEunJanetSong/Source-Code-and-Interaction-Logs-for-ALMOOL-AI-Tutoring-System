\SetValue{Module}{1}\SetValue{SectionAB}{A}\SetValue{MainChapter}{}\SetValue{SubChapter}{}\SetValue{Contents}{%%
    
The midpoints of the four sides of a rectangle are $(-3,0),(2,0),(5,4)$, and $(0,4)$. What is the area of the rectangle?

\ansFIVEs{%start
		 20 }{%<--- (A)
		 25 }{%<--- (B)
		 40 }{%<--- (C)
		 50 }{%<---- (D)
		 80 }%<---- (D)
}\SetValue{Concept}{%



}\SetValue{AltText}{%



}\SetValue{Solution}{%

We are given the midpoints of the sides of a rectangle:

- $A=(-3,0)$

- $B=(2,0)$

- $C=(5,4)$

- $D=(0,4)$

Key ldea:
- These midpoints form a rhombus.

- The diagonals of this rhombus are equal in length to the sides of the rectangle.

- So, area of rectangle = product of diagonals of the rhombus

Step 1: Find lengths of diagonals

- Diagonal AC:

$$
\sqrt{(5-(-3))^2+(4-0)^2}=\sqrt{8^2+4^2}=\sqrt{64+16}=\sqrt{80}
$$

- Diagonal BD:

$$
\sqrt{(0-2)^2+(4-0)^2}=\sqrt{(-2)^2+4^2}=\sqrt{4+16}=\sqrt{20}
$$


Step 2: Area of rhombus

Area of rhombus with diagonals $d_1$ and $d_2$ :

$$
\text { Area }=\frac{1}{2} d_1 d_2=\frac{1}{2} \cdot \sqrt{80} \cdot \sqrt{20}=\frac{1}{2} \cdot \sqrt{1600}=\frac{1}{2} \cdot 40=20
$$


Step 3: Area of rectangle

Since rhombus formed by midpoints has half the area of the rectangle:

Area of rectangle $=2 \cdot 20=40$

Answer: (C) 40

}\SetValue{Rubric}{%Markdown



}\SetValue{Hint}{%
Solution Goes Here
}\SetValue{Answer}{%

}
\ProcessDATA



