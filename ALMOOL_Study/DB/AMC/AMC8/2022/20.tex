\SetValue{Module}{1}\SetValue{SectionAB}{A}\SetValue{MainChapter}{}\SetValue{SubChapter}{}\SetValue{Contents}{%%
    
The grid below is to be filled with integers in such a way that the sum of the numbers in each row and the sum of the numbers in each column are the same. Four numbers are missing. The number $x$ in the lower left corner is larger than the other three missing numbers. What is the smallest possible value of $x$?

\begin{center}		
    \includegraphics[scale=0.6]{AMC-8-pics/2022-20.png}
	\end{center}

\ansFIVEs{%start
		-1 }{%<--- (A)
		5 }{%<--- (B)
		6 }{%<--- (C)
		8 }{%<---- (D)
		9 }%<---- (D)
}\SetValue{Concept}{%



}\SetValue{AltText}{%



}\SetValue{Solution}{%

Step 1: Use known row and column sums

From the grid:

- Top row sum: $-2+9+5=12$

- Right column sum: $5+(-1)+8=12$

So, each row and column must sum to 12.

Step 2: Let variables represent missing entries

Let the missing values be:

\begin{tabular}{|c|c|c|}
\hline-2 & 9 & 5 \\
\hline$y+10$ & -1 & $x-1$ \\
\hline$x$ & $y$ & 8 \\
\hline
\end{tabular}

Step 3: Use row and column totals

- Bottom row:

$$
x+y+8=12 \Rightarrow x+y=4
$$

- Middle row:

$$
(y+10)+(-1)+(x-1)=12 \Rightarrow x+y+8=12 \quad \text { (same as above) }
$$

- So all constraints reduce to:

$$
x+y=4
$$

Step 4: Use condition that $x$ is the largest among the 4 missing numbers

We compare $x$ with:

- $x-1$

- $y+10$

- $y$

Use equation (1): $y=4-x$

Then:

- $y+10=4-x+10=14-x$

Now compare:

- $x>x-1 \rightarrow$ always true

- $x>y \rightarrow x>4-x \Rightarrow 2 x>4 \Rightarrow x>2$

- $x>y+10 \rightarrow x>14-x \Rightarrow 2 x>14 \Rightarrow x>7$

So, smallest integer satisfying all:

$$
x>7 \Rightarrow x=8
$$

(D) 8.	
}\SetValue{Rubric}{%Markdown



}\SetValue{Hint}{%
Solution Goes Here
}\SetValue{Answer}{%

}
\ProcessDATA



