\SetValue{Module}{1}\SetValue{SectionAB}{A}\SetValue{MainChapter}{}\SetValue{SubChapter}{}\SetValue{Contents}{%%
    
The letter $\mathbf{M}$ in the figure below is first reflected over the line $q$ and then reflected over the line $p$. What is the resulting image?

\begin{center}		
    \includegraphics[scale=0.6]{AMC-8-pics/2022-04.png}
	\end{center}

\ansFIVEsT{%start
\includegraphics[scale=0.6]{AMC-8-pics/2022-04-a.png}}{%<--- (A)
\includegraphics[scale=0.6]{AMC-8-pics/2022-04-b.png}}{%<--- (B)
\includegraphics[scale=0.6]{AMC-8-pics/2022-04-c.png}}{%<--- (C)
\includegraphics[scale=0.6]{AMC-8-pics/2022-04-d.png}}{%<---- (D)
\includegraphics[scale=0.6]{AMC-8-pics/2022-04-e.png}}%<---- (D)

}\SetValue{Concept}{%



}\SetValue{AltText}{%



}\SetValue{Solution}{%

Step 1: Reflect the letter M over line $q$

- Line $q$ is vertical.

- Reflecting over a vertical line flips the figure horizontally.

\begin{center}		
    \includegraphics[scale=0.6]{AMC-8-pics/2022-04-s-1.png}
	\end{center}

Step 2: Reflect the new image over line $p$

- Line $p$ is horizontal.

- Reflecting over a horizontal line flips the figure vertically.
    
\begin{center}		
    \includegraphics[scale=0.6]{AMC-8-pics/2022-04-s-2.png}
	\end{center}

Final result:

- A horizontal flip, followed by a vertical flip, results in a $180^{\circ}$ rotation.

- The image is upside-down and reversed.

Therefore, the answer is (E). 
}\SetValue{Rubric}{%Markdown



}\SetValue{Hint}{%
Solution Goes Here
}\SetValue{Answer}{%

}
\ProcessDATA



