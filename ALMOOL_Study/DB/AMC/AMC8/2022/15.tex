\SetValue{Module}{1}\SetValue{SectionAB}{A}\SetValue{MainChapter}{}\SetValue{SubChapter}{}\SetValue{Contents}{%%
    
Laszlo went online to shop for black pepper and found thirty different black pepper options varying in weight and price, shown in the scatter plot below. In ounces, what is the weight of the pepper that offers the lowest price per ounce?

\begin{center}		
    \includegraphics[scale=0.6]{AMC-8-pics/2022-15.png}
	\end{center}

\ansFIVEs{%start
		1 }{%<--- (A)
		2 }{%<--- (B)
		3 }{%<--- (C)
		4 }{%<---- (D)
		5 }%<---- (D)
		
}\SetValue{Concept}{%

라즐로는 후추를 구입하기 위해 온라인에 접속하여 아래 산점도와 같이 무게와 가격이 다른 30가지의 후추 옵션을 발견했습니다. 온스당 가장 낮은 가격을 제공하는 후추의 무게는 온스 단위로 얼마인가요?

}\SetValue{AltText}{%



}\SetValue{Solution}{%

\begin{center}		
    \includegraphics[scale=0.6]{AMC-8-pics/2022-15-s.png}
	\end{center}

	We are looking for a black point, that when connected to the origin, yields the lowest slope. The slope represents the price per ounce. We can visually find that the point with the lowest slope is the blue point. Furthermore, it is the only one with a price per ounce significantly less than 1. Finally, we see that the blue point is in the category with a weight of (C) 3 ounces.	
}\SetValue{Rubric}{%Markdown



}\SetValue{Hint}{%
Solution Goes Here
}\SetValue{Answer}{%

}
\ProcessDATA



