\SetValue{Module}{1}\SetValue{SectionAB}{A}\SetValue{MainChapter}{}\SetValue{SubChapter}{}\SetValue{Contents}{%%
    
One sunny day, Ling decided to take a hike in the mountains. She left her house at 8 AM , drove at a constant speed of 45 miles per hour, and arrived at the hiking trail at 10 Am . After hiking for 3 hours, Ling drove home at a constant speed of 60 miles per hour. Which of the following graphs best illustrates the distance between Ling's car and her house over the course of her trip?

\ansFIVEsT{%start
	\includegraphics[scale=0.6]{AMC-8-pics/2022-10-a.png} }{%<--- (A)
	\includegraphics[scale=0.6]{AMC-8-pics/2022-10-b.png} }{%<--- (B)
	\includegraphics[scale=0.6]{AMC-8-pics/2022-10-c.png}}{%<--- (C)
	\includegraphics[scale=0.6]{AMC-8-pics/2022-10-d.png}}{%<---- (D)
	\includegraphics[scale=0.6]{AMC-8-pics/2022-10-e.png}}%<---- (D)
	
}\SetValue{Concept}{%



}\SetValue{AltText}{%



}\SetValue{Solution}{%

8:00 AM - 10:00 AM

- Drove away from home at 45 mph for 2 hours

- Distance from home: $45 \times 2=90$ miles

- Graph: increasing line from 0 to 90 miles

10:00 AM - 1:00 PM

- Hiking $\rightarrow$ car didn't move

- Distance stayed at 90 miles

- Graph: horizontal line at 90 miles

1:00 PM-2:30 PM

- Drove home at 60 mph, total distance = 90 miles

- Time $=90 \div 60=1.5$ hours

- Graph: decreasing line from 90 to 0 miles

Correct graph:

Starts at $0 \rightarrow$ rises steadily to 90 by 10 AM $\rightarrow$ flat until 1 PM $\rightarrow$ falls steadily back to 0 by 2:30 PM

Answer: (E)
}\SetValue{Rubric}{%Markdown



}\SetValue{Hint}{%
Solution Goes Here
}\SetValue{Answer}{%

}
\ProcessDATA



