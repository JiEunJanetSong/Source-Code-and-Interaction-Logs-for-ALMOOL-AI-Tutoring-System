\SetValue{Module}{1}\SetValue{SectionAB}{A}\SetValue{MainChapter}{}\SetValue{SubChapter}{}\SetValue{Contents}{%%
    
The arrows on the two spinners shown below are spun. Let the number $N$ equal 10 times the number on Spinner A, added to the number on Spinner B. What is the probability that $N$ is a perfect square number?

\begin{center}		
    \includegraphics[scale=0.6]{AMC-8-pics/2022-12.png}
	\end{center}

\ansFIVEs{%start
		\frac{1}{16} }{%<--- (A)
		\frac{1}{8} }{%<--- (B)
		\frac{1}{4} }{%<--- (C)
		\frac{3}{8} }{%<---- (D)
		\frac{1}{2} }%<---- (D)
		
}\SetValue{Concept}{%

아래에 표시된 두 개의 스피너의 화살표가 회전하고 있습니다. 스피너 A의 숫자에 스피너 B의 숫자를 더한 값의 10배인 숫자 $N$이 완벽한 제곱수일 확률은 얼마인가?





}\SetValue{AltText}{%



}\SetValue{Solution}{%

We are given:

- Spinner A has values: 5, 6, 7, 8

- Spinner B has values: 1, 2, 3, 4

- Define $N=10 \times(\mathrm{A})+(\mathrm{B})$

- Find the probability that $N$ is a perfect square

Step 1: Total possible outcomes

4 options for $\mathrm{A} \times 4$ options for $\mathrm{B}=16$ outcomes

Step 2: List all possible $N=10 A+B$

- $A=5 \rightarrow N=51,52,53,54$

- $A=6 \rightarrow N=61,62,63,64$

- $\mathrm{A}=7 \rightarrow \mathrm{~N}=71,72,73,74$

- $A=8 \rightarrow N=81,82,83,84$

Now identify the perfect squares among these:

- $64=8^2$

- $81=9^2$

Only 2 perfect squares

Step 3: Probability

$$
\frac{2}{16}=\frac{1}{8}
$$


Answer: (B) $\frac{1}{8}$

}\SetValue{Rubric}{%Markdown



}\SetValue{Hint}{%
Solution Goes Here
}\SetValue{Answer}{%

}
\ProcessDATA



