\SetValue{Module}{1}\SetValue{SectionAB}{A}\SetValue{MainChapter}{}\SetValue{SubChapter}{}\SetValue{Contents}{%%
    
Malcolm wants to visit Isabella after school today and knows the street where she lives but doesn't know her house number. She tells him, ``My house number has two digits, and exactly three of the following four statements about it are true."

\begin{center}		
    \includegraphics[scale=0.6]{AMC-8-pics/2017-08.png}
	\end{center}
\ansFOURsT{%start
		It is prime. }{%<--- (A)
		It is even. }{%<--- (B)
		It is divisible by 7. }{%<--- (C)
		One of its digits is 9. }
}\SetValue{Concept}{%






}\SetValue{AltText}{%



}\SetValue{Solution}{%

Answer (A): We are told that exactly three statements are true, and one is false.

We need to find a two-digit number that fits this condition.

Let's consider statement (A) and (B):

- If a number is even and prime, it must be 2.

- But 2 is not a two-digit number, so both $(\mathrm{A})$ and $(\mathrm{B})$ cannot be true together for any two-digit number.

(E) Therefore, at least one of (A) or (B) must be false.

Let's test the number 98:

- Is 98 prime?

No. $98=2 \times 49$, so it is not prime. $\rightarrow(\mathrm{A})$ is false.

- Is 98 even?

Yes. $\rightarrow$ (B) is true.

- Is 98 divisible by 7?

Yes. $98 \div 7=14 \rightarrow$ (C) is true.

- Does 98 have a digit 9?

Yes. $\rightarrow$ (D) is true.

So for 98 , the truth values are:

- (A) $>$ False

- (B) $\sqrt{ }$ True

- (C) $\sqrt{ }$ True

- (D) $\sqrt{ }$ True

Exactly three statements are true, and one (A) is false.

This satisfies Isabella's condition.

}\SetValue{Rubric}{%Markdown



}\SetValue{Hint}{%
Solution Goes Here
}\SetValue{Answer}{%

}
\ProcessDATA



