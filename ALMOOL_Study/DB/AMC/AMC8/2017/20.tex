\SetValue{Module}{1}\SetValue{SectionAB}{A}\SetValue{MainChapter}{}\SetValue{SubChapter}{}\SetValue{Contents}{%%
    
An integer between 1000 and 9999, inclusive, is chosen at random. What is the probability that it is an odd integer whose digits are all distinct?

\ansFIVEs{%start
		\frac{14}{75} }{%<--- (A)
		\frac{56}{225} }{%<--- (B)
		\frac{107}{400} }{%<--- (C)
		\frac{7}{25} }{%<---- (D)
		\frac{9}{25} }%<---- (D)
}\SetValue{Concept}{%

1000 이상 9999 이하의 네 자리 정수 가운데 하나를 임의로 고른다.
이 수가 자릿수가 모두 서로 다르면서 홀수일 확률은 얼마인가?

1. 전체 경우의 수
- 1000 부터 9999 까지의 정수는 모두 $9999-1000+1=9000$ 개이다.
2. 조건을 만족하는 수 세기
1. 일의 자리(홀수 조건)
- 홀수가 되려면 일의 자리는 $1,3,5,7,9$ 중 하나여야 한다.
$\rightarrow 5$ 가지 선택.
2. 천의 자리 (첫 자리는 0 이 아님)
- 천의 자리는 $1 \sim 9$ 중에서 일의 자리와 겹치지 않는 숫자를 고른다.
$\rightarrow 9-1=8$ 가지 선택.
3. 백의 자리
- 0~9 중에서 이미 사용한 두 자리(천•일 자리)를 제외한 숫자를 고른다.
$\rightarrow 10-2=8$ 가지 선택.
4. 십의 자리
- 0~9 중에서 이미 사용한 세 자리(천•백•일 자리)를 제외한 숫자를 고른다.
$\rightarrow 10-3=7$ 가지 선택.
따라서 조건을 만족하는 수의 개수는

$$
5 \times 8 \times 8 \times 7=2240 .
$$

3. 확률 계산

$$
\frac{2240}{9000}=\frac{56}{225} .
$$

}\SetValue{AltText}{%



}\SetValue{Solution}{%

Answer (B): There are 9000 integers between 1000 and 9999 inclusive. For an integer to be odd it must end in $1,3,5,7$, or 9. So there are 5 choices for the units digit. For a number to be between 1000 and 9999 the thousands digit must be nonzero and so there are now 8 choices for the thousands digit. For the hundreds digit there are 8 choices and for the tens digit there are 7 choices for a total number of $5 \cdot 8 \cdot 8 \cdot 7=2240$ choices. So the probability is $\frac{2240}{9000}=\frac{56}{225}$.
}\SetValue{Rubric}{%Markdown



}\SetValue{Hint}{%
Solution Goes Here
}\SetValue{Answer}{%

}
\ProcessDATA



