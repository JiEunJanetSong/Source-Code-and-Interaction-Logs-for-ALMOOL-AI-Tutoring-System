\SetValue{Module}{1}\SetValue{SectionAB}{A}\SetValue{MainChapter}{}\SetValue{SubChapter}{}\SetValue{Contents}{%%
    
In the right triangle $A B C, A C=12, B C=5$, and angle $C$ is a right angle. A semicircle is inscribed in the triangle as shown. What is the radius of the semicircle?

\begin{center}		
    \includegraphics[scale=0.6]{AMC-8-pics/2017-22.png}
	\end{center}

\ansFIVEs{%start
		\frac{7}{6} }{%<--- (A)
		\frac{13}{5} }{%<--- (B)
		\frac{59}{18} }{%<--- (C)
		\frac{10}{3} }{%<---- (D)
		\frac{60}{13} }%<---- (D)
}\SetValue{Concept}{%



}\SetValue{AltText}{%



}\SetValue{Solution}{%

Answer (D): Let $O$ be the center of the inscribed semicircle on $\overline{A C}$, and let $D$ be the point at which $\overline{A B}$ is tangent to the semicircle. Because $\overline{O D}$ is a radius of the semicircle it is perpendicular to $\overline{A B}$, making $\overline{O D}$ an altitude of $\triangle A O B$. By the Pythagorean Theorem, $A B=13$. In the diagram, $\overline{O B}$ partitions $\triangle A B C$ so that

$$
\operatorname{Area}(\triangle A B C)=\operatorname{Area}(\triangle B O C)+\operatorname{Area}(\triangle A O B)
$$


Since we know $\triangle A B C$ has area 30 , we have

$$
\begin{aligned}
30 & =\operatorname{Area}(\triangle B O C)+\operatorname{Area}(\triangle A O B) \\
& =\frac{1}{2}(B C) r+\frac{1}{2}(A B) r=\frac{5}{2} r+\frac{13}{2} r=9 r
\end{aligned}
$$


Therefore $r=\frac{30}{9}=\frac{10}{3}$.

\begin{center}		
    \includegraphics[scale=0.6]{AMC-8-pics/2017-22.png}
	\end{center}

	OR

	Because $\overline{O D}$ is a radius of the semicircle, it is perpendicular to $\overline{A B}$, making $\triangle A D O$ similar to $\triangle A C B$. Because $\overline{B C}$ and $\overline{B D}$ are both tangent to the semicircle, they are congruent. So $B D=5$ and $A D=8$. It follows that $\frac{r}{8}=\frac{5}{12}$ and so $r=\frac{40}{12}=\frac{10}{3}$.

}\SetValue{Rubric}{%Markdown



}\SetValue{Hint}{%
Solution Goes Here
}\SetValue{Answer}{%

}
\ProcessDATA



