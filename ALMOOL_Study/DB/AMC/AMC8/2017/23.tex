\SetValue{Module}{1}\SetValue{SectionAB}{A}\SetValue{MainChapter}{}\SetValue{SubChapter}{}\SetValue{Contents}{%%
    
Each day for four days, Linda traveled for one hour at a speed that resulted in her traveling one mile in an integer number of minutes. Each day after the first, her speed decreased so that the number of minutes to travel one mile increased by 5 minutes over the preceding day. Each of the four days, her distance traveled was also an integer number of miles. What was the total number of miles for the four trips?

\ansFIVEs{%start
		10 }{%<--- (A)
		15 }{%<--- (B)
		25 }{%<--- (C)
		50 }{%<---- (D)
		82 }%<---- (D)
}\SetValue{Concept}{%



}\SetValue{AltText}{%



}\SetValue{Solution}{%

Answer (C): Her time for each trip was 60 minutes. The factors of 60 are $1,2,3,4,5,6$, $10,12,15,20,30$, and 60. Her daily number of minutes for a mile form a sequence of four numbers where each number is 5 more than the previous number. Also, these numbers must each be a factor of 60 since the number of miles traveled must be an integer. The only such sequence from among the factors of 60 is $5,10,15,20$. So her rates in miles per minute for the four days were $\frac{1}{5}, \frac{1}{10}, \frac{1}{15}, \frac{1}{20}$, and multiplying each by 60 minutes gives her distances in miles as $12,6,4$, and 3, for a total distance of 25 miles.


}\SetValue{Rubric}{%Markdown



}\SetValue{Hint}{%
Solution Goes Here
}\SetValue{Answer}{%

}
\ProcessDATA



