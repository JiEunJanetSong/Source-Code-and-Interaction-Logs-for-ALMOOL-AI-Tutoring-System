\SetValue{Module}{1}\SetValue{SectionAB}{A}\SetValue{MainChapter}{}\SetValue{SubChapter}{}\SetValue{Contents}{%%
    
A box contains five cards, numbered $1$, $2$, $3$, $4$, and $5$. Three cards are selected randomly without replacement from the box. What is the probability that 4 is the largest value selected?

\ansFIVEs{%start
		\tfrac{1}{10} }{%<--- (A)
		\tfrac{1}{5} }{%<--- (B)
		\tfrac{3}{10} }{%<--- (C)
		\tfrac{2}{5} }{%<---- (D)
		\tfrac{1}{2} }%<---- (D)
}\SetValue{Concept}{%



}\SetValue{AltText}{%



}\SetValue{Solution}{%

Answer (C): There are 10 possible equally likely outcomes:

\begin{center}		
    \includegraphics[scale=0.6]{AMC-8-pics/2017-10-s.png}
	\end{center}

The three highlighted outcomes have 4 as the largest value selected. Hence the probability is $\frac{3}{10}$.

OR

There are 10 ways to select 3 cards without replacement from a box of 5 cards. If the largest value selected is 4, then the remaining two cards can be selected from the cards 1, 2, and 3 in 3 ways. So, the probability that 4 is the largest value selected is $\frac{3}{10}$.
11. 
}\SetValue{Rubric}{%Markdown



}\SetValue{Hint}{%
Solution Goes Here
}\SetValue{Answer}{%

}
\ProcessDATA



