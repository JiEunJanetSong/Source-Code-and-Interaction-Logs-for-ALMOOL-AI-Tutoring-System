\SetValue{Module}{1}\SetValue{SectionAB}{A}\SetValue{MainChapter}{}\SetValue{SubChapter}{}\SetValue{Contents}{%%
    
In the figure shown, $\overline{U S}$ and $\overline{U T}$ are line segments each of length 2, and $m \angle T U S=60^{\circ}$. Arcs ${T R}$ and ${S R}$ are each one-sixth of a circle with radius 2. What is the area of the region shown?

\begin{center}		
    \includegraphics[scale=0.6]{AMC-8-pics/2017-25.png}
	\end{center}
	
\ansFIVEs{%start
		3 \sqrt{3}-\pi }{%<--- (A)
		4 \sqrt{3}-\frac{4 \pi}{3} }{%<--- (B)
		2 \sqrt{3} }{%<--- (C)
		4 \sqrt{3}-\frac{2 \pi}{3} }{%<---- (D)
		4+\frac{4 \pi}{3} }%<---- (D)
}\SetValue{Concept}{%



}\SetValue{AltText}{%



}\SetValue{Solution}{%

Answer (B): The region shown is what remains when two one-sixth sectors of a circle of radius 2 are removed from an equilateral triangle with side length 4.

\begin{center}		
    \includegraphics[scale=0.6]{AMC-8-pics/2017-25-s.png}
	\end{center}

The area of an equaliateral triangle with side length $s$ is $\frac{\sqrt{3}}{4} s^2$. Thus the area of the region is $4 \sqrt{3}-2\left(\frac{1}{6} \cdot 4 \pi\right)=4 \sqrt{3}-\frac{4 \pi}{3}$.


}\SetValue{Rubric}{%Markdown



}\SetValue{Hint}{%
Solution Goes Here
}\SetValue{Answer}{%

}
\ProcessDATA



