\SetValue{Module}{1}\SetValue{SectionAB}{A}\SetValue{MainChapter}{}\SetValue{SubChapter}{}\SetValue{Contents}{%%
    
Mrs. Sanders has three grandchildren, who call her regularly. One calls her every three days, one calls her every four days, and one calls her every five days. All three called her on December 31, 2016. On how many days during the next year did she not receive a phone call from any of her grandchildren?

\ansFIVEs{%start
		78 }{%<--- (A)
		80 }{%<--- (B)
		144 }{%<--- (C)
		146 }{%<---- (D)
		152 }%<---- (D)
}\SetValue{Concept}{%



}\SetValue{AltText}{%



}\SetValue{Solution}{%

Answer (D): During a 60-day cycle, there are 20 days that the first one calls, 15 days that the second one calls, and 12 days that the third one calls. The sum $20+15+12=47$ overcounts the number of days when more than one grandchild called. There were $60 \div 12=5$ days when the first and second called. There were $60 \div 15=4$ days when the first and third called. There were $60 \div 20=3$ days when the second and third called. Subtracting $5+4+3$ from 47 leaves 35 days. But the $60^{\text {th }}$ day was added in three times and subtracted out three times, so there were 36 days in which she received at least one phone call. Thus, in each 60 -day cycle, there were $60-36=24$ days without a phone call. In a year, there are six full cycles. Additionally, she receives no phone call on the $361^{\text {st }}$ or $362^{\text {nd }}$ day. Therefore, the total number of days that she does not receive a phone call is $6 \cdot 24+2=146$ days.

OR

In the Venn diagram below, let $A$ be the set of days in the year in which the first grandchild calls her, let $B$ be the set of days in which the second calls her, and let $C$ be the set of days in which the third calls her. Then the region common to the 3 sets represents the days on which all 3 call, which are every $3 \cdot 4 \cdot 5=60$ days, or 6 days in the year.

\begin{center}		
    \includegraphics[scale=0.6]{AMC-8-pics/2017-24-s.png}
	\end{center}

	The region common to $A$ and $B$ represents the days on which the first and second grandchild call, which are every $3 \cdot 4=12$ days, or 30 days in the year. Since 6 of those days have already been counted, we label the region below the 6 with $30-6=24$. Similarly, the region common to $A$ and $C$ is 24 , so the region to the left of the common region is $24-6=18$, and the region to the right is $18-6=12$.

	Now the first grandchild calls on 121 days, of which 48 have already been counted. Thus the lower left region contains 73 days. Similarly, $B$ contains 91 days, so the lower right region contains 49 days, and the top region has 37 days.

All the regions total 219 days, so the number of days without a call is $365-219=146$ days. Note that in leap years, the lower left region increases to 74, so the answer is still $366-220=146$ days.


}\SetValue{Rubric}{%Markdown



}\SetValue{Hint}{%
Solution Goes Here
}\SetValue{Answer}{%

}
\ProcessDATA



