\SetValue{Module}{1}\SetValue{SectionAB}{A}\SetValue{MainChapter}{}\SetValue{SubChapter}{}\SetValue{Contents}{%%
    
Let $Z$ be a 6-digit positive integer, such as 247247, whose first three digits are the same as its last three digits taken in the same order. Which of the following numbers must be a factor of $Z$?

\ansFIVEs{%start
		11 }{%<--- (A)
		19 }{%<--- (B)
		101 }{%<--- (C)
		111 }{%<---- (D)
		1111 }%<---- (D)
}\SetValue{Concept}{%



}\SetValue{AltText}{%



}\SetValue{Solution}{%

Answer (A): Assume $Z$ has the form $a b c a b c$. Then

$$
Z=1001 \cdot a b c=7 \cdot 11 \cdot 13 \cdot a b c
$$


So 11 must be a factor of $Z$.

OR

A positive integer is divisible by 11 if and only if the difference of the sums of the digits in the even and odd positions in the number is divisible by 11 . For $Z=a b c a b c$ the sum of the digits in the even positions is equal to the sum of the digits in the odd positions, so the difference of the two sums is 0. Hence, 11 divides $Z$.
}\SetValue{Rubric}{%Markdown



}\SetValue{Hint}{%
Solution Goes Here
}\SetValue{Answer}{%

}
\ProcessDATA



