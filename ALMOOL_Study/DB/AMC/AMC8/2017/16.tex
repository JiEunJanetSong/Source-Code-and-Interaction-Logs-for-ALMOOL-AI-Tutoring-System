\SetValue{Module}{1}\SetValue{SectionAB}{A}\SetValue{MainChapter}{}\SetValue{SubChapter}{}\SetValue{Contents}{%%
    
In the figure shown below, choose point $D$ on side $\overline{B C}$ so that $\triangle A C D$ and $\triangle A B D$ have equal perimeters. What is the area of $\triangle A B D$?

\begin{center}		
    \includegraphics[scale=0.6]{AMC-8-pics/2017-16.png}
	\end{center}

\ansFIVEs{%start
		\frac{3}{4} }{%<--- (A)
		\frac{3}{2} }{%<--- (B)
		2 }{%<--- (C)
		\frac{12}{5} }{%<---- (D)
		\frac{5}{2} }%<---- (D)
}\SetValue{Concept}{%



}\SetValue{AltText}{%



}\SetValue{Solution}{%

Answer (D): Because the perimeters of $\triangle A D C$ and $\triangle A D B$ are equal, $C D=3$ and $B D=2$.

\begin{center}		
    \includegraphics[scale=0.6]{AMC-8-pics/2017-16-s.png}
	\end{center}

$\triangle A D C$ and $\triangle A D B$ have the same altitude from $A$, so the area of $\triangle A D C$ will be $3 / 5$ of the area of $\triangle A B C$, and $\triangle A D B$ will be $\frac{2}{5}$ of the area of $\triangle A B C$. The area of $\triangle A B C$ is $\frac{1}{2} \cdot 3 \cdot 4=6$, so the area of $\triangle A D B$ is $\frac{2}{5} \cdot 6=12 / 5$.	
}\SetValue{Rubric}{%Markdown



}\SetValue{Hint}{%
Solution Goes Here
}\SetValue{Answer}{%

}
\ProcessDATA



