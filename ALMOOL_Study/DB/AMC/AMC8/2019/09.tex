\SetValue{Module}{1}\SetValue{SectionAB}{A}\SetValue{MainChapter}{}\SetValue{SubChapter}{}\SetValue{Contents}{%%
    
Alex and Felicia each have cats as pets. Alex buys cat food in cylindrical cans that are 6 cm in diameter and 12 cm high. Felicia buys cat food in cylindrical cans that are 12 cm in diameter and 6 cm high. What is the ratio of the volume of one of Alex's cans to the volume one of Felicia's cans?

	\ansFIVEs{%start
		1: 4}{%<--- (A)
		1: 2 }{%<--- (B)
		1: 1 }{%<--- (C)
		2: 1 }{%<---- (D)
	4: 1}%<---- (D)

}\SetValue{Concept}{%

알렉스와 펠리시아는 각각 고양이를 반려동물로 키우고 있습니다. 알렉스는 지름 6cm, 높이 12cm의 원통형 캔에 담긴 고양이 사료를 구입합니다. 펠리시아는 지름 12cm, 높이 6cm의 원통형 캔에 담긴 고양이 사료를 구입합니다. 알렉스의 캔 중 하나의 부피와 펠리시아의 캔 중 하나의 부피의 비율은 얼마인가요?

}\SetValue{AltText}{%



}\SetValue{Solution}{%

The volume of a cylinder with radius $r$ and height $h$ is $V=\pi r^2 h$. Alex's cans have radius 3 cm and height 12 cm , so the volume of one of his cans is $\pi \cdot 3^2 \cdot 12=108 \pi \mathrm{~cm}^3$. Felicia's cans have radius 6 cm and height 6 cm , so the volume of one of her cans is $\pi \cdot 6^2 \cdot 6=$ $216 \pi \mathrm{~cm}^3$. Thus the ratio of the volume of one of Alex's cans to the volume of one of Felicia's cans is $108 \pi: 216 \pi=1: 2$.

OR

Because the volume of a cylindrical can depends upon the height and the square of the radius, when the height is doubled, the volume is multiplied by 2, and when the radius is divided by 2, the volume is divided by 4. Thus doubling the height and halving the radius results in Alex's cans having a volume that is one-half the volume of Felicia's cans. The requested ratio is therefore $1: 2$.
}\SetValue{Rubric}{%Markdown



}\SetValue{Hint}{%
Solution Goes Here
}\SetValue{Answer}{%
Answer (B)
}
\ProcessDATA



