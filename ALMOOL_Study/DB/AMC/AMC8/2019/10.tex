\SetValue{Module}{1}\SetValue{SectionAB}{A}\SetValue{MainChapter}{}\SetValue{SubChapter}{}\SetValue{Contents}{%%
  
The diagram shows the number of students at soccer practice each weekday during last week. After computing the mean and median values, Coach discovers that there were actually 21 participants on Wednesday. Which of the following statements describes the change in the mean and median after the correction is made?

\begin{center}		
    \includegraphics[scale=0.6]{AMC-8-pics/2019-10.png}
	\end{center}
	
	\ansFIVEsT{%start
		The mean increases by 1 and the median does not change.}{%<--- (A)
		The mean increases by 1 and the median increases by 1. }{%<--- (B)
		The mean increases by 1 and the median increases by 5. }{%<--- (C)
		The mean increases by 5 and the median increases by 1. }{%<---- (D)
	The mean increases by 5 and the median increases by 5.}%<---- (D)

}\SetValue{Concept}{%



}\SetValue{AltText}{%



}\SetValue{Solution}{%

The original values in increasing order are 16, 16, 20, 22, and 26, and thus the median (middle) value is 20. After replacing one of the 16 s with 21, the new values are 16,20, 21,22, and 26, with a median of 21. Therefore the median value increases by 1.

An increase of 5 in the number of participants on one day is equivalent to an increase of 1 participant per day. Therefore the mean (average) value increases by 1. (The mean values could also be directly computed. The mean is 20 for the original set of values and 21 after the mistake has been corrected.)
}\SetValue{Rubric}{%Markdown



}\SetValue{Hint}{%
Solution Goes Here
}\SetValue{Answer}{%
Answer (B)
}
\ProcessDATA



