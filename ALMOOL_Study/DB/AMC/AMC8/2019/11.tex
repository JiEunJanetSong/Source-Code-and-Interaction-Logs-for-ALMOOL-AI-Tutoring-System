\SetValue{Module}{1}\SetValue{SectionAB}{A}\SetValue{MainChapter}{}\SetValue{SubChapter}{}\SetValue{Contents}{%%
    
The third-grade class at Lincoln Elementary School has 93 students. Each student takes a math class or a foreign language class or both. There are 70 third graders taking a math class, and there are 54 third graders taking a foreign language class. How many third graders take only a math class and not a foreign language class?

	\ansFIVEs{%start
		16}{%<--- (A)
		23 }{%<--- (B)
		31 }{%<--- (C)
		39 }{%<---- (D)
	70}%<---- (D)

}\SetValue{Concept}{%



}\SetValue{AltText}{%



}\SetValue{Solution}{%

The total enrollment in math classes and foreign language classes is $70+54=$ 124 students. There are 93 students, so there must be $124-93=31$ students who are counted twice because they are taking both a math class and a foreign language class. There are therefore $70-31=39$ students taking only a math class and not a foreign language class.

OR

In the Venn diagram below, the circle on the left represents the students taking a math class and the circle on the right represents the students taking a foreign language class. Let $x$ represent the number of students taking both a math class and a foreign language class. Then $70-x$ students take only a math class and $54-x$ students take only a foreign language class. Because $(70-x)+x+(54-x)=93$, it follows that $x=31$. There are therefore $70-31=39$ students who take only a math class and not a foreign language class.

\begin{center}		
    \includegraphics[scale=0.6]{AMC-8-pics/2019-11-s.png}
	\end{center}
}\SetValue{Rubric}{%Markdown



}\SetValue{Hint}{%
Solution Goes Here
}\SetValue{Answer}{%
Answer (D)
}
\ProcessDATA



