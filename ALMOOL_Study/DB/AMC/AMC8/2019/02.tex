\SetValue{Module}{1}\SetValue{SectionAB}{A}\SetValue{MainChapter}{}\SetValue{SubChapter}{}\SetValue{Contents}{%%
    
Three identical rectangles are put together to form rectangle $ABCD$, as shown in the figure below. Given that the length of the shorter side of each of the smaller rectangles is 5 feet, what is the area in square feet of rectangle $ABCD$?
    
\begin{center}		
    \includegraphics[scale=0.6]{AMC-8-pics/2019-02.png}
	\end{center}

\ansFIVEs{%start
45}{%<--- (A)
 75}{%<--- (B)
 100}{%<--- (C)
 125}{%<---- (D)
150}%<---- (D)


}\SetValue{Concept}{%

아래 그림과 같이 세 개의 동일한 직사각형이 합쳐져 직사각형 $ABCD$를 형성합니다. 각각의 작은 직사각형의 짧은 변의 길이가 5피트라고 가정할 때, 직사각형 $ABCD$의 면적(평방 피트)은 얼마인가요?

}\SetValue{AltText}{%



}\SetValue{Solution}{%

The length of side $\overline{A D}$ is 10 feet because it is the sum of the lengths of the shorter sides of two of the small rectangles. It follows that the small rectangles are 5 feet by 10 feet. Therefore rectangle $A B C D$ is 10 feet by 15 feet with an area of 150 square feet.

\begin{center}		
    \includegraphics[scale=0.6]{AMC-8-pics/2019-02-s.png}
	\end{center}


}\SetValue{Rubric}{%Markdown



}\SetValue{Hint}{%
Solution Goes Here
}\SetValue{Answer}{%
Answer (E)
}
\ProcessDATA


