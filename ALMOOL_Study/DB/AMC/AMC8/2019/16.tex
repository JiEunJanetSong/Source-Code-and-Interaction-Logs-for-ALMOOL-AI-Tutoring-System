\SetValue{Module}{1}\SetValue{SectionAB}{A}\SetValue{MainChapter}{}\SetValue{SubChapter}{}\SetValue{Contents}{%%
    
Qiang drives 15 miles at an average speed of 30 miles per hour. How many additional miles will he have to drive at 55 miles per hour to average 50 miles per hour for the entire trip?

	\ansFIVEs{%start
		 45}{%<--- (A)
		 62 }{%<--- (B)
		 90 }{%<--- (C)
		 110 }{%<---- (D)
	 135}%<---- (D)

}\SetValue{Concept}{%

치앙은 시속 30마일의 평균 속도로 15마일을 운전합니다. 전체 여정 동안 평균 시속 50마일을 주행하려면 시속 55마일로 몇 마일을 더 주행해야 할까요?

}\SetValue{AltText}{%



}\SetValue{Solution}{%

At 30 mph, Qiang travels the first 15 miles in $\frac{15 \mathrm{miles}}{30 \mathrm{mph}}=\frac{1}{2}$ hour. If he had driven the first half hour at 50 mph, he would have traveled 25 miles. So he needs to make up the extra $25-15=10$ miles in the second part of the trip, in which he is driving 5 mph faster than his overall average. He will require $\frac{10 \text { miles }}{5 \mathrm{mph}}=2$ hours to make up the extra 10 miles. In that time, he will travel $55 \cdot 2=110$ miles.

OR

Let $t$ represent the time in hours for the second part of the trip. The distance for the first part of the trip is 15 miles, and the distance for the second part of the trip is the average speed, 55 mph, times $t$. The distance for the entire trip is the average speed, 50 mph, times the total time, $\frac{1}{2}+t$. Thus

$$
15+55 t=50\left(\frac{1}{2}+t\right)
$$

and it follows that $t=2$ hours. The distance traveled in 2 hours at 55 mph is 110 miles.
}\SetValue{Rubric}{%Markdown



}\SetValue{Hint}{%
Solution Goes Here
}\SetValue{Answer}{%
Answer (D)
}
\ProcessDATA



