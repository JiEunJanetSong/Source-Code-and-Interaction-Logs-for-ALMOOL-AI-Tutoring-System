\SetValue{Module}{1}\SetValue{SectionAB}{A}\SetValue{MainChapter}{}\SetValue{SubChapter}{}\SetValue{Contents}{%%
    
Quadrilateral $ABCD$ is a rhombus with perimeter $52$ meters. The length of diagonal $\overline{AC}$ is $24$ meters. What is the area in square meters of rhombus $ABCD$?
    
\begin{center}		
    \includegraphics[scale=0.6]{AMC-8-pics/2019-04.png}
	\end{center}

    \ansFIVEs{%start
	60 }{%<--- (A)
	90 }{%<--- (B)
	105 }{%<--- (C)
	120 }{%<---- (D)
    144}
}\SetValue{Concept}{%

사변형의 $ABCD$는 둘레가 $52$ 미터인 마름모입니다. 대각선 $\overline{AC}$의 길이는 $24$ 미터입니다. 마름모 $ABCD$의 면적(제곱미터)은 얼마인가요?

}\SetValue{AltText}{%



}\SetValue{Solution}{%

Let $M$ be the midpoint of $\overline{A C}$. Then $A M=12$. Because the diagonals of a rhombus are perpendicular and bisect each other, $\overline{B M}$ is perpendicular to $\overline{A C}$. Because all four sides of a rhombus have the same length, $A B=\frac{52}{4}=13$. By the Pythagorean Theorem, $B M=\sqrt{13^2-12^2}=5$. The area of $\triangle A B M$ is $\frac{1}{2} \cdot 12 \cdot 5=30$ square meters. Thus the area of rhombus $A B C D$ is $4 \cdot 30=120$ square meters.

\begin{center}		
    \includegraphics[scale=0.6]{AMC-8-pics/2019-04-s.png}
	\end{center}

	OR

As above, $B M=5$, and so $B D=10$. The area of a rhombus is one-half the product of the lengths of the diagonals. The area of $A B C D$ is therefore $\frac{1}{2} \cdot 24 \cdot 10=120$ square meters.
}\SetValue{Rubric}{%Markdown



}\SetValue{Hint}{%
Solution Goes Here
}\SetValue{Answer}{%
Answer (D)
}
\ProcessDATA


