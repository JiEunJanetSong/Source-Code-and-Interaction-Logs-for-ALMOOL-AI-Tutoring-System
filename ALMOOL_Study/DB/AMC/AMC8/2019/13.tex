\SetValue{Module}{1}\SetValue{SectionAB}{A}\SetValue{MainChapter}{}\SetValue{SubChapter}{}\SetValue{Contents}{%%

A palindrome is a number that has the same value when read from left to right or from right to left. (For example, 12321 is a palindrome.) Let $N$ be the least three-digit integer which is not a palindrome but which is the sum of three distinct two-digit palindromes. What is the sum of the digits of $N$ ?

\ansFIVEs{%start
2}{%<--- (A)
 3}{%<--- (B)
 4}{%<--- (C)
 5}{%<---- (D)
6}%<---- (D)

}\SetValue{Concept}{%



}\SetValue{AltText}{%



}\SetValue{Solution}{%

The two-digit palindromes are $11,22,33, \ldots, 99$, all of which are multiples of 11. A sum of three of these palindromes will also be a multiple of 11 . The smallest three-digit multiple of 11 is 110 , and 110 can be expressed as the sum of three two-digit palindromes. For example, $110=22+33+55$. The sum of the digits of 110 is $1+1+0=2$.
Note: Javier Cilleruelo, Florian Luca, and Lewis Baxter recently proved that any positive integer can be written as the sum of three palindromes. 
}\SetValue{Rubric}{%Markdown



}\SetValue{Hint}{%
Solution Goes Here
}\SetValue{Answer}{%
Answer (A)
}
\ProcessDATA



