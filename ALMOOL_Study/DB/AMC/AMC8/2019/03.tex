\SetValue{Module}{1}\SetValue{SectionAB}{A}\SetValue{MainChapter}{}\SetValue{SubChapter}{}\SetValue{Contents}{%%
    
Which of the following is the correct order of the fractions $\frac{15}{11}$, $\frac{19}{15}$, and $\frac{17}{13}$, from least to greatest?

\ansFIVEs{%start
	\frac{15}{11}< \frac{17}{13}< \frac{19}{15}  }{%<--- (A)
	\frac{15}{11}< \frac{19}{15}<\frac{17}{13}   }{%<--- (B)
	\frac{17}{13}<\frac{19}{15}<\frac{15}{11}  }{%<--- (C)
	\frac{19}{15}<\frac{15}{11}<\frac{17}{13}  }{%<---- (D)
\frac{19}{15}<\frac{17}{13}<\frac{15}{11}}%<---- (E)
}\SetValue{Concept}{%



}\SetValue{AltText}{%



}\SetValue{Solution}{%

To determine the order of $\frac{19}{15}$ and $\frac{17}{13}$, rewrite the fractions using a common denominator: $\frac{19 \cdot 13}{15 \cdot 13}$ and $\frac{17 \cdot 15}{13 \cdot 15}$. Because

$$
19 \cdot 13=(16+3)(16-3)=16^2-3^2
$$

$$
17 \cdot 15=(16+1)(16-1)=16^2-1^2
$$

and $16^2-3^2<16^2-1^2$, it follows that $\frac{19}{15}<\frac{17}{13}$.
Similarly, to determine the order of $\frac{17}{13}$ and $\frac{15}{11}$, rewrite the fractions using a common denominator: $\frac{17 \cdot 11}{13 \cdot 11}$ and $\frac{15 \cdot 13}{11 \cdot 13}$. Because

$$
\begin{aligned}
& 17 \cdot 11=(14+3)(14-3)=14^2-3^2 \\
& 15 \cdot 13=(14+1)(14-1)=14^2-1^2
\end{aligned}
$$

and $14^2-3^2<14^2-1^2$, it follows that $\frac{17}{13}<\frac{15}{11}$.
OR

Subtracting 1 from each fraction results in the fractions $\frac{4}{11}, \frac{4}{15}$, and $\frac{4}{13}$. Because $\frac{4}{15}<\frac{4}{13}<\frac{4}{11}$, it follows that $\frac{19}{15}<\frac{17}{13}<\frac{15}{11}$.

OR

Given a fraction $\frac{a}{b}$ where $0<b<a$, if $n$ is a positive integer, then $b n<a n$ and so $b(a+n)=$ $a b+b n<a b+a n=a(b+n)$. Thus $\frac{a+n}{b+n}<\frac{a}{b}$. Therefore $\frac{19}{15}<\frac{17}{13}<\frac{15}{11}$.

}\SetValue{Rubric}{%Markdown



}\SetValue{Hint}{%
Solution Goes Here
}\SetValue{Answer}{%
Answer (E)
}
\ProcessDATA




   
