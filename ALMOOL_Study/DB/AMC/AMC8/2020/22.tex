\SetValue{Module}{1}\SetValue{SectionAB}{A}\SetValue{MainChapter}{}\SetValue{SubChapter}{}\SetValue{Contents}{%%
    
When a positive integer $N$ is fed into a machine, the output is a number calculated according to the rule shown below.

\begin{center}		
    \includegraphics[scale=0.6]{AMC-8-pics/2020-22.png}
	\end{center}
	
	For example, starting with an input of $N=7$, the machine will output $3 \cdot 7+1=22$. Then if the output is repeatedly inserted into the machine five more times, the final output is 26.

$$
7 \rightarrow 22 \rightarrow 11 \rightarrow 34 \rightarrow 17 \rightarrow 52 \rightarrow 26
$$


When the same 6 -step process is applied to a different starting value of $N$, the final output is 1. What is the sum of all such integers $N$ ?

$$
N \rightarrow \underline{\quad} \rightarrow \underline{\quad}\rightarrow \underline{\quad} \rightarrow \underline{\quad} \rightarrow \underline{\quad}\rightarrow 1
$$

\ansFIVEs{%start
		 73 }{%<--- (A)
		 74 }{%<--- (B)
		 75 }{%<--- (C)
		 82 }{%<---- (D)
		 83 }%<---- (D)

}\SetValue{Concept}{%



}\SetValue{AltText}{%



}\SetValue{Solution}{%

Answer (E): Working backwards from the number 1, as shown below, there are 4 possible starting values of $N: 1,8,10$, and 64. The sum of these values is $1+8+10+64=83$.

\begin{center}		
    \includegraphics[scale=0.6]{AMC-8-pics/2020-22-s.png}
	\end{center}

	Note: The Collatz Conjecture states that the number 1 will appear eventually no matter which positive integer $N$ is chosen as the starting value. For example if the $N=7$ sequence is extended past 26, the outputs will be

	$$
	26 \rightarrow 13 \rightarrow 40 \rightarrow 20 \rightarrow 10 \rightarrow 5 \rightarrow 16 \rightarrow 8 \rightarrow 4 \rightarrow 2 \rightarrow 1
	$$
	
	
	It is not known whether this conjecture is true.
	
	
}\SetValue{Rubric}{%Markdown



}\SetValue{Hint}{%
Solution Goes Here
}\SetValue{Answer}{%

}
\ProcessDATA



