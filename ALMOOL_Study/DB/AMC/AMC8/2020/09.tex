\SetValue{Module}{1}\SetValue{SectionAB}{A}\SetValue{MainChapter}{}\SetValue{SubChapter}{}\SetValue{Contents}{%%
   
Akash's birthday cake is in the form of a $4 \times 4 \times 4$ inch cube. The cake has icing on the top and the four side faces, and no icing on the bottom. Suppose the cake is cut into 64 smaller cubes, each measuring $1 \times 1 \times 1$ inch, as shown below. How many small pieces will have icing on exactly two sides?

\begin{center}		
    \includegraphics[scale=0.6]{AMC-8-pics/2020-09.png}
	\end{center}
	
	\ansFIVEs{%start
	12}{%<--- (A)
	16}{%<--- (B)
	18}{%<--- (C)
	20}{%<---- (D)
	24}%<---- (D)

}\SetValue{Concept}{%



}\SetValue{AltText}{%



}\SetValue{Solution}{%

Answer (D): 

\begin{center}		
    \includegraphics[scale=0.6]{AMC-8-pics/2020-09-s.png}
	\end{center}

	The cuts divide the cake into four $4 \times 4 \times 1$ inch horizontal layers. In the top layer there are 8 pieces having exactly two sides with icing, namely the non-corner edge pieces that are part of both the top face and a side face. In each of the next 3 layers, the 4 corner pieces are the only ones having two sides with icing. This gives a total of $8+3 \cdot 4=20$ small pieces with icing on exactly two sides.

	OR
	
	First consider the 8 edges of the cake that are not along the bottom face. Each of those edges has 2 middle pieces with icing on exactly two sides. In addition to these pieces, the 4 corner pieces of the bottommost layer also have two sides with icing. This gives a total of $8 \cdot 2+4=20$ small pieces with icing on exactly two sides.	
}\SetValue{Rubric}{%Markdown



}\SetValue{Hint}{%
Solution Goes Here
}\SetValue{Answer}{%

}
\ProcessDATA



