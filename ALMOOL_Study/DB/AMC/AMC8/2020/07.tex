\SetValue{Module}{1}\SetValue{SectionAB}{A}\SetValue{MainChapter}{}\SetValue{SubChapter}{}\SetValue{Contents}{%%
    
How many integers between 2020 and 2400 have four distinct digits arranged in increasing order? (For example, 2347 is one integer.)

\ansFIVEs{%start
		 9 }{%<--- (A)
		 10 }{%<--- (B)
		 15 }{%<--- (C)
		 21 }{%<---- (D)
		 28 }%<---- (D)

}\SetValue{Concept}{%



}\SetValue{AltText}{%



}\SetValue{Solution}{%

The digits are in increasing order so the first digit must be 2, the second digit must be 3, and the third digit must be $4,5,6,7$, or 8. The possible integers are listed below.

\begin{center}		
	\includegraphics[scale=0.6]{AMC-8-pics/2020-07-s.png}
	\end{center}

	The total is $5+4+3+2+1=15$ integers.

	OR

	The first digit must be 2, the second digit must be 3, and the remaining two digits must be chosen from among the six digits $4,5,6,7,8$, and 9. There are ${ }_6 C_2=\binom{6}{2}=\frac{6!}{4!2!}=\frac{6 \cdot 5}{2}=15$ pairs of those digits and only one possible order for each pair, so there are 15 such integers.	
}\SetValue{Rubric}{%Markdown



}\SetValue{Hint}{%
Solution Goes Here
}\SetValue{Answer}{%
Answer (C): 
}
\ProcessDATA



