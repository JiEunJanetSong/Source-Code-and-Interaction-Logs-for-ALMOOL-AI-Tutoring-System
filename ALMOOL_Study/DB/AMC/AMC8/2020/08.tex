\SetValue{Module}{1}\SetValue{SectionAB}{A}\SetValue{MainChapter}{}\SetValue{SubChapter}{}\SetValue{Contents}{%%
    
Ricardo has 2020 coins, some of which are pennies (1-cent coins) and the rest of which are nickels ( 5 -cent coins). He has at least one penny and at least one nickel. What is the difference in cents between the greatest possible and least possible amounts of money that Ricardo can have?

\ansFIVEs{%start
		 8062 }{%<--- (A)
		 8068 }{%<--- (B)
		 8072 }{%<--- (C)
		 8076 }{%<---- (D)
		 8082 }%<---- (D)

}\SetValue{Concept}{%



}\SetValue{AltText}{%



}\SetValue{Solution}{%

Answer (C): To obtain the greatest possible amount of money, use as many nickels as possible: 2019 nickels and 1 penny. This gives $2019 \cdot 5+1 \cdot 1=10,096$ cents. Similarly, to obtain the least possible amount of money, use as many pennies as possible: 1 nickel and 2019 pennies. This gives $1 \cdot 5+2019 \cdot 1=2024$ cents. Therefore the difference between the greatest possible and least possible amounts of money is $10,096-2024=8072$ cents.

OR

One penny and one nickel will be used in either case so they can be removed, leaving 2018 coins. The greatest possible sum will result from having 2018 nickels, and the least possible sum will result from having 2018 pennies. The difference in amounts is $2018 \cdot 5-2018 \cdot 1=2018 \cdot 4=8072$ cents.
}\SetValue{Rubric}{%Markdown



}\SetValue{Hint}{%
Solution Goes Here
}\SetValue{Answer}{%

}
\ProcessDATA



