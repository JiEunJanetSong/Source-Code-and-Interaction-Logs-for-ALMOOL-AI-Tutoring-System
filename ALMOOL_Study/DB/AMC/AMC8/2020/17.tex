\SetValue{Module}{1}\SetValue{SectionAB}{A}\SetValue{MainChapter}{}\SetValue{SubChapter}{}\SetValue{Contents}{%%
  
How many factors of 2020 have more than 3 factors? (As an example, 12 has 6 factors, namely $1,2,3,4,6$, and 12 .)

\ansFIVEs{%start
		6 }{%<--- (A)
		7 }{%<--- (B)
		8 }{%<--- (C)
		9 }{%<---- (D)
		10 }%<---- (D)

}\SetValue{Concept}{%



}\SetValue{AltText}{%



}\SetValue{Solution}{%

Answer (B): The number 2020 has 12 factors.

$$
\begin{aligned}
2020 & =1 \cdot 2020 \\
& =2 \cdot 1010 \\
& =4 \cdot 505 \\
& =5 \cdot 404 \\
& =10 \cdot 202 \\
& =20 \cdot 101
\end{aligned}
$$


These factors may be classified as follows:

- The number 1 has exactly 1 factor.

- The numbers 2,5, and 101 are primes, so each has exactly 2 factors.

- The number 4 has exactly 3 factors, namely 1, 2, and 4.

- The remaining 7 numbers, $10,20,202,404,505,1010$, and 2020, each have more than 3 factors.

Thus 7 of the factors of 2020 have more than 3 factors.

OR

In order to determine the number of factors of 2020, note that the prime factorization of 2020 is

$$
2020=2^2 \cdot 5^1 \cdot 101^1
$$


This means that each factor of 2020 has the form $2^a \cdot 5^b \cdot 101^c$, where $a=0,1$, or $2 ; b=0$ or 1 ; and $c=0$ or 1. Thus the number of factors of 2020 is $3 \cdot 2 \cdot 2=12$.

By similar reasoning, any positive integer with at least 2 distinct prime factors, say $p$ and $q$, has at least 4 factors, namely $1, p, q$, and $p q$.
The factors of 2020 that have no more than 1 distinct prime factor are 1, 2, 4, 5, and 101. Each of these has at most 3 factors. The remaining 7 factors, 10, 20, 202, 404, 505, 1010, and 2020, each have at least 2 distinct prime factors. Therefore 7 of the factors of 2020 have more than 3 factors.
}\SetValue{Rubric}{%Markdown



}\SetValue{Hint}{%
Solution Goes Here
}\SetValue{Answer}{%

}
\ProcessDATA



