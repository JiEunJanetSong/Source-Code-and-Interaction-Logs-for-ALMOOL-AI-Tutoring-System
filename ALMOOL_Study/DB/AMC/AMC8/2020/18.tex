\SetValue{Module}{1}\SetValue{SectionAB}{A}\SetValue{MainChapter}{}\SetValue{SubChapter}{}\SetValue{Contents}{%%
  
Rectangle $A B C D$ is inscribed in a semicircle with diameter $\overline{F E}$, as shown in the figure. Let $D A=16$, and let $F D=A E=9$. What is the area of $A B C D ?$

\begin{center}		
    \includegraphics[scale=0.6]{AMC-8-pics/2020-18.png}
	\end{center}
	
\ansFIVEs{%start
		240 }{%<--- (A)
		248 }{%<--- (B)
		256 }{%<--- (C)
		264 }{%<---- (D)
		272 }%<---- (D)

}\SetValue{Concept}{%







}\SetValue{AltText}{%



}\SetValue{Solution}{%

Answer (A):

\begin{center}		
    \includegraphics[scale=0.6]{AMC-8-pics/2020-18-s.png}
	\end{center}

	Let $O$ be the center of the circle. Then $D O=O A=8$ and radius $O E=O A+A E=8+9=17$. Draw radius $\overline{O B}$ which also has length 17. Then $\triangle O A B$ is a right triangle with hypotenuse 17 and base leg 8 . By the Pythagorean Theorem $A B=\sqrt{17^2-8^2}=\sqrt{289-64}=\sqrt{225}=15$. The area of rectangle $A B C D$ is $D A \cdot A B=16 \cdot 15=240$.	
}\SetValue{Rubric}{%Markdown



}\SetValue{Hint}{%
Solution Goes Here
}\SetValue{Answer}{%

}
\ProcessDATA



