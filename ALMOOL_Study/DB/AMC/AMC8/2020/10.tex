\SetValue{Module}{1}\SetValue{SectionAB}{A}\SetValue{MainChapter}{}\SetValue{SubChapter}{}\SetValue{Contents}{%%
  
Zara has a collection of 4 marbles: an Aggie, a Bumblebee, a Steelie, and a Tiger. She wants to display them in a row on a shelf, but does not want to put the Steelie and the Tiger next to one another. In how many ways can she do this?

\ansFIVEs{%start
		 6 }{%<--- (A)
		 8 }{%<--- (B)
		 12 }{%<--- (C)
		 18 }{%<---- (D)
		 24 }%<---- (D)

}\SetValue{Concept}{%



}\SetValue{AltText}{%



}\SetValue{Solution}{%

Answer (C): There are 6 ways Zara can place the Steelie and the Tiger so that they are not next to one another. Those arrangements are shown below.

\begin{center}		
    \includegraphics[scale=0.6]{AMC-8-pics/2020-10-s.png}
	\end{center}

	For each arrangement, there are 2 ways to place the remaining marbles: either the Aggie appears to the left of the Bumblebee or it appears to the right. This makes $6 \cdot 2=12$ arrangements with the Steelie and the Tiger not next to one another.

	OR
	
	The number of arrangements can be determined by using complementary counting. That is, begin with the total number of possible arrangements, which is $4!=4 \cdot 3 \cdot 2 \cdot 1=24$, then subtract the number of ways to place the Steelie and the Tiger next to each other. If the Steelie and the Tiger are considered as a single object, then there are 3 objects to arrange and $3!=3 \cdot 2 \cdot 1=6$ ways to do so. There are 2 ways to arrange the Steelie and the Tiger, so there are $6 \cdot 2=12$ arrangements with the Steelie and the Tiger adjacent to each other. Thus there are $24-12=12$ arrangements with the Steelie and the Tiger not next to one another.

}\SetValue{Rubric}{%Markdown



}\SetValue{Hint}{%
Solution Goes Here
}\SetValue{Answer}{%

}
\ProcessDATA



