\SetValue{Module}{1}\SetValue{SectionAB}{A}\SetValue{MainChapter}{}\SetValue{SubChapter}{}\SetValue{Contents}{%%
   
A number is called flippy if its digits alternate between two distinct digits. For example, 2020 and 37373 are flippy, but 3883 and 123123 are not. How many five-digit flippy numbers are divisible by 15?

\ansFIVEs{%start
		3 }{%<--- (A)
		4 }{%<--- (B)
		5 }{%<--- (C)
		6 }{%<---- (D)
		8 }%<---- (D)

}\SetValue{Concept}{%



}\SetValue{AltText}{%



}\SetValue{Solution}{%

Answer (B): A five-digit flippy number looks like $\underline{A} \underline{B} \underline{A} \underline{B} \underline{A}$ for some distinct digits $A$ and $B$ with $A \neq 0$ (otherwise it would not be a five-digit number). In order for this number to be divisible by 15, it must be divisible by 5 and 3.
Any number divisible by 5 will have a units digit of 0 or 5. Because $A \neq 0, A$ must equal 5. This means the flippy number must have the form $\underline{5} \underline{B} \underline{5} \underline{B} \underline{5}$.

For a number to be divisible by 3, the sum of its digits must be divisible by 3. The sum of the digits of $\underline{5} \underline{B} \underline{5} \underline{B} \underline{5}$ is $15+2 B$. This means $2 B$ must be divisible by 3, so $B$ is divisible by 3. There are 4 possible digits for $B: 0,3,6$, or 9. Therefore there are 4 five-digit flippy numbers divisible by 15 : $50505,53535,56565$, and 59595.
}\SetValue{Rubric}{%Markdown



}\SetValue{Hint}{%
Solution Goes Here
}\SetValue{Answer}{%

}
\ProcessDATA



