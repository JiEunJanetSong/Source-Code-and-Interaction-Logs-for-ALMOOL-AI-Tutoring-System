\SetValue{Module}{1}\SetValue{SectionAB}{A}\SetValue{MainChapter}{}\SetValue{SubChapter}{}\SetValue{Contents}{%%
    
Five distinct integers from 1 to 10 are chosen, and five distinct integers from 11 to 20 are chosen. No two numbers differ by exactly 10. What is the sum of the ten chosen numbers?

\ansFIVEs{%start
		95 }{%<--- (A)
		100 }{%<--- (B)
		105 }{%<--- (C)
		110 }{%<---- (D)
		115 }%<---- (D)
}\SetValue{Concept}{%



}\SetValue{AltText}{%



}\SetValue{Solution}{%

Solution
Note that for no two numbers to differ by 10 , every number chosen must have a different units digit. To make computations easier, we can choose ( $1,2,3,4,5$ ) from the first group and $(16,17,18,19,20)$ from the second group. Then the sum evaluates to

$$
1+2+3+4+5+16+17+18+19+20=(\mathrm{C}) 105 .
$$

Another Way To Compute
For $1+2+3+4+5+16+17+18+19+20$, we can add the first term and the last term, which is 21 . If we add the second term and the second-to-last term it is also 21 . There are 5 pairs that sum to 21 , so the answer is $21 \times 5$ which equals $(\mathrm{C}) 105$.


Similar solution
One efficient method is to quickly add ( $1,2,3,4,5,6,7,8,9,10$ ), which is 55 . Then because you took 50 in total away from (16, 17, 18, 19, 20), you add 50 .

$$
55+50=\text { (C) } 105 .
$$



Solution 4 (effecient)
To solve this problem, I started with the easiest/smallest case possible. In my opinion, that was

$$
1+2+3+4+5+16+17+18+19+20
$$


I then solved this equation quickly using Little Gauss's method, rearranging that into

$$
1+19+2+18+3+17+4+16+5+20
$$


I simplified this into

$$
20+20+20+20+25
$$

}\SetValue{Rubric}{%Markdown



}\SetValue{Hint}{%
Solution Goes Here
}\SetValue{Answer}{%

}
\ProcessDATA



