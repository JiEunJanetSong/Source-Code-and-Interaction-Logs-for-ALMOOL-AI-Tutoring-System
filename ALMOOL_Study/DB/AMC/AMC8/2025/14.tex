\SetValue{Module}{1}\SetValue{SectionAB}{A}\SetValue{MainChapter}{}\SetValue{SubChapter}{}\SetValue{Contents}{%%
    
A number $N$ is inserted into the list $2,6,7,7,28$. The mean is now twice as great as the median. What is $N$?

\ansFIVEs{%start
		7 }{%<--- (A)
		14 }{%<--- (B)
		20 }{%<--- (C)
		28 }{%<---- (D)
		34 }%<---- (D)
}\SetValue{Concept}{%



}\SetValue{AltText}{%



}\SetValue{Solution}{%

Solution 1
The median of the list is 7 , so the mean of the new list will be $7 \cdot 2=14$. Since there are 6 numbers in the new list, the sum of the 6 numbers will be $14 \cdot 6=84$. Therefore,

$$
2+6+7+7+28+N=84 \Longrightarrow N=\boxed{(\mathrm{E}) 34}
$$


Solution 2
Since the average right now is 10 , and the median is 7 , we see that N must be larger than 10 , which means that the median of the 6 resulting numbers should be 7 , making the mean of these 14. We can do $2+6+7+7+28+N=14$ * $6=84$. $50+N=84$, so $N=$ $\square$
(E) 34

Solution 3
We try out every option by inserting each number into the list. After trying out each number, we get (E) 34

Note that this is very time-consuming and it is not the most practical solution.

Solution 4
We could use answer choices to solve this problem. The sum of the 5 numbers is 50 . If you add 7 to the list, 57 is not divisible by 6 , therefore it will not work. Same thing applies to 14 and 20 . The only possible choices left are 28 and 34 . Now you check 28 . You see that 28 doesn't work because $(28+50) \div 6=13$ and 13 is not twice of the median, which is still 7. Therefore, only choice left is $\square$ (E) 34
}\SetValue{Rubric}{%Markdown



}\SetValue{Hint}{%
Solution Goes Here
}\SetValue{Answer}{%

}
\ProcessDATA



