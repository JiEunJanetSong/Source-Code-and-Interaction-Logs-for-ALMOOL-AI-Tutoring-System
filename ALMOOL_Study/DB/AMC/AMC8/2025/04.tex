\SetValue{Module}{1}\SetValue{SectionAB}{A}\SetValue{MainChapter}{}\SetValue{SubChapter}{}\SetValue{Contents}{%%
    
Lucius is counting backward by 7 s. His first three numbers are 100, 93, and 86. What is his 10 th number?

\ansFIVEs{%start
		30 }{%<--- (A)
		37 }{%<--- (B)
		42 }{%<--- (C)
		44 }{%<---- (D)
		47 }%<---- (D)
}\SetValue{Concept}{%



}\SetValue{AltText}{%



}\SetValue{Solution}{%

Solution 1
We plug $a=100, d=-7$ and $n=10$ into the formula $a+d(n-1)$ for the $n$th term of an arithmetic sequence whose first term is $a$ and common difference is $d$ to get

$$
100-7(10-1)=\boxed{(\mathrm{B}) 37 .}
$$


Solution 2
Since we want to find the 9 th number Lucius says after he says 100,7 is subtracted from his number 9 times, so our answer is $100-(9 \cdot 7)=(\mathrm{B}) 37$

Solution 3
Using brute force and counting backward by 7 s , we have

$$
100,93,86,79,72,65,58,51,44,(\mathrm{~B}) 37 \text {. }
$$

}\SetValue{Rubric}{%Markdown



}\SetValue{Hint}{%
Solution Goes Here
}\SetValue{Answer}{%

}
\ProcessDATA



