\SetValue{Module}{1}\SetValue{SectionAB}{A}\SetValue{MainChapter}{}\SetValue{SubChapter}{}\SetValue{Contents}{%%
    
The table below shows the ancient Egyptian hieroglyphs that were used to represent different numbers.

\begin{center}		
    \includegraphics[scale=0.6]{AMC-8-pics/2025-02-1.png}
	\end{center}

	For example, the number 32 was represented by the hieroglyphs $\cap \cap \cap \|$. What number is represented by the following combination of hieroglyphs?

	\begin{center}		
		\includegraphics[scale=0.6]{AMC-8-pics/2025-02-2.png}
		\end{center}	
		
\ansFIVEs{%start
		1,423 }{%<--- (A)
		10,423 }{%<--- (B)
		14,023 }{%<--- (C)
		14,203 }{%<---- (D)
		14,230 }%<---- (D)
}\SetValue{Concept}{%



}\SetValue{AltText}{%



}\SetValue{Solution}{%

Solution 1
The first hieroglyph is worth 10,000 , the next 4 are worth $100 \cdot 4=400$, the next 2 are worth $10 \cdot 2=20$, and the last 3 are worth $1 \cdot 3=3$. Therefore, the answer is $10,000+400+20+3=$ (B) 10,423

Solution 2
Simply notice that the first hieroglyph represents 10,000 , and the next ones represent 4 hundreds. They only answer choice with a 1 in the thousands place and a 4 in the hundreds place is answer choice (B) 10,423
}\SetValue{Rubric}{%Markdown



}\SetValue{Hint}{%
Solution Goes Here
}\SetValue{Answer}{%

}
\ProcessDATA



