\SetValue{Module}{1}\SetValue{SectionAB}{A}\SetValue{MainChapter}{}\SetValue{SubChapter}{}\SetValue{Contents}{%%
  
A tetromino consists of four squares connected along their edges. There are five possible tetromino shapes, $I, O, L, T$, and $S$, shown below, which can be rotated or flipped over. Three tetrominoes are used to completely cover a $3 \times 4$ rectangle. At least one of the tiles is an $S$ tile. What are the other two tiles?

\begin{center}		
    \includegraphics[scale=0.6]{AMC-8-pics/2025-11.png}
	\end{center}

\ansFIVEs{%start
		I and L }{%<--- (A)
		I and T }{%<--- (B)
		L and L }{%<--- (C)
		L and S }{%<---- (D)
		O and T }%<---- (D)
}\SetValue{Concept}{%



}\SetValue{AltText}{%



}\SetValue{Solution}{%

Solution 1
The $3 \times 4$ rectangle allows for 7 possible places to put the S piece, with each possible placement having an inverted version. One of the cases looks like this:

\begin{center}		
    \includegraphics[scale=0.6]{AMC-8-pics/2025-11-s-1.png}
	\end{center}

As you can see, there is a hole in the top left corner of the board, which would be impossible to fill using the tetrominos. There are three cases in which a hole isn't created; the $S$ lies flat in the bottom left corner, it lies flat in the top right corner, or it stands upright in the center. All three tilings are shown below.

\begin{center}		
    \includegraphics[scale=0.6]{AMC-8-pics/2025-11-s-2.png}
	\end{center}

	For each of the inverted cases, the $L$ pieces can be inverted along with the $S$ piece. Because the only cases that fill the rectangle after the $S$ is placed are the ones that use two $L$ pieces, the answer must be (C) $L$ and $L$. 
}\SetValue{Rubric}{%Markdown



}\SetValue{Hint}{%
Solution Goes Here
}\SetValue{Answer}{%

}
\ProcessDATA



