\SetValue{Module}{1}\SetValue{SectionAB}{A}\SetValue{MainChapter}{}\SetValue{SubChapter}{}\SetValue{Contents}{%%
    
On the most recent exam on Prof. Xochi's class,

5 students earned a score of at least $95 \%$,
13 students earned a score of at least $90 \%$,
27 students earned a score of at least $85 \%$,
50 students earned a score of at least $80 \%$,

How many students earned a score of at least $80 \%$ and less than $90 \%$?

\ansFIVEs{%start
		8 }{%<--- (A)
		14 }{%<--- (B)
		22 }{%<--- (C)
		37 }{%<---- (D)
		45 }%<---- (D)
}\SetValue{Concept}{%



}\SetValue{AltText}{%



}\SetValue{Solution}{%

Solution 1
50 people scored at least $80 \%$, and out of these 50 people, 13 of them earned a score that was not less than $90 \%$, so the number of people that scored in between at least $80 \%$ and less than $90 \%$ is $50-13=$ (D) 37 .

Solution 2
Let $a$ denote the number of people who had a score of at least 85 , but less than 90 , and let $b$ denote the number of people who had a score of at least 80 but less than 85 . Our answer is equal to $a+b$. We find $a=27-13=14$, while $b=50-27=23$. Thus, the answer is $23+14=$ (D) 37 .
}\SetValue{Rubric}{%Markdown



}\SetValue{Hint}{%
Solution Goes Here
}\SetValue{Answer}{%

}
\ProcessDATA



