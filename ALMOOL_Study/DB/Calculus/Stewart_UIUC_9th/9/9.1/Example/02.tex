\SetValue{Module}{1}\SetValue{SectionAB}{A}\SetValue{MainChapter}{}\SetValue{SubChapter}{}\SetValue{Contents}{%%
    
Show that every member of the family of functions

$$
y=\frac{1+c e^t}{1-c e^t}
$$

is a solution of the differential equation $y^{\prime}=\frac{1}{2}\left(y^2-1\right)$.

}\SetValue{Concept}{%



}\SetValue{AltText}{%



}\SetValue{Solution}{%

We use the Quotient Rule to differentiate the expression for $y$ :

$$
\begin{aligned}
y^{\prime} & =\frac{\left(1-c e^t\right)\left(c e^t\right)-\left(1+c e^t\right)\left(-c e^t\right)}{\left(1-c e^t\right)^2} \\
& =\frac{c e^t-c^2 e^{2 t}+c e^t+c^2 e^{2 t}}{\left(1-c e^t\right)^2}=\frac{2 c e^t}{\left(1-c e^t\right)^2}
\end{aligned}
$$


The right side of the differential equation becomes

$$
\begin{aligned}
\frac{1}{2}\left(y^2-1\right) & =\frac{1}{2}\left[\left(\frac{1+c e^t}{1-c e^t}\right)^2-1\right] \\
& =\frac{1}{2}\left[\frac{\left(1+c e^t\right)^2-\left(1-c e^t\right)^2}{\left(1-c e^t\right)^2}\right] \\
& =\frac{1}{2} \frac{4 c e^t}{\left(1-c e^t\right)^2}=\frac{2 c e^t}{\left(1-c e^t\right)^2}
\end{aligned}
$$


This shows that the left and right sides of the differential equation are equal. Therefore, for every value of $c$, the given function is a solution of the differential equation.


}\SetValue{Rubric}{%Markdown



}\SetValue{Hint}{%
Solution Goes Here
}\SetValue{Answer}{%

}
\ProcessDATA



