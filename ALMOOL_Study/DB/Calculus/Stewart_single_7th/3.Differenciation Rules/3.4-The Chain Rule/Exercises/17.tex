\SetValue{Module}{1}\SetValue{SectionAB}{A}\SetValue{MainChapter}{}\SetValue{SubChapter}{}\SetValue{Contents}{%%
    
Find the derivative of the function.

$f(x)=(2 x-3)^4\left(x^2+x+1\right)^5$

}\SetValue{Concept}{%



}\SetValue{AltText}{%



}\SetValue{Solution}{%

We can differentiate $f(x)$ by viewing it as a product of two functions and then applying the product rule:

$$
f(x) = \underbrace{(2x - 3)^4}_{u(x)} \cdot \underbrace{\bigl(x^2 + x + 1\bigr)^5}_{v(x)}.
$$

The product rule states

$$
\dfrac{d}{dx}\bigl[u(x)\,v(x)\bigr] = u'(x)\,v(x) + u(x)\,v'(x).
$$

\textbf{Step 1.} Find $u'(x)$

$$
u(x) = (2x - 3)^4.
$$

Use the chain rule:

1. Inner function: $2x - 3$, whose derivative is $2$.

2. Outer function: $(\cdot)^4$, whose derivative is $4(\cdot)^3$.

Hence,

$$
u'(x) = 4(2x - 3)^3 \cdot 2 = 8(2x - 3)^3.
$$

\textbf{Step 2.} Find $v'(x)$

$$
v(x) = \bigl(x^2 + x + 1\bigr)^5.
$$

Again use the chain rule:

1. Inner function: $x^2 + x + 1$, whose derivative is $2x + 1$.

2. Outer function: $(\cdot)^5$, whose derivative is $5(\cdot)^4$.

Hence,

$$
v'(x) 
= 5(x^2 + x + 1)^4 \cdot (2x + 1).
$$

\textbf{Step 3.} Combine via the product rule

$$
f'(x) 
= u'(x)\,v(x) + u(x)\,v'(x)
$$

$$
= \Bigl[\,8(2x - 3)^3\Bigr]\,\Bigl[(x^2 + x + 1)^5\Bigr] 
\;+\; \Bigl[(2x - 3)^4\Bigr]\,\Bigl[5(x^2 + x + 1)^4 (2x + 1)\Bigr].
$$

You can leave the derivative in this expanded sum form or factor out common terms if you wish. A perfectly acceptable final form is:

$$
\boxed{
f'(x) 
= 8(2x - 3)^3\,(x^2 + x + 1)^5 \;+\; 5(2x - 3)^4\,(x^2 + x + 1)^4 \,(2x + 1).
}
$$

}\SetValue{Rubric}{%Markdown



}\SetValue{Hint}{%
Solution Goes Here
}\SetValue{Answer}{%

}
\ProcessDATA



