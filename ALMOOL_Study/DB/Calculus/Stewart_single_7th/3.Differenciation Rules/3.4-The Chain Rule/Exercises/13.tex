\SetValue{Module}{1}\SetValue{SectionAB}{A}\SetValue{MainChapter}{}\SetValue{SubChapter}{}\SetValue{Contents}{%%
    
Find the derivative of the function.

$y=\cos \left(a^3+x^3\right)$

}\SetValue{Concept}{%



}\SetValue{AltText}{%



}\SetValue{Solution}{%

To find the derivative of 

$$
y = \cos\bigl(a^3 + x^3\bigr),
$$

treat $a$ as a constant. Using the chain rule:

1. Let $u = a^3 + x^3$. Then $y = \cos(u)$.

2. $\dfrac{du}{dx} = 3x^2$ (since $a^3$ is constant and its derivative is 0).

3. $\dfrac{dy}{dx} = -\sin(u) \cdot \dfrac{du}{dx} = -\sin(a^3 + x^3)\, (3x^2)$.

Hence,

$$
\boxed{\dfrac{dy}{dx} = -3x^2 \sin\bigl(a^3 + x^3\bigr).}
$$

}\SetValue{Rubric}{%Markdown



}\SetValue{Hint}{%
Solution Goes Here
}\SetValue{Answer}{%

}
\ProcessDATA



