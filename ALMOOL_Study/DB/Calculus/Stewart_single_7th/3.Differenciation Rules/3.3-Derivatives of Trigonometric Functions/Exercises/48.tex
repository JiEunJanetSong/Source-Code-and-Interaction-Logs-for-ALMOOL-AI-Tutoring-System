\SetValue{Module}{1}\SetValue{SectionAB}{A}\SetValue{MainChapter}{}\SetValue{SubChapter}{}\SetValue{Contents}{%%
    
Find the limit.

$\lim _{x \rightarrow 1} \dfrac{\sin (x-1)}{x^2+x-2}$

}\SetValue{Concept}{%



}\SetValue{AltText}{%



}\SetValue{Solution}{%

We want to evaluate
$
\lim_{x \to 1} \dfrac{\sin(x-1)}{x^2 + x - 2}.
$

1. Check for indeterminate form:  

   As $x \to 1$, the numerator $\sin(x-1) \to \sin(0) = 0$. 
   
   The denominator $x^2 + x - 2$ at $x=1$ is $1 + 1 - 2 = 0$. 
   
   So we have a $0/0$ indeterminate form.

2. Factor the denominator:  
   $
   x^2 + x - 2 = (x - 1)(x + 2).
   $

   Thus,
   $
   \dfrac{\sin(x-1)}{x^2 + x - 2} 
   = \dfrac{\sin(x-1)}{(x - 1)(x + 2)}.
   $

3. Rewrite and use the standard limit $\sin u / u \to 1$ as $u \to 0$:  

   Let $u = x - 1$. Then as $x \to 1$, $u \to 0$. 
   
   We can rewrite
   $
   \dfrac{\sin(x-1)}{(x-1)(x + 2)}
   = \dfrac{\sin(x-1)}{x-1} \cdot \dfrac{1}{x + 2}.
   $

   Now, $\lim_{x \to 1} \dfrac{\sin(x-1)}{x-1} = 1$, and $\lim_{x \to 1} \dfrac{1}{x+2} = \dfrac{1}{3}$. 
   
   Therefore,

   $
   \lim_{x \to 1} \dfrac{\sin(x-1)}{x^2 + x - 2}
   = \lim_{x \to 1} \left(\dfrac{\sin(x-1)}{x-1}\right) \cdot \lim_{x \to 1}\left(\dfrac{1}{x + 2}\right)
   = 1 \cdot \dfrac{1}{3}
   = \dfrac{1}{3}.
   $

Hence,

$
\boxed{\dfrac{1}{3}}.
$

}\SetValue{Rubric}{%Markdown



}\SetValue{Hint}{%
Solution Goes Here
}\SetValue{Answer}{%

}
\ProcessDATA



