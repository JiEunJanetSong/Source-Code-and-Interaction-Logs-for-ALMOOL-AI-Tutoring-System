\SetValue{Module}{1}\SetValue{SectionAB}{A}\SetValue{MainChapter}{}\SetValue{SubChapter}{}\SetValue{Contents}{%%
    
Differentiate.

$y=\dfrac{1-\sec x}{\tan x}$

}\SetValue{Concept}{%



}\SetValue{AltText}{%



}\SetValue{Solution}{%

Using the Quotient Rule:

Given
$
y = \dfrac{1 - \sec x}{\tan x},
$
let 
$
u = 1 - \sec x, 
\quad 
v = \tan x.
$

Then
$
u' = \dfrac{d}{dx}(1 - \sec x) = -\sec x \tan x,
\quad
v' = \dfrac{d}{dx}(\tan x) = \sec^2 x.
$

By the quotient rule,
$
\biggl(\dfrac{u}{v}\biggr)' 
= \dfrac{u'v \;-\; u\,v'}{v^2},
$

we get

$
y' 
= \dfrac{(-\sec x \tan x)\,(\tan x) \;-\; (1 - \sec x)\,\sec^2 x}{(\tan x)^2}.
$

Numerator Simplification

1. First term: 
   $-\sec x \tan x \cdot \tan x = -\sec x \,\tan^2 x.$

2. Second term: 
   $- (1 - \sec x)\,\sec^2 x = -\sec^2 x + \sec^3 x$ (distributing the minus sign).

Hence the numerator is $-\sec x \,\tan^2 x \;-\; \sec^2 x \;+\; \sec^3 x.$

Use the identity $\tan^2 x = \sec^2 x - 1$ to rewrite 
$\sec x \,\tan^2 x = \sec x(\sec^2 x - 1) = \sec^3 x - \sec x.$

Thus $-\sec x \tan^2 x = -(\sec^3 x - \sec x) 
= -\sec^3 x + \sec x.$

So the numerator becomes
$
(-\sec^3 x + \sec x) \;-\; \sec^2 x \;+\; \sec^3 x 
= -\sec^3 x + \sec^3 x \;+\; \sec x \;-\; \sec^2 x 
= \sec x - \sec^2 x 
= \sec x \,(1 - \sec x).
$

Therefore, the derivative is
$
\boxed{
y' = \dfrac{\sec x\,(1 - \sec x)}{\tan^2 x}.
}
$

}\SetValue{Rubric}{%Markdown



}\SetValue{Hint}{%
Solution Goes Here
}\SetValue{Answer}{%

}
\ProcessDATA



