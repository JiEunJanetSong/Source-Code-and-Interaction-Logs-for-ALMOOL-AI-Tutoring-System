\SetValue{Module}{1}\SetValue{SectionAB}{A}\SetValue{MainChapter}{}\SetValue{SubChapter}{}\SetValue{Contents}{%%
    
Differentiate.

$g(\theta)=e^\theta(\tan \theta-\theta)$

}\SetValue{Concept}{%



}\SetValue{AltText}{%



}\SetValue{Solution}{%

Let
$
g(\theta) = e^\theta (\tan \theta - \theta).
$
Use the product rule $(uv)' = u'v + uv'$. Set $u = e^\theta$ and $v = \tan \theta - \theta$. 

Then

$
u' = e^\theta, 
\quad
v' = \dfrac{d}{d\theta}(\tan \theta - \theta) = \sec^2 \theta - 1.
$

Hence,
$
g'(\theta) 
= e^\theta(\tan \theta - \theta) + e^\theta(\sec^2 \theta - 1) 
= e^\theta \bigl(\tan \theta - \theta + \sec^2 \theta - 1\bigr).
$

}\SetValue{Rubric}{%Markdown



}\SetValue{Hint}{%
Solution Goes Here
}\SetValue{Answer}{%

}
\ProcessDATA



