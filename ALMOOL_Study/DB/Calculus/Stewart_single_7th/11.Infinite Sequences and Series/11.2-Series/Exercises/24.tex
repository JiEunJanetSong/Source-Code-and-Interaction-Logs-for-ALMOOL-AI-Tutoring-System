\SetValue{Module}{1}\SetValue{SectionAB}{A}\SetValue{MainChapter}{}\SetValue{SubChapter}{}\SetValue{Contents}{%%
    
Determine whether the geometric series is convergent or divergent. If it is convergent, find its sum.

$$\sum_{n=0}^{\infty} \dfrac{1}{(\sqrt{2})^n}$$

}\SetValue{Concept}{%



}\SetValue{AltText}{%



}\SetValue{Solution}{%

Step 1: Identify the first term $a$ and common ratio $r$
We can rewrite the general term as:

$$
\left(\dfrac{1}{\sqrt{2}}\right)^n
$$


This is a geometric series of the form:

$$
\sum_{n=0}^{\infty} a r^n \quad \text { with } a=1, \quad r=\dfrac{1}{\sqrt{2}}
$$


Step 2: Convergence Test

A geometric series $\sum a r^n$ converges if:

$$
|r|<1
$$


Here:

$$
r=\dfrac{1}{\sqrt{2}} \approx 0.7071<1 \Rightarrow \text { converges }
$$

Step 3: Use the geometric series sum formula

$$
S=\dfrac{a}{1-r}=\dfrac{1}{1-\dfrac{1}{\sqrt{2}}}=\dfrac{1}{\dfrac{\sqrt{2}-1}{\sqrt{2}}}=\dfrac{\sqrt{2}}{\sqrt{2}-1}
$$


Rationalize the denominator:

$$
\dfrac{\sqrt{2}}{\sqrt{2}-1} \cdot \dfrac{\sqrt{2}+1}{\sqrt{2}+1}=\dfrac{\sqrt{2}(\sqrt{2}+1)}{(\sqrt{2})^2-1^2}=\dfrac{\sqrt{2}(\sqrt{2}+1)}{2-1}=\sqrt{2}(\sqrt{2}+1)=2+\sqrt{2}
$$


Final Answer:

- The series is convergent.

- The sum is $2+\sqrt{2}$.

}\SetValue{Rubric}{%Markdown



}\SetValue{Hint}{%
Solution Goes Here
}\SetValue{Answer}{%

}
\ProcessDATA



