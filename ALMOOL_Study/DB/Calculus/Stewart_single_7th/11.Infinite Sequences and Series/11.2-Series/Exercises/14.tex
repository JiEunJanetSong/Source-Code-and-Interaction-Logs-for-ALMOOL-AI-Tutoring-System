\SetValue{Module}{1}\SetValue{SectionAB}{A}\SetValue{MainChapter}{}\SetValue{SubChapter}{}\SetValue{Contents}{%%
    
Find at least 10 partial sums of the series. Graph both the sequence of terms and the sequence of partial sums on the same screen. Does it appear that the series is convergent or divergent? If it is convergent, find the sum. If it is divergent, explain why.

$$\sum_{n=2}^{\infty} \dfrac{1}{n(n+2)}$$

}\SetValue{Concept}{%



}\SetValue{AltText}{%



}\SetValue{Solution}{%

Here are the first 10 partial sums of the series:

\begin{tabular}{|l|l|}
\hline & $\sum_{n=2}^{\infty} \dfrac{1}{n(n+2)}$ \\
\hline $n$ & Partial Sum $s_n$ (rounded) \\
\hline 2 & 0.125000 \\
\hline 3 & 0.191667 \\
\hline 4 & 0.233333 \\
\hline 5 & 0.261905 \\
\hline 6 & 0.282738 \\
\hline 7 & 0.298611 \\
\hline 8 & 0.311111 \\
\hline 9 & 0.321212 \\
\hline 10 & 0.329545 \\
\hline 11 & 0.336538 \\
\hline
\end{tabular}

Observation from the graph:
- The terms $\dfrac{1}{n(n+2)}$ are decreasing.
- The partial sums are increasing and approaching a limit, indicating convergence.

Analytical Insight:

We can use partial fractions to simplify:

$$
\dfrac{1}{n(n+2)}=\dfrac{1}{2}\left(\dfrac{1}{n}-\dfrac{1}{n+2}\right)
$$


This makes it a telescoping series:

$$
\sum_{n=2}^{\infty} \dfrac{1}{n(n+2)}=\dfrac{1}{2} \sum_{n=2}^{\infty}\left(\dfrac{1}{n}-\dfrac{1}{n+2}\right)
$$


Writing out a few terms:

$$
\dfrac{1}{2}\left[\left(\dfrac{1}{2}-\dfrac{1}{4}\right)+\left(\dfrac{1}{3}-\dfrac{1}{5}\right)+\left(\dfrac{1}{4}-\dfrac{1}{6}\right)+\cdots\right]
$$


Most terms cancel out. What's left is:

$$
\dfrac{1}{2}\left(\dfrac{1}{2}+\dfrac{1}{3}-\dfrac{1}{n+1}-\dfrac{1}{n+2}\right) \rightarrow \dfrac{1}{2}\left(\dfrac{1}{2}+\dfrac{1}{3}\right)=\dfrac{1}{2} \cdot \dfrac{5}{6}=\dfrac{5}{12}
$$

Final Answer:

- The series converges.

- The sum is $\dfrac{5}{12} \approx 0.4167$.

}\SetValue{Rubric}{%Markdown



}\SetValue{Hint}{%
Solution Goes Here
}\SetValue{Answer}{%

}
\ProcessDATA



