\SetValue{Module}{1}\SetValue{SectionAB}{A}\SetValue{MainChapter}{}\SetValue{SubChapter}{}\SetValue{Contents}{%%
    
Determine whether the series is convergent or divergent by expressing $s_n$ as a telescoping sum. If it is convergent, find its sum.

$$\sum_{n=2}^{\infty} \dfrac{1}{n^3-n}$$

}\SetValue{Concept}{%



}\SetValue{AltText}{%



}\SetValue{Solution}{%

Step 1: Factor the Denominator

Factor the expression in the denominator:

$$
n^3-n=n\left(n^2-1\right)=n(n-1)(n+1)
$$


So the term becomes:

$$
\dfrac{1}{n(n-1)(n+1)}
$$


Step 2: Partial Fraction Decomposition

We decompose:

$$
\dfrac{1}{n(n-1)(n+1)}=\dfrac{A}{n-1}+\dfrac{B}{n}+\dfrac{C}{n+1}
$$


Multiply both sides by $n(n-1)(n+1)$ :

$$
1=A n(n+1)+B(n-1)(n+1)+C n(n-1)
$$


Now expand and simplify each term:

- $A n(n+1)=A\left(n^2+n\right)$

- $B(n-1)(n+1)=B\left(n^2-1\right)$

- $C n(n-1)=C\left(n^2-n\right)$

So,

$$
1=A\left(n^2+n\right)+B\left(n^2-1\right)+C\left(n^2-n\right)
$$


Group like terms:

$$
1=(A+B+C) n^2+(A-C) n+(-B)
$$


Set up equations by matching coefficients:

- $A+B+C=0$

- $A-C=0$

- $-B=1$

From the third equation:

$$
B=-1
$$


From the second:

$$
A=C
$$


From the first:

$$
A+(-1)+A=0 \Rightarrow 2 A=1 \Rightarrow A=\dfrac{1}{2}
$$


So:

$$
A=\dfrac{1}{2}, B=-1, C=\dfrac{1}{2}
$$

Step 3: Rewrite the Series

$$
\dfrac{1}{n(n-1)(n+1)}=\dfrac{1}{2(n-1)}-\dfrac{1}{n}+\dfrac{1}{2(n+1)}
$$


So the series becomes:

$$
\sum_{n=2}^{\infty}\left(\dfrac{1}{2(n-1)}-\dfrac{1}{n}+\dfrac{1}{2(n+1)}\right)
$$


Step 4: Telescoping the Sum

Let's write out the partial sum $s_N$ for $n=2$ to $N$:

$$
s_N=\sum_{n=2}^N\left(\dfrac{1}{2(n-1)}-\dfrac{1}{n}+\dfrac{1}{2(n+1)}\right)
$$


Break into three separate sums:

$$
s_N=\dfrac{1}{2} \sum_{n=2}^N \dfrac{1}{n-1}-\sum_{n=2}^N \dfrac{1}{n}+\dfrac{1}{2} \sum_{n=2}^N \dfrac{1}{n+1}
$$


Now reindex each sum:

- $\sum_{n=2}^N \dfrac{1}{n-1}=\sum_{k=1}^{N-1} \dfrac{1}{k}$

- $\sum_{n=2}^N \dfrac{1}{n}$ stays as is

- $\sum_{n=2}^N \dfrac{1}{n+1}=\sum_{k=3}^{N+1} \dfrac{1}{k}$

Putting it all together:

$$
s_N=\dfrac{1}{2} \sum_{k=1}^{N-1} \dfrac{1}{k}-\sum_{k=2}^N \dfrac{1}{k}+\dfrac{1}{2} \sum_{k=3}^{N+1} \dfrac{1}{k}
$$


Let's write a few terms to see the cancellation. Combining all three sums:

$$
s_N=\left(\dfrac{1}{2} \cdot \sum_{k=1}^{N-1} \dfrac{1}{k}\right)-\sum_{k=2}^N \dfrac{1}{k}+\left(\dfrac{1}{2} \cdot \sum_{k=3}^{N+1} \dfrac{1}{k}\right)
$$


Now combine the sums carefully. Most terms will cancel telescopically as $N \rightarrow \infty$, and the remaining terms are:

- From $\dfrac{1}{2} \sum_{k=1}^{N-1} \dfrac{1}{k}$ : contributes $\dfrac{1}{2}(1+\cdots)$

- From $-\sum_{k=2}^N \dfrac{1}{k}$ : cancels with overlapping $k \geq 2$

- From $\dfrac{1}{2} \sum_{k=3}^{N+1} \dfrac{1}{k}$ : cancels with overlapping $k \geq 3$

After cancellation, the remaining terms are:

$$
\dfrac{1}{2} \cdot 1+\left(-\dfrac{1}{2}\right)+\left(\dfrac{1}{2(N+1)}\right)
$$


But as $N \rightarrow \infty, \dfrac{1}{2(N+1)} \rightarrow 0$

So the limit of the partial sums is:

$$
\dfrac{1}{2}
$$

Final Answer:

The series converges, and its sum is: $\dfrac{1}{2}$

}\SetValue{Rubric}{%Markdown



}\SetValue{Hint}{%
Solution Goes Here
}\SetValue{Answer}{%

}
\ProcessDATA



