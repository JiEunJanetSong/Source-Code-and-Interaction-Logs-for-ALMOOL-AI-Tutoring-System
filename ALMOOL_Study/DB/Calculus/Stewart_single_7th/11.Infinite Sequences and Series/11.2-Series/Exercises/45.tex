\SetValue{Module}{1}\SetValue{SectionAB}{A}\SetValue{MainChapter}{}\SetValue{SubChapter}{}\SetValue{Contents}{%%
    
Determine whether the series is convergent or divergent by expressing $s_n$ as a telescoping sum. If it is convergent, find its sum.

$$\sum_{n=1}^{\infty} \dfrac{3}{n(n+3)}$$


}\SetValue{Concept}{%



}\SetValue{AltText}{%



}\SetValue{Solution}{%

Step 1: Partial fraction decomposition

We want to write:

$$
\dfrac{3}{n(n+3)}=\dfrac{A}{n}+\dfrac{B}{n+3}
$$


Multiply both sides by $n(n+3)$ :

$$
3=A(n+3)+B n=A n+3 A+B n=(A+B) n+3 A
$$


Match coefficients:

- $A+B=0$

- $3 A=3 \Rightarrow A=1$

Then $B=-1$

So:

$$
\dfrac{3}{n(n+3)}=\dfrac{1}{n}-\dfrac{1}{n+3}
$$

Step 2: Rewrite the series using partial fractions

$$
\sum_{n=1}^{\infty}\left(\dfrac{1}{n}-\dfrac{1}{n+3}\right)
$$


Write the first few terms:

$$
\left(\dfrac{1}{1}-\dfrac{1}{4}\right)+\left(\dfrac{1}{2}-\dfrac{1}{5}\right)+\left(\dfrac{1}{3}-\dfrac{1}{6}\right)+\left(\dfrac{1}{4}-\dfrac{1}{7}\right)+\left(\dfrac{1}{5}-\dfrac{1}{8}\right)+\cdots
$$


Step 3: Observe telescoping

In the sum:

$$
\sum_{n=1}^N\left(\dfrac{1}{n}-\dfrac{1}{n+3}\right)
$$

many terms cancel out. Let's compute the partial sum:

$$
s_N=\left(\dfrac{1}{1}+\dfrac{1}{2}+\dfrac{1}{3}\right)-\left(\dfrac{1}{N+1}+\dfrac{1}{N+2}+\dfrac{1}{N+3}\right)
$$


This happens because for each $n$, the term $-\dfrac{1}{n+3}$ cancels with a later $\dfrac{1}{n}$, except for the first three terms.

So:

$$
s_N=\sum_{n=1}^N\left(\dfrac{1}{n}-\dfrac{1}{n+3}\right)=\dfrac{1}{1}+\dfrac{1}{2}+\dfrac{1}{3}-\dfrac{1}{N+1}-\dfrac{1}{N+2}-\dfrac{1}{N+3}
$$

Step 4: Take the limit

As $N \rightarrow \infty$, the last three terms go to 0:

$$
\lim _{N \rightarrow \infty} s_N=\dfrac{1}{1}+\dfrac{1}{2}+\dfrac{1}{3}=\dfrac{6}{6}+\dfrac{3}{6}+\dfrac{2}{6}=\dfrac{11}{6}
$$

Final Answer:

- The series converges.

- The sum is: $\dfrac{11}{6}$

}\SetValue{Rubric}{%Markdown



}\SetValue{Hint}{%
Solution Goes Here
}\SetValue{Answer}{%

}
\ProcessDATA



