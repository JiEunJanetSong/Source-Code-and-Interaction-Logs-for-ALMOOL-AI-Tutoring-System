\SetValue{Module}{1}\SetValue{SectionAB}{A}\SetValue{MainChapter}{}\SetValue{SubChapter}{}\SetValue{Contents}{%%
    
Determine whether the series is convergent or divergent. If it is convergent, find its sum.

$$\sum_{n=1}^{\infty} \dfrac{1}{1+\left(\dfrac{2}{3}\right)^n}$$


}\SetValue{Concept}{%



}\SetValue{AltText}{%



}\SetValue{Solution}{%

Step 1: Analyze the Behavior of $a_n$

As $n \rightarrow \infty$, we have:

$$
\left(\dfrac{2}{3}\right)^n \rightarrow 0 \quad \Rightarrow \quad a_n \rightarrow \dfrac{1}{1+0}=1
$$


So the general term does not tend to 0. Instead, it approaches 1.

Step 2: Apply the Divergence Test

The divergence (nth-term) test says:

If $\lim _{n \rightarrow \infty} a_n \neq 0$, then $\sum a_n$ diverges.

Since:

$$
\lim _{n \rightarrow \infty} a_n=1 \neq 0
$$


The series diverges.

Final Answer: The series diverges.

}\SetValue{Rubric}{%Markdown



}\SetValue{Hint}{%
Solution Goes Here
}\SetValue{Answer}{%

}
\ProcessDATA



