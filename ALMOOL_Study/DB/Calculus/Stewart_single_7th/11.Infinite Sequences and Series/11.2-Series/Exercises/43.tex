\SetValue{Module}{1}\SetValue{SectionAB}{A}\SetValue{MainChapter}{}\SetValue{SubChapter}{}\SetValue{Contents}{%%
    
Determine whether the series is convergent or divergent by expressing $s_n$ as a telescoping sum. If it is convergent, find its sum.

$$\sum_{n=2}^{\infty} \dfrac{2}{n^2-1}$$



}\SetValue{Concept}{%



}\SetValue{AltText}{%



}\SetValue{Solution}{%

Step 1: Factor the denominator

Note that:

$$
n^2-1=(n-1)(n+1)
$$


So the term becomes:

$$
\dfrac{2}{n^2-1}=\dfrac{2}{(n-1)(n+1)}
$$


Step 2: Use partial fraction decomposition

We want to express:

$$
\dfrac{2}{(n-1)(n+1)}=\dfrac{A}{n-1}+\dfrac{B}{n+1}
$$


Multiply both sides by $(n-1)(n+1)$:

$$
2=A(n+1)+B(n-1)
$$


Expand:

$$
2=A n+A+B n-B=(A+B) n+(A-B)
$$


Equating coefficients:

- $A+B=0$

- $A-B=2$

Solving:

- From $A+B=0$, we get $B=-A$

- Plug into $A-B=2: A-(-A)=2 \Rightarrow 2 A=2 \Rightarrow A=1$

- Then $B=-1$

So:

$$
\dfrac{2}{(n-1)(n+1)}=\dfrac{1}{n-1}-\dfrac{1}{n+1}
$$


Step 3: Rewrite the series as a telescoping sum

$$
\sum_{n=2}^{\infty}\left(\dfrac{1}{n-1}-\dfrac{1}{n+1}\right)
$$


Let's write out the first few terms:

- For $n=2: \dfrac{1}{1}-\dfrac{1}{3}$

- For $n=3: \dfrac{1}{2}-\dfrac{1}{4}$

- For $n=4: \dfrac{1}{3}-\dfrac{1}{5}$

- For $n=5: \dfrac{1}{4}-\dfrac{1}{6}$

So the partial sum $s_N$ becomes:

$$
s_N=\left(1-\dfrac{1}{3}\right)+\left(\dfrac{1}{2}-\dfrac{1}{4}\right)+\left(\dfrac{1}{3}-\dfrac{1}{5}\right)+\cdots+\left(\dfrac{1}{N-1}-\dfrac{1}{N+1}\right)
$$


The terms cancel in a telescoping fashion:

$$
s_N=1+\dfrac{1}{2}-\dfrac{1}{N}-\dfrac{1}{N+1}
$$


Step 4: Take the limit as $N \rightarrow \infty$

$$
\lim _{N \rightarrow \infty} s_N=1+\dfrac{1}{2}-\lim _{N \rightarrow \infty}\left(\dfrac{1}{N}+\dfrac{1}{N+1}\right)=1+\dfrac{1}{2}-0=\dfrac{3}{2}
$$

Final Answer:

- The series converges.

- Its sum is: $\dfrac{3}{2}$

}\SetValue{Rubric}{%Markdown



}\SetValue{Hint}{%
Solution Goes Here
}\SetValue{Answer}{%

}
\ProcessDATA



