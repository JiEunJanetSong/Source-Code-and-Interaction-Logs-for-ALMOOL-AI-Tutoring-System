\SetValue{Module}{1}\SetValue{SectionAB}{A}\SetValue{MainChapter}{}\SetValue{SubChapter}{}\SetValue{Contents}{%%
    
Determine whether the sequence converges or diverges. If it converges, find the limit.

$$a_n=\frac{(-1)^n}{2\sqrt n}$$

}\SetValue{Concept}{%



}\SetValue{AltText}{%



}\SetValue{Solution}{%

We are given the sequence:

$
a_n = \frac{(-1)^n}{2\sqrt{n}}
$

Step 1: Understand the Behavior

- The numerator $(-1)^n$ alternates between $+1$ and $-1$, so the sequence oscillates in sign.

- The denominator $2\sqrt{n}$ increases without bound, so the magnitude of the terms gets smaller.

So the sequence looks like:

$
a_1 = \frac{-1}{2}, \quad a_2 = \frac{1}{2\sqrt{2}}, \quad a_3 = \frac{-1}{2\sqrt{3}}, \quad \dots
$

Each term gets closer to 0, and oscillates above and below 0.

Step 2: Take the Limit

We take the limit of the sequence:

$
\lim_{n \to \infty} \frac{(-1)^n}{2\sqrt{n}} = 0
$

- Because $\sqrt{n} \to \infty$

- And $\frac{1}{\sqrt{n}} \to 0$

- The factor $(-1)^n$ just makes the sequence alternate in sign but doesn’t affect convergence to zero.

Final Answer:

- The sequence converges

- The limit is:

$
\boxed{0}
$

}\SetValue{Rubric}{%Markdown



}\SetValue{Hint}{%
Solution Goes Here
}\SetValue{Answer}{%

}
\ProcessDATA



