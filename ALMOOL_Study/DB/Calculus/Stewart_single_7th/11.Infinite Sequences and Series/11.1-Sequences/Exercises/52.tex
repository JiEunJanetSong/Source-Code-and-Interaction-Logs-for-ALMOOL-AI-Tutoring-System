\SetValue{Module}{1}\SetValue{SectionAB}{A}\SetValue{MainChapter}{}\SetValue{SubChapter}{}\SetValue{Contents}{%%
    
Determine whether the sequence converges or diverges. If it converges, find the limit.

$$a_n=n-\sqrt{n+1}\sqrt{n+3}$$

}\SetValue{Concept}{%



}\SetValue{AltText}{%



}\SetValue{Solution}{%

We are given the sequence:

$
a_n = n - \sqrt{n+1}\sqrt{n+3}
$

We want to determine whether this sequence converges or diverges, and if it converges, find the limit as $n \to \infty$.


 Step 1: Simplify the expression

We can write the square root product:

$
\sqrt{n+1}\sqrt{n+3} = \sqrt{(n+1)(n+3)} = \sqrt{n^2 + 4n + 3}
$

So:

$
a_n = n - \sqrt{n^2 + 4n + 3}
$


 Step 2: Factor out $n$ inside the square root to get dominant behavior

Factor $n^2$ from inside the root:

$
a_n = n - \sqrt{n^2(1 + \frac{4}{n} + \frac{3}{n^2})}
= n - n \sqrt{1 + \frac{4}{n} + \frac{3}{n^2}}
$

Factor out $n$:

$
a_n = n\left(1 - \sqrt{1 + \frac{4}{n} + \frac{3}{n^2}}\right)
$


 Step 3: Use binomial expansion or approximation

As $n \to \infty$, we can use the Taylor expansion:

$
\sqrt{1 + x} \approx 1 + \frac{x}{2} - \frac{x^2}{8} + \dots
$

Let $x = \frac{4}{n} + \frac{3}{n^2}$, then:

$
\sqrt{1 + \frac{4}{n} + \frac{3}{n^2}} \approx 1 + \frac{1}{2} \left( \frac{4}{n} + \frac{3}{n^2} \right)
= 1 + \frac{2}{n} + \frac{3}{2n^2}
$

So:

$
a_n \approx n \left(1 - \left[1 + \frac{2}{n} + \frac{3}{2n^2}\right]\right)
= n \left(-\frac{2}{n} - \frac{3}{2n^2}\right)
= -2 - \frac{3}{2n}
$


 Step 4: Take the limit

$
\lim_{n \to \infty} a_n = \lim_{n \to \infty} \left(-2 - \frac{3}{2n} \right) = -2
$

Final Answer:

The sequence converges, and its limit is:

$
\boxed{-2}
$
}\SetValue{Rubric}{%Markdown



}\SetValue{Hint}{%
Solution Goes Here
}\SetValue{Answer}{%

}
\ProcessDATA



