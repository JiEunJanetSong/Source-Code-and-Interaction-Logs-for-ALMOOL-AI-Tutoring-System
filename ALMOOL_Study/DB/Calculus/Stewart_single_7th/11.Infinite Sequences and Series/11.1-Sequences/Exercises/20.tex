\SetValue{Module}{1}\SetValue{SectionAB}{A}\SetValue{MainChapter}{}\SetValue{SubChapter}{}\SetValue{Contents}{%%
    
Calculate, to four decimal places, the first ten terms of the sequence and use them to plot the graph of the sequence by hand. Does the sequence appear to have a limit? If so, calculate it. If not, explain why.

$$a_n=2+\frac{(-1)^n}{n}$$

}\SetValue{Concept}{%



}\SetValue{AltText}{%



}\SetValue{Solution}{%

We are given the sequence:

$
a_n = 2 + \frac{(-1)^n}{n}
$

 Step 1: Calculate the First 10 Terms (to 4 decimal places)

We’ll compute values of $a_n$ for $n = 1$ to $10$:

$
\begin{align}
a_1 &= 2 + \frac{(-1)^1}{1} = 2 - 1 = 1.0000 \\
a_2 &= 2 + \frac{1}{2} = 2.5000 \\
a_3 &= 2 + \frac{-1}{3} \approx 1.6667 \\
a_4 &= 2 + \frac{1}{4} = 2.2500 \\
a_5 &= 2 + \frac{-1}{5} = 1.8000 \\
a_6 &= 2 + \frac{1}{6} \approx 2.1667 \\
a_7 &= 2 + \frac{-1}{7} \approx 1.8571 \\
a_8 &= 2 + \frac{1}{8} = 2.1250 \\
a_9 &= 2 + \frac{-1}{9} \approx 1.8889 \\
a_{10} &= 2 + \frac{1}{10} = 2.1000 \\
\end{align}
$

 Step 2: Plot by Hand (Description)

To plot this sequence by hand:

- Plot the points $(n, a_n)$ from $n = 1$ to $10$.

- You’ll see the points oscillating above and below 2:

  - Odd $n$: values are below 2.

  - Even $n$: values are above 2.

- The oscillations get smaller over time.

Step 3: Does the Sequence Have a Limit?

Let’s examine:

$
a_n = 2 + \frac{(-1)^n}{n}
$

As $n \to \infty$:

- $\frac{1}{n} \to 0$

- $\frac{(-1)^n}{n} \to 0$, since numerator alternates ±1 but denominator grows without bound

So,

$
\lim_{n \to \infty} a_n = 2 + 0 = 2
$

Even though the sequence oscillates, the oscillations shrink and settle toward 2.

 Final Answer:

- First 10 terms (to 4 decimals):
  $
  \boxed{1.0000,\ 2.5000,\ 1.6667,\ 2.2500,\ 1.8000,\ 2.1667,\ 1.8571,\ 2.1250,\ 1.8889,\ 2.1000}
  $

- Limit:  
  $
  \boxed{2}
  $
- The sequence has a limit, and it is 2.

}\SetValue{Rubric}{%Markdown



}\SetValue{Hint}{%
Solution Goes Here
}\SetValue{Answer}{%

}
\ProcessDATA



