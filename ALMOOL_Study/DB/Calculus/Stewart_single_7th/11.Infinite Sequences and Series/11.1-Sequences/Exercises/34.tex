\SetValue{Module}{1}\SetValue{SectionAB}{A}\SetValue{MainChapter}{}\SetValue{SubChapter}{}\SetValue{Contents}{%%
    
Determine whether the sequence converges or diverges. If it converges, find the limit.

$$a_n=\frac{(-1)^{n+1}n}{n+\sqrt n}$$

}\SetValue{Concept}{%



}\SetValue{AltText}{%



}\SetValue{Solution}{%

We are given the sequence:

$
a_n = \frac{(-1)^{n+1}n}{n + \sqrt{n}}
$

Step 1: Analyze the Absolute Value of the Sequence

Focus on the magnitude first:

$
\left| a_n \right| = \left| \frac{n}{n + \sqrt{n}} \right| = \frac{n}{n + \sqrt{n}}
$

Divide numerator and denominator by $n$:

$
\left| a_n \right| = \frac{1}{1 + \frac{1}{\sqrt{n}}}
$

As $n \to \infty$, $\frac{1}{\sqrt{n}} \to 0$, so:

$
\left| a_n \right| \to \frac{1}{1 + 0} = 1
$

So the magnitude of the terms approaches 1.

Step 2: Consider the Sign

The sign alternates because of $(-1)^{n+1}$, so the sequence oscillates:

- When $n$ is odd: $a_n > 0$

- When $n$ is even: $a_n < 0$

And the values approach ±1 alternately.

Step 3: Does the Sequence Converge?

The sequence oscillates between values near 1 and -1 as $n$ increases. Since the values do not settle to a single number (they keep switching signs), the sequence does not converge.

Final Answer:

- The sequence diverges

- It oscillates and does not settle to a single limit

$
\boxed{\text{The sequence diverges}}
$
}\SetValue{Rubric}{%Markdown



}\SetValue{Hint}{%
Solution Goes Here
}\SetValue{Answer}{%

}
\ProcessDATA



