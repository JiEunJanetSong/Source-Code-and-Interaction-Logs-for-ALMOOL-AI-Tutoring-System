\SetValue{Module}{1}\SetValue{SectionAB}{A}\SetValue{MainChapter}{}\SetValue{SubChapter}{}\SetValue{Contents}{%%
    
List the first five terms of the sequence.

$$a_n=\frac{(-1)^n n}{n!+1}$$

}\SetValue{Concept}{%



}\SetValue{AltText}{%



}\SetValue{Solution}{%

Let's calculate the first five terms of the sequence $a_n=\frac{(-1)^n n}{n!+1}$ by substituting n = 1, 2, 3, 4, and 5.

1.  For n = 1:

    $$a_1 = \frac{(-1)^1 \cdot 1}{1!+1} = \frac{-1 \cdot 1}{1+1} = \frac{-1}{2} = -\frac{1}{2}$$

2.  For n = 2:

    $$a_2 = \frac{(-1)^2 \cdot 2}{2!+1} = \frac{1 \cdot 2}{2+1} = \frac{2}{3}$$

3.  For n = 3:

    $$a_3 = \frac{(-1)^3 \cdot 3}{3!+1} = \frac{-1 \cdot 3}{6+1} = \frac{-3}{7} = -\frac{3}{7}$$

4.  For n = 4:

    $$a_4 = \frac{(-1)^4 \cdot 4}{4!+1} = \frac{1 \cdot 4}{24+1} = \frac{4}{25}$$

5.  For n = 5:

    $$a_5 = \frac{(-1)^5 \cdot 5}{5!+1} = \frac{-1 \cdot 5}{120+1} = \frac{-5}{121} = -\frac{5}{121}$$

The first five terms of the sequence are:

$$-\frac{1}{2}, \frac{2}{3}, -\frac{3}{7}, \frac{4}{25}, -\frac{5}{121}$$

}\SetValue{Rubric}{%Markdown



}\SetValue{Hint}{%
Solution Goes Here
}\SetValue{Answer}{%

}
\ProcessDATA



