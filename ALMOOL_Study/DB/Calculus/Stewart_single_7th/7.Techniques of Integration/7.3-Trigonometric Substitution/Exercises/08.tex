\SetValue{SectionAB}{}\SetValue{MainChapter}{}\SetValue{SubChapter}{}\SetValue{Contents}{%

Evaluate the integral.

$$\int \dfrac{d t}{t^2 \sqrt{t^2-16}}$$

}\SetValue{Solution}{%

1) Trigonometric Substitution

Set

$$
t=4 \sec (\theta)
$$


Then:
1. $t^2-16=16 \sec ^2(\theta)-16=16\left(\sec ^2(\theta)-1\right)=16 \tan ^2(\theta)$.

Hence,

$$
\sqrt{t^2-16}=4|\tan (\theta)|
$$

(For $t>4, \tan (\theta) \geq 0$, so $\sqrt{t^2-16}=4 \tan (\theta)$.)
2. $d t=4 \sec (\theta) \tan (\theta) d \theta$.
3. $t^2=(4 \sec (\theta))^2=16 \sec ^2(\theta)$.

Substitute into the integrand:

$$
\dfrac{1}{t^2 \sqrt{t^2-16}} d t=\dfrac{1}{\left(16 \sec ^2(\theta)\right)(4 \tan (\theta))} \times(4 \sec (\theta) \tan (\theta)) d \theta
$$


Let us simplify step by step inside the product:
- The denominator in front is $16 \sec ^2(\theta) \cdot 4 \tan (\theta)=64 \sec ^2(\theta) \tan (\theta)$.
- The factor from $d t$ is $4 \sec (\theta) \tan (\theta)$.

Thus,

$$
\dfrac{1}{64 \sec ^2(\theta) \tan (\theta)} \times 4 \sec (\theta) \tan (\theta)=\dfrac{4}{64} \dfrac{\sec (\theta) \tan (\theta)}{\sec ^2(\theta) \tan (\theta)}=\dfrac{4}{64} \dfrac{1}{\sec (\theta)}=\dfrac{1}{16} \cos (\theta)
$$


Hence the integral becomes

$$
\int \dfrac{d t}{t^2 \sqrt{t^2-16}}=\int \dfrac{1}{16} \cos (\theta) d \theta=\dfrac{1}{16} \int \cos (\theta) d \theta=\dfrac{1}{16} \sin (\theta)+C
$$

2) Back-Substitute in Terms of $t$

From $t=4 \sec (\theta)$, we have $\sec (\theta)=\dfrac{t}{4}$. Then

$$
\cos (\theta)=\dfrac{4}{t}, \quad \sin ^2(\theta)=1-\cos ^2(\theta)=1-\left(\dfrac{4}{t}\right)^2=1-\dfrac{16}{t^2}=\dfrac{t^2-16}{t^2}
$$


Thus

$$
\sin (\theta)=\dfrac{\sqrt{t^2-16}}{|t|}
$$


If we assume $t>0$ and $|t| \geq 4$, then $|t|=t$. Hence

$$
\sin (\theta)=\dfrac{\sqrt{t^2-16}}{t}
$$

Therefore,

$$
\dfrac{1}{16} \sin (\theta)=\dfrac{1}{16} \dfrac{\sqrt{t^2-16}}{t}
$$


Final Result

$$
\int \dfrac{d t}{t^2 \sqrt{t^2-16}}=\dfrac{1}{16} \dfrac{\sqrt{t^2-16}}{t}+C.
$$

}\SetValue{Rubric}{%

# General Scoring Notes


## Model Solution (a)


## Scoring (a)

For $\text { Area }=\int_0^2(f(x)-g(x)) d x$,
Integrand; 1 point, Answer; 1 point.

## Scoring notes (a)


}\SetValue{Answer}{%

}
\ProcessDATA


