\SetValue{SectionAB}{}\SetValue{MainChapter}{}\SetValue{SubChapter}{}\SetValue{Contents}{%

From 5 A.M. to 10 A.M., the rate at which vehicles arrive at a certain toll plaza is given by $A(t)=450 \sqrt{\sin (0.62 t)}$, where $t$ is the number of hours after 5 A.M. and $A(t)$ is measured in vehicles per hour. Traffic is flowing smoothly at 5 A.M. with no vehicles waiting in line.

(d) A line forms whenever $A(t) \geq 400$. The number of vehicles in line at time $t$, for $a \leq t \leq 4$, is given by $N(t)=\int_a^t(A(x)-400) d x$, where $a$ is the time when a line first begins to form. To the nearest whole number, find the greatest number of vehicles in line at the toll plaza in the time interval $a \leq t \leq 4$. Justify your answer.

}\SetValue{Solution}{%

$$
\begin{aligned}
& N^{\prime}(t)=A(t)-400=0 \\
& \Rightarrow A(t)=400 \Rightarrow t=1.469372, t=3.597713
\end{aligned}
$$


$$
\begin{aligned}
& a=1.469372 \\
& b=3.597713
\end{aligned}
$$

\begin{tabular}{c|c}
    $t$ & $N(t)=\int_a^t(A(x)-400) d x$ \\
    \hline$a$ & 0 \\
    $b$ & 71.254129 \\
    4 & 62.338346
    \end{tabular}
    
The greatest number of vehicles in line is 71.

}\SetValue{Rubric}{%

# General Scoring Notes

The model solution is presented using standard mathematical notation.

Answers (numeric or algebraic) need not be simplified. Answers given as a decimal approximation should be correct to three places after the decimal point. Within each individual free-response question, at most one point is not earned for inappropriate rounding.

From 5 A.M. to 10 A.M., the rate at which vehicles arrive at a certain toll plaza is given by $A(t)=450 \sqrt{\sin (0.62 t)}$, where $t$ is the number of hours after 5 A.M. and $A(t)$ is measured in vehicles per hour. Traffic is flowing smoothly at 5 A.M. with no vehicles waiting in line.

## Model Solution (d)

$$
\begin{aligned}
& N^{\prime}(t)=A(t)-400=0 \\
& \Rightarrow A(t)=400 \Rightarrow t=1.469372, t=3.597713
\end{aligned}
$$


$$
\begin{aligned}
& a=1.469372 \\
& b=3.597713
\end{aligned}
$$

\begin{tabular}{c|c}
    $t$ & $N(t)=\int_a^t(A(x)-400) d x$ \\
    \hline$a$ & 0 \\
    $b$ & 71.254129 \\
    4 & 62.338346
    \end{tabular}
    
The greatest number of vehicles in line is 71.

## Scoring (d)

- For $N^{\prime}(t)=A(t)-400=0$
    - Considers $N^{\prime}(t)=0$; 1 point

- For $A(t)=400 \Rightarrow t=1.469372, t=3.597713, a=1.469372, b=3.597713$
    - $t=a$ and $t=b$; 1 point

- For The greatest number of vehicles in line is 71.
    - Answer $71$; 1 point

- For \begin{tabular}{c|c}
    $t$ & $N(t)=\int_a^t(A(x)-400) d x$ \\
    \hline$a$ & 0 \\
    $b$ & 71.254129 \\
    4 & 62.338346
    \end{tabular}
    - Justification; 1 point    

## Scoring notes (d)

- It is not necessary to indicate that $A(t)=400$ to earn the first point, although this statement alone would earn the first point.

- A response of ``$A(t) \geq 400$ when $1.469372 \leq t \leq 3.597713$'' will earn the first 2 points. A response of ``$A(t) \geq 400$'' along with the presence of exactly one of the two numbers above will earn the first point, but not the second. A response of ``$A(t) \geq 400$'' by itself will not earn either of the first 2 points.

- To earn the second point the values for $a$ and $b$ must be accurate to the number of decimals presented, with at least one and up to three decimal places. These may appear only in a candidates table, as limits of integration, or on a number line.

- A response with incorrect notation involving $t$ or $x$ is eligible to earn all 4 points.

- A response that does not earn the first point is still eligible for the remaining 3 points.

- To earn the third point, a response must present the greatest number of vehicles. This point is earned for answers of either 71 or $71.254 * * *$ only.

- A correct justification earns the fourth point, even if the third point is not earned because of a decimal presentation error.

- When using a Candidates Test, the response must include the values for $N(a), N(b)$, and $N(4)$ to earn the fourth point. These values must be correct to the number of decimals presented, with up to three decimal places. (Correctly rounded integer values are acceptable.)

- Alternate solution for the third and fourth points:

For $a \leq t \leq b, A(t) \geq 400$. For $b \leq t \leq 4, A(t) \leq 400$.
Thus, $N(t)=\int_a^t(A(x)-400) d x$ is greatest at $t=b$.
$N(b)=71.254129$, and the greatest number of vehicles in line is 71.

- Degree mode: The response is only eligible to earn the first point because in degree mode $A(t)<400$.

Total for part (d) 4 points

}\SetValue{Answer}{%

}
\ProcessDATA


