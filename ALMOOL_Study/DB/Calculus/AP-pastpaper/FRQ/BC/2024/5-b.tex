\SetValue{SectionAB}{}\SetValue{MainChapter}{}\SetValue{SubChapter}{}\SetValue{Contents}{%

\begin{tabular}{|c|c|c|c|}
    \hline$x$ & 0 & $\pi$ & $2 \pi$ \\
    \hline$f^{\prime}(x)$ & 5 & 6 & 0 \\
    \hline
    \end{tabular}
    
The function $f$ is twice differentiable for all $x$ with $f(0)=0$. Values of $f^{\prime}$, the derivative of $f$, are given in the table for selected values of $x$.

(b) What information does $\int_0^\pi \sqrt{1+\left(f^{\prime}(x)\right)^2} d x$ provide about the graph of $f$?

}\SetValue{Solution}{%

$\int_0^\pi \sqrt{1+\left(f^{\prime}(x)\right)^2} d x$ is the are length of the graph of $f$ on $[0, \pi]$.

}\SetValue{Rubric}{%

# General Scoring Notes

The model solution is presented using standard mathematical notation.
Answers (numeric or algebraic) need not be simplified. Answers given as a decimal approximation should be correct to three places after the decimal point. Within each individual free-response question, at most one point is not earned for inappropriate rounding.

\begin{tabular}{|c|c|c|c|}
\hline$x$ & 0 & $\pi$ & $2 \pi$ \\
\hline$f^{\prime}(x)$ & 5 & 6 & 0 \\
\hline
\end{tabular}

The function $f$ is twice differentiable for all $x$ with $f(0)=0$. Values of $f^{\prime}$, the derivative of $f$, are given in the table for selected values of $x$.

## Model Solution (b)

$\int_0^\pi \sqrt{1+\left(f^{\prime}(x)\right)^2} d x$ is the are length of the graph of $f$ on $[0, \pi]$.

## Scoring (b)

- For $\int_0^\pi \sqrt{1+\left(f^{\prime}(x)\right)^2} d x$ 
    - Arc length of $f$; 1 point

- For on $[0, \pi]$.  
    - Interval $[0, \pi]$; 1 point

## Scoring notes (b)

- A response of ``arc length" or ``length" earns the first point. Such a response does not need to reference $f$. However, if the response references a different function, the response does not earn the first point and is eligible to earn the second point.

- A response referring to distance explicitly connected to the graph or $f$ (or equivalent) earns the first point. For example, a response of ``distance along the curve" or ``distance traveled by a particle moving along $f^{\prime \prime}$ earns the first point and is eligible to earn the second point.

- A response referring to distance that is not explicitly connected to the graph of $f$ does not earn the first point but is eligible to earn the second point. For example, a response of ``distance" or ``distance traveled" does not earn the first point but is eligible to earn the second point.

- To earn the second point a response must connect the interval $[0, \pi]$ to are length, length, or distance.

Total for part (b) 2 points

}\SetValue{Answer}{%

}
\ProcessDATA


