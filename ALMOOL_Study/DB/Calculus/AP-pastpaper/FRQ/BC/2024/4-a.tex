\SetValue{SectionAB}{}\SetValue{MainChapter}{}\SetValue{SubChapter}{}\SetValue{Contents}{%

\begin{center}		
    \includegraphics[scale=0.4]{APpics/2024BCF04.png}
        \end{center}

The graph of the differentiable function $f$, shown for $-6 \leq x \leq 7$, has a horizontal tangent at $x=-2$ and is linear for $0 \leq x \leq 7$. Let $R$ be the region in the second quadrant bounded by the graph of $f$, the vertical line $x=-6$, and the $x$ - and $y$-axes. Region $R$ has area 12.

(a) The function $g$ is defined by $g(x)=\int_0^x f(t) d t$. Find the values of $g(-6), g(4)$, and $g(6)$.
  
}\SetValue{AltText}{%



}\SetValue{Solution}{%

\begin{align*}
    & g(-6)=\int_0^{-6} f(t) d t=-\int_{-6}^0 f(t) d t=-12 \\
    & g(4)=\int_0^4 f(t) d t=\frac{1}{2} \cdot 4 \cdot 2=4 \\
    & g(6)=\int_0^6 f(t) d t=\frac{1}{2} \cdot 4 \cdot 2-\frac{1}{2} \cdot 2 \cdot 1=3
    \end{align*}

}\SetValue{Rubric}{%

# General Scoring Notes

The model solution is presented using standard mathematical notation.

Answers (numeric or algebraic) need not be simplified. Answers given as a decimal approximation should be correct to three places after the decimal point. Within each individual free-response question, at most one point is not earned for inappropriate rounding.

The graph of the differentiable function $f$, shown for $-6 \leq x \leq 7$, has a horizontal tangent at $x=-2$ and is linear for $0 \leq x \leq 7$. Let $R$ be the region in the second quadrant bounded by the graph of $f$, the vertical line $x=-6$, and the $x$ - and $y$-axes. Region $R$ has area 12.

## Model Solution (a)

\begin{align*}
    & g(-6)=\int_0^{-6} f(t) d t=-\int_{-6}^0 f(t) d t=-12 \\
    & g(4)=\int_0^4 f(t) d t=\frac{1}{2} \cdot 4 \cdot 2=4 \\
    & g(6)=\int_0^6 f(t) d t=\frac{1}{2} \cdot 4 \cdot 2-\frac{1}{2} \cdot 2 \cdot 1=3
    \end{align*}

## Scoring (a)

- For $g(-6)=\int_0^{-6} f(t) d t=-\int_{-6}^0 f(t) d t=-12$ 
    -  $g(-6)$; 1 point

- For $g(4)=\int_0^4 f(t) d t=\frac{1}{2} \cdot 4 \cdot 2=4$ 
     - $g(4)$; 1 point  
    
- For $g(6)=\int_0^6 f(t) d t=\frac{1}{2} \cdot 4 \cdot 2-\frac{1}{2} \cdot 2 \cdot 1=3$
    - $g(6)$; 1 point 

## Scoring notes (a)

- Supporting work is not required for any of these values. However, any supporting work that is shown must be correct to earn the corresponding point.

- Special case: A response that explicitly presents $g(x)=\int_{-6}^x f(t) d t$ does not earn the first point it would have otherwise earned. The response is eligible for all subsequent points for correct answers, or for consistent answers with supporting work.

- Note: $\int_{-6}^{-6} f(t) d t=0, \int_{-6}^4 f(t) d t=16, \int_{-6}^6 f(t) d t=15$

- Labeled values may be presented in any order. Unlabeled values are read from left to right and from top to bottom as $g(-6), g(4)$, and $g(6)$, respectively. A response that presents only 1 or 2 values must label them to earn any points.

Total for part (a) 3 points
}\SetValue{Answer}{%

}
\ProcessDATA


