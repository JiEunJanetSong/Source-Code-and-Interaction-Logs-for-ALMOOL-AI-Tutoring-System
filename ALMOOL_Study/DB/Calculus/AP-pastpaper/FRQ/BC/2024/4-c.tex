\SetValue{SectionAB}{}\SetValue{MainChapter}{}\SetValue{SubChapter}{}\SetValue{Contents}{%

\begin{center}		
   \includegraphics[scale=0.4]{APpics/2024BCF04.png}
       \end{center}

The graph of the differentiable function $f$, shown for $-6 \leq x \leq 7$, has a horizontal tangent at $x=-2$ and is linear for $0 \leq x \leq 7$. Let $R$ be the region in the second quadrant bounded by the graph of $f$, the vertical line $x=-6$, and the $x$ - and $y$-axes. Region $R$ has area 12.

(c) The function $h$ is defined by $h(x)=\int_{-6}^x f^{\prime}(t) d t$. Find the values of $h(6), h^{\prime}(6)$, and $h^{\prime \prime}(6)$. Show the work that leads to your answers.

}\SetValue{AltText}{%



}\SetValue{Solution}{%

\begin{align*}
   &h(6)=\int_{-6}^6 f^{\prime}(t) d t=f(6)-f(-6)=-1-0.5=-1.5\\
   & h^{\prime}(x)=f^{\prime}(x), \text { so } h^{\prime}(6)=f^{\prime}(6)=-\frac{1}{2} \\
   & h^{\prime \prime}(x)=f^{\prime \prime}(x), \text { so } h^{\prime \prime}(6)=f^{\prime \prime}(6)=0
   \end{align*}
 
}\SetValue{Rubric}{%

# General Scoring Notes

The model solution is presented using standard mathematical notation.

Answers (numeric or algebraic) need not be simplified. Answers given as a decimal approximation should be correct to three places after the decimal point. Within each individual free-response question, at most one point is not earned for inappropriate rounding.

The graph of the differentiable function $f$, shown for $-6 \leq x \leq 7$, has a horizontal tangent at $x=-2$ and is linear for $0 \leq x \leq 7$. Let $R$ be the region in the second quadrant bounded by the graph of $f$, the vertical line $x=-6$, and the $x$ - and $y$-axes. Region $R$ has area 12.

## Model Solution (c)

\begin{align*}
   &h(6)=\int_{-6}^6 f^{\prime}(t) d t=f(6)-f(-6)=-1-0.5=-1.5\\
   & h^{\prime}(x)=f^{\prime}(x), \text { so } h^{\prime}(6)=f^{\prime}(6)=-\frac{1}{2} \\
   & h^{\prime \prime}(x)=f^{\prime \prime}(x), \text { so } h^{\prime \prime}(6)=f^{\prime \prime}(6)=0
   \end{align*}
 
## Scoring (c)

- For $h(6)=\int_{-6}^6 f^{\prime}(t) d t=f(6)-f(-6)=-1-0.5=-1.5$ 
   - Uses Fundamental Theorem of Calculus; 1 point   
   - $h(6)$ with supporting work; 1 point 

- For $h^{\prime}(x)=f^{\prime}(x), \text { so } h^{\prime}(6)=f^{\prime}(6)=-\frac{1}{2}$
   - $h^{\prime}(6)$; 1 point 

- For $h^{\prime \prime}(x)=f^{\prime \prime}(x), \text { so } h^{\prime \prime}(6)=f^{\prime \prime}(6)=0$
   - $h^{\prime}(6)$; 1 point

## Scoring notes (c)

- Labeled values may be presented in any order.

- Unlabeled values are read from left to right and from top to bottom as $h(6), h^{\prime}(6)$, and $h^{\prime \prime}(6)$, respectively. A response that presents only 1 or 2 values must label them in order to earn any points.

- A response of $h(6)=-1.5$ does not earn either of the first 2 points. A response of $h(6)=f(6)-f(-6)$ earns the first point but not yet the second point.

- A response of $h(6)=-1-0.5$ is the minimum work required to earn both of the first 2 points.

- To earn the third point a response must state either $h^{\prime}(x)=f^{\prime}(x)$ or $h^{\prime}(6)=f^{\prime}(6)$, and provide an answer of $-\frac{1}{2}$.

- The fourth point is earned for a response of $h^{\prime \prime}(6)=0$, with or without supporting work.

- A response that has one or more linkage errors does not earn the first point it would have otherwise earned. For example, $h^{\prime}(x)=f^{\prime}(6)=-\frac{1}{2}$ does not earn the third point but is eligible for the fourth point even in the presence of another linkage error, such as $h^{\prime \prime}(x)=f^{\prime \prime}(6)=0$.

Total for part (c) 4 points
}\SetValue{Answer}{%

}
\ProcessDATA


