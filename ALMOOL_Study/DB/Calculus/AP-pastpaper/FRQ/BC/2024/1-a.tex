\SetValue{SectionAB}{}\SetValue{MainChapter}{}\SetValue{SubChapter}{}\SetValue{Contents}{%

\begin{center}
    \begin{tabular}{|c|c|c|c|c|}
        \hline
        $t$  (minutes) & 0 & 3 & 7 & 12 \\ 
        \hline
        $C(t)$  (degrees Celsius) & 100 & 85 & 69 & 55 \\
        \hline
    \end{tabular}
\end{center}
    
The temperature of coffee in a cup at time $t$ minutes is modeled by a decreasing differentiable function $C$, where $C(t)$ is measured in degrees Celsius. For $0 \leq t \leq 12$, selected values of $C(t)$ are given in the table shown.
    
(a) Approximate $C^{\prime}(5)$ using the average rate of change of $C$ over the interval $3 \leq t \leq 7$. Show the work that leads to your answer and include units of measure.
   
}\SetValue{Solution}{%

$
C^{\prime}(5) \approx \frac{C(7)-C(3)}{7-3}=\frac{69-85}{4}=-4 \text { degrees Celsius per minute}
$

}\SetValue{Rubric}{%

# General Scoring Notes

The model solution is presented using standard mathematical notation.

Answers (numeric or algebraic) need not be simplified. Answers given as a decimal approximation should be correct to three places after the decimal point. Within each individual free-response question, at most one point is not earned for inappropriate rounding.

| $t$ <br> (minutes) | 0 | 3 | 7 | 12 |
| :---: | :---: | :---: | :---: | :---: |
| $C(t)$ <br> (degrees Celsius) | 100 | 85 | 69 | 55 |

The temperature of coffee in a cup at time $t$ minutes is modeled by a decreasing differentiable function $C$, where $C(t)$ is measured in degrees Celsius. For $0 \leq t \leq 12$, selected values of $C(t)$ are given in the table shown.

## Model Solution (a)

$
C^{\prime}(5) \approx \frac{C(7)-C(3)}{7-3}=\frac{69-85}{4}=-4 \text { degrees Celsius per minute}
$

## Scoring (a)

- For $C^{\prime}(5) \approx \frac{C(7)-C(3)}{7-3}=\frac{69-85}{4}=-4 $ 
    - Estimate with supporting work: 1 point \\
 
- For $\text { degrees Celsius per minute}$
   - Units; 1 point 

## Scoring notes (a)

- To earn the first point a response must include a difference and a quotient as the supporting work.

- $\frac{-16}{7-3}, \frac{69-85}{7-3}$, or $\frac{69-85}{4}$ is sufficient to earn the first point.

- A response that presents only units without a numerical approximation for $C^{\prime}(5)$ does not earn the second point.

- The second point is also earned for ``degrees per minute" attached to a numerical value.

Total for part (a) 2 points

}\SetValue{Answer}{%

}
\ProcessDATA


