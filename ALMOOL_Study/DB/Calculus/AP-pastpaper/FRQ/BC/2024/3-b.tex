\SetValue{SectionAB}{}\SetValue{MainChapter}{}\SetValue{SubChapter}{}\SetValue{Contents}{%

The depth of seawater at a location can be modeled by the function $H$ that satisfies the differential equation $\frac{d H}{d t}=\frac{1}{2}(H-1) \cos \left(\frac{t}{2}\right)$, where $H(t)$ is measured in feet and $t$ is measured in hours after noon $(t=0)$. It is known that $H(0)=4$.
 
(b) For $0<t<5$, it can be shown that $H(t)>1$. Find the value of $t$, for $0<t<5$, at which $H$ has a critical point. Determine whether the critical point corresponds to a relative minimum, a relative maximum, or neither a relative minimum nor a relative maximum of the depth of seawater at the location. Justify your answer.

}\SetValue{Solution}{%

Because $H(t)>1$, then $\frac{d H}{d t}=0$ implies $\cos \left(\frac{t}{2}\right)=0$.

This implies that $t=\pi$ is a critical point.
For $0<t<\pi, \frac{d H}{d t}>0$ and for $\pi<t<5, \frac{d H}{d t}<0$. Therefore, $t=\pi$ is the location of a relative maximum value of $H$.

}\SetValue{Rubric}{%

# General Scoring Notes

The model solution is presented using standard mathematical notation.

Answers (numeric or algebraic) need not be simplified. Answers given as a decimal approximation should be correct to three places after the decimal point. Within each individual free-response question, at most one point is not earned for inappropriate rounding.

The depth of seawater at a location can be modeled by the function $H$ that satisfies the differential equation $\frac{d H}{d t}=\frac{1}{2}(H-1) \cos \left(\frac{t}{2}\right)$, where $H(t)$ is measured in feet and $t$ is measured in hours after noon $(t=0)$. It is known that $H(0)=4$.

## Model Solution (b)

Because $H(t)>1$, then $\frac{d H}{d t}=0$ implies $\cos \left(\frac{t}{2}\right)=0$.

This implies that $t=\pi$ is a critical point.
For $0<t<\pi, \frac{d H}{d t}>0$ and for $\pi<t<5, \frac{d H}{d t}<0$. Therefore, $t=\pi$ is the location of a relative maximum value of $H$.

## Scoring (b)

If either the mathematical expression or the sentence is correct, it should be considered correct.

- For Because $H(t)>1$, then $\frac{d H}{d t}=\frac{1}{2}(H-1)\cos \left(\frac{t}{2}\right)=0$ implies $\cos \left(\frac{t}{2}\right)=0$.
    - Considers sign of $\frac{d H}{d t}$; 1 point

- For This implies that $t=\pi$ is a critical point.
    - Identifies $t=\pi$;  1 point

- For $0<t<\pi, \frac{d H}{d t}>0$ and for $\pi<t<5, \frac{d H}{d t}<0$. Therefore, $t=\pi$ is the location of a relative maximum value of $H$.
    - Answer with justification; 1 point

## Scoring notes (b)

- The first point is earned for considering $\frac{d H}{d t}=0, \frac{d H}{d t}>0, \frac{d H}{d t}<0, \cos \left(\frac{t}{2}\right)=0, \cos \left(\frac{t}{2}\right)>0$, or $\cos \left(\frac{t}{2}\right)<0$.

- The second point is earned for identifying $t=\pi$, with or without supporting work. A response may consider $H=1$ or $t=1$ as potential critical points without penalty.

- The third point cannot be carned without the first point. The third point is earned only for a correct justification and a correct answer of ``relative maximum."

- The justification can be shown by determining the $\operatorname{sign}$ of $\frac{d H}{d t}\left(\right.$ (or $\cos \left(\frac{t}{2}\right)$) at a single value in $0<t<\pi$ and at a single value in $\pi<t<5$. It is not necessary to state that $\frac{d H}{d t}$ does not change sign on these intervals.

- The third point can also be carned by using the Second Derivative Test. For example:

\begin{align*}
    \frac{d^2 H}{d t^2} &= \frac{1}{2}(H-1)\left(-\frac{1}{2} \sin \left(\frac{t}{2}\right)\right) 
    + \cos \left(\frac{t}{2}\right) \cdot \frac{1}{2} \cdot \frac{d H}{d t}, \\
    \left.\frac{d^2 H}{d t^2}\right|_{t=\pi} &< 0
    \end{align*}
    
Therefore, $t=\pi$ is the location of a relative maximum value of $H$.

Total for part (b) 3 points

}\SetValue{Answer}{%

}
\ProcessDATA


