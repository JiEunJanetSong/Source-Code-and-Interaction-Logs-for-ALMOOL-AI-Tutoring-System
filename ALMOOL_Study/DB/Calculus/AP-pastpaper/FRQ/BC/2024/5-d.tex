\SetValue{SectionAB}{}\SetValue{MainChapter}{}\SetValue{SubChapter}{}\SetValue{Contents}{%

\begin{tabular}{|c|c|c|c|}
    \hline$x$ & 0 & $\pi$ & $2 \pi$ \\
    \hline$f^{\prime}(x)$ & 5 & 6 & 0 \\
    \hline
    \end{tabular}
    
The function $f$ is twice differentiable for all $x$ with $f(0)=0$. Values of $f^{\prime}$, the derivative of $f$, are given in the table for selected values of $x$.

(d) Find $\int(t+5) \cos \left(\frac{t}{4}\right) d t$. Show the work that leads to your answer.

}\SetValue{Solution}{%

$\begin{array}{ll}u=t+5 & d v=\cos \left(\frac{t}{4}\right) d t \\ d u=d t & v=4 \sin \left(\frac{t}{4}\right)\end{array}$ 

$\int(t+5) \cos \left(\frac{t}{4}\right) d t=4(t+5) \sin \left(\frac{t}{4}\right)-\int 4 \sin \left(\frac{t}{4}\right) d t$

$=4(t+5) \sin \left(\frac{t}{4}\right)+16 \cos \left(\frac{t}{4}\right)+C$

}\SetValue{Rubric}{%

# General Scoring Notes

The model solution is presented using standard mathematical notation.
Answers (numeric or algebraic) need not be simplified. Answers given as a decimal approximation should be correct to three places after the decimal point. Within each individual free-response question, at most one point is not earned for inappropriate rounding.

\begin{tabular}{|c|c|c|c|}
\hline$x$ & 0 & $\pi$ & $2 \pi$ \\
\hline$f^{\prime}(x)$ & 5 & 6 & 0 \\
\hline
\end{tabular}

The function $f$ is twice differentiable for all $x$ with $f(0)=0$. Values of $f^{\prime}$, the derivative of $f$, are given in the table for selected values of $x$.

## Model Solution (d)

$\begin{array}{ll}u=t+5 & d v=\cos \left(\frac{t}{4}\right) d t \\ d u=d t & v=4 \sin \left(\frac{t}{4}\right)\end{array}$ 

$\int(t+5) \cos \left(\frac{t}{4}\right) d t=4(t+5) \sin \left(\frac{t}{4}\right)-\int 4 \sin \left(\frac{t}{4}\right) d t$

$=4(t+5) \sin \left(\frac{t}{4}\right)+16 \cos \left(\frac{t}{4}\right)+C$

## Scoring (d)

- For $\begin{array}{ll}u=t+5 & d v=\cos \left(\frac{t}{4}\right) d t \\ d u=d t & v=4 \sin \left(\frac{t}{4}\right)\end{array}$ 
    - $u$ and $dv$; 1 point

- For $\int(t+5) \cos \left(\frac{t}{4}\right) d t=4(t+5) \sin \left(\frac{t}{4}\right)-\int 4 \sin \left(\frac{t}{4}\right) d t$
    - $uv-\int v du$; 1 point

- For $=4(t+5) \sin \left(\frac{t}{4}\right)+16 \cos \left(\frac{t}{4}\right)+C$
    - Answer; 1 point

## Scoring notes (d)

- The first and second points are earned with an implied $u$ and $d v$ in the presence of $4(t+5) \sin \left(\frac{t}{4}\right)-\int 4 \sin \left(\frac{t}{4}\right) d t$ or a mathematically equivalent expression.

- The tabular method may be used to show integration by parts. In this case, the first point is earned by columns (labeled or unlabeled) that begin with $t+5$ and $\cos \left(\frac{t}{4}\right)$. The second point is earned for $4(t+5) \sin \left(\frac{t}{4}\right)-\int 4 \sin \left(\frac{t}{4}\right) d t$ or a mathematically equivalent expression.

- The third point is eamed only for an expression mathematically equivalent to $4(t+5) \sin \left(\frac{t}{4}\right)+16 \cos \left(\frac{t}{4}\right)+C$ (such as $4 t \sin \left(\frac{t}{4}\right)+20 \sin \left(\frac{t}{4}\right)+16 \cos \left(\frac{t}{4}\right)+C$ ) in the presence of correct supporting work.

- To earn the third point a response must have a final answer that includes a constant of integration.

- Alternate solution:

$$
\begin{aligned}
& \int(t+5) \cos \left(\frac{t}{4}\right) d t=\int t \cos \left(\frac{t}{4}\right) d t+\int 5 \cos \left(\frac{t}{4}\right) d t \\
& u=t \quad d v=\cos \left(\frac{t}{4}\right) d t \\
& d u=d t \quad v=4 \sin \left(\frac{t}{4}\right) \\
& \int t \cos \left(\frac{t}{4}\right) d t+\int 5 \cos \left(\frac{t}{4}\right) d t=4 t \sin \left(\frac{t}{4}\right)-\int 4 \sin \left(\frac{t}{4}\right) d t+\int 5 \cos \left(\frac{t}{4}\right) d t \\
& =4 t \sin \left(\frac{t}{4}\right)+16 \cos \left(\frac{t}{4}\right)+20 \sin \left(\frac{t}{4}\right)+C
\end{aligned}
$$

- A response can earn the first and second points for correctly applying integration by parts to $\int t \cos \left(\frac{t}{4}\right) d t$. The tabular method may be used to show integration by parts. The third point is earned for the correct answer.

Total for part (d) 3 points

}\SetValue{Answer}{%

}
\ProcessDATA


