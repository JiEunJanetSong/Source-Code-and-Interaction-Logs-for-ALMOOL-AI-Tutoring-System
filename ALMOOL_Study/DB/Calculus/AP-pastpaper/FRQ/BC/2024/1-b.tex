\SetValue{SectionAB}{}\SetValue{MainChapter}{}\SetValue{SubChapter}{}\SetValue{Contents}{%

\begin{center}
  \begin{tabular}{|c|c|c|c|c|}
      \hline
      $t$  (minutes) & 0 & 3 & 7 & 12 \\ 
      \hline
      $C(t)$  (degrees Celsius) & 100 & 85 & 69 & 55 \\
      \hline
  \end{tabular}
\end{center}

The temperature of coffee in a cup at time $t$ minutes is modeled by a decreasing differentiable function $C$, where $C(t)$ is measured in degrees Celsius. For $0 \leq t \leq 12$, selected values of $C(t)$ are given in the table shown.

(b) Use a left Riemann sum with the three subintervals indicated by the data in the table to approximate the value of $\int_0^{12} C(t) d t$. Interpret the meaning of $\frac{1}{12} \int_0^{12} C(t) d t$ in the context of the problem.
     
}\SetValue{Solution}{%

$\int_0^{12} C(t) d t \approx(3-0) \cdot C(0)+(7-3) \cdot C(3)+(12-7) \cdot C(7)  =3 \cdot 100+4 \cdot 85+5 \cdot 69=985$
   
$\frac{1}{12} \int_0^{12} C(t) d t$ is the average temperature of the coffee (in degrees Celsius) over the interval from $t=0$ to $t=12$.

}\SetValue{Rubric}{%

# General Scoring Notes

The model solution is presented using standard mathematical notation.

Answers (numeric or algebraic) need not be simplified. Answers given as a decimal approximation should be correct to three places after the decimal point. Within each individual free-response question, at most one point is not earned for inappropriate rounding.

| $t$ <br> (minutes) | 0 | 3 | 7 | 12 |
| :---: | :---: | :---: | :---: | :---: |
| $C(t)$ <br> (degrees Celsius) | 100 | 85 | 69 | 55 |

The temperature of coffee in a cup at time $t$ minutes is modeled by a decreasing differentiable function $C$, where $C(t)$ is measured in degrees Celsius. For $0 \leq t \leq 12$, selected values of $C(t)$ are given in the table shown.

## Model Solution (b)

$\int_0^{12} C(t) d t \approx(3-0) \cdot C(0)+(7-3) \cdot C(3)+(12-7) \cdot C(7)  =3 \cdot 100+4 \cdot 85+5 \cdot 69=985$
   
$\frac{1}{12} \int_0^{12} C(t) d t$ is the average temperature of the coffee (in degrees Celsius) over the interval from $t=0$ to $t=12$.

## Scoring (b)

- For $\int_0^{12} C(t) d t \approx(3-0) \cdot C(0)+(7-3) \cdot C(3)+(12-7) \cdot C(7)$
  - Form of left Riemann sum; 1 point 

- For $3 \cdot 100+4 \cdot 85+5 \cdot 69=985$
  - Estimate; 1 point 
   
- For $\frac{1}{12} \int_0^{12} C(t) d t$ is the average temperature of the coffee (in degrees Celsius) over the interval from $t=0$ to $t=12$.
  - Interpretation; 1 point 

## Scoring notes (b)

- Read ``$=$ " as ``$\approx$ " for the first point.

- To earn the first point at least five of the six factors in the Riemann sum must be correct. If any of the six factors is incorrect, the response does not earn the second point.

- A response of $(3-0) \cdot C(0)+(7-3) \cdot C(3)+(12-7) \cdot C(7)$ earns the first point. Values must be pulled from the table to earn the second point.

- A response of $3 \cdot 100+4 \cdot 85+5 \cdot 69$ earns both the first and second points, unless there is a subsequent error in simplification, in which case the response would earn only the first point.

- A completely correct right Riemann sum (e.g., $3 \cdot 85+4 \cdot 69+5 \cdot 55$ ) earns 1 of the first 2 points. An unsupported answer of 806 does not earn either of the first 2 points.

- Units will not affect scoring for the second point.

- To earn the third point the interpretation must include both ``average temperature" and the time interval. The response need not include a reference to units. However, if incorrect units are given in the interpretation, the response does not earn the third point.
Total for part (b) 3 points

}\SetValue{Answer}{%

}
\ProcessDATA


