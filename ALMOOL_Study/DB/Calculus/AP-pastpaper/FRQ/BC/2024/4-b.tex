\SetValue{SectionAB}{}\SetValue{MainChapter}{}\SetValue{SubChapter}{}\SetValue{Contents}{%

\begin{center}		
    \includegraphics[scale=0.4]{APpics/2024BCF04.png}
        \end{center}

The graph of the differentiable function $f$, shown for $-6 \leq x \leq 7$, has a horizontal tangent at $x=-2$ and is linear for $0 \leq x \leq 7$. Let $R$ be the region in the second quadrant bounded by the graph of $f$, the vertical line $x=-6$, and the $x$ - and $y$-axes. Region $R$ has area 12.
       
(b) For the function $g$ defined in part (a), find all values of $x$ in the interval $0 \leq x \leq 6$ at which the graph of $g$ has a critical point. Give a reason for your answer.
 
}\SetValue{AltText}{%

This graph shows the function $f$ defined for $x$ from $-6$ to $6$, which can be divided into two main sections:

1. Left Section $[-6, 0]$  
   - A curve connecting the point $(-6, 0.5)$ to $(0, 2)$. 

   - Visually, it is concave up, with a maximum value (roughly $2 < y < 3$) near $x =-1$.  

   - The gray-shaded region $R$ represents the area between this curve and the $x$-axis from $x=-6$ to $x=0$.

2. Right Section $[0, 6]$  
   - A straight line extending from $(0, 2)$ to $(6, -1)$.  

   - The line has a negative slope, so as $x$ increases, $y$ decreases, reaching $-1$ when $x=6$.

In other words, on the left side, $f$ rises in a concave-up manner and reaches its peak near $x=-2$. From that point to the right, it decreases linearly with a constant negative slope. The region $R$ shown in the figure is bounded by the curve and the $x$-axis, and its area can be determined using integration or similar methods.

}\SetValue{Solution}{%

\begin{align*}
    &g^{\prime}(x)=f(x)\\
    &g^{\prime}(x)=f(x)=0 \Rightarrow x=4\\
    &\text { Therefore, the graph of } g \text { has a critical point at } x=4 \text {. }
    \end{align*}

}\SetValue{Rubric}{%
# General Scoring Notes

The model solution is presented using standard mathematical notation.

Answers (numeric or algebraic) need not be simplified. Answers given as a decimal approximation should be correct to three places after the decimal point. Within each individual free-response question, at most one point is not earned for inappropriate rounding.

The graph of the differentiable function $f$, shown for $-6 \leq x \leq 7$, has a horizontal tangent at $x=-2$ and is linear for $0 \leq x \leq 7$. Let $R$ be the region in the second quadrant bounded by the graph of $f$, the vertical line $x=-6$, and the $x$ - and $y$-axes. Region $R$ has area 12.

## Model Solution (b)

\begin{align*}
    &g^{\prime}(x)=f(x)\\
    &g^{\prime}(x)=f(x)=0 \Rightarrow x=4\\
    &\text { Therefore, the graph of } g \text { has a critical point at } x=4 \text {. }
    \end{align*}

## Scoring (b)

- For $g^{\prime}(x)=f(x)$
    - Fundamental Theorem of Calculus; 1 point
    
- For $g^{\prime}(x)=f(x)=0 \Rightarrow x=4$, Therefore, the graph of  $g$ has a critical point at $x=4$.
    - Answer with reason; 1 point

## Scoring notes (b)

- The first point is earned for explicitly making the connection $g^{\prime}=f$ in this part.

- A response that writes $g^{\prime \prime}=f^{\prime}$ earns the first point but can only earn the second point by reasoning from $f=0$.

- A response that does not earn the first point is eligible to earn the second point with an implied application of the FTC (e.g., ``Because $g^{\prime}(4)=0, x=4$ is a critical point").

- A response that reports any additional critical points in $0<x<6$ does not earn the second point.

- Any presented critical point outside the interval $0<x<6$ will not affect scoring.

Total for part (b) 2 points

}\SetValue{Answer}{%

}
\ProcessDATA


