\SetValue{SectionAB}{}\SetValue{MainChapter}{}\SetValue{SubChapter}{}\SetValue{Contents}{%

The depth of seawater at a location can be modeled by the function $H$ that satisfies the differential equation $\frac{d H}{d t}=\frac{1}{2}(H-1) \cos \left(\frac{t}{2}\right)$, where $H(t)$ is measured in feet and $t$ is measured in hours after noon $(t=0)$. It is known that $H(0)=4$.

(c) Use separation of variables to find $y=H(t)$, the particular solution to the differential equation $\frac{d H}{d t}=\frac{1}{2}(H-1) \cos \left(\frac{t}{2}\right)$ with initial condition $H(0)=4$.

}\SetValue{Solution}{%

\begin{align*}
    & \frac{d H}{H-1} = \frac{1}{2} \cos \left(\frac{t}{2}\right) d t, \\
    & \int \frac{d H}{H-1} = \int \frac{1}{2} \cos \left(\frac{t}{2}\right) d t, \\
    & \Rightarrow \ln |H-1| = \sin \left(\frac{t}{2}\right) + C, \\
    & \ln |4-1| = \sin \left(\frac{0}{2}\right) + C \Rightarrow C = \ln 3. \\
    & \text{Because } H(0) = 4, H > 1, \text{ so } |H-1| = H-1. \\
    & \ln (H-1) = \sin \left(\frac{t}{2}\right) + \ln 3, \\
    & H-1 = e^{\sin (t / 2) + \ln 3} = 3 e^{\sin (t / 2)}, \\
    & H(t) = 1 + 3 e^{\sin (t / 2)}.
\end{align*}

}\SetValue{Rubric}{%

# General Scoring Notes

The model solution is presented using standard mathematical notation.

Answers (numeric or algebraic) need not be simplified. Answers given as a decimal approximation should be correct to three places after the decimal point. Within each individual free-response question, at most one point is not earned for inappropriate rounding.

The depth of seawater at a location can be modeled by the function $H$ that satisfies the differential equation $\frac{d H}{d t}=\frac{1}{2}(H-1) \cos \left(\frac{t}{2}\right)$, where $H(t)$ is measured in feet and $t$ is measured in hours after noon $(t=0)$. It is known that $H(0)=4$.

## Model Solution (a)

\begin{align*}
    & \frac{d H}{H-1} = \frac{1}{2} \cos \left(\frac{t}{2}\right) d t, \\
    & \int \frac{d H}{H-1} = \int \frac{1}{2} \cos \left(\frac{t}{2}\right) d t, \\
    & \Rightarrow \ln |H-1| = \sin \left(\frac{t}{2}\right) + C, \\
    & \ln |4-1| = \sin \left(\frac{0}{2}\right) + C \Rightarrow C = \ln 3. \\
    & \text{Because } H(0) = 4, H > 1, \text{ so } |H-1| = H-1. \\
    & \ln (H-1) = \sin \left(\frac{t}{2}\right) + \ln 3, \\
    & H-1 = e^{\sin (t / 2) + \ln 3} = 3 e^{\sin (t / 2)}, \\
    & H(t) = 1 + 3 e^{\sin (t / 2)}.
\end{align*}


## Scoring (c)

- For $\frac{d H}{H-1} = \frac{1}{2} \cos \left(\frac{t}{2}\right) d t$
    - Separation of variables; 1 point
    
- For $\int \frac{d H}{H-1} = \int \frac{1}{2} \cos \left(\frac{t}{2}\right) d t$
    - One antiderivative; 1 point

- For $\Rightarrow \ln |H-1| = \sin \left(\frac{t}{2}\right) + C$

    - Second antiderivative; 1 point

- For $\ln |4-1| = \sin \left(\frac{0}{2}\right) + C \Rightarrow C = \ln 3.$ \\
    Because $H(0) = 4, H > 1, \text{ so } |H-1| = H-1$. \\
    $\ln (H-1) = \sin \left(\frac{t}{2}\right) + \ln 3$

    - Constant of integration and uses initial condition; 1 point
    
- For $H-1 = e^{\sin (t / 2) + \ln 3} = 3 e^{\sin (t / 2)}$, \\
    $H(t) = 1 + 3 e^{\sin (t / 2)}$.
    - Solves for $H$; 1 point

## Scoring notes (c)

- A response with no separation of variables earns 0 out of 5 points.

- A response that presents $\int \frac{d H}{H-1}=\ln (H-1)$ without absolute value symbols earns that antiderivative point.

- A response with no constant of integration can earn at most the first 3 points.

- A response is eligible for the fourth point only if it has earned the first point and at least 1 of the 2 antiderivative points.

- An eligible response earns the fourth point by correctly including the constant of integration in an equation and substituting 0 for $t$ and 4 for $H$.

- A response is eligible for the fifth point only if it has earned the first 4 points.

- A response earns the fifth point only for an answer of $H(t)=1+3 e^{\sin (t / 2)}$ or a mathematically equivalent expression for $H(t)$ such as $H(t)=1+e^{\sin (t / 2)+\ln 3}$.

- A response does not need to argue that $|H-1|=H-1$ in order to earn the fifth point.

- Special case: A response that presents an incorrect separation of variables of $\frac{1}{2} \cdot \frac{d H}{H-1}=\cos \left(\frac{t}{2}\right) d t$ does not earn the first point or the fifth point but is eligible for the 2 antiderivative points. If the response earns at least 1 of the 2 antiderivative points, then the response is eligible for the fourth point.

Total for part (c) 5 points

}\SetValue{Answer}{%


}
\ProcessDATA


