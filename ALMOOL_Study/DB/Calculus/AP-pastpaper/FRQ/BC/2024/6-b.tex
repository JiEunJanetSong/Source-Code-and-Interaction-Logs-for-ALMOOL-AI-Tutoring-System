\SetValue{SectionAB}{}\SetValue{MainChapter}{}\SetValue{SubChapter}{}\SetValue{Contents}{%

The Maclaurin series for a function $f$ is given by $\sum_{n=1}^{\infty} \frac{(n+1) x^n}{n^2 6^n}$ and converges to $f(x)$ for all $x$ in the interval of convergence. It can be shown that the Maclaurin series for $f$ has a radius of convergence of 6.

(b) It can be shown that $f(-3)=\sum_{n=1}^{\infty} \frac{(n+1)(-3)^n}{n^2 6^n}=\sum_{n=1}^{\infty} \frac{n+1}{n^2}\left(-\frac{1}{2}\right)^n$ and that the first three terms of this series sum to $S_3=-\frac{125}{144}$. Show that $\left|f(-3)-S_3\right|<\frac{1}{50}$.
   
}\SetValue{Solution}{%
 
$f(-3)=\sum_{n=1}^{\infty} \frac{(n+1)(-3)^n}{n^2 6^n}=\sum_{n=1}^{\infty} \frac{(n+1)}{n^2}\left(-\frac{1}{2}\right)^n$ is an alternating series with terms that decrease in magnitude to 0.

By the alternating series error bound, $\sum_{n=1}^3 \frac{n+1}{n^2}\left(-\frac{1}{2}\right)^n=-\frac{125}{144}$ approximates $f(-3)$ with error of at most $\left|\frac{4+1}{4^2}\left(-\frac{1}{2}\right)^4\right|=\frac{5}{256}<\frac{5}{250}=\frac{1}{50}$.
Thus, $\left|f(-3)-S_3\right|<\frac{1}{50}$.

}\SetValue{Rubric}{%

# General Scoring Notes

The model solution is presented using standard mathematical notation.

Answers (numeric or algebraic) need not be simplified. Answers given as a decimal approximation should be correct to three places after the decimal point. Within each individual free-response question, at most one point is not earned for inappropriate rounding.

The Maclaurin series for a function $f$ is given by $\sum_{n=1}^{\infty} \frac{(n+1) x^n}{n^2 6^n}$ and converges to $f(x)$ for all $x$ in the interval of convergence. It can be shown that the Maclaurin series for $f$ has a radius of convergence of 6.

## Model Solution (b)

$f(-3)=\sum_{n=1}^{\infty} \frac{(n+1)(-3)^n}{n^2 6^n}=\sum_{n=1}^{\infty} \frac{(n+1)}{n^2}\left(-\frac{1}{2}\right)^n$ is an alternating series with terms that decrease in magnitude to 0.

By the alternating series error bound, $\sum_{n=1}^3 \frac{n+1}{n^2}\left(-\frac{1}{2}\right)^n=-\frac{125}{144}$ approximates $f(-3)$ with error of at most $\left|\frac{4+1}{4^2}\left(-\frac{1}{2}\right)^4\right|=\frac{5}{256}<\frac{5}{250}=\frac{1}{50}$.
Thus, $\left|f(-3)-S_3\right|<\frac{1}{50}$.

## Scoring (b)

- For By the alternating series error bound, $\sum_{n=1}^3 \frac{n+1}{n^2}\left(-\frac{1}{2}\right)^n=-\frac{125}{144}$ approximates $f(-3)$ with error of at most $\left|\frac{4+1}{4^2}\left(-\frac{1}{2}\right)^4\right|=\frac{5}{256}<\frac{5}{250}=\frac{1}{50}$.
        - Uses fourth term; 1 point

- For Thus, $\left|f(-3)-S_3\right|<\frac{1}{50}$.
        - Verification; 1 point

## Scoring notes (b)

- The first point is earned for correctly using $x=-3$ in the fourth term. (Listing the fourth term as part of a polynomial is not sufficient.) Using $x=-3$ in any term of degree five or higher does not earn this point.

- The expression $\frac{4+1}{4^2}\left(-\frac{1}{2}\right)^4$ carns the first point, but just $\frac{5}{256}$ does not eam the first point.

- A response including the expression $\frac{4+1}{4^2}\left(-\frac{1}{2}\right)^4$ that is subsequently simplified incorrectly earns the first point but not the second.

- To earn the second point the response must state that the series for $f(-3)$ is alternating or that the alternating series error bound is being used.

        - A response of just ``Error $\leq \frac{4+1}{4^2}\left(-\frac{1}{2}\right)^4<\frac{1}{50}{ }^{\text {" }}$ (or any equivalent mathematical expression) earns both points, provided it is accompanied by an indication that the series is alternating.
        
- A response that declares the error is equal to $\frac{5}{256}$ (or any equivalent form of this value) does not earn the second point.

Total for part (b) 2 points

}\SetValue{Answer}{%

}
\ProcessDATA


