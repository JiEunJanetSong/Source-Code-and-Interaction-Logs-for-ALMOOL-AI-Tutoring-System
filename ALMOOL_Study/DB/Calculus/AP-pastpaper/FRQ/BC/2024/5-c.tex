\SetValue{SectionAB}{}\SetValue{MainChapter}{}\SetValue{SubChapter}{}\SetValue{Contents}{%

\begin{tabular}{|c|c|c|c|}
    \hline$x$ & 0 & $\pi$ & $2 \pi$ \\
    \hline$f^{\prime}(x)$ & 5 & 6 & 0 \\
    \hline
    \end{tabular}
    
The function $f$ is twice differentiable for all $x$ with $f(0)=0$. Values of $f^{\prime}$, the derivative of $f$, are given in the table for selected values of $x$.

(c) Use Euler's method, starting at $x=0$ with two steps of equal size, to approximate $f(2 \pi)$. Show the computations that lead to your answer.
       
}\SetValue{Solution}{%

$$
f(\pi) \approx f(0)+\pi f^{\prime}(0)=0+5 \pi=5 \pi
$$

$$
f(2 \pi) \approx f(\pi)+\pi f^{\prime}(\pi)
$$

$\approx 5 \pi+6 \pi=11 \pi$

}\SetValue{Rubric}{%

# General Scoring Notes

The model solution is presented using standard mathematical notation.
Answers (numeric or algebraic) need not be simplified. Answers given as a decimal approximation should be correct to three places after the decimal point. Within each individual free-response question, at most one point is not earned for inappropriate rounding.

\begin{tabular}{|c|c|c|c|}
\hline$x$ & 0 & $\pi$ & $2 \pi$ \\
\hline$f^{\prime}(x)$ & 5 & 6 & 0 \\
\hline
\end{tabular}

The function $f$ is twice differentiable for all $x$ with $f(0)=0$. Values of $f^{\prime}$, the derivative of $f$, are given in the table for selected values of $x$.

## Model Solution (c)

$$
f(\pi) \approx f(0)+\pi f^{\prime}(0)=0+5 \pi=5 \pi
$$

$$
f(2 \pi) \approx f(\pi)+\pi f^{\prime}(\pi)
$$

$\approx 5 \pi+6 \pi=11 \pi$

## Scoring (c)

- For $f(\pi) \approx f(0)+\pi f^{\prime}(0)=0+5 \pi=5 \pi$,
$f(2 \pi) \approx f(\pi)+\pi f^{\prime}(\pi)$
    - Euler's method; 1 point

- For $\approx 5 \pi+6 \pi=11 \pi$
    - Answer; 1 point

## Scoring notes (c)

- To earn the first point a response must demonstrate two Euler's steps, with use of the correct expression for $\frac{d y}{d x}$, and at most one error. If there is an error, the second point is not earned.

- In order to earn the first point, a response that presents a single error in computing the approximation of $f(\pi)$ must import the incorrect value in computing the approximation of $f(2 \pi)$.

- The two Euler's steps may be explicit expressions or may be presented in a table. For example:
\begin{tabular}{c|c|c}
$x$ & $y$ & $\frac{d y}{d x} \cdot \Delta x\left(\right.$ or $\left.\frac{d y}{d x} \cdot \pi\right)$ \\
\hline 0 & 0 & $5 \pi$ \\
$\pi$ & $5 \pi$ & $6 \pi$ \\
$2 \pi$ & $11 \pi$ &
\end{tabular}

- In the presence of a correct answer, a table does not need to be labeled in order to eam both points. In the presence of no answer or an incorrect answer, such a table must be correctly labeled in order to carn the first point.

- Both points are earned for $5 \pi+6 \pi$.

- The response may report the final answer as $(2 \pi, 11 \pi)$.

Total for part (c) 2 points

}\SetValue{Answer}{%

}
\ProcessDATA


