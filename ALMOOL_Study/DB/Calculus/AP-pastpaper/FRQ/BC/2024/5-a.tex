\SetValue{SectionAB}{}\SetValue{MainChapter}{}\SetValue{SubChapter}{}\SetValue{Contents}{%

\begin{tabular}{|c|c|c|c|}
    \hline$x$ & 0 & $\pi$ & $2 \pi$ \\
    \hline$f^{\prime}(x)$ & 5 & 6 & 0 \\
    \hline
    \end{tabular}
    
The function $f$ is twice differentiable for all $x$ with $f(0)=0$. Values of $f^{\prime}$, the derivative of $f$, are given in the table for selected values of $x$.

(a) For $x \geq 0$, the function $h$ is defined by $h(x)=\int_0^x \sqrt{1+\left(f^{\prime}(t)\right)^2} d t$. Find the value of $h^{\prime}(\pi)$. Show the work that leads to your answer.

}\SetValue{Solution}{%

$h^{\prime}(x)=\sqrt{1+\left(f^{\prime}(x)\right)^2}$

$h^{\prime}(\pi)=\sqrt{1+\left(f^{\prime}(\pi)\right)^2}=\sqrt{1+6^2}=\sqrt{37}$
 
}\SetValue{Rubric}{%

# General Scoring Notes

The model solution is presented using standard mathematical notation.
Answers (numeric or algebraic) need not be simplified. Answers given as a decimal approximation should be correct to three places after the decimal point. Within each individual free-response question, at most one point is not earned for inappropriate rounding.

\begin{tabular}{|c|c|c|c|}
\hline$x$ & 0 & $\pi$ & $2 \pi$ \\
\hline$f^{\prime}(x)$ & 5 & 6 & 0 \\
\hline
\end{tabular}

The function $f$ is twice differentiable for all $x$ with $f(0)=0$. Values of $f^{\prime}$, the derivative of $f$, are given in the table for selected values of $x$.

## Model Solution (a)

$h^{\prime}(x)=\sqrt{1+\left(f^{\prime}(x)\right)^2}$

$h^{\prime}(\pi)=\sqrt{1+\left(f^{\prime}(\pi)\right)^2}=\sqrt{1+6^2}=\sqrt{37}$

## Scoring (a)

- For $h^{\prime}(x)=\sqrt{1+\left(f^{\prime}(x)\right)^2}$
    - Fundamental Theorem of Calculus; 1 point

- For $h^{\prime}(\pi)=\sqrt{1+\left(f^{\prime}(\pi)\right)^2}=\sqrt{1+6^2}=\sqrt{37}$
    - Answer; 1 point 

## Scoring notes (a)

- A response of $\sqrt{1+\left(f^{\prime}(\pi)\right)^2}$ earns the first point.

- A response of $\sqrt{1+6^2}$ alone carns both points.

- A response such as $h^{\prime}(x)=\sqrt{1+\left(f^{\prime}(x)\right)^2}=\sqrt{37}$, that equates a variable expression to a numeric value, earns at most 1 of the 2 points.

- A response that equates $h^{\prime}(x)$ or $h^{\prime}(\pi)$ to a derivative of a constant, such as $h^{\prime}(x)=\frac{d}{d x} \int_0^\pi \sqrt{1+\left(f^{\prime}(t)\right)^2} d t$, carns at most 1 of the 2 points.

Total for part (a) 2 points

}\SetValue{Answer}{%

}
\ProcessDATA


