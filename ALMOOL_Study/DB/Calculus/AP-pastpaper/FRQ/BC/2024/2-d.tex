\SetValue{SectionAB}{}\SetValue{MainChapter}{}\SetValue{SubChapter}{}\SetValue{Contents}{%

A particle moving along a curve in the $x y$-plane has position $(x(t), y(t))$ at time $t$ seconds, where $x(t)$ and $y(t)$ are measured in centimeters. It is known that $x^{\prime}(t)=8 t-t^2$ and $y^{\prime}(t)=-t+\sqrt{t^{1.2}+20}$. At time $t=2$ seconds, the particle is at the point $(3,6)$.

(d) For $2 \leq t \leq 8$, the particle remains in the first quadrant. Find all times $t$ in the interval $2 \leq t \leq 8$ when the particle is moving toward the $x$-axis. Give a reason for your answer.

}\SetValue{Solution}{%

Because $y(t)>0$ when $2 \leq t \leq 8$, the particle will be moving toward the $x$-axis when $y^{\prime}(t)<0$. This occurs when 5.222 (or 5.221 ) $<t<8$.

}\SetValue{Rubric}{%

# General Scoring Notes

The model solution is presented using standard mathematical notation.

Answers (numeric or algebraic) need not be simplified. Answers given as a decimal approximation should be correct to three places after the decimal point. Within each individual free-response question, at most one point is not earned for inappropriate rounding.

A particle moving along a curve in the $x y$-plane has position $(x(t), y(t))$ at time $t$ seconds, where $x(t)$ and $y(t)$ are measured in centimeters. It is known that $x^{\prime}(t)=8 t-t^2$ and $y^{\prime}(t)=-t+\sqrt{t^{1.2}+20}$. At time $t=2$ seconds, the particle is at the point $(3,6)$.

## Model Solution (d)

Because $y(t)>0$ when $2 \leq t \leq 8$, the particle will be moving toward the $x$-axis when $y^{\prime}(t)<0$. This occurs when 5.222 (or 5.221 ) $<t<8$.

## Scoring (d)

- For $y^{\prime}(t)<0$. 
    - Consider sign of $y^{\prime}(t)$; 1 point

- For Because $y(t)>0$ when $2 \leq t \leq 8$, the particle will be moving toward the $x$-axis when $y^{\prime}(t)<0$. This occurs when 5.222 (or 5.221 ) $<t<8$.
    - Answer with reason; 1 point

## Scoring notes (d)

- The first point can be earned by stating $y^{\prime}(t)=0, y^{\prime}(t)<0, y^{\prime}(t)>0$, or $t=5.222$ (or 5.221 ). Note: $y^{\prime}(t)$ may be written as $\frac{d y}{d t}$.

- The second point cannot be earned without the first.

- To earn the second point, a response must identify the correct interval (and no additional intervals in $[2,8]$ ) and explicitly state the need for $y^{\prime}(t)<0$. The interval can be open, closed, or half-open.

Total for part (d) 2 points

}\SetValue{Answer}{%

}
\ProcessDATA


