\SetValue{SectionAB}{}\SetValue{MainChapter}{}\SetValue{SubChapter}{}\SetValue{Contents}{%

The function $f$ has derivatives of all orders for all real numbers. It is known that $f(0)=2, f^{\prime}(0)=3$, $f^{\prime \prime}(x)=-f\left(x^2\right)$, and $f^{\prime \prime \prime}(x)=-2 x \cdot f^{\prime}\left(x^2\right)$

(a) Find $f^{(4)}(x)$, the fourth derivative of $f$ with respect to $x$. Write the fourth-degree Taylor polynomial for $f$ about $x=0$. Show the work that leads to your answer.

}\SetValue{Solution}{%

$f^{(4)}(x)=-2 \cdot f^{\prime}\left(x^2\right)+(-2 x) f^{\prime \prime}\left(x^2\right) \cdot 2 x$

$$
\begin{aligned}
& f^{\prime \prime}(0)=-f(0)=-2 \\
& f^{\prime \prime \prime}(0)=-2(0) \cdot f^{\prime}(0)=0 \\
& f^{(4)}(0)=-2 \cdot f^{\prime}(0)+0 \cdot f^{\prime \prime}(0) \cdot 0=-2 \cdot 3+0=-6
\end{aligned}
$$


The fourth-degree Taylor polynomial for $f$ about $x=0$ is

$$
\begin{aligned}
T_4(x) & =2+3 x+\frac{-2}{2!} x^2+\frac{0}{3!} x^3+\frac{-6}{4!} x^4 \\
& =2+3 x-x^2-\frac{1}{4} x^4
\end{aligned}
$$

}\SetValue{Rubric}{%

# General Scoring Notes

The model solution is presented using standard mathematical notation.

Answers (numeric or algebraic) need not be simplified. Answers given as a decimal approximation should be correct to three places after the decimal point. Within each individual free-response question, at most one point is not earned for inappropriate rounding.

The function $f$ has derivatives of all orders for all real numbers. It is known that $f(0)=2, f^{\prime}(0)=3$, $f^{\prime \prime}(x)=-f\left(x^2\right)$, and $f^{\prime \prime \prime}(x)=-2 x \cdot f^{\prime}\left(x^2\right)$.

## Model Solution (a)

$f^{(4)}(x)=-2 \cdot f^{\prime}\left(x^2\right)+(-2 x) f^{\prime \prime}\left(x^2\right) \cdot 2 x$

$$
\begin{aligned}
& f^{\prime \prime}(0)=-f(0)=-2 \\
& f^{\prime \prime \prime}(0)=-2(0) \cdot f^{\prime}(0)=0 \\
& f^{(4)}(0)=-2 \cdot f^{\prime}(0)+0 \cdot f^{\prime \prime}(0) \cdot 0=-2 \cdot 3+0=-6
\end{aligned}
$$


The fourth-degree Taylor polynomial for $f$ about $x=0$ is

$$
\begin{aligned}
T_4(x) & =2+3 x+\frac{-2}{2!} x^2+\frac{0}{3!} x^3+\frac{-6}{4!} x^4 \\
& =2+3 x-x^2-\frac{1}{4} x^4
\end{aligned}
$$

## Scoring (a)

- For $f^{(4)}(x)=-2 \cdot f^{\prime}\left(x^2\right)+(-2 x) f^{\prime \prime}\left(x^2\right) \cdot 2 x$
    - Form of product rule; 1 point
    - $f^{(4)}(x)$; 1 point

- For $f^{(4)}(0)=-2 \cdot f^{\prime}(0)+0 \cdot f^{\prime \prime}(0) \cdot 0=-2 \cdot 3+0=-6$
    - Two terms of polynomial; 1 point

- For $$
\begin{aligned}
T_4(x) & =2+3 x+\frac{-2}{2!} x^2+\frac{0}{3!} x^3+\frac{-6}{4!} x^4 \\
& =2+3 x-x^2-\frac{1}{4} x^4
\end{aligned}
$$
    - Remaining terms; 1 point

## Scoring notes (a)

- The first point is earned for a correct fourth derivative or for $f^{(4)}(x)=-2 \cdot f^{\prime}\left(x^2\right)+(-2 x) f^{\prime \prime}\left(x^2\right)$.

- The second point is earned only for a completely correct expression for $f^{(4)}(x)$.

- A response that earns the first point but not the second may evaluate the presented expression for $f^{(4)}(x)$ at $x=0$ and use the consistent nonzero value in computing the coefficient of $x^4$ in the fourth-degree Taylor polynomial.

- A polynomial that includes a nonzero third-degree term, any terms of degree greater than four, or $+\ldots$ does not earn the fourth point.

Total for part (a) 4 points

}\SetValue{Answer}{%

}
\ProcessDATA


