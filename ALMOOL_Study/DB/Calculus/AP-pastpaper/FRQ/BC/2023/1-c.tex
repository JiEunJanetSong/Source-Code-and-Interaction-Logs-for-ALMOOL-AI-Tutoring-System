\SetValue{SectionAB}{}\SetValue{MainChapter}{}\SetValue{SubChapter}{}\SetValue{Contents}{%

\begin{tabular}{|c|c|c|c|c|c|c|}
    \hline
    $t$ 
    (seconds)
    & 0 & 60 & 90 & 120 & 135 & 150 \\
    \hline 
    $f(t)$  
    (gallons per second)
    & 0 & 0.1 & 0.15 & 0.1 & 0.05 & 0 \\
    \hline
    \end{tabular}

A customer at a gas station is pumping gasoline into a gas tank. The rate of flow of gasoline is modeled by a differentiable function $f$, where $f(t)$ is measured in gallons per second and $t$ is measured in seconds since pumping began. Selected values of $f(t)$ are given in the table.

(c) The rate of flow of gasoline, in gallons per second, can also be modeled by $g(t)=\left(\frac{t}{500}\right) \cos \left(\left(\frac{t}{120}\right)^2\right)$ for $0 \leq t \leq 150$. Using this model, find the average rate of flow of gasoline over the time interval $0 \leq t \leq 150$.
            
}\SetValue{Solution}{%LaTeX

$\begin{aligned} & \frac{1}{150-0} \int_0^{150} g(t) d t \\ & =0.0959967\end{aligned}$

The average rate of flow of gasoline, in gallons per second, is 0.096 (or 0.095).

}\SetValue{Rubric}{%Markdown

# General Scoring Notes

The model solution is presented using standard mathematical notation.
Answers (numeric or algebraic) need not be simplified. Answers given as a decimal approximation should be correct to three places after the decimal point. Within each individual free-response question, at most one point is not earned for inappropriate rounding.

| $t$ <br> (seconds) | 0 | 60 | 90 | 120 | 135 | 150 |
| :---: | :---: | :---: | :---: | :---: | :---: | :---: |
| $f(t)$ <br> (gallons per second) | 0 | 0.1 | 0.15 | 0.1 | 0.05 | 0 |

A customer at a gas station is pumping gasoline into a gas tank. The rate of flow of gasoline is modeled by a differentiable function $f$, where $f(t)$ is measured in gallons per second and $t$ is measured in seconds since pumping began. Selected values of $f(t)$ are given in the table.

## Model Solution (c)

$\begin{aligned} & \frac{1}{150-0} \int_0^{150} g(t) d t \\ & =0.0959967\end{aligned}$

The average rate of flow of gasoline, in gallons per second, is 0.096 (or 0.095).

## Scoring (c)

- For $\frac{1}{150-0} \int_0^{150} g(t) d t$ 
        - Average value formula; 1 point

- For  $0.0959967$ or $0.096$ or $0.095$
        - Answer; 1 point

## Scoring notes (c)

- The exact value of $\frac{1}{150} \int_0^{150} g(t) d t$ is $\frac{12}{125} \sin \left(\frac{25}{16}\right)$.

- A response may present the average value formula in single or multiple steps. For example, the following response earns both points: $\int_0^{150} g(t) d t=14.399504$ so the average rate is 0.0959967.

- A response that presents the average value formula in multiple steps but provides incorrect or incomplete communication (e.g., $\int_0^{150} g(t) d t=\frac{14.399504}{150}=0.0959967$ ) earns 1 out of 2 points.

- A response of $\int_0^{150} g(t) d t=0.0959967$ does not earn either point.

- Degree mode: A response that presents answers obtained by using a calculator in degree mode does not earn the first point it would have otherwise earned. The response is generally eligible for all subsequent points (unless no answer is possible in degree mode or the question is made simpler by using degree mode). In degree mode, $\frac{1}{150} \int_0^{150} g(t) d t=0.149981$ or 0.002618.

Total for part (c) 2 points

}\SetValue{Answer}{%

}
\ProcessDATA


