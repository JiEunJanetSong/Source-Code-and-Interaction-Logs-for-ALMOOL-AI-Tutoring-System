\SetValue{SectionAB}{}\SetValue{MainChapter}{}\SetValue{SubChapter}{}\SetValue{Contents}{%

\begin{center}		
    \includegraphics[scale=0.4]{APpics/2023BCF02.png}
        \end{center}

For $0 \leq t \leq \pi$, a particle is moving along the curve shown so that its position at time $t$ is $(x(t), y(t))$, where $x(t)$ is not explicitly given and $y(t)=2 \sin t$. It is known that $\frac{d x}{d t}=e^{\cos t}$. At time $t=0$, the particle is at position (1,0).

(d) Find the total distance traveled by the particle over the time interval $0 \leq t \leq \pi$. Show the setup for your calculations.

}\SetValue{AltText}{%



}\SetValue{Solution}{%

$$
\begin{aligned}
& \int_0^\pi \sqrt{\left(e^{\cos t}\right)^2+(2 \cos t)^2} d t \\
& =6.034611
\end{aligned}
$$


The total distance traveled by the particle over $0 \leq t \leq \pi$ is 6.035 (or 6.034).

}\SetValue{Rubric}{%

# General Scoring Notes

The model solution is presented using standard mathematical notation.

Answers (numeric or algebraic) need not be simplified. Answers given as a decimal approximation should be correct to three places after the decimal point. Within each individual free-response question, at most one point is not earned for inappropriate rounding.

\begin{center}		
    \includegraphics[scale=0.4]{APpics/2023BCF02.png}
        \end{center}

For $0 \leq t \leq \pi$, a particle is moving along the curve shown so that its position at time $t$ is $(x(t), y(t))$, where $x(t)$ is not explicitly given and $y(t)=2 \sin t$. It is known that $\frac{d x}{d t}=e^{\cos t}$. At time $t=0$, the particle is at position (1,0).    

## Model Solution (d)

$$
\begin{aligned}
& \int_0^\pi \sqrt{\left(e^{\cos t}\right)^2+(2 \cos t)^2} d t \\
& =6.034611
\end{aligned}
$$


The total distance traveled by the particle over $0 \leq t \leq \pi$ is 6.035 (or 6.034).

## Scoring (d)

- For $\int_0^\pi \sqrt{\left(e^{\cos t}\right)^2+(2 \cos t)^2} d t $
    - Integral; 1 point

- For $6.034611$ or $6.035$ or $6.034$.
    - Answer; 1 point

## Scoring notes (d)

- The first point is earned for presenting the correct integrand in a definite integral.
- Parentheses errors were assessed in part (b) and, therefore, will not affect the scoring in part (d).
- If the integrand is an incorrect speed function imported from part (b), the response earns the first point and does not earn the second point.
- An unsupported answer of 6.035 (or 6.034 ) does not earn either point.
- Degree mode: In degree mode, the total distance is 10.596835 or 8.536161.

Total for part (d) 2 points

}\SetValue{Answer}{%

}
\ProcessDATA


