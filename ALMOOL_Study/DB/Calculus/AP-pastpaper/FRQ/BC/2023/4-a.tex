\SetValue{SectionAB}{}\SetValue{MainChapter}{}\SetValue{SubChapter}{}\SetValue{Contents}{%

\begin{center}		
    \includegraphics[scale=0.4]{APpics/2023BCF04.png}
        \end{center}

The function $f$ is defined on the closed interval $[-2,8]$ and satisfies $f(2)=1$. The graph of $f^{\prime}$, the derivative of $f$, consists of two line segments and a semicircle, as shown in the figure.

(a) Does $f$ have a relative minimum, a relative maximum, or neither at $x=6$? Give a reason for your answer.

}\SetValue{AltText}{%



}\SetValue{Solution}{%LaTeX

$f^{\prime}(x)>0$ on $(2,6)$ and $f^{\prime}(x)>0$ on $(6,8)$.

$f^{\prime}(x)$ does not change sign at $x=6$, so there is neither a relative maximum nor a relative minimum at this location.

}\SetValue{Rubric}{%Markdown

# General Scoring Notes

The model solution is presented using standard mathematical notation.

Answers (numeric or algebraic) need not be simplified. Answers given as a decimal approximation should be correct to three places after the decimal point. Within each individual free-response question, at most one point is not earned for inappropriate rounding.

\begin{center}		
    \includegraphics[scale=0.4]{APpics/2023BCF04.png}
        \end{center}

The function $f$ is defined on the closed interval $[-2,8]$ and satisfies $f(2)=1$. The graph of $f^{\prime}$, the derivative of $f$, consists of two line segments and a semicircle, as shown in the figure.

## Model Solution (b)

$f^{\prime}(x)>0$ on $(2,6)$ and $f^{\prime}(x)>0$ on $(6,8)$.

$f^{\prime}(x)$ does not change sign at $x=6$, so there is neither a relative maximum nor a relative minimum at this location.

## Scoring (b)

- For $f^{\prime}(x)>0$ on $(2,6)$ and $f^{\prime}(x)>0$ on $(6,8)$. $f^{\prime}(x)$ does not change sign at $x=6$, so there is neither a relative maximum nor a relative minimum at this location.
        - Answer with reason ; 1 point

## Scoring notes (b)

- A response that declares $f^{\prime}(x)$ does not change sign at $x=6$, so neither, is sufficient to earn the point.

- A response does not have to present intervals on which $f^{\prime}(x)$ is positive or negative, but if any are given, they must be correct. 
Any presented intervals may include none, one, or both endpoints.

- A response that declares $f^{\prime}(x)>0$ before and after $x=6$ does not earn the point.

Total for part (a) 1 point

}\SetValue{Answer}{%

}
\ProcessDATA


