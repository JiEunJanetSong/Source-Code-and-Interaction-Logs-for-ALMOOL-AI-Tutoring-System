\SetValue{SectionAB}{}\SetValue{MainChapter}{}\SetValue{SubChapter}{}\SetValue{Contents}{%

\begin{center}		
    \includegraphics[scale=0.4]{APpics/2023BCF04.png}
        \end{center}

The function $f$ is defined on the closed interval $[-2,8]$ and satisfies $f(2)=1$. The graph of $f^{\prime}$, the derivative of $f$, consists of two line segments and a semicircle, as shown in the figure.

(c) Find the value of $\lim _{x \rightarrow 2} \dfrac{6 f(x)-3 x}{x^2-5 x+6}$, or show that it does not exist. Justify your answer.

}\SetValue{AltText}{%



}\SetValue{Solution}{%LaTeX

Because $f$ is differentiable at $x=2, f$ is continuous at $x=2$,

$$
\begin{aligned}
& \text { so } \lim _{x \rightarrow 2} f(x)=f(2)=1 \\
& \lim _{x \rightarrow 2}(6 f(x)-3 x)=6 \cdot 1-3 \cdot 2=0 \\
& \lim _{x \rightarrow 2}\left(x^2-5 x+6\right)=0
\end{aligned}
$$

Because $\lim _{x \rightarrow 2} \frac{6 f(x)-3 x}{x^2-5 x+6}$ is of indeterminate form $\frac{0}{0}$,
L'Hospital's Rule can be applied.

Using L'Hospital's Rule,

$$
\lim _{x \rightarrow 2} \frac{6 f(x)-3 x}{x^2-5 x+6}=\lim _{x \rightarrow 2} \frac{6 f^{\prime}(x)-3}{2 x-5}=\frac{6 \cdot 0-3}{2 \cdot 2-5}=3 .
$$

}\SetValue{Rubric}{%Markdown

# General Scoring Notes

The model solution is presented using standard mathematical notation.

Answers (numeric or algebraic) need not be simplified. Answers given as a decimal approximation should be correct to three places after the decimal point. Within each individual free-response question, at most one point is not earned for inappropriate rounding.

\begin{center}		
    \includegraphics[scale=0.4]{APpics/2023BCF04.png}
        \end{center}

The function $f$ is defined on the closed interval $[-2,8]$ and satisfies $f(2)=1$. The graph of $f^{\prime}$, the derivative of $f$, consists of two line segments and a semicircle, as shown in the figure.

## Model Solution (c)

Because $f$ is differentiable at $x=2, f$ is continuous at $x=2$,

$$
\begin{aligned}
& \text { so } \lim _{x \rightarrow 2} f(x)=f(2)=1 \\
& \lim _{x \rightarrow 2}(6 f(x)-3 x)=6 \cdot 1-3 \cdot 2=0 \\
& \lim _{x \rightarrow 2}\left(x^2-5 x+6\right)=0
\end{aligned}
$$

Because $\lim _{x \rightarrow 2} \frac{6 f(x)-3 x}{x^2-5 x+6}$ is of indeterminate form $\frac{0}{0}$,
L'Hospital's Rule can be applied.

Using L'Hospital's Rule,

$$
\lim _{x \rightarrow 2} \frac{6 f(x)-3 x}{x^2-5 x+6}=\lim _{x \rightarrow 2} \frac{6 f^{\prime}(x)-3}{2 x-5}=\frac{6 \cdot 0-3}{2 \cdot 2-5}=3 .
$$

## Scoring (c)

- For Because $f$ is differentiable at $x=2, f$ is continuous at $x=2$,
       
$$
\begin{aligned}
& \text { so } \lim _{x \rightarrow 2} f(x)=f(2)=1 \\
& \lim _{x \rightarrow 2}(6 f(x)-3 x)=6 \cdot 1-3 \cdot 2=0 \\
& \lim _{x \rightarrow 2}\left(x^2-5 x+6\right)=0
\end{aligned}
$$
        - Limits of numerator and denominator ; 1 point

- Uses L'Hospital's Rule; 1 point

- For Using L'Hospital's Rule,

$$
\lim _{x \rightarrow 2} \frac{6 f(x)-3 x}{x^2-5 x+6}=\lim _{x \rightarrow 2} \frac{6 f^{\prime}(x)-3}{2 x-5}=\frac{6 \cdot 0-3}{2 \cdot 2-5}=3 .
$$
        - Answer; 1 point

## Scoring notes (c)

- The first point is earned by the presentation of two separate limits for the numerator and denominator.

- A response that presents a limit explicitly equal to $\frac{0}{0}$ does not earn the first point.

- The second point is earned by applying L'Hospital's Rule, that is, by presenting at least one correct derivative in the limit of a ratio of derivatives.

- The third point is earned for the correct answer with supporting work.

Total for part (c) 3 points

}\SetValue{Answer}{%

}
\ProcessDATA


