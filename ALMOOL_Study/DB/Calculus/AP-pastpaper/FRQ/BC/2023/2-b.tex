\SetValue{SectionAB}{}\SetValue{MainChapter}{}\SetValue{SubChapter}{}\SetValue{Contents}{%

\begin{center}		
    \includegraphics[scale=0.4]{APpics/2023BCF02.png}
        \end{center}

For $0 \leq t \leq \pi$, a particle is moving along the curve shown so that its position at time $t$ is $(x(t), y(t))$, where $x(t)$ is not explicitly given and $y(t)=2 \sin t$. It is known that $\frac{d x}{d t}=e^{\cos t}$. At time $t=0$, the particle is at position (1,0).

(b) For $0 \leq t \leq \pi$, find the first time $t$ at which the speed of the particle is 1.5 . Show the work that leads to your answer.

}\SetValue{AltText}{%



}\SetValue{Solution}{%

$$
\text { Speed }=\sqrt{\left(\frac{d x}{d t}\right)^2+\left(\frac{d y}{d t}\right)^2}=\sqrt{\left(e^{\cos t}\right)^2+(2 \cos t)^2}
$$


$$
0 \leq t \leq \pi \text { and } \sqrt{\left(e^{\cos t}\right)^2+(2 \cos t)^2}=1.5
$$

$$
\Rightarrow t=1.254472, t=2.358077
$$


The first time at which the speed of the particle is 1.5 is $t=1.254$.


}\SetValue{Rubric}{%

# General Scoring Notes

The model solution is presented using standard mathematical notation.

Answers (numeric or algebraic) need not be simplified. Answers given as a decimal approximation should be correct to three places after the decimal point. Within each individual free-response question, at most one point is not earned for inappropriate rounding.

\begin{center}		
    \includegraphics[scale=0.4]{APpics/2023BCF02.png}
        \end{center}

For $0 \leq t \leq \pi$, a particle is moving along the curve shown so that its position at time $t$ is $(x(t), y(t))$, where $x(t)$ is not explicitly given and $y(t)=2 \sin t$. It is known that $\frac{d x}{d t}=e^{\cos t}$. At time $t=0$, the particle is at position (1,0).    

## Model Solution (b)

$$
\text { Speed }=\sqrt{\left(\frac{d x}{d t}\right)^2+\left(\frac{d y}{d t}\right)^2}=\sqrt{\left(e^{\cos t}\right)^2+(2 \cos t)^2}
$$


$$
0 \leq t \leq \pi \text { and } \sqrt{\left(e^{\cos t}\right)^2+(2 \cos t)^2}=1.5
$$

$$
\Rightarrow t=1.254472, t=2.358077
$$


The first time at which the speed of the particle is 1.5 is $t=1.254$.

## Scoring (b)

- For $\text { Speed }=\sqrt{\left(\frac{d x}{d t}\right)^2+\left(\frac{d y}{d t}\right)^2}=\sqrt{\left(e^{\cos t}\right)^2+(2 \cos t)^2}=1.5$
        - $\sqrt{\left(e^{\cos t}\right)^2+(2 \cos t)^2}=1.5$; 1 point

$t=1.254472$ or $t=1.254$.
        - Answer; 1 point

## Scoring notes (b)

- A response with an implied equation is eligible for both points. For example, a response of ``Speed $=\sqrt{\left(e^{\cos t}\right)^2+(2 \cos t)^2}$ and is first equal to 1.5 at $t=1.254$ " earns both points.

- $\sqrt{\left(\frac{d x}{d t}\right)^2+\left(\frac{d y}{d t}\right)^2}=1.5$ earns the first point. Speed $=1.5$ by itself does not earn the first point.

Both of these responses are eligible to earn the second point.
- A response need not consider the value $t=2.358077$.

- A response of $t=1.254$ alone does not eam either point.

- A response with a parenthesis error(s) in either $\left(e^{\cos t}\right)^2$ or $(2 \cos t)^2$ does not earn the first point but does earn the second point for the correct answer. Note: $\sqrt{\frac{d x^2}{d t}+\frac{d y^2}{d t}}$ is not considered a parenthesis error.

- Degree mode: In degree mode, $\sqrt{\left(e^{\cos t}\right)^2+(2 \cos t)^2}=1.5$ has no solution for $0 \leq t \leq \pi$. A response that finds no time $t$ at which the speed of the particle is 1.5 cannot be assumed to be working in degree mode.

Total for part (b) 2 points

}\SetValue{Answer}{%

}
\ProcessDATA


