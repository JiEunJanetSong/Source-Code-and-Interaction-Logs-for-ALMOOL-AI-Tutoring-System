\SetValue{SectionAB}{}\SetValue{MainChapter}{}\SetValue{SubChapter}{}\SetValue{Contents}{%

A bottle of milk is taken out of a refrigerator and placed in a pan of hot water to be warmed. The increasing function $M$ models the temperature of the milk at time $t$, where $M(t)$ is measured in degrees Celsius $\left({ }^{\circ} \mathrm{C}\right)$ and $t$ is the number of minutes since the bottle was placed in the pan. $M$ satisfies the differential equation $\frac{d M}{d t}=\frac{1}{4}(40-M)$. At time $t=0$, the temperature of the milk is $5^{\circ} \mathrm{C}$. It can be shown that $M(t)<40$ for all values of $t$.

(c) Write an expression for $\frac{d^2 M}{d t^2}$ in terms of $M$. Use $\frac{d^2 M}{d t^2}$ to determine whether the approximation from part (b) is an underestimate or an overestimate for the actual value of $M(2)$. Give a reason for your answer.

}\SetValue{Solution}{%LaTeX

$\frac{d^2 M}{d t^2}=-\frac{1}{4} \frac{d M}{d t}=-\frac{1}{4}\left(\frac{1}{4}(40-M)\right)=-\frac{1}{16}(40-M)$

Because $M(t)<40, \frac{d^2 M}{d t^2}<0$, so the graph of $M$ is concave down. Therefore, the tangent line approximation of $M(2)$ is an overestimate.

}\SetValue{Rubric}{%Markdown

# General Scoring Notes

The model solution is presented using standard mathematical notation.

Answers (numeric or algebraic) need not be simplified. Answers given as a decimal approximation should be correct to three places after the decimal point. Within each individual free-response question, at most one point is not earned for inappropriate rounding.

A bottle of milk is taken out of a refrigerator and placed in a pan of hot water to be warmed. The increasing function $M$ models the temperature of the milk at time $t$, where $M(t)$ is measured in degrees Celsius $\left({ }^{\circ} \mathrm{C}\right)$ and $t$ is the number of minutes since the bottle was placed in the pan. $M$ satisfies the differential equation $\frac{d M}{d t}=\frac{1}{4}(40-M)$. At time $t=0$, the temperature of the milk is $5^{\circ} \mathrm{C}$. It can be shown that $M(t)<40$ for all values of $t$.


## Model Solution (c)

$\frac{d^2 M}{d t^2}=-\frac{1}{4} \frac{d M}{d t}=-\frac{1}{4}\left(\frac{1}{4}(40-M)\right)=-\frac{1}{16}(40-M)$

Because $M(t)<40, \frac{d^2 M}{d t^2}<0$, so the graph of $M$ is concave down. Therefore, the tangent line approximation of $M(2)$ is an overestimate.

## Scoring (c)

- For $\frac{d^2 M}{d t^2}=-\frac{1}{4} \frac{d M}{d t}=-\frac{1}{4}\left(\frac{1}{4}(40-M)\right)=-\frac{1}{16}(40-M)$
    - $\frac{d^2 M}{d t^2}$; 1 point

- For ``Because $M(t)<40, \frac{d^2 M}{d t^2}<0$, so the graph of $M$ is concave down. Therefore, the tangent line approximation of $M(2)$ is an overestimate."
    - Overestimate with reason; 1 point

## Scoring notes (c)

- The first point is earned for either $\frac{d^2 M}{d t^2}=-\frac{1}{4}\left(\frac{1}{4}(40-M)\right)$ or $\frac{d^2 M}{d t^2}=-\frac{1}{16}(40-M)$ (or equivalent). A response that presents any subsequent simplification error does not earn the second point.

- A response that presents an expression for $\frac{d^2 M}{d t^2}$ in terms of $\frac{d M}{d t}$ but fails to continue to an expression in terms of $M$ (i.e., $\frac{d^2 M}{d t^2}=-\frac{1}{4} \frac{d M}{d t}$ ) does not earn the first point but is eligible for the second point.

- If the response presents an expression for $\frac{d^2 M}{d t^2}$ that is incorrect, the response is eligible for the second point only if the expression is a nonconstant linear function that is negative for $5<M<40$. 

    - Special case: A response that presents $\frac{d^2 M}{d t^2}=\frac{1}{16}(40-M)$ does not earn the first point but is eligible to earn the second point for a consistent answer and reason.

- To earn the second point a response must include $\frac{d^2 M}{d t^2}<0$, or $\frac{d M}{d t}$ is decreasing, or the graph of $M$ is concave down, as well as the conclusion that the approximation is an overestimate.

- A response that presents an argument based on $\frac{d^2 M}{d t^2}$ or concavity at a single point does not earn the second point.

Total for part (c) 2 points

}\SetValue{Answer}{%

}
\ProcessDATA


