\SetValue{SectionAB}{}\SetValue{MainChapter}{}\SetValue{SubChapter}{}\SetValue{Contents}{%

\begin{tabular}{|c|c|c|c|c|c|c|}
    \hline
    $t$ 
    (seconds)
    & 0 & 60 & 90 & 120 & 135 & 150 \\
    \hline 
    $f(t)$  
    (gallons per second)
    & 0 & 0.1 & 0.15 & 0.1 & 0.05 & 0 \\
    \hline
    \end{tabular}

A customer at a gas station is pumping gasoline into a gas tank. The rate of flow of gasoline is modeled by a differentiable function $f$, where $f(t)$ is measured in gallons per second and $t$ is measured in seconds since pumping began. Selected values of $f(t)$ are given in the table.

(b) Must there exist a value of $c$, for $60<c<120$, such that $f^{\prime}(c)=0$? Justify your answer.

}\SetValue{Solution}{%LaTeX

$f$ is differentiable. $\Rightarrow f$ is continuous on $[60,120]$.

$$
\frac{f(120)-f(60)}{120-60}=\frac{0.1-0.1}{60}=0
$$


By the Mean Value Theorem, there must exist a $c$, for $60<c<120$, such that $f^{\prime}(c)=0$.

}\SetValue{Rubric}{%Markdown

# General Scoring Notes

The model solution is presented using standard mathematical notation.
Answers (numeric or algebraic) need not be simplified. Answers given as a decimal approximation should be correct to three places after the decimal point. Within each individual free-response question, at most one point is not earned for inappropriate rounding.

| $t$ <br> (seconds) | 0 | 60 | 90 | 120 | 135 | 150 |
| :---: | :---: | :---: | :---: | :---: | :---: | :---: |
| $f(t)$ <br> (gallons per second) | 0 | 0.1 | 0.15 | 0.1 | 0.05 | 0 |

A customer at a gas station is pumping gasoline into a gas tank. The rate of flow of gasoline is modeled by a differentiable function $f$, where $f(t)$ is measured in gallons per second and $t$ is measured in seconds since pumping began. Selected values of $f(t)$ are given in the table.

## Model Solution (b)

$f$ is differentiable. $\Rightarrow f$ is continuous on $[60,120]$.

$$
\frac{f(120)-f(60)}{120-60}=\frac{0.1-0.1}{60}=0
$$


By the Mean Value Theorem, there must exist a $c$, for $60<c<120$, such that $f^{\prime}(c)=0$.

## Scoring (b)

- For $f$ is differentiable. $\Rightarrow f$ is continuous on $[60,120]$.
    -  $f(120)-f(60)=0$; 1 point

- For $\frac{f(120)-f(60)}{120-60}=\frac{0.1-0.1}{60}=0$, By the Mean Value Theorem, there must exist a $c$, for $60<c<120$, such that $f^{\prime}(c)=0$.
    - Answer with justification; 1 point

## Scoring notes (b)

- To earn the first point a response must present either $f(120)-f(60)=0,0.1-0.1=0$ (perhaps as the numerator of a quotient), or $f(60)=f(120)$.

- To earn the second point a response must:

    - have earned the first point,

    - state that $f$ is continuous because $f$ is differentiable (or equivalent), and

    - answer ``yes" in some way.

- A response may reference either the Mean Value Theorem or Rolle's Theorem.

- A response that references the Intermediate Value Theorem cannot earn the second point.

Total for part (b) 2 points

}\SetValue{Answer}{%

}
\ProcessDATA


