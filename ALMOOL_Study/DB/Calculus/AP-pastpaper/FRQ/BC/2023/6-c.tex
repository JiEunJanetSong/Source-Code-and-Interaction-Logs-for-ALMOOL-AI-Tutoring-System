\SetValue{SectionAB}{}\SetValue{MainChapter}{}\SetValue{SubChapter}{}\SetValue{Contents}{%

The function $f$ has derivatives of all orders for all real numbers. It is known that $f(0)=2, f^{\prime}(0)=3$, $f^{\prime \prime}(x)=-f\left(x^2\right)$, and $f^{\prime \prime \prime}(x)=-2 x \cdot f^{\prime}\left(x^2\right)$

(c) Let $g$ be the function such that $g(0)=4$ and $g^{\prime}(x)=e^x f(x)$. Write the second-degree Taylor polynomial for $g$ about $x=0$.

}\SetValue{Solution}{%

$g^{\prime \prime}(x)=e^x \cdot f(x)+e^x \cdot f^{\prime}(x)$

$$
\begin{aligned}
& g^{\prime}(0)=e^0 \cdot f(0)=2 \\
& g^{\prime \prime}(0)=e^0 \cdot f(0)+e^0 \cdot f^{\prime}(0)=2+3=5
\end{aligned}
$$

The second-degree Taylor polynomial for $g$ about $x=0$ is $T_2(x)=4+2 x+\frac{5}{2} x^2$

}\SetValue{Rubric}{%

# General Scoring Notes

The model solution is presented using standard mathematical notation.

Answers (numeric or algebraic) need not be simplified. Answers given as a decimal approximation should be correct to three places after the decimal point. Within each individual free-response question, at most one point is not earned for inappropriate rounding.

The function $f$ has derivatives of all orders for all real numbers. It is known that $f(0)=2, f^{\prime}(0)=3$, $f^{\prime \prime}(x)=-f\left(x^2\right)$, and $f^{\prime \prime \prime}(x)=-2 x \cdot f^{\prime}\left(x^2\right)$.

## Model Solution (c)

$g^{\prime \prime}(x)=e^x \cdot f(x)+e^x \cdot f^{\prime}(x)$

$$
\begin{aligned}
& g^{\prime}(0)=e^0 \cdot f(0)=2 \\
& g^{\prime \prime}(0)=e^0 \cdot f(0)+e^0 \cdot f^{\prime}(0)=2+3=5
\end{aligned}
$$


The second-degree Taylor polynomial for $g$ about $x=0$ is $T_2(x)=4+2 x+\frac{5}{2} x^2$

## Scoring (c)

- For $g^{\prime \prime}(x)=e^x \cdot f(x)+e^x \cdot f^{\prime}(x)$
    - $g^{\prime \prime}(x)$; 1 point
- For $g^{\prime}(0)=e^0 \cdot f(0)=2$, $g^{\prime \prime}(0)=e^0 \cdot f(0)+e^0 \cdot f^{\prime}(0)=2+3=5$
    - First two terms of polynomial; 1 point

- For The second-degree Taylor polynomial for $g$ about $x=0$ is $T_2(x)=4+2 x+\frac{5}{2} x^2$
    - Taylor polynomial; 1 point
    
## Scoring notes (c)

- The first point is earned for $g^{\prime \prime}(x)=e^x \cdot f(x)+e^x \cdot f^{\prime}(x), g^{\prime \prime}(0)=e^0 \cdot f(0)+e^0 \cdot f^{\prime}(0)$, or $g^{\prime \prime}(0)=f(0)+f^{\prime}(0)$.

- A presented polynomial of the form $4+2 x+a x^2$ earns the second point with or without any supporting work for the first two terms.

- A response that earned neither the first nor the second point only earns the third point for a polynomial of the form $a+b x+\frac{c}{2} x^2$, where $c \neq 0$ is declared to be $g^{\prime \prime}(0)$.

- A presented polynomial with no support for the coefficient of $x^2$ does not earn the third point.

- A polynomial that includes any terms of degree greater than two, or $+\ldots$, does not earn the third point.
- Alternate solution:

$$
\begin{aligned}
& e^x=1+x+\frac{x^2}{2}+\cdots \\
& e^x f(x)=\left(1+x+\frac{x^2}{2}+\cdots\right)\left(2+3 x-x^2+\cdots\right)=2+5 x+\cdots \\
& g(x)=\int e^x f(x) d x=C+2 x+\frac{5}{2} x^2+\cdots \\
& g(0)=4 \Rightarrow C=4 \\
& g(x) \approx 4+2 x+\frac{5}{2} x^2
\end{aligned}
$$

    - A response that is using this alternate solution method earns the first point for $e^x f(x)=2+5 x+\cdots$, the second point for any two correct terms in a second-degree polynomial, and the third point for a completely correct second-degree Taylor polynomial with supporting work.

    - Note: There is not enough information to conclude that $f(x)$ is equal to its Maclaurin series on its interval of convergence. The second and third lines of the alternate solution are being accepted as identifications of the Maclaurin series for $e^x f(x)$ and $g(x)$, respectively.

Total for part (c) 3 points

}\SetValue{Answer}{%

}
\ProcessDATA


