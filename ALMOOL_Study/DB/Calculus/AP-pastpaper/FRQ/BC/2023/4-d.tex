\SetValue{SectionAB}{}\SetValue{MainChapter}{}\SetValue{SubChapter}{}\SetValue{Contents}{%

\begin{center}		
    \includegraphics[scale=0.4]{APpics/2023BCF04.png}
        \end{center}

The function $f$ is defined on the closed interval $[-2,8]$ and satisfies $f(2)=1$. The graph of $f^{\prime}$, the derivative of $f$, consists of two line segments and a semicircle, as shown in the figure.

(d) Find the absolute minimum value of $f$ on the closed interval $[-2,8]$. Justify your answer.

}\SetValue{AltText}{%



}\SetValue{Solution}{%LaTeX

$f^{\prime}(x)=0 \Rightarrow x=-1, x=2, x=6$

The function $f$ is continuous on $[-2,8]$, so the candidates for the location of an absolute minimum for $f$ are $x=-2, x=-1$, $x=2, x=6$, and $x=8$.
\begin{tabular}{c|c}
$x$ & $f(x)$ \\
\hline-2 & 3 \\
-1 & 4 \\
2 & 1 \\
6 & $7-\pi$ \\
8 & $11-2 \pi$
\end{tabular}

The absolute minimum value of $f$ is $f(2)=1$.

}\SetValue{Rubric}{%Markdown

# General Scoring Notes

The model solution is presented using standard mathematical notation.

Answers (numeric or algebraic) need not be simplified. Answers given as a decimal approximation should be correct to three places after the decimal point. Within each individual free-response question, at most one point is not earned for inappropriate rounding.

\begin{center}		
    \includegraphics[scale=0.4]{APpics/2023BCF04.png}
        \end{center}

The function $f$ is defined on the closed interval $[-2,8]$ and satisfies $f(2)=1$. The graph of $f^{\prime}$, the derivative of $f$, consists of two line segments and a semicircle, as shown in the figure.

## Model Solution (d)

$f^{\prime}(x)=0 \Rightarrow x=-1, x=2, x=6$

The function $f$ is continuous on $[-2,8]$, so the candidates for the location of an absolute minimum for $f$ are $x=-2, x=-1$, $x=2, x=6$, and $x=8$.
\begin{tabular}{c|c}
$x$ & $f(x)$ \\
\hline-2 & 3 \\
-1 & 4 \\
2 & 1 \\
6 & $7-\pi$ \\
8 & $11-2 \pi$
\end{tabular}

The absolute minimum value of $f$ is $f(2)=1$.

## Scoring (d)

- Considers $f'(x)=0$; 1 point
- Justification ; 1 point
- Answer ; 1 point

## Scoring notes (d)

- To earn the first point a response must state $f^{\prime}=0$ or equivalent. Listing the zeros of $f^{\prime}$ is not sufficient.

- A response that presents any error in evaluating $f$ at any critical point or endpoint will not earn the justification point.

- A response need not present the value of $f(-1)$ provided $x=-1$ is eliminated because it is the location of a local maximum. A response need not present the value of $f(6)$ provided $x=6$ is eliminated by reference to part (a) or eliminated through analysis.

- A response need not present the value of $f(8)$ provided there is a presentation that argues $f^{\prime}(x) \geq 0$ for $x>2$ and, therefore, $f(8)>f(2)$.

- A response that does not consider both endpoints does not earn the justification point.

- The answer point is earned only for indicating that the minimum value is 1. It is not earned for noting that the minimum occurs at $x=2$.
Total for part (d) 3 points

}\SetValue{Answer}{%

}
\ProcessDATA


