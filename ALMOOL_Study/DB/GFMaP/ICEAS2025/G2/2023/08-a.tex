\SetValue{SectionAB}{}\SetValue{MainChapter}{}\SetValue{SubChapter}{}\SetValue{Contents}{%

\begin{center}		
    \includegraphics[scale=0.12]{GFMaP-pics/2023-G2-08.png}
        \end{center}

In the diagram, $M N R$ is right angle triangle. $|\overline{M N}|=15 \mathrm{~m},|\overline{M R}|=10 \sqrt{3} \mathrm{~m}$ and $\angle M Q N=72^{\circ}$. Calculate, correct to the nearest whole number:

(a) the value of the angle marked $x$;

}\SetValue{Translation}{%

}\SetValue{Concept}{%



}\SetValue{GPTSolution}{%


}\SetValue{AltText}{%

The diagram depicts a right-angled triangle $\triangle MNR$ with a right angle at $N$. The points are labeled $M$ at the left vertex, $N$ at the bottom right vertex, and $R$ at the top vertex, forming the hypotenuse $MR$. The side $MN$ is the base, measuring 15 meters, while the hypotenuse $MR$ measures $10\sqrt{3}$ meters. There is another point $Q$ on the side $NR$ forming $\angle MQN = 72^\circ$. The angle $\angle NMR$ is labeled $x=12^{\circ}$. $QR=3.78$. Area of the ${\triangle MRQ=QR \times MN} \div 2$. 
angle $QMN$ is $18$ degree. $x$ is $12$ degree. angle $MRN=MRQ=60$ degree. $|\overline{Q R}| = 3.7865 \text{ m}\approx 4$

$\angle R M Q + \angle Q M N = 30^{\circ}$

}\SetValue{Solution}{%

(a) From $\triangle M N Q$

    $\angle N M Q + 72^{\circ} + 90^{\circ} = 180^{\circ}$

    $\angle  N M Q  + 162^{\circ} = 180^{\circ}$

    $\angle  N M Q  = 180^{\circ} - 162^{\circ}$

    $\angle  N M Q  = 18^{\circ}$

    Let $\angle M R N  = \theta$. From NMR

    $\sin \theta = \dfrac{15}{10 \sqrt{3}}$

    $\theta = \sin^{-1}\left(\dfrac{15}{10 \sqrt{3}}\right)$

    $\theta = 60^{\circ}$

    $\angle  R M Q + 18^{\circ} + x + 60^{\circ} + 90^{\circ} = 180^{\circ}$

    $x + 168^{\circ} = 180^{\circ}$

    $x = 180^{\circ} - 168^{\circ}$

    $x = 12^{\circ}$

}\SetValue{Answer}{%

12

}
\ProcessDATA

