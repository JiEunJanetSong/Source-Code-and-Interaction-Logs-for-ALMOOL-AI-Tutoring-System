\SetValue{SectionAB}{}\SetValue{MainChapter}{}\SetValue{SubChapter}{}\SetValue{Contents}{%

In the frustrum of a cone, the bottom diameter is thrice the top diameter.

(a) If the height of the frustrum is $12 \mathrm{~cm}$. Calculate the height of the whole cone.

}\SetValue{Translation}{%

}\SetValue{Hint}{%



}\SetValue{GPTSolution}{%

}\SetValue{AltText}{%

The diagram shows a frustum of a cone. 
1. The frustum has a circular top and bottom with different diameters.
2. The diameter of the bottom circle is three times the diameter of the top circle.
3. The vertical height of the frustum, measured from the center of the bottom circle to the center of the top circle, is labeled as 12 cm.

}\SetValue{Solution}{%

\begin{center}		
    \includegraphics[scale=0.12]{GFMaP-pics/2023-G2-03.png}
        \end{center}

Radius, $r=\dfrac{1}{2} d$

Let the height of the whole cone be h.

(a) $\dfrac{x+12}{3 d / 2}=\dfrac{x}{d / 2}$

    $\Rightarrow \dfrac{2 x+24}{3 d}=\dfrac{2 x}{d}$

    $\Rightarrow 2 x+24=6 x$

    $6 x-2 x=24$

    $\dfrac{4 x}{4}=\dfrac{24}{4}$

    $x=6 \mathrm{~cm}$

    $h=(6+12)=18 \mathrm{~cm}$

}\SetValue{Answer}{%

18

}
\ProcessDATA

