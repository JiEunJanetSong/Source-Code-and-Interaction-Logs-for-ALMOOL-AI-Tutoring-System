\SetValue{SectionAB}{}\SetValue{MainChapter}{}\SetValue{SubChapter}{}\SetValue{Contents}{%

\begin{center}		
    \includegraphics[scale=0.12]{GFMaP-pics/2023-G2-08.png}
        \end{center}

In the diagram, $M N R$ is right angle triangle. $|\overline{M N}|=15 \mathrm{~m},|\overline{M R}|=10 \sqrt{3} \mathrm{~m}$ and $\angle M Q N=72^{\circ}$. Calculate, correct to the nearest whole number:

(b) $|\overline{Q R}|$;

}\SetValue{Translation}{%

}\SetValue{Concept}{%



}\SetValue{GPTSolution}{%


}\SetValue{AltText}{%

The diagram depicts a right-angled triangle $\triangle MNR$ with a right angle at $N$. The points are labeled $M$ at the left vertex, $N$ at the bottom right vertex, and $R$ at the top vertex, forming the hypotenuse $MR$. The side $MN$ is the base, measuring 15 meters, while the hypotenuse $MR$ measures $10\sqrt{3}$ meters. There is another point $Q$ on the side $NR$ forming $\angle MQN = 72^\circ$. The angle $\angle NMR$ is labeled $x=12^{\circ}$. $QR=3.78$. Area of the ${\triangle MRQ=QR \times MN} \div 2$. 
angle $QMN$ is $18$ degree. $x$ is $12$ degree. angle $MRN=MRQ=60$ degree. $|\overline{Q R}| = 3.7865 \text{ m}\approx 4$

$\angle R M Q + \angle Q M N = 30^{\circ}$

}\SetValue{Solution}{%

(b) From $\triangle M Q N$

    $\tan 72^{\circ} = \dfrac{15}{|\overline{N Q}|}$

    $|\overline{N Q}| = \dfrac{15}{\tan 72^{\circ}} = 4.8738 \text{ m}$

    From $\triangle M R N$, using the Pythagorean theorem:

    $(10 \sqrt{3})^2 = 15^2 + |\overline{N R}|^2$

    $|\overline{N R}|^2 = (10 \sqrt{3})^2 - 15^2$

    $|\overline{N R}|^2 = 300 - 225$

    $|\overline{N R}|^2 = 75$

    $|\overline{N R}| = \sqrt{75} = 5 \sqrt{3} \text{ m}$

    $|\overline{N R}| = |\overline{N Q}| + |\overline{Q R}|$

    $5 \sqrt{3} = 4.8738 + |\overline{Q R}|$

    $|\overline{Q R}| = 5 \sqrt{3} - 4.8738$

    $|\overline{Q R}| = 3.7865 \text{ m}$

    $|\overline{Q R}| = 4 \text{ m}$

}\SetValue{Answer}{%

4

}
\ProcessDATA

