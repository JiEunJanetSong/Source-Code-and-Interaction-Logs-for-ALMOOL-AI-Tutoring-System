\SetValue{SectionAB}{}\SetValue{MainChapter}{}\SetValue{SubChapter}{}\SetValue{Contents}{%

\begin{center}		
    \includegraphics[scale=0.12]{GFMaP-pics/2023-G2-09.png}
        \end{center}

In the diargram, $\mathrm{A}, \mathrm{B}, \mathrm{C}, \mathrm{D}$ are points on a circle. $\mathrm{ADE}$ and $\mathrm{BCE}$ are straight lines. $\angle D C E=56^{\circ}, \angle A B E=(5 x-16)^{\circ}$ and $\angle B A E=\left(\frac{3}{2} x+4 y\right)^{\circ}$. If $\angle D C E: \angle C D E=7: 8$; find:

(a) $\angle A E B$;

}\SetValue{Translation}{%

}\SetValue{Hint}{%



}\SetValue{GPTSolution}{%

}\SetValue{AltText}{%

The diagram shows a circle with points $A$, $B$, $C$, and $D$ on its circumference, forming a quadrilateral ABCD, with BC being horizontal. 

Starting from the upper-left corner and moving clockwise, the vertices are $A$, $B$, $C$, and $D$. The angles are marked as follows:

- $\angle B = (5x - 16)^\circ$

- $\angle A = \left(\frac{3}{2}x + 4y \right)^\circ$

In the bottom-right corner of this diagram, Point E outside the circle satisfies:

- two straight line AD and BC meet at E

- $\angle DCE = 56^\circ$.

}\SetValue{Solution}{%

(a) $\triangle A E B$

    $\angle AEB + \angle DCE + \angle CDE = 180^{\circ}$

    $\angle AEB + 56^{\circ} + 64^{\circ} = 180^{\circ}$

    $\angle AEB + 120^{\circ} = 180^{\circ}$

    $\angle AEB = 180^{\circ} - 120^{\circ}$

    $\angle AEB = 60^{\circ}$

}\SetValue{Answer}{%

60

}
\ProcessDATA

