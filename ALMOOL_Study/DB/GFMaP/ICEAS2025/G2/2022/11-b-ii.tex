\SetValue{SectionAB}{}\SetValue{MainChapter}{}\SetValue{SubChapter}{}\SetValue{Contents}{%

\begin{center}		
    \includegraphics[scale=0.4]{GFMaP-pics/2022-G2-11.png}
        \end{center}

In the diagram, $A B C F$ is a straight line, $\overline{B D} / / \overline{C E}, \angle A B D=(2 x+34)^{\circ}$. If $\angle B C D=\frac{1}{3} \angle B D C$ and $\angle F C E=48^{\circ}$, 

find:

(ii) $\angle B D C$.

}\SetValue{Translation}{%

}\SetValue{Hint}{%



}\SetValue{GPTSolution}{%


}\SetValue{AltText}{%

 Diagram Explanation and Solution

The given diagram represents a geometric setup with a straight line $ABCF$ and parallel lines $BD \parallel CE$. Various angles are labeled, and we are given relationships between these angles.


 Key Given Information:
1. $\angle ABD = (2x + 34)^\circ$
2. $BD \parallel CE$
3. $\angle FCE = 48^\circ$
4. $\angle CBD$ is equal to $\angle FCE$ because they are corresponding angles, so:

   $$
   \angle CBD = 48^\circ
   $$

5. Since $ABCF$ is a straight line, we use the linear pair property, which states that:

   $$
   \angle ABD + \angle CBD = 180^\circ
   $$

   Substituting the given values:

   $$
   (2x + 34) + 48 = 180
   $$

   Solving for $x$:

   $$
   2x + 82 = 180
   $$

   $$
   2x = 98
   $$

   $$
   x = 49
   $$

}\SetValue{Solution}{%

$\begin{aligned} \angle B C D+\angle B D C+\angle C B D & =180 \\ \frac{1}{3} \angle B D C+\angle B D C+48 & =180 \\ \frac{4}{3} \angle B D C & =180-48 \\ \frac{4}{3} \angle B D C & =132 \\ \angle B D C & =\frac{132 \times 3}{4} \\ & =99^{\circ}\end{aligned}$

}\SetValue{Answer}{%

99

}
\ProcessDATA

