\SetValue{SectionAB}{}\SetValue{MainChapter}{}\SetValue{SubChapter}{}\SetValue{Contents}{%

\begin{center}		
    \includegraphics[scale=0.4]{GFMaP-pics/2022-G2-04.png}
        \end{center}

In the diagram $\overline{P Q} / / \overline{R S}$. Given that $x: y=2: 7$, find the value of $m$.

}\SetValue{Translation}{%

}\SetValue{Hint}{%



}\SetValue{GPTSolution}{%



}\SetValue{AltText}{%

This diagram shows two parallel lines, $PQ$ and $RS$, with a transversal crossing between them, forming multiple angles.

 Diagram Analysis  
1. Given Elements  
   - $PQ$ and $RS$ are parallel lines.  
   - The transversal intersects these two lines, forming angles $x$, $y$, and $(7m - 16)^\circ$.  

2. Analysis of Angle Relationships  
   - Alternate Interior Angles  
     - When a transversal intersects two parallel lines, the alternate interior angles are always equal.  
     - Therefore, $x = 7m - 16$.  

   - Supplementary Angles  
     - The sum of two angles on a straight line is always $180^\circ$.  
     - Therefore, $y + (7m - 16)^\circ = 180^\circ$.  

 Setting Up an Equation for Problem Solving  
- The relationship $x + y = 180$ can be used.  
- Additionally, by applying the property of supplementary angles, we derive the following equation:  

$$
x = (7m - 16) = 40
$$

Solving this equation will determine the value of $m$, which can then be used to calculate the given angle measures.

 Conclusion  
This diagram presents a problem involving the properties of alternate interior angles and supplementary angles formed by a transversal intersecting parallel lines. Through this, one can practice mathematically calculating angle measures and analyzing their relationships.

}\SetValue{Solution}{%


$$
\begin{aligned}
& x=7 m-16 \\
& x+y=180  \\
& 2a+7a=180  \\
& a=20 \\
& 7 m-16=2a=40 \\
& 7 m=56 \\
& m=8
\end{aligned}
$$

}\SetValue{Answer}{%

8

}
\ProcessDATA

