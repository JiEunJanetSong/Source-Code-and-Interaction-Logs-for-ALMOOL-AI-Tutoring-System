\SetValue{Module}{1}\SetValue{SectionAB}{A}\SetValue{MainChapter}{}\SetValue{SubChapter}{}\SetValue{Contents}{%%
    
Express the differential equation

$$
y^{\prime}=6 y+1
$$

as a total derivative of a potential function $\psi(t, y)$, that is, find $\psi$ satisfying

$$
y^{\prime}=6 y+1 \quad \Leftrightarrow \quad \psi^{\prime}=0
$$


Integrate the equation for the potential function $\psi$ to find all solutions $y$ of Eq. (1.1.14).



}\SetValue{Concept}{%



}\SetValue{AltText}{%



}\SetValue{Solution}{%

Step 1: Rewrite as a first-order linear ODE

We write:

$$
y^{\prime}-6 y=1
$$


This is in the standard form:

$$
y^{\prime}+P(t) y=Q(t), \quad \text { with } P(t)=-6, \quad Q(t)=1
$$


Step 2: Find the integrating factor

$$
\mu(t)=e^{\int-6 d t}=e^{-6 t}
$$


Multiply the entire equation by $\mu(t)=e^{-6 t}$ :

$$
e^{-6 t} y^{\prime}-6 e^{-6 t} y=e^{-6 t} \Rightarrow \frac{d}{d t}\left(e^{-6 t} y\right)=e^{-6 t}
$$


Step 3: Define the potential function $\psi(t, y)$

Let:

$$
\psi(t, y):=e^{-6 t} y \Rightarrow \frac{d}{d t} \psi(t, y(t))=e^{-6 t}
$$


This is not yet of the form $\psi^{\prime}=0$, so we proceed as follows:

We integrate both sides:

$$
\frac{d}{d t}\left(e^{-6 t} y\right)=e^{-6 t} \Rightarrow e^{-6 t} y=\int e^{-6 t} d t=\frac{-1}{6} e^{-6 t}+C
$$


So we define:

$$
\psi(t, y):=e^{-6 t} y+\frac{1}{6} e^{-6 t}=e^{-6 t}\left(y+\frac{1}{6}\right) \Rightarrow \frac{d}{d t} \psi(t, y(t))=0
$$


Hence, we've expressed the differential equation as:

$$
y^{\prime}=6 y+1 \quad \Leftrightarrow \quad \frac{d}{d t}\left(e^{-6 t}\left(y+\frac{1}{6}\right)\right)=0
$$


That is,

$$
\psi(t, y)=e^{-6 t}\left(y+\frac{1}{6}\right)=\mathrm{constant}
$$

Step 4: Solve for $y(t)$

From $\psi(t, y)=C$ :

$$
e^{-6 t}\left(y+\frac{1}{6}\right)=C \Rightarrow y+\frac{1}{6}=C e^{6 t} \Rightarrow y(t)=C e^{6 t}-\frac{1}{6}
$$


Final Answer:

- Potential function:

$$
\psi(t, y)=e^{-6 t}\left(y+\frac{1}{6}\right) \quad \text { so that } \quad \psi^{\prime}=0
$$

- General solution:

$$
y(t)=C e^{6 t}-\frac{1}{6}, \quad C \in \mathbb{R}
$$


}\SetValue{Rubric}{%Markdown



}\SetValue{Hint}{%
Solution Goes Here
}\SetValue{Answer}{%

}
\ProcessDATA



