\SetValue{Module}{1}\SetValue{SectionAB}{A}\SetValue{MainChapter}{}\SetValue{SubChapter}{}\SetValue{Contents}{%%
    
Follow the steps below to solve

$$
y^{\prime}=-3 y+5, \quad y(0)=1
$$

(b) Write the differential equation as a total derivative of a potential function $\psi$.

}\SetValue{Concept}{%



}\SetValue{AltText}{%



}\SetValue{Solution}{%

To write the differential equation $y^{\prime}=-3 y+5$ as a total derivative of a potential function $\psi$, we follow the steps of the integrating factor method.

First, rewrite the differential equation in the standard linear first-order form:

$$
y^{\prime}+3 y=5
$$


From part (a), the integrating factor is $\mu(t)=e^{3 t}$.
Multiply the standard form of the differential equation by the integrating factor:

$$
e^{3 t}\left(y^{\prime}+3 y\right)=e^{3 t}(5)
$$


$$
e^{3 t} y^{\prime}+3 e^{3 t} y=5 e^{3 t}
$$


The left side of this equation is the derivative of the product of the integrating factor and $y(t)$ :

$$
\frac{d}{d t}\left(e^{3 t} y\right)=e^{3 t} y^{\prime}+3 e^{3 t} y
$$


So, the multiplied equation can be written as:

$$
\frac{d}{d t}\left(e^{3 t} y\right)=5 e^{3 t}
$$


To express this equation in the form $\psi^{\prime}=0$, move the term on the right side to the left side:

$$
\frac{d}{d t}\left(e^{3 t} y\right)-5 e^{3 t}=0
$$


Now, we need to find a function $\psi(t, y)$ such that its total derivative with respect to $t$ is the left side of this equation. The left side is $\frac{d}{d t}\left(e^{3 t} y\right)-5 e^{3 t}$.

We can see that $\frac{d}{d t}\left(e^{3 t} y\right)$ is the derivative of $e^{3 t} y$.

We need to find a function whose derivative is $-5 e^{3 t}$. 

This is the antiderivative of $-5 e^{3 t}$ with respect to $t$:

$$
\int-5 e^{3 t} d t=-5 \int e^{3 t} d t=-5\left(\frac{1}{3} e^{3 t}\right)+C=-\frac{5}{3} e^{3 t}+C
$$

Ignoring the constant of integration for defining $\psi$, the function whose derivative is $-5 e^{3 t}$ is $-\frac{5}{3} e^{3 t}$.

Therefore, the expression $\frac{d}{d t}\left(e^{3 t} y\right)-5 e^{3 t}$ is the derivative of $e^{3 t} y-\frac{5}{3} e^{3 t}$.

Let the potential function $\psi(t, y)$ be:

$$
\psi(t, y)=e^{3 t} y-\frac{5}{3} e^{3 t}
$$


With this definition, the equation $\frac{d}{d t}\left(e^{3 t} y\right)-5 e^{3 t}=0$ is equivalent to $\frac{d}{d t}(\psi(t, y))=0$, which is $\psi^{\prime}(t)=0$.

Thus, the differential equation $y^{\prime}=-3 y+5$ can be written as a total derivative of the potential function $\psi(t, y)=e^{3 t} y-\frac{5}{3} e^{3 t}$, such that $\psi^{\prime}=0$.

The final answer is $\psi(t, y)=e^{3 t} y-\frac{5}{3} e^{3 t}$.
}\SetValue{Rubric}{%Markdown



}\SetValue{Hint}{%
Solution Goes Here
}\SetValue{Answer}{%

}
\ProcessDATA



