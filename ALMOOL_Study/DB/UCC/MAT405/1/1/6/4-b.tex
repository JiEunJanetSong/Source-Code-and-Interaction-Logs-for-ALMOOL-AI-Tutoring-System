\SetValue{Module}{1}\SetValue{SectionAB}{A}\SetValue{MainChapter}{}\SetValue{SubChapter}{}\SetValue{Contents}{%%
    
Follow the steps below to find all solutions of

$$
y^{\prime}=-4 y+2
$$

(b) Write the equations as a total derivative of a function $\psi$, that is

$$
y^{\prime}=-4 y+2 \Leftrightarrow \psi^{\prime}=0.
$$

}\SetValue{Concept}{%



}\SetValue{AltText}{%



}\SetValue{Solution}{%

To write the equation $y^{\prime}=-4 y+2$ as a total derivative of a function $\psi$ such that $\psi^{\prime}=0$, we first rewrite the given differential equation in the standard linear first-order form and multiply by the integrating factor found in the previous step.

The differential equation is $y^{\prime}=-4 y+2$.
Rearranging it to the standard form $y^{\prime}+P(t) y=Q(t)$, we get:

$$
y^{\prime}+4 y=2
$$


From part (a), the integrating factor is $\mu(t)=e^{4 t}$.
Multiply the standard form of the equation by the integrating factor:

$$
\begin{aligned}
& e^{4 t}\left(y^{\prime}+4 y\right)=e^{4 t}(2) \\
& e^{4 t} y^{\prime}+4 e^{4 t} y=2 e^{4 t}
\end{aligned}
$$


The left side of this equation is the exact derivative of the product of the integrating factor and $y(t)$.
That is, $\frac{d}{d t}\left(e^{4 t} y\right)=e^{4 t} y^{\prime}+4 e^{4 t} y$.
So, the equation becomes:

$$
\frac{d}{d t}\left(e^{4 t} y\right)=2 e^{4 t}
$$


To get the equation in the form $\psi^{\prime}=0$, move the term on the right side to the left side:

$$
\frac{d}{d t}\left(e^{4 t} y\right)-2 e^{4 t}=0
$$


Now, we need to express the left side as the total derivative of some function $\psi(t, y)$. We can see that the first term is the derivative of $e^{4 t} y$. The second term, $-2 e^{4 t}$, is the derivative of $-\int 2 e^{4 t} d t$.

$$
\int-2 e^{4 t} d t=-2 \cdot \frac{1}{4} e^{4 t}+C_2=-\frac{1}{2} e^{4 t}+C_2
$$


So, the left side of the equation is the derivative of $e^{4 t} y-\frac{1}{2} e^{4 t}$ (we can ignore the constant of integration $C_2$ when defining $\psi$ ).

Let $\psi(t, y)=e^{4 t} y-\frac{1}{2} e^{4 t}$.
Then, the total derivative of $\psi$ with respect to $t$ is:

$$
\begin{gathered}
\psi^{\prime}(t)=\frac{d}{d t}\left(e^{4 t} y-\frac{1}{2} e^{4 t}\right)=\frac{d}{d t}\left(e^{4 t} y\right)-\frac{d}{d t}\left(\frac{1}{2} e^{4 t}\right) \\
\psi^{\prime}(t)=\left(e^{4 t} y^{\prime}+4 e^{4 t} y\right)-\left(\frac{1}{2} \cdot 4 e^{4 t}\right) \\
\psi^{\prime}(t)=e^{4 t} y^{\prime}+4 e^{4 t} y-2 e^{4 t}
\end{gathered}
$$


Since we have shown that $y^{\prime}=-4 y+2$ is equivalent to $e^{4 t} y^{\prime}+4 e^{4 t} y-2 e^{4 t}=0$, we can conclude that the original differential equation is equivalent to $\psi^{\prime}(t)=0$, where $\psi(t, y)=e^{4 t} y-\frac{1}{2} e^{4 t}$.

The function $\psi$ is $\psi(t, y)=e^{4 t} y-\frac{1}{2} e^{4 t}$.

The final answer is $\psi(t, y)=e^{4 t} y-\frac{1}{2} e^{4 t}$.


}\SetValue{Rubric}{%Markdown



}\SetValue{Hint}{%
Solution Goes Here
}\SetValue{Answer}{%

}
\ProcessDATA



