\SetValue{Module}{1}\SetValue{SectionAB}{A}\SetValue{MainChapter}{}\SetValue{SubChapter}{}\SetValue{Contents}{%%
    
Find the unit vector in the direction of each of the following vectors:

$$\underline{r}=-10 i+24 j$$


}\SetValue{Concept}{%



}\SetValue{AltText}{%



}\SetValue{Solution}{%

To find the unit vector in the direction of a given vector $\underline{r}$, we follow the same procedure: divide the vector by its magnitude. The formula for the unit vector $\hat{r}$ is:

$$
\hat{r}=\frac{\underline{r}}{\|\underline{r}\|}
$$


The given vector is $\underline{r}=-10 i+24 j$.
First, calculate the magnitude (or norm) of the vector $\underline{r}$ :

$$
\begin{gathered}
\|\underline{r}\|=\sqrt{(-10)^2+(24)^2} \\
\|\underline{r}\|=\sqrt{100+576}
\end{gathered}
$$

$\$\left|\mid \underline{r} \|=\sqrt{676} \$\right.$ To find the square root of 676 , we can recognize that $20^2=400$ and $30^2=900$. Since 676 ends in 6 , its square root must end in 4 or 6 . Testing $26^2: 26 \times 26=676$.

$$
\|\underline{r}\|=26
$$


Now, divide the vector $\underline{r}$ by its magnitude 26 :

$$
\begin{aligned}
& \hat{r}=\frac{-10 i+24 j}{26} \\
& \hat{r}=\frac{-10}{26} i+\frac{24}{26} j
\end{aligned}
$$


Simplify the fractions by dividing the numerator and denominator by their greatest common divisor, which is 2 :

$$
\hat{r}=-\frac{5}{13} i+\frac{12}{13} j
$$


So, the unit vector in the direction of $\underline{r}=-10 i+24 j$ is $\hat{r}=-\frac{5}{13} i+\frac{12}{13} j$.
}\SetValue{Rubric}{%Markdown



}\SetValue{Hint}{%
Solution Goes Here
}\SetValue{Answer}{%

}
\ProcessDATA



